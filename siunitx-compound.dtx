% \iffalse meta-comment
%
% File: siunitx-compound.dtx Copyright (C) 2018-2021 Joseph Wright
%
% It may be distributed and/or modified under the conditions of the
% LaTeX Project Public License (LPPL), either version 1.3c of this
% license or (at your option) any later version.  The latest version
% of this license is in the file
%
%    https://www.latex-project.org/lppl.txt
%
% This file is part of the "siunitx bundle" (The Work in LPPL)
% and all files in that bundle must be distributed together.
%
% The released version of this bundle is available from CTAN.
%
% -----------------------------------------------------------------------
%
% The development version of the bundle can be found at
%
%    https://github.com/josephwright/siunitx
%
% for those people who are interested.
%
% -----------------------------------------------------------------------
%
%<*driver>
\documentclass{l3doc}
\ProvideDocumentCommand\foreign{m}{\textit{#1}}
% The next line is needed so that \GetFileInfo will be able to pick up
% version data
\usepackage{siunitx}
\begin{document}
  \DocInput{\jobname.dtx}
\end{document}
%</driver>
% \fi
%
% \GetFileInfo{siunitx.sty}
%
% \title{^^A
%   \pkg{siunitx-compound} -- Compound numbers and quantities^^A
%   \thanks{This file describes \fileversion,
%     last revised \filedate.}^^A
% }
%
% \author{^^A
%  Joseph Wright^^A
%  \thanks{^^A
%    E-mail:
%    \href{mailto:joseph.wright@morningstar2.co.uk}
%      {joseph.wright@morningstar2.co.uk}^^A
%   }^^A
% }
%
% \date{Released \filedate}
%
% \maketitle
%
% \begin{documentation}
%
% \begin{function}{\siunitx_compound_number:n}
%   \begin{syntax}
%     \cs{siunitx_compound_number:n} \Arg{entries}
%   \end{syntax}
%   Prints a set of numbers in the \meta{entries}, each of which should
%   be given as a \meta{balanced text}. Unlike \cs{siunitx_number_list:nn}, this
%   function may semantically take any form
% \end{function}
%
% \begin{function}{\siunitx_compound_quantity:nn}
%   \begin{syntax}
%     \cs{siunitx_compound_quantity:nn} \Arg{entries} \Arg{unit}
%   \end{syntax}
%   Prints a set of quantities in the \meta{entries}, each of which should
%   be given as a \meta{balanced text}. Unlike \cs{siunitx_quantity_list:nn}, this
%   function may semantically take any form
% \end{function}
%
% \begin{function}{\siunitx_number_list:nn}
%   \begin{syntax}
%     \cs{siunitx_number_list:nn} \Arg{entries}
%   \end{syntax}
%   Prints the list of numbers in the \meta{entries}, each of which should
%   be given as a \meta{balanced text}.
% \end{function}
%
% \begin{function}{\siunitx_quantity_list:nn}
%   \begin{syntax}
%     \cs{siunitx_quantity_list:nn} \Arg{entries} \Arg{unit}
%   \end{syntax}
%   Prints the list of quantities in the \meta{entries}, each of which should
%   be given as a \meta{balanced text}.
% \end{function}
%
% \begin{function}{\siunitx_number_product:n}
%   \begin{syntax}
%     \cs{siunitx_number_product:n} \Arg{entries}
%   \end{syntax}
%   Prints the series of numbers in the \meta{entries}, each of which should
%   be given as a \meta{balanced text}.
% \end{function}
%
% \begin{function}{\siunitx_quantity_product:nn}
%   \begin{syntax}
%     \cs{siunitx_number_product:n} \Arg{entries} \Arg{unit}
%   \end{syntax}
%   Prints the series of quantities in the \meta{entries}, each of which should
%   be given as a \meta{balanced text}.
% \end{function}
%
% \begin{function}{\siunitx_number_range:nn}
%   \begin{syntax}
%     \cs{siunitx_number_range:nn} \Arg{start} \Arg{end}
%   \end{syntax}
%   Prints the range of numbers from the \meta{start} to the \meta{end}.
% \end{function}
%
% \begin{function}{\siunitx_quantity_range:nnn}
%   \begin{syntax}
%     \cs{siunitx_number_range:nn} \Arg{start} \Arg{end} \Arg{unit}
%   \end{syntax}
%   Prints the range of quantities from the \meta{start} to the \meta{end}.
% \end{function}
%
% \begin{variable}
%   {
%     \l_siunitx_list_separator_pair_tl,
%     \l_siunitx_list_separator_tl,
%     \l_siunitx_list_separator_final_tl
%   }
%   Separators for lists of numbers and quantities.
% \end{variable}
%
% \begin{variable}{\l_siunitx_range_phrase_tl}
%   Phrase (or similar) used between limits of a range.
% \end{variable}
%
% \begin{function}{compound-exponents}
%   \begin{syntax}
%     |compound-exponents| = |combine|\verb"|"|combine-bracket|\verb"|"|individual|
%   \end{syntax}
% \end{function}
%
% \begin{function}{compound-final-separator}
%   \begin{syntax}
%     |compound-final-separator| = \meta{text}
%   \end{syntax}
% \end{function}
%
% \begin{function}{compound-pair-separator}
%   \begin{syntax}
%     |compound-pair-separator| = \meta{text}
%   \end{syntax}
% \end{function}
%
% \begin{function}{compound-separator}
%   \begin{syntax}
%     |compound-separator| = \meta{text}
%   \end{syntax}
% \end{function}
%
% \begin{function}{compound-separator-mode}
%   \begin{syntax}
%     |compound-separator-mode| = |number|\verb"|"|text|
%   \end{syntax}
% \end{function}
%
% \begin{function}{compound-units}
%   \begin{syntax}
%     |compound-units| = |bracket|\verb"|"|repeat|\verb"|"|single|
%   \end{syntax}
% \end{function}
%
% \begin{function}{list-exponents}
%   \begin{syntax}
%     |list-exponents| = |combine|\verb"|"|combine-bracket|\verb"|"|individual|
%   \end{syntax}
% \end{function}
%
% \begin{function}{list-final-separator}
%   \begin{syntax}
%     |list-final-separator| = \meta{text}
%   \end{syntax}
% \end{function}
%
% \begin{function}{list-pair-separator}
%   \begin{syntax}
%     |list-pair-separator| = \meta{text}
%   \end{syntax}
% \end{function}
%
% \begin{function}{list-separator}
%   \begin{syntax}
%     |list-separator| = \meta{text}
%   \end{syntax}
% \end{function}
%
% \begin{function}{list-units}
%   \begin{syntax}
%     |list-units| = |bracket|\verb"|"|repeat|\verb"|"|single|
%   \end{syntax}
% \end{function}
%
% \begin{function}{product-exponents}
%   \begin{syntax}
%     |product-exponents| = |combine|\verb"|"|combine-bracket|\verb"|"|individual|
%   \end{syntax}
% \end{function}
%
% \begin{function}{product-mode}
%   \begin{syntax}
%     |product-mode| = |phrase|\verb"|"|choice|
%   \end{syntax}
% \end{function}
%
% \begin{function}{product-phrase}
%   \begin{syntax}
%     |product-phrase| = \meta{text}
%   \end{syntax}
% \end{function}
%
% \begin{function}{product-symbol}
%   \begin{syntax}
%     |product-symbol| = \meta{symbol}
%   \end{syntax}
% \end{function}
%
% \begin{function}{range-exponents}
%   \begin{syntax}
%     |range-exponents| = |combine|\verb"|"|combine-bracket|\verb"|"|individual|
%   \end{syntax}
% \end{function}
%
% \begin{function}{range-phrase}
%   \begin{syntax}
%     |range-phrase| = \meta{text}
%   \end{syntax}
% \end{function}
%
% \begin{function}{range-units}
%   \begin{syntax}
%     |range-units| = |bracket|\verb"|"|repeat|\verb"|"|single|
%   \end{syntax}
% \end{function}
%
% \end{documentation}
%
% \begin{implementation}
%
% Start the \pkg{DocStrip} guards.
%    \begin{macrocode}
%<*package>
%    \end{macrocode}
%
% \section{\pkg{siunitx-compound} implementation}
%
%    \begin{macrocode}
\cs_generate_variant:Nn \keys_set:nn { nx }
%    \end{macrocode}
%
% \subsection{General mechanism}
%
% Identify the internal prefix (\LaTeX3 \pkg{DocStrip} convention): only
% internal material in this \emph{submodule} should be used directly.
%    \begin{macrocode}
%<@@=siunitx_compound>
%    \end{macrocode}
%
% Typesetting lists, ranges and products of numbers or quantities has shared
% features which mean they are best handled using a common mechanism. The aim
% therefore is to abstract out enough of the process such that output-specific
% aspects can be left to separate processes.
%
% \begin{variable}{\l_@@_tmp_fp, \l_@@_tmp_seq, \l_@@_tmp_tl}
%   Scratch space.
%    \begin{macrocode}
\fp_new:N \l_@@_tmp_fp
\seq_new:N \l_@@_tmp_seq
\tl_new:N \l_@@_tmp_tl
%    \end{macrocode}
% \end{variable}
%
% \begin{variable}{\l_@@_first_tl}
%   The first number in the list in internal format.
%    \begin{macrocode}
\tl_new:N \l_@@_first_tl
%    \end{macrocode}
% \end{variable}
%
% \begin{variable}{\l_@@_exp_tl}
%   For storing the combined exponent, if present.
%    \begin{macrocode}
\tl_new:N \l_@@_exp_tl
%    \end{macrocode}
% \end{variable}
%
% \begin{variable}{\l_@@_start_tl, \l_@@_end_tl}
%   Data on the end-of-list.
%    \begin{macrocode}
\tl_new:N \l_@@_start_tl
\tl_new:N \l_@@_end_tl
%    \end{macrocode}
% \end{variable}
%
% \begin{variable}{\l_@@_count_int}
%   Data on the length-of-list.
%    \begin{macrocode}
\int_new:N \l_@@_count_int
%    \end{macrocode}
% \end{variable}
%
% \begin{variable}{\l_@@_unit_tl}
%    \begin{macrocode}
\tl_new:N \l_@@_unit_tl
%    \end{macrocode}
% \end{variable}
%
% \begin{variable}{\l_@@_bracket_close_tl, \l_@@_bracket_open_tl}
%   Purely internal for the present.
%    \begin{macrocode}
\tl_new:N \l_@@_bracket_close_tl
\tl_new:N \l_@@_bracket_open_tl
\tl_set:Nn \l_@@_bracket_open_tl { ( }
\tl_set:Nn \l_@@_bracket_close_tl { ) }
%    \end{macrocode}
% \end{variable}
%
% \begin{variable}
%   {
%     \l_@@_separator_final_tl  ,
%     \l_@@_separator_pair_tl   ,
%     \l_@@_separator_tl        ,
%     \l_@@_separator_text_bool ,
%     \l_@@_exp_bracket_bool    ,
%     \l_@@_exp_combine_bool    ,
%     \l_@@_unit_bracket_bool   ,
%     \l_@@_unit_repeat_bool    ,
%     \l_@@_unit_repeat_bool
%   }
%   List options.
%    \begin{macrocode}
\bool_new:N \l_@@_exp_bracket_bool
\bool_new:N \l_@@_exp_combine_bool
\bool_new:N \l_@@_separator_text_bool
\bool_new:N \l_@@_unit_bracket_bool
\bool_new:N \l_@@_unit_power_bool
\bool_new:N \l_@@_unit_repeat_bool
\keys_define:nn { siunitx }
  {
    compound-exponents .choice: ,
    compound-exponents / combine .code:n =
      {
        \bool_set_false:N \l_@@_exp_bracket_bool
        \bool_set_true:N \l_@@_exp_combine_bool
      } ,
    compound-exponents / combine-bracket .code:n =
      {
        \bool_set_true:N \l_@@_exp_bracket_bool
        \bool_set_true:N \l_@@_exp_combine_bool
      } ,
    compound-exponents / individual .code:n =
      {
        \bool_set_false:N \l_@@_exp_bracket_bool
        \bool_set_false:N \l_@@_exp_combine_bool
      } ,
    compound-final-separator .tl_set:N =
      \l_@@_separator_final_tl ,
    compound-pair-separator .tl_set:N =
      \l_@@_separator_pair_tl ,
    compound-separator .tl_set:N =
      \l_@@_separator_tl ,
    compound-separator-mode .choice: ,
    compound-separator-mode / number .code:n =
      { \bool_set_false:N \l_@@_separator_text_bool } ,
    compound-separator-mode / text .code:n =
      { \bool_set_true:N \l_@@_separator_text_bool } ,
    compound-units .choice: ,
    compound-units / bracket .code:n =
      {
        \bool_set_true:N  \l_@@_unit_bracket_bool
        \bool_set_false:N \l_@@_unit_power_bool
        \bool_set_false:N \l_@@_unit_repeat_bool
      } ,
    compound-units / bracket-power .code:n =
      {
        \bool_set_true:N  \l_@@_unit_bracket_bool
        \bool_set_true:N  \l_@@_unit_power_bool
        \bool_set_false:N \l_@@_unit_repeat_bool
      } ,
    compound-units / power .code:n =
      {
        \bool_set_false:N \l_@@_unit_bracket_bool
        \bool_set_true:N  \l_@@_unit_power_bool
        \bool_set_false:N \l_@@_unit_repeat_bool
      } ,
    compound-units / repeat .code:n   =
      {
        \bool_set_false:N \l_@@_unit_bracket_bool
        \bool_set_false:N \l_@@_unit_power_bool
        \bool_set_true:N  \l_@@_unit_repeat_bool
      } ,
    compound-units / single .code:n   =
      {
        \bool_set_false:N \l_@@_unit_bracket_bool
        \bool_set_false:N \l_@@_unit_power_bool
        \bool_set_false:N \l_@@_unit_repeat_bool
      }
  }
%    \end{macrocode}
% \end{variable}
%
% \begin{macro}{\siunitx_compound_number:n}
% \begin{macro}{\@@_format:n}
% \begin{macro}{\@@_format:nnn}
%   Printing a generic set starts with the question of whether we want to
%   extract exponents. If we do, then there is the work to do with extraction.
%   Either way, the printing is handed off to a common function. We do a quick
%   count up-front to avoid excess work when there is not actually a list.
%    \begin{macrocode}
\cs_new_protected:Npn \siunitx_compound_number:n #1
  {
    \group_begin:
      \@@_format:nn {#1} { }
      \@@_print:N \@@_print_number:n
    \group_end:
  }
\cs_new_protected:Npn \@@_format:nn #1#2
  {
    \seq_clear:N \l_@@_tmp_seq
    \bool_if:NTF \l_siunitx_number_parse_bool
      {
        \exp_args:Nxx \@@_format:nnn
          { \tl_head:n {#1} }
          { \tl_tail:n {#1} }
          {#2}
      }
      { \tl_map_function:nN {#1} \@@_unparsed:n }
  }
%    \end{macrocode}
%   Formatting at a low level needs to know about units and numbers: we have to
%   exchange data between the two. Most of the business of handling the units
%   is left to a dedicated auxiliary.
%    \begin{macrocode}
\cs_new_protected:Npn \@@_format:nnn #1#2#3
  {
    \siunitx_number_parse:nN {#1} \l_@@_tmp_tl
    \tl_if_blank:nTF {#3}
      { \siunitx_number_process:NN \l_@@_tmp_tl \l_@@_first_tl }
      { \@@_format_units:nn {#2} {#3} }
    \bool_lazy_and:nnTF
      { \l_@@_exp_combine_bool }
      { \int_compare_p:nNn { \tl_count:n {#2} } > 0 }
      { \@@_extract_exponents: }
      {
        \tl_set:Nx \l_@@_tmp_tl
          { \siunitx_number_output:N \l_@@_first_tl }
        \seq_put_right:NV \l_@@_tmp_seq \l_@@_tmp_tl
      }
    \tl_map_function:nN {#2} \@@_parsed:n
  }
%    \end{macrocode}
% \end{macro}
% \end{macro}
% \end{macro}
%
% \begin{macro}{\@@_extract_exponents:N}
% \begin{macro}{\@@_extract_exponents_auxi:w}
% \begin{macro}{\@@_extract_exponents_auxii:nw}
% \begin{macro}{\@@_extract_exponents_auxiii:nnnnnnn}
%   Extracting exponents means dealing with the first entry as a special case.
%   After that, apply fixed processing to all other entries, tidying up using
%   the number formatter.
%    \begin{macrocode}
\cs_new_protected:Npn \@@_extract_exponents:
  {
    \tl_set:Nx \l_@@_tmp_tl
      { \siunitx_number_output:NN \l_@@_first_tl \q_nil }
    \exp_after:wN \@@_extract_exponents_auxi:w
      \l_@@_tmp_tl \q_stop
  }
\cs_new_protected:Npn \@@_extract_exponents_auxi:w
  #1 \q_nil #2 \q_nil #3 \q_nil #4 \q_nil #5 \q_nil #6 \q_nil #7 \q_nil #8
  \q_nil #9 \q_stop
  {
    \@@_extract_exponents_auxii:nw {#1#2#3#4#5#6#7#8} #9 \q_stop
  }
\cs_new_protected:Npn \@@_extract_exponents_auxii:nw
  #1#2 \q_nil #3 \q_nil #4 \q_stop
  {
    \seq_put_right:Nn \l_@@_tmp_seq { #1#2 }
    \tl_set:Nn \l_@@_exp_tl { #3#4 }
    \exp_after:wN \@@_extract_exponents_auxiii:nnnnnnn
      \l_@@_first_tl
  }
\cs_new_protected:Npn \@@_extract_exponents_auxiii:nnnnnnn
  #1#2#3#4#5#6#7
  {
    \keys_set:nn { siunitx }
      {
        drop-exponent  = true  ,
        exponent-mode  = fixed ,
        fixed-exponent = #6#7
      }
  }
%    \end{macrocode}
% \end{macro}
% \end{macro}
% \end{macro}
% \end{macro}
%
% \begin{macro}{\@@_parsed:n, \@@_unparsed:n}
%   The simple cases for parsing (or not) all entries.
%    \begin{macrocode}
\cs_new_protected:Npn \@@_parsed:n #1
  {
    \siunitx_number_format:nN {#1} \l_@@_tmp_tl
    \seq_put_right:NV \l_@@_tmp_seq \l_@@_tmp_tl
  }
\cs_new_protected:Npn \@@_unparsed:n #1
  {
    \seq_put_right:Nn \l_@@_tmp_seq { \ensuremath {#1} }
  }
%    \end{macrocode}
% \end{macro}
%
% \begin{macro}{\@@_format_units:nn}
% \begin{macro}
%   {
%     \@@_format_combine-exponent:n ,
%     \@@_format_extract-exponent:n ,
%     \@@_format_input:n
%   }
% \begin{macro}
%   {
%     \@@_format_combine-exponent:nn ,
%     \@@_format_extract-exponent:nn
%   }
% \begin{macro}
%   {
%     \@@_format_combine-exponent_aux:n ,
%     \@@_format_extract-exponent_aux:n
%   }
% \begin{macro}{\@@_extract_exp:nN}
% \begin{macro}{\@@_extract_exp:nnnnnnnN}
%   Actually formatting the units is much the same as is done in
%   the quantities module, except that we have to cover the multiplication
%   cases too: gets a bit repetitive. Notice that when combining exponents,
%   there is no adjustment to the original exponent: we purely need to
%   extract it.
%    \begin{macrocode}
\cs_new_protected:Npn \@@_format_units:nn #1#2
  {
    \bool_if:NTF \l_@@_unit_power_bool
      {
        \use:c { @@_format_ \l_siunitx_quantity_prefix_mode_tl :nn }
          {#2} { \tl_count:n {#1} + 1 }
      }
      {
        \use:c { @@_format_ \l_siunitx_quantity_prefix_mode_tl :n } {#2}
      }
  }
\cs_new_protected:cpx { @@_format_combine-exponent:n } #1
  {
    \exp_not:c { @@_format_combine-exponent_aux:n }
      {
        \exp_not:N \siunitx_unit_format_combine_exponent:nnN
          {#1}
      }
  }
\cs_new_protected:cpx { @@_format_combine-exponent:nn } #1#2
  {
    \exp_not:c { @@_format_combine-exponent_aux:n }
      {
        \exp_not:N \siunitx_unit_format_multiply_combine_exponent:nnnN
          {#1} {#2}
      }
  }
\cs_new_protected:cpn { @@_format_combine-exponent_aux:n } #1
  {
    \bool_set_true:N \l_@@_exp_combine_bool
    \siunitx_number_process:NN \l_@@_tmp_tl \l_@@_first_tl
    \exp_args:NV \@@_extract_exp:nN
      \l_@@_first_tl \l_@@_tmp_fp
    #1 \l_@@_tmp_fp \l_@@_unit_tl
  }
\cs_new_protected:cpx { @@_format_extract-exponent:n } #1
  {
    \exp_not:c { @@_format_extract-exponent_aux:n }
      { \exp_not:N \siunitx_unit_format_extract_prefixes:nNN {#1} }
  }
\cs_new_protected:cpx { @@_format_extract-exponent:nn } #1#2
  {
    \exp_not:c { @@_format_extract-exponent_aux:n }
      {
        \exp_not:N \siunitx_unit_format_multiply_extract_prefixes:nnNN
          {#1} {#2}
      }
  }
\cs_new_protected:cpn { @@_format_extract-exponent_aux:n } #1
  {
    #1 \l_@@_unit_tl \l_@@_tmp_fp
    \tl_set:Nx \l_@@_tmp_tl
      { \siunitx_number_adjust_exponent:Nn \l_@@_tmp_tl \l_@@_tmp_fp }
    \siunitx_number_process:NN \l_@@_tmp_tl \l_@@_first_tl
    \bool_set_true:N \l_@@_exp_combine_bool
  }
\cs_new_protected:Npn \@@_format_input:n #1
  {
    \siunitx_number_process:NN \l_@@_tmp_tl \l_@@_first_tl
    \siunitx_unit_format:nN {#1} \l_@@_unit_tl
  }
\cs_new_protected:Npn \@@_format_input:nn #1#2
  {
    \siunitx_number_process:NN \l_@@_tmp_tl \l_@@_first_tl
    \siunitx_unit_format_multiply:nnN {#1} {#2} \l_@@_unit_tl
  }
\cs_new_protected:Npn \@@_extract_exp:nN #1#2
  { \@@_extract_exp:nnnnnnnN #1 #2 }
\cs_new_protected:Npn \@@_extract_exp:nnnnnnnN #1#2#3#4#5#6#7#8
  { \fp_set:Nn #8 {#6#7} }
%    \end{macrocode}
% \end{macro}
% \end{macro}
% \end{macro}
% \end{macro}
% \end{macro}
% \end{macro}
%
% \begin{macro}{\siunitx_compound_quantity:nn}
%   For quantities, life is more complex as there are interactions between the
%   options for exponents and units.
%    \begin{macrocode}
\cs_new_protected:Npn \siunitx_compound_quantity:nn #1#2
  {
    \group_begin:
      \bool_if:NT \l_@@_unit_bracket_bool
        { \bool_set_true:N \l_@@_exp_bracket_bool }
      \bool_if:NT \l_@@_unit_repeat_bool
        { \bool_set_false:N \l_@@_exp_combine_bool }
      \bool_lazy_or:nnT
        { \l_@@_unit_bracket_bool }
        { ! \l_@@_unit_repeat_bool }
        { \bool_set_false:N \l_siunitx_number_bracket_ambiguous_bool }
      \@@_format:nn {#1} {#2}
      \str_if_eq:VnT \l_siunitx_quantity_prefix_mode_tl { combine-exponent }
        { \tl_clear:N \l_@@_exp_tl }
      \bool_if:NTF \l_@@_unit_repeat_bool
        { \@@_print:N \@@_print_quantity:n }
        {
          \bool_lazy_and:nnTF
            { \l_@@_unit_bracket_bool }
            { \tl_if_empty_p:N \l_@@_exp_tl }
            {
              \siunitx_print:nV { number } \l_@@_bracket_open_tl
              \@@_print:N \@@_print_number:n
              \siunitx_print:nV { number } \l_@@_bracket_close_tl
            }
            { \@@_print:N \@@_print_number:n }
          \@@_print_quantity:n { }
        }
    \group_end:
  }
%    \end{macrocode}
% \end{macro}
%
% \begin{macro}{\@@_print:N}
% \begin{macro}{\@@_print:nnN, \@@_print:xxN}
% \begin{macro}{\@@_print:nnnN}
% \begin{macro}{\@@_print_aux:n}
% \begin{macro}{\@@_print_aux:nn}
% \begin{macro}{\@@_print_number:n}
%   We now need to know how many entries there are: the reason we don't use
%   \cs{seq_use:Nnnn} is that we want to be able to insert \cs{siunitx_print:nn}
%   in a controlled way.
%    \begin{macrocode}
\cs_new_protected:Npn \@@_print:N #1
  {
    \bool_lazy_and:nnTF
      { \l_@@_exp_bracket_bool }
      { ! \tl_if_empty_p:N \l_@@_exp_tl }
      {
        \@@_print:xxN
          { \exp_not:V \l_@@_bracket_open_tl }
          {
            \exp_not:V \l_@@_bracket_close_tl
            \exp_not:V \l_@@_exp_tl
          }
          #1
      }
      { \@@_print:xxN { } { \exp_not:V \l_@@_exp_tl } #1 }
  }
\cs_new_protected:Npn \@@_print:nnN #1#2#3
  {
    \exp_args:Nx \@@_print:nnnN
      { \seq_count:N \l_@@_tmp_seq } {#1} {#2} #3
  }
\cs_generate_variant:Nn \@@_print:nnN { xx }
%    \end{macrocode}
%   A rather long auxiliary as we want a way to have the brackets/exponent
%   available. The actual flow is simple enough: see how many entries there are
%   and print as required. To keep everything generic, we have some slightly
%   tricky saving of data to allow everything to go to the mapping.
%    \begin{macrocode}
\cs_new_protected:Npn \@@_print:nnnN #1#2#3#4
  {
    \int_case:nnF {#1}
      {
        { 0 } { }
        { 1 }
          {
            #4
              { \seq_item:Nn \l_@@_tmp_seq { 1 } }
          }
        { 2 }
          {
            #4
              {
                \exp_not:n {#2}
                \seq_item:Nn \l_@@_tmp_seq { 1 }
              }
            \@@_print_separator:V \l_@@_separator_pair_tl
            #4
              {
                \seq_item:Nn \l_@@_tmp_seq { 2 }
                \exp_not:n {#3}
              }
          }
      }
      {
        \int_set:Nn \l_@@_count_int {#1}
        \tl_set:Nn \l_@@_start_tl {#2}
        \tl_set:Nn \l_@@_end_tl {#3}
        \cs_set_eq:NN \@@_print_aux:n #4
        \seq_map_indexed_function:NN
          \l_@@_tmp_seq
          \@@_print_aux:nn
      }
  }
\cs_new_protected:Npn \@@_print_aux:n #1 { }
\cs_new_protected:Npn \@@_print_aux:nn #1#2
  {
    \int_case:nnF {#1}
      {
        { 1 }
          {
            \@@_print_aux:n
              {
                \exp_not:V \l_@@_start_tl
                \exp_not:n {#2}
              }
            \@@_print_separator:V \l_@@_separator_tl
          }
        { \l_@@_count_int - 1 }
          {
            \@@_print_aux:n { \exp_not:n {#2} }
            \@@_print_separator:V \l_@@_separator_final_tl
          }
        { \l_@@_count_int }
          {
            \@@_print_aux:n
              {
                \exp_not:n {#2}
                \exp_not:V \l_@@_end_tl
              }
          }
      }
      {
        \@@_print_aux:n { \exp_not:n {#2} }
        \@@_print_separator:V \l_@@_separator_tl
      }
  }
\cs_new_protected:Npn \@@_print_number:n #1
  { \siunitx_print:nx { number } {#1} }
\cs_new_protected:Npn \@@_print_quantity:n #1
  { \siunitx_quantity_print:nV {#1} \l_@@_unit_tl }
\cs_new_protected:Npn \@@_print_separator:n #1
  {
    \bool_if:NTF \l_@@_separator_text_bool
      { #1 }
      { \siunitx_print:nn { number } {#1} }
  }
\cs_generate_variant:Nn \@@_print_separator:n { V }
%    \end{macrocode}
% \end{macro}
% \end{macro}
% \end{macro}
% \end{macro}
% \end{macro}
% \end{macro}
%
% \subsection{Lists}
%
% Identify the internal prefix (\LaTeX3 \pkg{DocStrip} convention): only
% internal material in this \emph{submodule} should be used directly.
%    \begin{macrocode}
%<@@=siunitx_list>
%    \end{macrocode}
%
% \begin{variable}
%   {
%     \l_siunitx_list_separator_tl       ,
%     \l_siunitx_list_separator_final_tl ,
%     \l_siunitx_list_separator_pair_tl  ,
%     \l_@@_exp_tl                       ,
%     \l_@@_units_tl
%   }
%   Options for products.
%    \begin{macrocode}
\tl_new:N \l_@@_exp_tl
\tl_new:N \l_@@_units_tl
\keys_define:nn { siunitx }
  {
    list-exponents .choices:nn =
      { combine , combine-bracket , individual }
      { \tl_set_eq:NN \l_@@_exp_tl \l_keys_choice_tl } ,
    list-final-separator .tl_set:N = \l_siunitx_list_separator_final_tl ,
    list-pair-separator .tl_set:N = \l_siunitx_list_separator_pair_tl ,
    list-separator .tl_set:N = \l_siunitx_list_separator_tl ,
    list-units .choices:nn =
      { bracket , repeat , single }
      { \tl_set_eq:NN \l_@@_units_tl \l_keys_choice_tl }
  }
%    \end{macrocode}
% \end{variable}
%
% \begin{macro}{\siunitx_number_list:nn}
% \begin{macro}{\siunitx_quantity_list:nn}
% \begin{macro}{\@@_aux:}
%   Simply recover the settings and use as a list.
%    \begin{macrocode}
\cs_new_protected:Npn \siunitx_number_list:nn #1
  {
    \group_begin:
      \@@_aux:
      \siunitx_compound_number:n {#1}
    \group_end:
  }
\cs_new_protected:Npn \siunitx_quantity_list:nn #1#2
  {
    \group_begin:
      \@@_aux:
      \siunitx_compound_quantity:nn {#1} {#2}
    \group_end:
  }
\cs_new_protected:Npn \@@_aux:
  {
    \keys_set:nx { siunitx }
      {
        compound-exponents       = \l_@@_exp_tl ,
        compound-final-separator =
          { \exp_not:V \l_siunitx_list_separator_final_tl } ,
        compound-pair-separator  =
          { \exp_not:V \l_siunitx_list_separator_pair_tl } ,
        compound-separator       =
          { \exp_not:V \l_siunitx_list_separator_tl } ,
        compound-separator-mode  = text ,
        compound-units           = \l_@@_units_tl
      }
  }
%    \end{macrocode}
% \end{macro}
% \end{macro}
% \end{macro}
%
% \subsection{Products}
%
% Identify the internal prefix (\LaTeX3 \pkg{DocStrip} convention): only
% internal material in this \emph{submodule} should be used directly.
%    \begin{macrocode}
%<@@=siunitx_product>
%    \end{macrocode}
%
% \begin{variable}
%   {
%     \l_@@_exp_tl      ,
%     \l_@@_phrase_bool ,
%     \l_@@_phrase_tl   ,
%     \l_@@_symbol_tl   ,
%     \l_@@_units_tl
%   }
%   Options for products.
%    \begin{macrocode}
\tl_new:N \l_@@_exp_tl
\bool_new:N \l_@@_phrase_bool
\tl_new:N \l_@@_units_tl
\keys_define:nn { siunitx }
  {
    product-exponents .choices:nn =
      { combine , combine-bracket , individual }
      { \tl_set_eq:NN \l_@@_exp_tl \l_keys_choice_tl } ,
    product-mode .choice: ,
    product-mode / phrase .code:n =
      { \bool_set_true:N \l_@@_phrase_bool } ,
    product-mode / symbol .code:n =
      { \bool_set_false:N \l_@@_phrase_bool } ,
    product-phrase .tl_set:N = \l_@@_phrase_tl ,
    product-symbol .tl_set:N = \l_@@_symbol_tl ,
    product-units .choices:nn =
      { bracket , bracket-power , power , repeat , single }
      { \tl_set_eq:NN \l_@@_units_tl \l_keys_choice_tl }
  }
%    \end{macrocode}
% \end{variable}
%
% \begin{macro}{\siunitx_number_product:n}
% \begin{macro}{\siunitx_quantity_product:nn}
% \begin{macro}{\@@_aux:}
% \begin{macro}{\@@_aux:n, \@@_aux:x}
%   Simply recover the settings and use as a list.
%    \begin{macrocode}
\cs_new_protected:Npn \siunitx_number_product:n #1
  {
    \group_begin:
      \@@_aux:
      \siunitx_compound_number:n {#1}
    \group_end:
  }
\cs_new_protected:Npn \siunitx_quantity_product:nn #1#2
  {
    \group_begin:
      \@@_aux:
      \siunitx_compound_quantity:nn {#1} {#2}
    \group_end:
  }
\cs_new_protected:Npn \@@_aux:
  {
    \bool_if:NTF \l_@@_phrase_bool
      { \@@_aux:x { \exp_not:V \l_@@_phrase_tl } }
      { \@@_aux:x { { } \exp_not:V \l_@@_symbol_tl { } } }
  }
\cs_new_protected:Npn \@@_aux:n #1
  {
    \keys_set:nx { siunitx }
      {
        compound-exponents       = \l_@@_exp_tl ,
        compound-final-separator = { \exp_not:n {#1} } ,
        compound-pair-separator  = { \exp_not:n {#1} } ,
        compound-separator       = { \exp_not:n {#1} } ,
        compound-separator-mode  =
          \bool_if:NTF \l_@@_phrase_bool { text } { number } ,
        compound-units           = \l_@@_units_tl
      }
  }
\cs_generate_variant:Nn \@@_aux:n { x }
%    \end{macrocode}
% \end{macro}
% \end{macro}
% \end{macro}
% \end{macro}
%
% \subsection{Ranges}
%
% Identify the internal prefix (\LaTeX3 \pkg{DocStrip} convention): only
% internal material in this \emph{submodule} should be used directly.
%    \begin{macrocode}
%<@@=siunitx_range>
%    \end{macrocode}
%
% \begin{variable}
%   {
%     \l_@@_exp_tl               ,
%     \l_siunitx_range_phrase_tl ,
%     \l_@@_units_tl
%   }
%   Options for products.
%    \begin{macrocode}
\tl_new:N \l_@@_exp_tl
\tl_new:N \l_@@_units_tl
\keys_define:nn { siunitx }
  {
    range-exponents .choices:nn =
      { combine , combine-bracket , individual }
      { \tl_set_eq:NN \l_@@_exp_tl \l_keys_choice_tl } ,
    range-phrase .tl_set:N = \l_siunitx_range_phrase_tl ,
    range-units .choices:nn =
      { bracket , repeat , single }
      { \tl_set_eq:NN \l_@@_units_tl \l_keys_choice_tl }
  }
%    \end{macrocode}
% \end{variable}
%
% \begin{macro}{\siunitx_number_range:nn}
% \begin{macro}{\siunitx_quantity_range:nnn}
% \begin{macro}{\@@_aux:}
%   Simply recover the settings and use as a list.
%    \begin{macrocode}
\cs_new_protected:Npn \siunitx_number_range:nn #1#2
  {
    \group_begin:
      \@@_aux:
      \siunitx_compound_number:n { {#1} {#2} }
    \group_end:
  }
\cs_new_protected:Npn \siunitx_quantity_range:nnn #1#2#3
  {
    \group_begin:
      \@@_aux:
      \siunitx_compound_quantity:nn { {#1} {#2} } {#3}
    \group_end:
  }
\cs_new_protected:Npn \@@_aux:
  {
    \keys_set:nx { siunitx }
      {
        compound-exponents       = \l_@@_exp_tl ,
        compound-pair-separator  = { \exp_not:V \l_siunitx_range_phrase_tl } ,
        compound-separator-mode  = text ,
        compound-units           = \l_@@_units_tl
      }
  }
%    \end{macrocode}
% \end{macro}
% \end{macro}
% \end{macro}
%
% \subsection{Standard settings for module options}
%
% Some of these follow naturally from the point of definition
% (\foreign{e.g.}~boolean variables are always |false| to begin with),
% but for clarity everything is set here.
%    \begin{macrocode}
\keys_set:nn { siunitx }
  {
    compound-exponents       = individual  ,
    compound-final-separator =
      {
        \ifmmode \  \else \space \fi
        \text { and }
        \ifmmode \  \else \space \fi
      } ,
    compound-pair-separator  =
      {
        \ifmmode \  \else \space \fi
        \text { and }
        \ifmmode \  \else \space \fi
      } ,
    compound-separator       =
      { , \ifmmode \  \else \space \fi } ,
    compound-separator-mode  = text        ,
    compound-units           = repeat      ,
    list-exponents           = individual  ,
    list-final-separator     =
      {
        \ifmmode \  \else \space \fi
        \text { and }
        \ifmmode \  \else \space \fi
      } ,
    list-pair-separator      =
      {
        \ifmmode \  \else \space \fi
        \text { and }
        \ifmmode \  \else \space \fi
      } ,
    list-separator           =
      { , \ifmmode \  \else \space \fi } ,
    list-units               = repeat      ,
    product-exponents        = individual  ,
    product-mode             = symbol      ,
    product-phrase           =
      {
        \ifmmode \  \else \space \fi
        \text { by }
        \ifmmode \  \else \space \fi
      } ,
    product-symbol           = \times      ,
    product-units            = repeat      ,
    range-exponents          = individual  ,
    range-phrase             =
      {
        \ifmmode \  \else \space \fi
        \text { to }
        \ifmmode \  \else \space \fi
      } ,
    range-units              = repeat
  }
%    \end{macrocode}
%
%    \begin{macrocode}
%</package>
%    \end{macrocode}
%
% \end{implementation}
%
% \PrintIndex

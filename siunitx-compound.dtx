% \iffalse meta-comment
%
% File: siunitx-compound.dtx Copyright (C) 2018-2021 Joseph Wright
%
% It may be distributed and/or modified under the conditions of the
% LaTeX Project Public License (LPPL), either version 1.3c of this
% license or (at your option) any later version.  The latest version
% of this license is in the file
%
%    https://www.latex-project.org/lppl.txt
%
% This file is part of the "siunitx bundle" (The Work in LPPL)
% and all files in that bundle must be distributed together.
%
% The released version of this bundle is available from CTAN.
%
% -----------------------------------------------------------------------
%
% The development version of the bundle can be found at
%
%    https://github.com/josephwright/siunitx
%
% for those people who are interested.
%
% -----------------------------------------------------------------------
%
%<*driver>
\documentclass{l3doc}
\ProvideDocumentCommand\foreign{m}{\textit{#1}}
% The next line is needed so that \GetFileInfo will be able to pick up
% version data
\usepackage{siunitx}
\begin{document}
  \DocInput{\jobname.dtx}
\end{document}
%</driver>
% \fi
%
% \GetFileInfo{siunitx.sty}
%
% \title{^^A
%   \pkg{siunitx-quantity} -- Multi-part numbers^^A
%   \thanks{This file describes \fileversion,
%     last revised \filedate.}^^A
% }
%
% \author{^^A
%  Joseph Wright^^A
%  \thanks{^^A
%    E-mail:
%    \href{mailto:joseph.wright@morningstar2.co.uk}
%      {joseph.wright@morningstar2.co.uk}^^A
%   }^^A
% }
%
% \date{Released \filedate}
%
% \maketitle
%
% \begin{documentation}
%
% \begin{function}{\siunitx_compound:nN}
%   \begin{syntax}
%     \cs{siunitx_compound:nN} \Arg{entries} \meta{seq}
%   \end{syntax}
%   Formats each \meta{entry} in the \meta{entries}, each of which should be
%   a \meta{balanced text}, and places the result in the \meta{seq}.
% \end{function}
%
% \begin{function}{\siunitx_compound_extract_exponents:nNN}
%   \begin{syntax}
%     \cs{siunitx_compound_extract_exponents:nNN} \Arg{entries} \meta{seq} \meta{tl~var}
%   \end{syntax}
%   Formats each \meta{entry} in the \meta{entries} as described for
%   \cs{siunitx_compound:nN}. Places each formatted \meta{entry} minus any
%   exponent in the \meta{seq}, and places the common (formatted) exponent in
%   the \meta{tl~var}.
% \end{function}
%
% \begin{function}{\siunitx_list:n}
%   \begin{syntax}
%     \cs{siunitx_list:n} \Arg{entries}
%   \end{syntax}
%   Prints the list of numbers in the \meta{entries}, each of which should
%   be given as a \meta{balanced text}.
% \end{function}
%
% \begin{function}{\siunitx_list:nn}
%   \begin{syntax}
%     \cs{siunitx_list:n} \Arg{entries} \Arg{unit}
%   \end{syntax}
%   Prints the list of quantities in the \meta{entries}, each of which should
%   be given as a \meta{balanced text}.
% \end{function}
%
% \begin{function}{\siunitx_product:n}
%   \begin{syntax}
%     \cs{siunitx_product:n} \Arg{entries}
%   \end{syntax}
%   Prints the series of numbers in the \meta{entries}, each of which should
%   be given as a \meta{balanced text}.
% \end{function}
%
% \begin{function}{\siunitx_range:nn}
%   \begin{syntax}
%     \cs{siunitx_range:nn} \Arg{start} \Arg{end}
%   \end{syntax}
%   Prints the range of numbers from the \meta{start} to the \meta{end}.
% \end{function}
%
% \begin{variable}
%   {
%     \l_siunitx_list_separator_pair_tl,
%     \l_siunitx_list_separator_tl,
%     \l_siunitx_list_separator_final_tl
%   }
%   Separators for lists of numbers and quantities.
% \end{variable}
%
% \begin{variable}{\l_siunitx_range_phrase_tl}
%   Phrase (or similar) used between limits of a range.
% \end{variable}
%
% \begin{function}{list-exponents}
%   \begin{syntax}
%     |list-exponents| = |combine|\verb"|"|combine-bracket|\verb"|"|individual|
%   \end{syntax}
% \end{function}
%
% \begin{function}{list-final-separator}
%   \begin{syntax}
%     |list-final-separator| = \meta{text}
%   \end{syntax}
% \end{function}
%
% \begin{function}{list-pair-separator}
%   \begin{syntax}
%     |list-pair-separator| = \meta{text}
%   \end{syntax}
% \end{function}
%
% \begin{function}{list-separator}
%   \begin{syntax}
%     |list-separator| = \meta{text}
%   \end{syntax}
% \end{function}
%
% \begin{function}{list-units}
%   \begin{syntax}
%     |list-units| = |bracket|\verb"|"|repeat|\verb"|"|single|
%   \end{syntax}
% \end{function}
%
% \begin{function}{product-exponents}
%   \begin{syntax}
%     |product-exponents| = |combine|\verb"|"|combine-bracket|\verb"|"|individual|
%   \end{syntax}
% \end{function}
%
% \begin{function}{product-mode}
%   \begin{syntax}
%     |product-mode| = |phrase|\verb"|"|choice|
%   \end{syntax}
% \end{function}
%
% \begin{function}{product-phrase}
%   \begin{syntax}
%     |product-phrase| = \meta{text}
%   \end{syntax}
% \end{function}
%
% \begin{function}{product-symbol}
%   \begin{syntax}
%     |product-symbol| = \meta{symbol}
%   \end{syntax}
% \end{function}
%
% \begin{function}{range-exponents}
%   \begin{syntax}
%     |range-exponents| = |combine|\verb"|"|combine-bracket|\verb"|"|individual|
%   \end{syntax}
% \end{function}
%
% \begin{function}{range-phrase}
%   \begin{syntax}
%     |range-phrase| = \meta{text}
%   \end{syntax}
% \end{function}
%
% \begin{function}{range-units}
%   \begin{syntax}
%     |range-units| = |bracket|\verb"|"|repeat|\verb"|"|single|
%   \end{syntax}
% \end{function}
%
% \end{documentation}
%
% \begin{implementation}
%
% Start the \pkg{DocStrip} guards.
%    \begin{macrocode}
%<*package>
%    \end{macrocode}
%
% \section{\pkg{siunitx-compound} implementation}
%
% \subsection{General mechanism}
%
% Identify the internal prefix (\LaTeX3 \pkg{DocStrip} convention): only
% internal material in this \emph{submodule} should be used directly.
%    \begin{macrocode}
%<@@=siunitx_compound>
%    \end{macrocode}
%
% Typesetting lists, ranges and products of numbers or quantities has shared
% features which mean they are best handled using a common mechanism. The aim
% therefore is to abstract out enough of the process such that output-specific
% aspects can be left to separate processes.
%
% \begin{variable}{\l_@@_tmp_seq, \l_@@_tmp_tl}
%   Scratch space.
%    \begin{macrocode}
\seq_new:N \l_@@_tmp_seq
\tl_new:N \l_@@_tmp_tl
%    \end{macrocode}
% \end{variable}
%
% \begin{variable}{\l_@@_first_tl}
%   The first number in the list in internal format.
%    \begin{macrocode}
\tl_new:N \l_@@_first_tl
%    \end{macrocode}
% \end{variable}
%
% \begin{variable}{\l_@@_exp_tl}
%   For storing the combined exponent, if present.
%    \begin{macrocode}
\tl_new:N \l_@@_exp_tl
%    \end{macrocode}
% \end{variable}
%
% \begin{variable}{\l_@@_extract_exp_bool}
%   To allow extraction or not using a single auxiliary.
%    \begin{macrocode}
\bool_new:N \l_@@_extract_exp_bool
%    \end{macrocode}
% \end{variable}
%
% \begin{macro}{\siunitx_compound:nN}
% \begin{macro}{\siunitx_compound_extract_exponents:nNN}
% \begin{macro}{\@@_end:}
% \begin{macro}[EXP]{\@@_end_aux:Nn}
%   A single internal route deals with both cases. Most of this is obvious,
%   other than we have a bit of fun as using a group is best here but
%   getting a sequence back out is non-trivial.
%    \begin{macrocode}
\cs_new_protected:Npn \siunitx_compound:nN #1#2
  {
    \bool_set_false:N \l_@@_extract_exp_bool
    \@@_extract_exponents:nNN {#1} #2 \l_@@_exp_tl
  }
\cs_new_protected:Npn \siunitx_compound_extract_exponents:nNN #1#2#3
  {
    \bool_set_true:N \l_@@_extract_exp_bool
    \@@_extract_exponents:nNN {#1} #2 #3
  }
\cs_new_protected:Npn \@@_extract_exponents:nNN #1#2#3
  {
    \group_begin:
      \seq_clear:N \l_@@_tmp_seq
      \bool_if:NTF \l_siunitx_number_parse_bool
        {
          \bool_if:NTF \l_@@_extract_exp_bool
            {
              \exp_args:Nxx \@@_extract_exponents:nn
                { \tl_head:n {#1} }
                { \tl_tail:n {#1} }
             }
            { \tl_map_function:nN {#1} \@@_parsed:n }
        }
        { \tl_map_function:nN {#1} \@@_unparsed:n }
      \cs_set_protected:Npx \@@_end:
        {
          \seq_clear:N \exp_not:N #2
          \seq_map_tokens:Nn \l_@@_tmp_seq
            { \@@_end_aux:Nn #2 }
          \tl_set:Nn \exp_not:N #3
            { \exp_not:V \l_@@_exp_tl }
        }
    \exp_after:wN \group_end:
    \@@_end:
  }
\cs_new_protected:Npn \@@_end: { }
\cs_new:Npn \@@_end_aux:Nn #1#2
  {
    \seq_put_right:Nn \exp_not:N #1 { \exp_not:n {#2} }
  }
%    \end{macrocode}
% \end{macro}
% \end{macro}
% \end{macro}
% \end{macro}
%
% \begin{macro}{\@@_parsed:n, \@@_unparsed:n}
%   The simple cases for parsing (or not) all entries.
%    \begin{macrocode}
\cs_new_protected:Npn \@@_parsed:n #1
  {
    \siunitx_number_format:nN {#1} \l_@@_tmp_tl
    \seq_put_right:NV \l_@@_tmp_seq \l_@@_tmp_tl
  }
\cs_new_protected:Npn \@@_unparsed:n #1
  {
    \seq_put_right:Nn \l_@@_tmp_seq { \ensuremath {#1} }
  }
%    \end{macrocode}
% \end{macro}
%
% \begin{macro}{\@@_extract_exponents:nn}
% \begin{macro}{\@@_extract_exponents_aux:nnnnnnn}
%   Extracting exponents means dealing with the first entry as a special case.
%   After that, apply fixed processing to all other entries, tidying up using
%   the number formatter.
%    \begin{macrocode}
\cs_new_protected:Npn \@@_extract_exponents:nn #1#2
  {
    \siunitx_number_parse:nN {#1} \l_@@_tmp_tl
    \siunitx_number_process:NN \l_@@_tmp_tl \l_@@_first_tl
    \tl_set:Nx \l_@@_tmp_tl
      { \siunitx_number_output:NN \l_@@_first_tl \q_nil }  
    \exp_after:wN \@@_extract_exponents_auxi:w
      \l_@@_tmp_tl \q_stop {#2}
  }
\cs_new_protected:Npn \@@_extract_exponents_auxi:w
  #1 \q_nil #2 \q_nil #3 \q_nil #4 \q_nil #5 \q_nil #6 \q_nil #7 \q_nil #8
  \q_nil #9 \q_stop
  {
    \seq_put_right:Nn \l_@@_tmp_seq { #1#2#3#3#4#5#6#7#8 }
    \@@_extract_exponents_auxii:w #9 \q_stop
  }
\cs_new_protected:Npn \@@_extract_exponents_auxii:w
  #1 \q_nil #2 \q_stop #3
  {
    \tl_set:Nn \l_@@_exp_tl { #1#2 }
    \exp_after:wN \@@_extract_exponents_auxiii:nnnnnnnn
      \l_@@_first_tl {#3}
  }
\cs_new_protected:Npn \@@_extract_exponents_auxiii:nnnnnnnn
   #1#2#3#4#5#6#7#8
   {
    \keys_set:nn { siunitx }
      {
        drop-exponent  = true  ,
        exponent-mode  = fixed ,
        fixed-exponent = #6#7
      }
     \tl_map_function:nN {#8} \@@_parsed:n
  }
%    \end{macrocode}
% \end{macro}
% \end{macro}
%
% \subsection{Lists}
%
% Identify the internal prefix (\LaTeX3 \pkg{DocStrip} convention): only
% internal material in this \emph{submodule} should be used directly.
%    \begin{macrocode}
%<@@=siunitx_list>
%    \end{macrocode}
%
% \begin{variable}{\l_@@_tmp_seq, \l_@@_tmp_tl}
%   Scratch space.
%    \begin{macrocode}
\seq_new:N \l_@@_tmp_seq
\tl_new:N \l_@@_tmp_tl
%    \end{macrocode}
% \end{variable}
%
% \begin{variable}{\l_@@_start_tl, \l_@@_end_tl}
%   Data on the end-of-list.
%    \begin{macrocode}
\tl_new:N \l_@@_start_tl
\tl_new:N \l_@@_end_tl
%    \end{macrocode}
% \end{variable}
%
% \begin{variable}{\l_@@_count_int}
%   Data on the length-of-list.
%    \begin{macrocode}
\int_new:N \l_@@_count_int
%    \end{macrocode}
% \end{variable}
%
% \begin{variable}{\l_@@_unit_tl}
%    \begin{macrocode}
\tl_new:N \l_@@_unit_tl
%    \end{macrocode}
% \end{variable}
%
% \begin{variable}{\l_@@_bracket_close_tl, \l_@@_bracket_open_tl}
%   Purely internal for the present.
%    \begin{macrocode}
\tl_new:N \l_@@_bracket_close_tl
\tl_new:N \l_@@_bracket_open_tl
\tl_set:Nn \l_@@_bracket_open_tl { ( }
\tl_set:Nn \l_@@_bracket_close_tl { ) }
%    \end{macrocode}
% \end{variable}
%
% \begin{variable}
%   {
%     \l_siunitx_list_separator_tl       ,
%     \l_siunitx_list_separator_final_tl ,
%     \l_siunitx_list_separator_pair_tl  ,
%     \l_@@_exp_bracket_bool             ,
%     \l_@@_exp_combine_bool             ,
%     \l_@@_unit_bracket_bool            ,
%     \l_@@_unit_repeat_bool
%   }
%   List options.
%    \begin{macrocode}
\bool_new:N \l_@@_exp_bracket_bool
\bool_new:N \l_@@_exp_combine_bool
\bool_new:N \l_@@_unit_bracket_bool
\bool_new:N \l_@@_unit_repeat_bool
\keys_define:nn { siunitx }
  {
    list-exponents .choice: ,
    list-exponents / combine .code:n =
      {
        \bool_set_false:N \l_@@_exp_bracket_bool
        \bool_set_true:N \l_@@_exp_combine_bool
      } ,
    list-exponents / combine-bracket .code:n =
      {
        \bool_set_true:N \l_@@_exp_bracket_bool
        \bool_set_true:N \l_@@_exp_combine_bool
      } ,
    list-exponents / individual .code:n =
      {
        \bool_set_false:N \l_@@_exp_bracket_bool
        \bool_set_false:N \l_@@_exp_combine_bool
      } ,
    list-final-separator .tl_set:N = \l_siunitx_list_separator_final_tl ,
    list-pair-separator .tl_set:N = \l_siunitx_list_separator_pair_tl ,
    list-separator .tl_set:N = \l_siunitx_list_separator_tl ,
    list-units .choice: ,
    list-units / bracket .code:n =
      {
        \bool_set_true:N  \l_@@_unit_bracket_bool
        \bool_set_false:N \l_@@_unit_repeat_bool
      } ,
    list-units / repeat .code:n   =
      {
        \bool_set_false:N \l_@@_unit_bracket_bool
        \bool_set_true:N  \l_@@_unit_repeat_bool
      } ,
    list-units / single .code:n   =
      {
        \bool_set_false:N \l_@@_unit_bracket_bool
        \bool_set_false:N \l_@@_unit_repeat_bool
      }
  }
%    \end{macrocode}
% \end{variable}
%
% \begin{macro}{\siunitx_list:n}
% \begin{macro}{\siunitx_list:nn}
% \begin{macro}{\@@_format_numbers:n}
% \begin{macro}{\@@_print:N}
% \begin{macro}{\@@_print:nnN, \@@_print:xxN}
% \begin{macro}{\@@_print:nnnN}
% \begin{macro}{\@@_print_aux:n}
% \begin{macro}{\@@_print_aux:nn}
% \begin{macro}{\@@_print_number:n}
%   Printing a list starts with the question of whether we want to extract
%   exponents. If we do, then there is the work to do with extraction. Either
%   way, the printing is handed off to a common function. We do a quick count
%   up-front to avoid excess work when there is not actually a list.
%    \begin{macrocode}
\cs_new_protected:Npn \siunitx_list:n #1
  {
    \group_begin:
      \@@_format_numbers:n {#1}
      \@@_print:N \@@_print_number:n
    \group_end:
  }
%    \end{macrocode}
%   For quantities, life is more complex as there are interactions between the
%   options for exponents and units.
%    \begin{macrocode}
\cs_new_protected:Npn \siunitx_list:nn #1#2
  {
    \group_begin:
      \siunitx_unit_options_apply:n {#2}
      \bool_if:NT \l_@@_unit_bracket_bool
        { \bool_set_true:N \l_@@_exp_bracket_bool }
      \bool_if:NT \l_@@_unit_repeat_bool
        { \bool_set_false:N \l_@@_exp_combine_bool }
      \bool_lazy_or:nnT
        { \l_@@_unit_bracket_bool }
        { ! \l_@@_unit_repeat_bool }
        { \bool_set_false:N \l_siunitx_number_bracket_ambiguous_bool }
      \@@_format_numbers:n {#1}
      \siunitx_unit_format:nN {#2} \l_@@_unit_tl
      \bool_if:NTF \l_@@_unit_repeat_bool
        { \@@_print:N \@@_print_quantity:n }
        {
          \bool_lazy_and:nnT
            { \l_@@_unit_bracket_bool }
            { \tl_if_empty_p:N \l_@@_tmp_tl }
            {
              \siunitx_print:nV { number } \l_@@_bracket_open_tl
              \@@_print:N \@@_print_number:n
              \siunitx_print:nV { number } \l_@@_bracket_close_tl
            }
            { \@@_print:N \@@_print_number:n }
          \@@_print_quantity:n { }
        }
    \group_end:
  }
\cs_new_protected:Npn \@@_format_numbers:n #1
  {
    \bool_lazy_and:nnTF
      { \l_@@_exp_combine_bool }
      { \int_compare_p:nNn { \tl_count:n {#1} } > 1 }
      {
        \siunitx_compound_extract_exponents:nNN {#1}
          \l_@@_tmp_seq \l_@@_tmp_tl
      }
      {
        \tl_clear:N \l_@@_tmp_tl
        \siunitx_compound:nN {#1} \l_@@_tmp_seq
      }
  }
\cs_new_protected:Npn \@@_print:N #1
  {
    \bool_lazy_and:nnTF
      { \l_@@_exp_bracket_bool }
      { ! \tl_if_empty_p:N \l_@@_tmp_tl }
      {
        \@@_print:xxN
          { \exp_not:V \l_@@_bracket_open_tl }
          {
            \exp_not:V \l_@@_bracket_close_tl
            \exp_not:V \l_@@_tmp_tl
          }
          #1
      }
      { \@@_print:xxN { } { \exp_not:V \l_@@_tmp_tl } #1 }
  }
%    \end{macrocode}
%   We now need to know how many entries there are: the reason we don't use
%   \cs{seq_use:Nnnn} is that we want to be able to insert \cs{siunitx_print:nn}
%   in a controlled way.
%    \begin{macrocode}
\cs_new_protected:Npn \@@_print:nnN #1#2#3
  {
    \exp_args:Nx \@@_print:nnnN
      { \seq_count:N \l_@@_tmp_seq } {#1} {#2} #3
  }
\cs_generate_variant:Nn \@@_print:nnN { xx }
%    \end{macrocode}
%   A rather long auxiliary as we want a way to have the brackets/exponent
%   available. The actual flow is simple enough: see how many entries there are
%   and print as required. To keep everything generic, we have some slightly
%   tricky saving of data to allow everything to go to the mapping.
%    \begin{macrocode}
\cs_new_protected:Npn \@@_print:nnnN #1#2#3#4
  {
    \int_case:nnF {#1}
      {
        { 0 } { }
        { 1 }
          {
            #4
              { \seq_item:Nn \l_@@_tmp_seq { 1 } }
          }
        { 2 }
          {
            #4
              {
                \exp_not:n {#2}
                \seq_item:Nn \l_@@_tmp_seq { 1 }
              }
            \tl_use:N \l_siunitx_list_separator_pair_tl
            #4
              {
                \seq_item:Nn \l_@@_tmp_seq { 2 }
                \exp_not:n {#3}
              }
          }
      }
      {
        \int_set:Nn \l_@@_count_int {#1}
        \tl_set:Nn \l_@@_start_tl {#2}
        \tl_set:Nn \l_@@_end_tl {#3}
        \cs_set_eq:NN \@@_print_aux:n #4
        \seq_map_indexed_function:NN
          \l_@@_tmp_seq
          \@@_print_aux:nn
      }
  }
\cs_new_protected:Npn \@@_print_aux:n #1 { }
\cs_new_protected:Npn \@@_print_aux:nn #1#2
  {
    \int_case:nnF {#1}
      {
        { 1 }
          {
            \@@_print_aux:n
              {
                \exp_not:V \l_@@_start_tl
                \exp_not:n {#2}
              }
            \tl_use:N \l_siunitx_list_separator_tl
          }
        { \l_@@_count_int - 1 }
          {
            \@@_print_aux:n { \exp_not:n {#2} }
            \tl_use:N \l_siunitx_list_separator_final_tl
          }
        { \l_@@_count_int }
          {
            \@@_print_aux:n
              {
                \exp_not:n {#2}
                \exp_not:V \l_@@_end_tl
              }
          }
      }
      {
        \@@_print_aux:n { \exp_not:n {#2} }
        \tl_use:N \l_siunitx_list_separator_tl
      }
  }
\cs_new_protected:Npn \@@_print_number:n #1
  { \siunitx_print:nx { number } {#1} }
\cs_new_protected:Npn \@@_print_quantity:n #1
  { \siunitx_quantity_print:nV {#1} \l_@@_unit_tl }
%    \end{macrocode}
% \end{macro}
% \end{macro}
% \end{macro}
% \end{macro}
% \end{macro}
% \end{macro}
% \end{macro}
% \end{macro}
%
% \subsection{Products}
%
% Identify the internal prefix (\LaTeX3 \pkg{DocStrip} convention): only
% internal material in this \emph{submodule} should be used directly.
%    \begin{macrocode}
%<@@=siunitx_product>
%    \end{macrocode}
%
% \begin{variable}{\l_@@_tmp_seq, \l_@@_tmp_tl}
%   Scratch space.
%    \begin{macrocode}
\seq_new:N \l_@@_tmp_seq
\tl_new:N \l_@@_tmp_tl
%    \end{macrocode}
% \end{variable}
%
% \begin{variable}{\l_@@_bracket_close_tl, \l_@@_bracket_open_tl}
%   Purely internal for the present.
%    \begin{macrocode}
\tl_new:N \l_@@_bracket_close_tl
\tl_new:N \l_@@_bracket_open_tl
\tl_set:Nn \l_@@_bracket_open_tl { ( }
\tl_set:Nn \l_@@_bracket_close_tl { ) }
%    \end{macrocode}
% \end{variable}
%
% \begin{variable}
%   {
%     \l_@@_exp_bracket_bool  ,
%     \l_@@_exp_combine_bool  ,
%     \l_@@_mode_tl           ,
%     \l_@@_phrase_tl         ,
%     \l_@@_power_bool        ,
%     \l_@@_unit_bracket_bool ,
%     \l_@@_unit_repeat_bool  ,
%     \l_@@_symbol_tl
%   }
%   Options for products.
%    \begin{macrocode}
\bool_new:N \l_@@_exp_bracket_bool
\bool_new:N \l_@@_exp_combine_bool
\tl_new:N \l_@@_mode_tl
\bool_new:N \l_@@_power_bool
\bool_new:N \l_@@_unit_bracket_bool
\bool_new:N \l_@@_unit_repeat_bool
\keys_define:nn { siunitx }
  {
    product-exponents .choice: ,
    product-exponents / combine .code:n =
      {
        \bool_set_false:N \l_@@_exp_bracket_bool
        \bool_set_true:N \l_@@_exp_combine_bool
      } ,
    product-exponents / combine-bracket .code:n =
      {
        \bool_set_true:N \l_@@_exp_bracket_bool
        \bool_set_true:N \l_@@_exp_combine_bool
      } ,
    product-exponents / individual .code:n =
      {
        \bool_set_false:N \l_@@_exp_bracket_bool
        \bool_set_false:N \l_@@_exp_combine_bool
      } ,
    product-mode .choices:nn =
      { phrase , symbol }
      { \tl_set_eq:NN \l_@@_mode_tl \l_keys_choice_tl } ,
    product-phrase .tl_set:N = \l_@@_phrase_tl ,
    product-symbol .tl_set:N = \l_@@_symbol_tl ,
    product-units .choice: ,
    product-units / bracket .code:n =
      {
        \bool_set_true:N  \l_@@_unit_bracket_bool
        \bool_set_false:N \l_@@_power_bool
        \bool_set_false:N \l_@@_unit_repeat_bool
      } ,
    product-units / bracket-power .code:n =
      {
        \bool_set_true:N  \l_@@_unit_bracket_bool
        \bool_set_true:N \l_@@_power_bool
        \bool_set_false:N \l_@@_unit_repeat_bool
      } ,
    product-units / power .code:n =
      {
        \bool_set_false:N  \l_@@_unit_bracket_bool
        \bool_set_true:N \l_@@_power_bool
        \bool_set_false:N \l_@@_unit_repeat_bool
      } ,
    product-units / repeat .code:n =
      {
        \bool_set_false:N \l_@@_unit_bracket_bool
        \bool_set_false:N \l_@@_power_bool
        \bool_set_true:N  \l_@@_unit_repeat_bool
      } ,
    product-units / single .code:n =
      {
        \bool_set_false:N \l_@@_unit_bracket_bool
        \bool_set_false:N \l_@@_power_bool
        \bool_set_false:N \l_@@_unit_repeat_bool
      }
  }
%    \end{macrocode}
% \end{variable}
%
% \begin{macro}{\siunitx_product:n}
% \begin{macro}{\@@_print:nn, \@@_print:xx}
% \begin{macro}[EXP]{\@@_phrase:, \@@_symbol:}
%   This works much like lists in many ways. First we collect up the data.
%    \begin{macrocode}
\cs_new_protected:Npn \siunitx_product:n #1
  {
    \group_begin:
      \bool_lazy_and:nnTF
        { \l_@@_exp_combine_bool }
        { \int_compare_p:nNn { \tl_count:n {#1} } > 1 }
        {
          \siunitx_compound_extract_exponents:nNN {#1}
            \l_@@_tmp_seq \l_@@_tmp_tl
        }
        {
          \tl_clear:N \l_@@_tmp_tl
          \siunitx_compound:nN {#1} \l_@@_tmp_seq
        }
      \bool_lazy_and:nnTF
        { \l_@@_exp_bracket_bool }
        { ! \tl_if_empty_p:N \l_@@_tmp_tl }
        {
          \@@_print:xx
            { \exp_not:V \l_@@_bracket_open_tl }
            {
              \exp_not:V \l_@@_bracket_close_tl
              \exp_not:V \l_@@_tmp_tl
            }
        }
        { \@@_print:xx { } { \exp_not:V \l_@@_tmp_tl } }
    \group_end:
  }
%    \end{macrocode}
%   Now we need to print: unlike a list, we don't need different separators
%   so the only requirement is to extract the first and last entries. That
%   is conveniently achieved by popping them off and handling them separately.
%    \begin{macrocode}
\cs_new_protected:Npn \@@_print:nn #1#2
  {
    \int_compare:nNnTF { \seq_count:N \l_@@_tmp_seq } > 1
      {
        \seq_pop_left:NN \l_@@_tmp_seq \l_@@_tmp_tl
        \siunitx_print:nx { number }
          {
            \exp_not:n {#1}
            \exp_not:V \l_@@_tmp_tl
          }
        \seq_pop_right:NN \l_@@_tmp_seq \l_@@_tmp_tl
        \seq_map_inline:Nn \l_@@_tmp_seq 
          {
            \siunitx_print:nx { number }
              {
                \use:c { @@_ \l_@@_mode_tl : }
                \exp_not:n {##1}
              }
          }
        \siunitx_print:nx { number }
          {
            \use:c { @@_ \l_@@_mode_tl : }
            \exp_not:V \l_@@_tmp_tl
            \exp_not:n {#2}
          }
      }
      {
        \siunitx_print:nx { number }
          { \seq_item:Nn \l_@@_tmp_seq { 1 } }
      }
  }
\cs_generate_variant:Nn \@@_print:nn { xx }
\cs_new:Npn \@@_phrase:
  { \exp_not:V \l_@@_phrase_tl }
\cs_new:Npn \@@_symbol:
  { { } \exp_not:V \l_@@_symbol_tl { } }
%    \end{macrocode}
% \end{macro}
% \end{macro}
% \end{macro}
%
% \subsection{Ranges}
%
% Identify the internal prefix (\LaTeX3 \pkg{DocStrip} convention): only
% internal material in this \emph{submodule} should be used directly.
%    \begin{macrocode}
%<@@=siunitx_range>
%    \end{macrocode}
%
% \begin{variable}{\l_@@_tmp_seq, \l_@@_tmp_tl}
%   Scratch space.
%    \begin{macrocode}
\seq_new:N \l_@@_tmp_seq
\tl_new:N \l_@@_tmp_tl
%    \end{macrocode}
% \end{variable}
%
% \begin{variable}{\l_@@_bracket_close_tl, \l_@@_bracket_open_tl}
%   Purely internal for the present.
%    \begin{macrocode}
\tl_new:N \l_@@_bracket_close_tl
\tl_new:N \l_@@_bracket_open_tl
\tl_set:Nn \l_@@_bracket_open_tl { ( }
\tl_set:Nn \l_@@_bracket_close_tl { ) }
%    \end{macrocode}
% \end{variable}
%
% \begin{variable}
%   {
%     \l_siunitx_range_phrase_tl ,
%     \l_@@_exp_bracket_bool     ,
%     \l_@@_exp_combine_bool     ,
%     \l_@@_unit_bracket_bool    ,
%     \l_@@_unit_repeat_bool
%   }
%    \begin{macrocode}
\bool_new:N \l_@@_exp_bracket_bool
\bool_new:N \l_@@_exp_combine_bool
\bool_new:N \l_@@_unit_bracket_bool
\bool_new:N \l_@@_unit_repeat_bool
\keys_define:nn { siunitx }
  {
    range-exponents .choice: ,
    range-exponents / combine .code:n =
      {
        \bool_set_false:N \l_@@_exp_bracket_bool
        \bool_set_true:N \l_@@_exp_combine_bool
      } ,
    range-exponents / combine-bracket .code:n =
      {
        \bool_set_true:N \l_@@_exp_bracket_bool
        \bool_set_true:N \l_@@_exp_combine_bool
      } ,
    range-exponents / individual .code:n =
      {
        \bool_set_false:N \l_@@_exp_bracket_bool
        \bool_set_false:N \l_@@_exp_combine_bool
      } ,
    range-phrase .tl_set:N = \l_siunitx_range_phrase_tl ,
    range-units .choice: ,
    range-units / bracket .code:n =
      {
        \bool_set_true:N  \l_@@_unit_bracket_bool
        \bool_set_false:N \l_@@_unit_repeat_bool
      } ,
    range-units / repeat .code:n =
      {
        \bool_set_false:N \l_@@_unit_bracket_bool
        \bool_set_true:N  \l_@@_unit_repeat_bool
      } ,
    range-units / single .code:n =
      {
        \bool_set_false:N \l_@@_unit_bracket_bool
        \bool_set_false:N \l_@@_unit_repeat_bool
      }
  }
%    \end{macrocode}
% \end{variable}
%
% \begin{macro}{\siunitx_range:nn, \@@_aux:nn}
% \begin{macro}{\@@_print:nn, \@@_print:xx}
% \begin{macro}{\@@_single:n}
%   A range is relatively easy to deal with. After dealing with dodgy input,
%   we simply combine the two entries into a list then use the generic
%   mechanism. Printing here is reality easy as there is no loop.
%    \begin{macrocode}
\cs_new_protected:Npn \siunitx_range:nn #1#2
  {
    \tl_if_blank:nTF {#1}
      { \@@_single:n {#2} }
      {
        \tl_if_blank:nTF {#2}
          { \@@_single:n {#1} }
          { \@@_aux:nn {#1} {#2} }
      }
  }
\cs_new_protected:Npn \@@_aux:nn #1#2
  {
    \group_begin:
      \bool_if:NTF \l_@@_exp_combine_bool
        {
          \siunitx_compound_extract_exponents:nNN { {#1} {#2} }
            \l_@@_tmp_seq \l_@@_tmp_tl
        }
        {
          \tl_clear:N \l_@@_tmp_tl
          \siunitx_compound:nN { {#1} {#2} } \l_@@_tmp_seq
        }
      \bool_lazy_and:nnTF
        { \l_@@_exp_bracket_bool }
        { ! \tl_if_empty_p:N \l_@@_tmp_tl }
        {
          \@@_print:xx
            {
              \exp_not:V \l_@@_bracket_open_tl
              \seq_item:Nn \l_@@_tmp_seq { 1 }
            }
            {
              \seq_item:Nn \l_@@_tmp_seq { 2 }
              \exp_not:V \l_@@_bracket_close_tl
              \exp_not:V \l_@@_tmp_tl
            }
        }
        {
          \@@_print:xx
            { \seq_item:Nn \l_@@_tmp_seq { 1 } }
            {
              \seq_item:Nn \l_@@_tmp_seq { 2 }
              \exp_not:V \l_@@_tmp_tl
            }
        }
    \group_end:
  }
\cs_new_protected:Npn \@@_print:nn #1#2
  {
    \siunitx_print:nn { number } {#1}
    \tl_if_blank:nF {#2} { \tl_use:N \l_siunitx_range_phrase_tl }
    \siunitx_print:nn { number } {#2}
  }
\cs_generate_variant:Nn \@@_print:nn { xx }
\cs_new_protected:Npn \@@_single:n #1
  {
    \group_begin:
      \siunitx_numnber_format:nN {#1} \l_@@_tmp_tl
      \siunitx_print:nV { number } \l_@@_tmp_tl
    \group_end:
  }
%    \end{macrocode}
% \end{macro}
% \end{macro}
% \end{macro}
%
% \subsection{Standard settings for module options}
%
% Some of these follow naturally from the point of definition
% (\foreign{e.g.}~boolean variables are always |false| to begin with),
% but for clarity everything is set here.
%    \begin{macrocode}
\keys_set:nn { siunitx }
  {
    list-exponents       = individual        ,
    list-final-separator = \text { ~ and ~ } ,
    list-pair-separator  = \text { ~ and ~ } ,
    list-separator       = \text { , ~ }     ,
    list-units           = repeat            ,
    product-exponents    = individual        ,
    product-mode         = symbol            ,
    product-phrase       = \text { ~ by ~ }  ,
    product-symbol       = \times            ,
    product-units        = repeat            ,
    range-exponents      = individual        ,
    range-phrase         = \text { ~ to ~ }  ,
    range-units          = repeat
  }
%    \end{macrocode}
%
%    \begin{macrocode}
%</package>
%    \end{macrocode}
%
% \end{implementation}
%
% \PrintIndex

% \iffalse meta-comment
%
% File: siunitx-print.dtx Copyright (C) 2016-2019,2021 Joseph Wright
%
% It may be distributed and/or modified under the conditions of the
% LaTeX Project Public License (LPPL), either version 1.3c of this
% license or (at your option) any later version.  The latest version
% of this license is in the file
%
%    https://www.latex-project.org/lppl.txt
%
% This file is part of the "siunitx bundle" (The Work in LPPL)
% and all files in that bundle must be distributed together.
%
% The released version of this bundle is available from CTAN.
%
% -----------------------------------------------------------------------
%
% The development version of the bundle can be found at
%
%    https://github.com/josephwright/siunitx
%
% for those people who are interested.
%
% -----------------------------------------------------------------------
%
%<*driver>
\documentclass{l3doc}
\ProvideDocumentCommand\foreign{m}{\textit{#1}}
% The next line is needed so that \GetFileInfo will be able to pick up
% version data
\usepackage{siunitx}
\begin{document}
  \DocInput{\jobname.dtx}
\end{document}
%</driver>
% \fi
%
% \GetFileInfo{siunitx.sty}
%
% \title{^^A
%   \pkg{siunitx-print} -- Printing material with font control^^A
%   \thanks{This file describes \fileversion,
%     last revised \filedate.}^^A
% }
%
% \author{^^A
%  Joseph Wright^^A
%  \thanks{^^A
%    E-mail:
%    \href{mailto:joseph.wright@morningstar2.co.uk}
%      {joseph.wright@morningstar2.co.uk}^^A
%   }^^A
% }
%
% \date{Released \filedate}
%
% \maketitle
%
% \begin{documentation}
%
% \section{Printing quantities}
%
% This submodule is focussed on providing controlled printing for numbers and
% units. Key to this is control of font: conventions for printing quantities
% mean that the exact nature of the output is important. At the same time, this
% module provides flexibility for the user in terms of which aspects of the font
% are responsive to the surrounding general text. Printing material may also
% take place in text or math mode.
%
% The printing routines assume that normal \LaTeXe{} font selection commands
% are available, in particular \cs{bfseries}, \cs{mathrm}, \cs{mathversion},
% \cs{fontfamily}, \cs{fontseries} and \cs{fontshape}, \cs{familydefault},
% \cs{seriesdefault}, \cs{shapedefault} and \cs{selectfont}.
% It also requires the standard
% \LaTeXe{} kernel commands \cs{ensuremath}, \cs{mbox}, \cs{textsubscript} and
% \cs{textsuperscript} for printing in text mode. The following packages are
% also required to provide the functionality detailed.
% \begin{itemize}
%   \item \pkg{color}: support for color using \cs{textcolor}
%   \item \pkg{textcomp}: \cs{textminus}, \cs{textpm} \cs{texttimes} and
%     \cs{textcenteredperiod} for printing in text mode
%   \item \pkg{amstext}: the \cs{text} command for printing in text mode
% \end{itemize}
% For detection of math mode fonts, as well as \cs{mathrm}, the existence of
% \cs{symoperators} is assumed; other math font commands are not
% \emph{required} to exist.
%
% \begin{function}
%   {
%     \siunitx_print_number:n, \siunitx_print_number:V,
%       \siunitx_print_number:x,
%     \siunitx_print_unit:n, \siunitx_print_unit:V,
%       \siunitx_print_unit:x
%   }
%   \begin{syntax}
%     \cs{siunitx_print_number:n} \Arg{material}
%     \cs{siunitx_print_unit:n} \Arg{material}
%   \end{syntax}
%   Prints the \meta{material} according the the prevailing settings for the
%   submodule as applicable to the \meta{type} of content (|number| or |unit|).
%   The \meta{material} should comprise normal
%   \LaTeX{} mark-up for numbers or units. In particular, units will typically
%   use |\mathrm| to indicate material to be printed in the current upright
%   roman font, and |^| and |_| will typically be used to indicate super- and
%   subscripts, respectively. These elements will be correctly handled when
%   printing for example using |\mathsf| in math mode, or using only text
%   fonts.
% \end{function}
%
% \begin{function}
%   {\siunitx_print_match:n, \siunitx_print_math:n, \siunitx_print_text:n}
%   \begin{syntax}
%     \cs{siunitx_print_match:n} \Arg{material}
%     \cs{siunitx_print_math:n} \Arg{material}
%     \cs{siunitx_print_text:n} \Arg{material}
%   \end{syntax}
%   Prints the \meta{material} as described for \cs{siunitx_print_\dots:n} but
%   with a fixed text or math mode output. The printing does \emph{not} set
%   color (which is managed on a |unit|/|number| basis), but otherwise sets
%   the font as described above. The |match| function uses either the prevailing
%   math or text mode.
% \end{function}
%
% \subsection{Key--value options}
%
% The options defined by this submodule are available within the \pkg{l3keys}
% |siunitx| tree.
%
% \begin{function}{color}
%   \begin{syntax}
%     |color| = \meta{color}
%   \end{syntax}
%   Color to apply to printed output: the latter should be a named color
%   defined for use with \cs{textcolor}. The standard setting is empty (no
%   color).
% \end{function}
%
% \begin{function}{mode}
%   \begin{syntax}
%     |mode| = |match|\verb"|"|math|\verb"|"|text|
%   \end{syntax}
%   Selects which mode (math or text) the output is printed in: a choice
%   from the options |match|, |math| or |text|. The option |match| matches
%   the mode prevailing at the point \cs{siunitx_print_\dots:n} is called. The
%   |math| and |text| options choose the relevant \TeX{} mode for printing.
%   The standard setting is |math|.
% \end{function}
%
% \begin{function}{number-color}
%   \begin{syntax}
%     |number-color| = \meta{color}
%   \end{syntax}
%   Color to apply to numbers in output: the latter should be a named color
%   defined for use with \cs{textcolor}. The standard setting is empty (no
%   color).
% \end{function}
%
% \begin{function}{number-mode}
%   \begin{syntax}
%     |number-mode| = |match|\verb"|"|math|\verb"|"|text|
%   \end{syntax}
%   Selects which mode (math or text) the numbers are printed in: a choice
%   from the options |match|, |math| or |text|. The option |match| matches
%   the mode prevailing at the point \cs{siunitx_prin_number:n} is called. The
%   |math| and |text| options choose the relevant \TeX{} mode for printing.
%   The standard setting is |math|.
% \end{function}
%
% \begin{function}{propagate-math-font}
%   \begin{syntax}
%     |propagate-math-font| = |true|\verb"|"|false|
%   \end{syntax}
%   Switch to determine if the currently-active math font is applied within
%   printed output. This is relevant only when \cs{siunitx_print_\dots:n} is
%   called from within math mode: in text mode there is not active math
%   font. When not active, math mode material will be typeset using
%   standard math mode fonts without any changes being made to the
%   supplied argument. The standard setting is |false|.
% \end{function}
%
% \begin{function}{reset-math-version}
%   \begin{syntax}
%     |reset-math-version| = |true|\verb"|"|false|
%   \end{syntax}
%   Switch to determine whether the active \cs{mathversion} is reset to
%   |normal| when printing in math mode. Note that math version is typically
%   used to select \cs{boldmath}, though it is also be used by
%   \foreign{e.g.}~\pkg{sansmath}. The standard setting is |true|.
% \end{function}
%
% \begin{function}{reset-text-family}
%   \begin{syntax}
%     |reset-text-family| = |true|\verb"|"|false|
%   \end{syntax}
%   Switch to determine whether the active text family is reset to
%   \cs{rmfamily} when printing in text mode. The standard setting is |true|.
% \end{function}
%
% \begin{function}{reset-text-series}
%   \begin{syntax}
%     |reset-text-series| = |true|\verb"|"|false|
%   \end{syntax}
%   Switch to determine whether the active text series is reset to
%   \cs{mdseries} when printing in text mode. The standard setting is |true|.
% \end{function}
%
% \begin{function}{reset-text-shape}
%   \begin{syntax}
%     |reset-text-shape| = |true|\verb"|"|false|
%   \end{syntax}
%   Switch to determine whether the active text shape is reset to
%   \cs{upshape} when printing in text mode. The standard setting is |true|.
% \end{function}
%
% \begin{function}{text-family-to-math}
%   \begin{syntax}
%     |text-family-to-math| = |true|\verb"|"|false|
%   \end{syntax}
%   Switch to determine if the family of the current text font should be
%   applied (where possible) to printing in math mode. The standard setting is
%   |false|.
% \end{function}
%
% \begin{function}{text-font-command}
%   \begin{syntax}
%     |text-font-command| = \meta{cmd}
%   \end{syntax}
%   Command applied to text during output, inserted after any reset of
%   font set-up. This can therefore be used to apply non-standard font
%   set up when printing in text mode. The standard setting is empty.
% \end{function}
%
% \begin{function}{text-series-to-math}
%   \begin{syntax}
%     |text-series-to-math| = |true|\verb"|"|false|
%   \end{syntax}
%   Switch to determine if the weight of the current text font should be
%   applied (where possible) to printing in math mode. This is achieved by
%   setting the \cs{mathversion}, and so will override |reset-math-version|.
%   The mappings between text and math weight are set . The standard setting is |false|.
% \end{function}
%
% \begin{function}{unit-color}
%   \begin{syntax}
%     |unit-color| = \meta{color}
%   \end{syntax}
%   Color to apply to units in output: the latter should be a named color
%   defined for use with \cs{textcolor}. The standard setting is empty (no
%   color).
% \end{function}
%
% \begin{function}{unit-mode}
%   \begin{syntax}
%     |unit-mode| = |match|\verb"|"|math|\verb"|"|text|
%   \end{syntax}
%   Selects which mode (math or text) units are printed in: a choice
%   from the options |match|, |math| or |text|. The option |match| matches
%   the mode prevailing at the point \cs{siunitx_print_\dots:n} is called. The
%   |math| and |text| options choose the relevant \TeX{} mode for printing.
%   The standard setting is |math|.
% \end{function}
%
% \begin{function}{series-version-mapping}
%   \begin{syntax}
%     |series-version-mapping| |/| \meta{weight} = \meta{version}
%   \end{syntax}
%   Defines how \pkg{siunitx} maps from text font weight to math font
%   version. The pre-defined weights are those used as-standard by
%   \pkg{autoinst}:
%   \begin{itemize}
%     \item \texttt{ul}
%     \item \texttt{el}
%     \item \texttt{l}
%     \item \texttt{sl}
%     \item \texttt{m}
%     \item \texttt{sb}
%     \item \texttt{b}
%     \item \texttt{eb}
%     \item \texttt{ub}
%   \end{itemize}
%   As standard, the \texttt{m} weight maps to \texttt{normal} math version
%   whilst all of the \texttt{b} weights map to \texttt{bold} and all of the
%   \texttt{l} weights map to \texttt{light}.
% \end{function}
%
% \end{documentation}
%
% \begin{implementation}
%
% \section{\pkg{siunitx-print} implementation}
%
% Start the \pkg{DocStrip} guards.
%    \begin{macrocode}
%<*package>
%    \end{macrocode}
%
% Identify the internal prefix (\LaTeX3 \pkg{DocStrip} convention): only
% internal material in this \emph{submodule} should be used directly.
%    \begin{macrocode}
%<@@=siunitx_print>
%    \end{macrocode}
%
% \subsection{Initial set up}
%
% The printing routines depend on \pkg{amstext} for text mode working.
%    \begin{macrocode}
\RequirePackage { amstext }
%    \end{macrocode}
%
% Color support is always required.
%    \begin{macrocode}
\RequirePackage { color }
%    \end{macrocode}
%
% \begin{macro}{\tl_replace_all:NVn}
%   Required variants.
%    \begin{macrocode}
\cs_generate_variant:Nn \tl_replace_all:Nnn { NV }
%    \end{macrocode}
% \end{macro}
%
% \begin{variable}{\l_@@_tmp_tl}
%   Scratch space.
%    \begin{macrocode}
\tl_new:N \l_@@_tmp_tl
%    \end{macrocode}
% \end{variable}
%
% \subsection{Printing routines}
%
% \begin{variable}
%   {
%     \l_@@_number_color_tl   ,
%     \l_@@_number_mode_tl    ,
%     \l_@@_unit_color_tl     ,
%     \l_@@_unit_mode_tl      ,
%     \l_@@_math_font_bool    ,
%     \l_@@_math_version_bool ,
%     \l_@@_math_family_bool  ,
%     \l_@@_text_font_tl      ,
%     \l_@@_math_weight_bool
%   }
%   Options which apply to the main formatting routine, and so are not tied
%   to either symbolic or literal input.
%    \begin{macrocode}
\tl_new:N \l_@@_number_mode_tl
\tl_new:N \l_@@_unit_mode_tl
\keys_define:nn { siunitx }
  {
    color .meta:n =
      { number-color = #1 , unit-color = #1 } ,
    mode .meta:n =
      { number-mode = #1 , unit-mode = #1 } ,
    number-color .tl_set:N =
      \l_@@_number_color_tl ,
    number-mode .choices:nn =
      { match , math , text }
      {
        \tl_set_eq:NN
          \l_@@_number_mode_tl \l_keys_choice_tl
      } ,
    propagate-math-font .bool_set:N =
      \l_@@_math_font_bool ,
    reset-math-version .bool_set:N =
      \l_@@_math_version_bool ,
    reset-text-family .bool_set:N =
      \l_@@_text_family_bool ,
    reset-text-series .bool_set:N =
      \l_@@_text_series_bool ,
    reset-text-shape .bool_set:N =
      \l_@@_text_shape_bool ,
    text-family-to-math .bool_set:N =
      \l_@@_math_family_bool ,
    text-font-command .tl_set:N =
      \l_@@_text_font_tl ,
    text-series-to-math .bool_set:N =
      \l_@@_math_weight_bool ,
    unit-color .tl_set:N =
      \l_@@_unit_color_tl ,
    unit-mode .choices:nn =
      { match , math , text }
      {
        \tl_set_eq:NN
          \l_@@_unit_mode_tl \l_keys_choice_tl
      }
  }
%    \end{macrocode}
% \end{variable}
%
% \begin{variable}
%   {
%      \l_@@_version_ul_tl ,
%      \l_@@_version_el_tl ,
%      \l_@@_version_l_tl ,
%      \l_@@_version_sl_tl ,
%      \l_@@_version_m_tl ,
%      \l_@@_version_sb_tl ,
%      \l_@@_version_b_tl ,
%      \l_@@_version_eb_tl ,
%      \l_@@_version_ub_tl
%   }
%  One set of \enquote{focussed} options.
%    \begin{macrocode}
\keys_define:nn { siunitx / series-version-mapping }
  {
    ul . tl_set:N = \l_@@_version_ul_tl ,
    el . tl_set:N = \l_@@_version_el_tl ,
    l  . tl_set:N = \l_@@_version_l_tl ,
    sl . tl_set:N = \l_@@_version_sl_tl ,
    m  . tl_set:N = \l_@@_version_m_tl ,
    sb . tl_set:N = \l_@@_version_sb_tl ,
    b  . tl_set:N = \l_@@_version_b_tl ,
    eb . tl_set:N = \l_@@_version_eb_tl ,
    ub . tl_set:N = \l_@@_version_ub_tl
  }
%    \end{macrocode}
% \end{variable}
%
% \begin{macro}
%   {
%     \siunitx_print_number:n, \siunitx_print_number:V,
%       \siunitx_print_number:x,
%     \siunitx_print_unit:n, \siunitx_print_unit:V,
%       \siunitx_print_unit:x
%   }
% \begin{macro}{\@@_aux:nn}
%   The main printing function doesn't actually need to do very much: just set
%   the color and select the correct sub-function.
%    \begin{macrocode}
\cs_new_protected:Npn \siunitx_print_number:n #1
  { \@@_aux:nn { number } {#1} }
\cs_generate_variant:Nn \siunitx_print_number:n { V , x }
\cs_new_protected:Npn \siunitx_print_unit:n #1
  { \@@_aux:nn { unit } {#1} }
\cs_generate_variant:Nn \siunitx_print_unit:n { V , x }
\cs_new_protected:Npn \@@_aux:nn #1#2
  {
    \tl_if_empty:cTF { l_@@_ #1 _color_tl }
      { \use:n }
      { \exp_args:Nv \textcolor { l_@@_ #1 _color_tl } }
        {
          \use:c
            {
              siunitx_print_
              \tl_use:c { l_@@_ #1 _mode_tl } :n
            }
              {#2}
        }
  }
%    \end{macrocode}
% \end{macro}
% \end{macro}
%
% \begin{macro}{\siunitx_print_match:n}
%   When the \emph{output} mode should match the input, a simple selection of
%   route can be made.
%    \begin{macrocode}
\cs_new_protected:Npn \siunitx_print_match:n #1
  {
    \mode_if_math:TF
      { \siunitx_print_math:n {#1} }
      { \siunitx_print_text:n {#1} }
  }
%    \end{macrocode}
% \end{macro}
%
% \begin{macro}{\@@_replace_font:N}
%   A simple auxiliary for \enquote{zapping} the unit font.
%    \begin{macrocode}
\cs_new_protected:Npn \@@_replace_font:N #1
  {
    \tl_if_empty:NF \l_siunitx_unit_font_tl
      {
        \tl_replace_all:NVn #1
          \l_siunitx_unit_font_tl
          { \use:n }
      }
  }
%    \end{macrocode}
% \end{macro}
%
% \begin{variable}
%   {
%      \c_@@_weight_uc_tl ,
%      \c_@@_weight_ecl_tl ,
%      \c_@@_weight_c_tl ,
%      \c_@@_weight_sc_tl ,
%      \c_@@_weight_sx_tl ,
%      \c_@@_weight_x_tl ,
%      \c_@@_weight_ex_tl ,
%      \c_@@_weight_ux_tl
%   }
%  Font widths where the |m| for weight is omitted.
%    \begin{macrocode}
\clist_map_inline:nn { uc , ec , c , sc , sx , x , ex , ux }
  { \tl_const:cn { c_@@_weight_ #1 _tl } { m } }
%    \end{macrocode}
% \end{variable}
%
% \begin{variable}
%   {
%      \c_@@_weight_l_tl ,
%      \c_@@_weight_m_tl ,
%      \c_@@_weight_b_tl
%   }
%  Font widths with one letter.
%    \begin{macrocode}
\clist_map_inline:nn { l , m , b }
  { \tl_const:cn { c_@@_weight_ #1 _tl } { #1 } }
%    \end{macrocode}
% \end{variable}
%
% \begin{macro}{\siunitx_print_math:n}
% \begin{macro}[EXP]{\@@_extract_series:Nw}
% \begin{macro}[EXP]{\@@_convert_series:n, \@@_convert_series:v}
% \begin{macro}{\@@_math_version:nn, \@@_math_version:Vn}
% \begin{macro}
%   {
%     \@@_math_auxi:n,
%     \@@_math_auxii:n,
%     \@@_math_auxiii:n,
%     \@@_math_auxiv:n,
%     \@@_math_auxv:n
%   }
% \begin{macro}{\@@_math_aux:N}
% \begin{macro}{\@@_math_aux:w}
% \begin{macro}{\@@_math_aux:Nn, \@@_math_aux:cn}
% \begin{macro}{\@@_math_sub:n, \@@_math_super:n, \@@_math_script:n}
% \begin{macro}{\@@_math_text:n}
%   The first step in setting in math mode is to check on the math version.
%   The starting point is the question of whether text series needs to
%   propagate to math mode: if so, check on the mapping, otherwise check on
%   the current math version.
%    \begin{macrocode}
\cs_new_protected:Npn \siunitx_print_math:n #1
  {
    \bool_if:NTF \l_@@_math_weight_bool
      {
        \tl_set:Nx \l_@@_tmp_tl
          { \exp_after:wN \@@_extract_series:Nw \f@series ? \q_stop }
        \tl_if_empty:NTF \l_@@_tmp_tl
          { \@@_math_auxi:n {#1} }
          { \@@_math_version:Vn \l_@@_tmp_tl {#1} }
      }
      { \@@_math_auxi:n {#1} }
  }
%    \end{macrocode}
%   Look up the math version from the text series. The weight is omitted
%   if it is |m| plus there are either one or two letters, so we have a little
%   work to do. To keep things fast, we use a hash table based lookup rather
%   than a sequence or property list.
%    \begin{macrocode}
\cs_new:Npn \@@_extract_series:Nw  #1#2 ? #3 \q_stop
  {
    \cs_if_exist:cTF { c_@@_weight_ #1#2 _tl }
      { \@@_convert_series:v { c_@@_weight_ #1#2 _tl } }
      {
        \cs_if_exist:cTF { c_@@_weight_ #1 _tl }
          { \@@_convert_series:v { c_@@_weight_ #1 _tl } }
          { \@@_convert_series:n {#1#2} }
      }
  }
\cs_new:Npn \@@_convert_series:n #1
  { \tl_use:c { l_@@_version_ #1 _tl } }
\cs_generate_variant:Nn \@@_convert_series:n { v }
\cs_new_protected:Npn \@@_math_auxi:n #1
  {
    \bool_if:NTF \l_@@_math_version_bool
      { \@@_math_version:nn { normal } {#1} }
      { \@@_math_auxii:n {#1} }
  }
%    \end{macrocode}
%   Any setting which changes the math version can only be set from text mode
%   (as it applies at the level of a formula). As such, the first test is to
%   see if that needs to be to check if the math version has to be set: if so,
%   switch to text mode, sort it out and switch back. That of course means
%   that in such cases, line breaking will not be possible.
%    \begin{macrocode}
\cs_new_protected:Npn \@@_math_version:nn #1#2
  {
    \str_if_eq:VnTF \math@version { #1 }
      { \@@_math_auxii:n {#2} }
      {
        \mode_if_math:TF
          { \text }
          { \use:n }
            {
              \mathversion {#1}
              \@@_math_auxii:n {#2}
            }
       }
  }
\cs_generate_variant:Nn \@@_math_version:nn { V }
%    \end{macrocode}
%   At this point, force math mode then start dealing with setting math font
%   based on text family. If the text family is roman, life is slightly
%   different to if it is sanserif or monospaced. In all cases, the outcomes
%   can be handled using the same routines as for normal math mode treatment.
%   The test here is on a string basis as |\f@family| and the |\...default|
%   commands have different |\long| status.
%    \begin{macrocode}
\cs_new_protected:Npn \@@_math_auxii:n #1
  { \ensuremath { \@@_math_auxiii:n {#1} } }
\cs_new_protected:Npn \@@_math_auxiii:n #1
  {
    \bool_if:NTF \l_@@_math_family_bool
      {
        \str_case_e:nnF { \f@family }
          {
            { \rmdefault } { \@@_math_auxv:n }
            { \sfdefault } { \@@_math_aux:Nn \mathsf }
            { \ttdefault } { \@@_math_aux:Nn \mathtt }
          }
          { \@@_math_auxiv:n }
      }
      { \@@_math_auxiv:n }
        {#1}
  }
%    \end{macrocode}
%   Now we deal with the font selection in math mode. There are two possible
%   cases. First, we are retaining the current math font, and the active one is
%   \cs{mathsf} or \cs{mathtt}: that needs to be applied to the argument.
%   Alternatively, if the current font is not retained, ensure that
%   normal math mode rules are active.
%    \begin{macrocode}
\cs_new_protected:Npn \@@_math_auxiv:n #1
  {
    \bool_if:NTF \l_@@_math_font_bool
      {
        \@@_math_aux:N
          \mathbf \mathit \mathsf \mathtt
          \q_recursion_tail \q_recursion_stop
      }
      { \@@_math_auxv:n }
        {#1}
  }
\cs_new_protected:Npn \@@_math_auxv:n #1
  {
    \bool_lazy_or:nnTF
      { \int_compare_p:nNn \fam = { -1 } }
      { \int_compare_p:nNn \fam = \symoperators }
      { \use:n }
      { \mathrm }
        {#1}
  }
\cs_new_protected:Npn \@@_math_aux:N #1
  {
    \quark_if_recursion_tail_stop_do:Nn #1 { \use:n }
    \exp_after:wN \exp_after:wN \exp_after:wN \@@_math_aux:w
      \cs:w \cs_to_str:N #1 \c_space_tl \cs_end:
        \use@mathgroup ? { -2 } \q_stop #1
  }
\cs_new_protected:Npn \@@_math_aux:w #1 \use@mathgroup #2#3 #4 \q_stop #5
  {
    \int_compare:nNnTF \fam = {#3}
      { \use_i_delimit_by_q_recursion_stop:nw { \@@_math_aux:Nn #5 } }
      { \@@_math_aux:N }
  }
%    \end{macrocode}
%   Search-and-replace fun: deal with any font commands  in the argument and
%   also inside sub/superscripts.
%    \begin{macrocode}
\cs_new_protected:Npx \@@_math_aux:Nn #1#2
  {
    \group_begin:
      \tl_set:Nn \exp_not:N \l_@@_tmp_tl {#2}
      \@@_replace_font:N \exp_not:N \l_@@_tmp_tl
      \tl_replace_all:Nnn \exp_not:N \l_@@_tmp_tl
        { \char_generate:nn { `\_ } { 8 } }
        { \exp_not:N \@@_math_sub:n }
      \tl_replace_all:Nnn \exp_not:N \l_@@_tmp_tl
        { ^ }
        { \exp_not:N \@@_math_super:n }
      #1 { \exp_not:N \tl_use:N \exp_not:N \l_@@_tmp_tl }
    \group_end:
  }
\cs_generate_variant:Nn \@@_math_aux:Nn { c }
\cs_new_protected:Npx \@@_math_sub:n #1
  {
    \char_generate:nn { `\_ } { 8 }
      { \exp_not:N \@@_math_script:n {#1} }
  }
\cs_new_protected:Npn \@@_math_super:n #1
  { ^ { \@@_math_script:n {#1} } }
\cs_new_protected:Npn \@@_math_script:n #1
  {
    \group_begin:
      \tl_set:Nn \l_@@_tmp_tl {#1}
      \@@_replace_font:N \l_@@_tmp_tl
      \tl_use:N \l_@@_tmp_tl
    \group_end:
  }
%    \end{macrocode}
%   For \pkg{tex4ht}, we need to have category code $12$ |^| tokens in math
%   mode. We handle that by intercepting at the first auxiliary that makes
%   sense.
%    \begin{macrocode}
\AtBeginDocument
  {
    \@ifpackageloaded { tex4ht }
      {
        \cs_set_protected:Npn \@@_math_auxii:n #1
          {
            \tl_set:Nn \l_@@_tmp_tl {#1}
            \exp_args:NNnx \tl_replace_all:Nnn \l_@@_tmp_tl
              { ^ } { \token_to_str:N ^ }
            \ensuremath { \exp_args:NV \@@_math_auxiii:n \l_@@_tmp_tl }
          }
      }
      { }
  }
%    \end{macrocode}
% \end{macro}
% \end{macro}
% \end{macro}
% \end{macro}
% \end{macro}
% \end{macro}
% \end{macro}
% \end{macro}
% \end{macro}
% \end{macro}
%
% \begin{macro}{\siunitx_print_text:n}
% \begin{macro}{\@@_text_replace:n}
% \begin{macro}{\@@_text_replace:N}
% \begin{macro}{\@@_text_replace:NNn}
% \begin{macro}{\@@_text_replace:Nnnn}
% \begin{macro}{\@@_text_replace_frac:n}
% \begin{macro}{\@@_text_sub:n, \@@_text_super:n}
% \begin{macro}{\@@_text_scripts:NnN}
% \begin{macro}{\@@_text_scripts:}
% \begin{macro}{\@@_text_scripts_one:NnN}
% \begin{macro}{\@@_text_scripts_two:NnNn}
% \begin{macro}{\@@_text_scripts_two:nn}
% \begin{macro}{\@@_text_scripts_two:n}
% \begin{macro}{\@@_text_fraction:Nnn}
%   Typesetting in text mode is easy in font control terms but more tricky
%   in the manipulation of the input. The easy part comes first.
%    \begin{macrocode}
\cs_new_protected:Npn \siunitx_print_text:n #1
  {
    \text
      {
        \bool_if:NT \l_@@_text_family_bool
          { \fontfamily { \familydefault } }
        \bool_if:NT \l_@@_text_series_bool
          { \fontseries { \seriesdefault } }
        \bool_if:NT \l_@@_text_shape_bool
          { \fontshape { \shapedefault } }
        \bool_lazy_any:nT
          {
            { \l_@@_text_family_bool }
            { \l_@@_text_series_bool }
            { \l_@@_text_shape_bool }
          }
          { \selectfont }
        \tl_use:N \l_@@_text_font_tl
        \exp_args:NnV \tl_if_head_eq_meaning:nNTF {#1} \l_siunitx_unit_fraction_tl
          { \@@_text_fraction:Nnn #1 }
          { \@@_text_replace:n {#1} }
      }
  }
%    \end{macrocode}
%   To get math mode material to print in text mode, various search-and-replace
%   steps are needed.
%    \begin{macrocode}
\cs_new_protected:Npn \@@_text_replace:n #1
  {
    \group_begin:
    \tl_if_head_eq_meaning:nNTF {#1} \mathchoice
      { \@@_text_replace:Nnnnn #1 }
      {
        \tl_set:Nn \l_@@_tmp_tl {#1}
        \@@_text_replace:N \l_@@_tmp_tl
        \tl_use:N \l_@@_tmp_tl
      }
    \group_end:
  }
\cs_new_protected:Npx \@@_text_replace:N #1
  {
    \@@_replace_font:N #1
    \exp_not:N \@@_text_replace:NNn #1
      \exp_not:N \mathord { }
      \exp_not:N \pm
        { \exp_not:N \textpm }
      \exp_not:N \mp
        { \exp_not:n { \ensuremath { \mp } } }
      -
        { \exp_not:N \textminus }
      \exp_not:N \times
        { \exp_not:N \texttimes }
      \exp_not:N \cdot
        { \exp_not:N \textperiodcentered }
      \char_generate:nn { `\_ } { 8 }
        { \exp_not:N \@@_text_sub:n }
      ^
        { \exp_not:N \@@_text_super:n }
      \exp_not:N \q_recursion_tail
        { ? }
      \exp_not:N \q_recursion_stop
  }
\cs_new_protected:Npn \@@_text_replace:NNn #1#2#3
  {
    \quark_if_recursion_tail_stop:N #2
    \tl_replace_all:Nnn #1 {#2} {#3}
    \@@_text_replace:NNn #1
  }
\cs_new_protected:Npn \@@_text_replace:Nnnnn #1#2#3#4#5
  {
    \ensuremath
      {
        \mathchoice
          { \@@_print_replace_frac:n {#2} }
          { \@@_print_replace_frac:n {#3} }
          { \@@_print_replace_frac:n {#4} }
          { \@@_print_replace_frac:n {#5} }
      }
  }
%    \end{macrocode}
%   Almost the same as the lead-off but here we need to deal with re-inserting
%   a text mode shift.
%    \begin{macrocode}
\cs_new_protected:Npn \@@_print_replace_frac:n #1
  {
    \exp_args:NnV \tl_if_head_eq_meaning:nNTF {#1} \l_siunitx_unit_fraction_tl
      { \@@_text_fraction:Nnn #1 }
      { \mbox { \@@_text_replace:n {#1} } }
  }
%    \end{macrocode}
%   When the \pkg{bidi} package is loaded, we need to make sure
%   that \cs{text} is doing the correct thing.
%    \begin{macrocode}
\sys_if_engine_xetex:T
  {
    \AtBeginDocument
      {
         \@ifpackageloaded { bidi }
           {
             \cs_set_protected:Npn \@@_text_replace:n #1
               {
                 \group_begin:
                   \tl_set:Nn \l_@@_tmp_tl {#1}
                   \@@_text_replace:N \l_@@_tmp_tl
                   \LRE { \tl_use:N \l_@@_tmp_tl }
                 \group_end:
               }
           }
       { }
      }
  }
%    \end{macrocode}
%   Sub- and superscripts can be in any order in the source. The first step
%   of handling them is therefore to do a look-ahead to work out whether
%   only one or both are present.
%    \begin{macrocode}
\cs_new_protected:Npn \@@_text_sub:n #1
  {
    \@@_text_scripts:NnN
      \textsubscript {#1} \@@_text_super:n
  }
\cs_new_protected:Npn \@@_text_super:n #1
  {
    \@@_text_scripts:NnN
      \textsuperscript {#1} \@@_text_sub:n
  }
\cs_new_protected:Npn \@@_text_scripts:NnN #1#2#3
  {
    \cs_set_protected:Npn \@@_text_scripts:
      {
        \if_meaning:w \l_peek_token #3
          \exp_after:wN \@@_text_scripts_two:NnNn
        \else:
          \exp_after:wN \@@_text_scripts_one:Nn
        \fi:
          #1 {#2}
      }
    \peek_after:Nw \@@_text_scripts:
  }
\cs_new_protected:Npn \@@_text_scripts: { }
%    \end{macrocode}
%   In the simple case of one script item, we have to do a search-and-replace
%   to deal with anything inside the argument.
%    \begin{macrocode}
\cs_new_protected:Npn \@@_text_scripts_one:Nn #1#2
  {
    \group_begin:
      \tl_set:Nn \l_@@_tmp_tl {#2}
      \@@_text_replace:N \l_@@_tmp_tl
    \exp_args:NNV \group_end:
    #1 \l_@@_tmp_tl
  }
%    \end{macrocode}
%   For the two scripts case, we cannot use |\textsubscript|/|\textsuperscript|
%   as they don't stack directly. Instead, we sort out the ordering then use
%   an implementation for both parts that is the same as the kernel text
%   scripts.
%    \begin{macrocode}
\cs_new_protected:Npn \@@_text_scripts_two:NnNn #1#2#3#4
  {
    \cs_if_eq:NNTF #1 \textsubscript
      { \@@_text_scripts_two:nn {#4} {#2} }
      { \@@_text_scripts_two:nn {#2} {#4} }
  }
\cs_new_protected:Npx \@@_text_scripts_two:nn #1#2
  {
    \group_begin:
      \exp_not:N \m@th
      \exp_not:N \ensuremath
        {
          ^ { \exp_not:N \@@_text_scripts_two:n {#1} }
          \char_generate:nn { `\_ } { 8 }
            { \exp_not:N \@@_text_scripts_two:n {#2} }
        }
    \group_end:
  }
\cs_new_protected:Npn \@@_text_scripts_two:n #1
  {
    \mbox
      {
        \fontsize \sf@size \z@ \selectfont
        \@@_text_scripts_one:Nn \use:n {#1}
      }
  }
%    \end{macrocode}
%   Fraction commands are always math mode, so we have to go back and forth:
%   this is done after general font setting for performance reasons.
%    \begin{macrocode}
\cs_new_protected:Npn \@@_text_fraction:Nnn #1#2#3
  {
    \ensuremath
      { 
        #1
          { \mbox { \@@_text_replace:n {#2} } }
          { \mbox { \@@_text_replace:n {#3} } }
      }
  }
%    \end{macrocode}
% \end{macro}
% \end{macro}
% \end{macro}
% \end{macro}
% \end{macro}
% \end{macro}
% \end{macro}
% \end{macro}
% \end{macro}
% \end{macro}
% \end{macro}
% \end{macro}
% \end{macro}
% \end{macro}
%
% \subsection{Standard settings for module options}
%
% Some of these follow naturally from the point of definition
% (\foreign{e.g.}~boolean variables are always |false| to begin with),
% but for clarity everything is set here.
%    \begin{macrocode}
\keys_set:nn { siunitx }
  {
    color               =       ,
    mode                = math  ,
    number-color        =       ,
    number-mode         = math  ,
    propagate-math-font = false ,
    reset-math-version  = true  ,
    reset-text-shape    = true  ,
    reset-text-series   = true  ,
    reset-text-family   = true  ,
    text-family-to-math = false ,
    text-font-command   =       ,
    text-series-to-math = false ,
    unit-color          =       ,
    unit-mode           = math
  }
%    \end{macrocode}
%
% These are separate as they all fall inside the same key.
%    \begin{macrocode}
\keys_set:nn { siunitx / series-version-mapping }
  {
    ul = light  ,
    el = light  ,
    l  = light  ,
    sl = light  ,
    m  = normal ,
    sb = bold   ,
    b  = bold   ,
    eb = bold   ,
    ub = bold
  }
%    \end{macrocode}
%
%    \begin{macrocode}
%</package>
%    \end{macrocode}
%
% \end{implementation}
%
% \PrintIndex

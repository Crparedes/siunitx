% \iffalse meta-comment
%
% File: siunitx-locale.dtx Copyright (C) 2020 Joseph Wright
%
% It may be distributed and/or modified under the conditions of the
% LaTeX Project Public License (LPPL), either version 1.3c of this
% license or (at your option) any later version.  The latest version
% of this license is in the file
%
%    https://www.latex-project.org/lppl.txt
%
% This file is part of the "siunitx bundle" (The Work in LPPL)
% and all files in that bundle must be distributed together.
%
% The released version of this bundle is available from CTAN.
%
% -----------------------------------------------------------------------
%
% The development version of the bundle can be found at
%
%    https://github.com/josephwright/siunitx
%
% for those people who are interested.
%
% -----------------------------------------------------------------------
%
%<*driver>
\documentclass{l3doc}
% The next line is needed so that \GetFileInfo will be able to pick up
% version data
\usepackage{siunitx}
\begin{document}
  \DocInput{\jobname.dtx}
\end{document}
%</driver>
% \fi
%
% \GetFileInfo{siunitx.sty}
%
% \title{^^A
%   \pkg{siunitx-locale} -- Localisation^^A
%   \thanks{This file describes \fileversion,
%     last revised \filedate.}^^A
% }
%
% \author{^^A
%  Joseph Wright^^A
%  \thanks{^^A
%    E-mail:
%    \href{mailto:joseph.wright@morningstar2.co.uk}
%      {joseph.wright@morningstar2.co.uk}^^A
%   }^^A
% }
%
% \date{Released \filedate}
%
% \maketitle
%
% \begin{documentation}
%
% \end{documentation}
%
% \begin{implementation}
%
% \section{\pkg{siunitx-locale} implementation}
%
% Start the \pkg{DocStrip} guards.
%    \begin{macrocode}
%<*package>
%    \end{macrocode}
%
% Identify the internal prefix (\LaTeX3 \pkg{DocStrip} convention): only
% internal material in this \emph{submodule} should be used directly.
%    \begin{macrocode}
%<@@=siunitx_locale>
%    \end{macrocode}
%
% \subsection{Locales}
%
% The basics for defining locales are easy: these are just meta keys.
%    \begin{macrocode}
\keys_define:nn { siunitx }
  {
    locale .choice: ,
    locale / DE .meta:n =
      {
        exponent-product      = \cdot ,
        inter-unit-product    = \,    ,
        output-decimal-marker = { , }
      } ,
    locale / FR .meta:n =
      {
        exponent-product      = \times ,
        inter-unit-product    = \,     ,
        output-decimal-marker = { , }
      } ,
    locale / UK .meta:n =
      {
        exponent-product      = \times ,
        inter-unit-product    = \,     ,
        output-decimal-marker = .
      },
    locale / US .meta:n =
      {
        exponent-product      = \times ,
        inter-unit-product    = \,     ,
        output-decimal-marker = .
      } ,
    locale / ZA .meta:n =
      {
        exponent-product      = \times ,
        inter-unit-product    = \cdot  ,
        output-decimal-marker = { , }
      }
  }
%    \end{macrocode}
%
% \subsection{Localisation}
%
% Localisation makes use of the \pkg{translator} package. This only happens
% if it is available, and is transparent to the user.
%    \begin{macrocode}
\file_if_exist:nT { translations.sty }
  {
    \RequirePackage { translations }
    \DeclareTranslation { English } { to~(numerical~range) } { to }
    \DeclareTranslation { French }  { to~(numerical~range) } { \'{a} }
    \DeclareTranslation { German }  { to~(numerical~range) } { bis }
    \DeclareTranslation { Spanish } { to~(numerical~range) } { a }
    \keys_set:nn { siunitx }
      {
        list-final-separator = { ~ \GetTranslation { and } ~ },
        list-pair-separator  = { ~ \GetTranslation { and } ~ },
        range-phrase         = { ~ \GetTranslation { to~(numerical~range) } ~ }
      }
  }
%    \end{macrocode}
%
% \end{implementation}
%
% \PrintIndex

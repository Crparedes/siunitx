% \iffalse meta-comment
% !TEX program  = pdflatex
%<*internal>
\iffalse
%</internal>
%<*readme>
----------------------------------------------------------------
The siunitx package --- A comprehensive (SI) units package
Maintained by Joseph Wright
E-mail: joseph.wright@morningstar2.co.uk
Released under the LaTeX Project Public License v1.3c or later
See http://www.latex-project.org/lppl.txt
----------------------------------------------------------------

Typesetting values with units requires care to ensure that the
combined mathematical meaning of the value plus unit combination
is clear.  In particular, the SI units system lays down a
consistent set of units with rules on how these are to be used.
However, different countries and publishers have differing
conventions on the exact appearance of numbers (and units).

The siunitx package provides a set of tools for authors to 
typeset numbers and units in a consistent way.  The package has
an extended set of configuration options which make it possible 
to follow varying typographic conventions with the same input
syntax.  The package includes automated processing of numbers
and units, and the ability to control tabular alignment of
numbers.

A number of LaTeX packages have been developed in the past for
formatting units: SIunits, sistyle, unitsdef, units, fancyunits 
and fancynum.  Support for users of all of these packages is 
available as emulation modules in siunitx.  In addition, siunitx
can carry out many of the functions of the dcolumn, rccol and 
numprint packages
%</readme>
%<*internal>
\fi
\def\nameofplainTeX{plain}
\ifx\fmtname\nameofplainTeX\else
  \expandafter\begingroup
\fi
%</internal>
%<*install>
\input docstrip.tex
\keepsilent
\askforoverwritefalse
\preamble
---------------------------------------------------------------
The siunitx package --- A comprehensive (SI) units package
Maintained by Joseph Wright
E-mail: joseph.wright@morningstar2.co.uk
Released under the LaTeX Project Public License v1.3c or later
See http://www.latex-project.org/lppl.txt
---------------------------------------------------------------

\endpreamble
\postamble

Copyright (C) 2008-2009 by
  Joseph Wright <joseph.wright@morningstar2.co.uk>

It may be distributed and/or modified under the conditions of
the LaTeX Project Public License (LPPL), either version 1.3c of
this license or (at your option) any later version.  The latest
version of this license is in the file:

   http://www.latex-project.org/lppl.txt

This work is "maintained" (as per LPPL maintenance status) by
  Joseph Wright.

This work consists of the file  siunitx.dtx
          and the derived files siunitx.pdf,
                                siunitx.sty and
                                siunitx.ins.

\endpostamble
\usedir{tex/latex/siunitx}
\generate{
  \file{\jobname.sty}{\from{\jobname.dtx}{package}}
}
%</install>
%<install>\endbatchfile
%<*internal>
\usedir{source/latex/siunitx}
\generate{
  \file{\jobname.ins}{\from{\jobname.dtx}{install}}
}
\nopreamble\nopostamble
\usedir{doc/latex/siunitx}
\generate{
  \file{README.txt}{\from{\jobname.dtx}{readme}}
}
\ifx\fmtname\nameofplainTeX
  \expandafter\endbatchfile
\else
  \expandafter\endgroup
\fi
%</internal>
%<*driver|package>
\RequirePackage{expl3}
\GetIdInfo$Id$
  {A comprehensive (SI) units package}
%</driver|package>
%<*driver>
%\fi
\ProvidesFile{\filename.\filenameext}
  [\filedate\space v\fileversion\space\filedescription]
%\iffalse
\documentclass[full]{l3doc}
\usepackage{helvet,mathpazo}
\begin{document}
  \DocInput{\filename.\filenameext}
\end{document}
%</driver>
% \fi
% 
%\title{^^A
%  \textsf{siunitx} --- A comprehensive (SI) units package\thanks{^^A
%    This file describes version \fileversion, last revised 
%    \filedate.^^A
%  }^^A
%}
%\author{^^A
%  Joseph Wright\thanks{E-mail: joseph.wright@morningstar2.co.uk}^^A
%}
%\date{Released \filedate}
%
%\maketitle
%
%\changes{v1.0}{2008/06/15}{First official release}
%\changes{v1.1}{2008/09/28}{Package extended to a greater range of unit
%  types}
%\changes{v1.2}{2009/01/22}{Correct handling for ranges of numbers 
%  added}
%\changes{v2.0}{2009/06/29}{Complete re-write of package to add many new
%  features}
%  
%\begin{abstract}
% Typesetting values with units requires care to ensure that the
% combined mathematical meaning of the value plus unit combination is
% clear.  In particular, the SI units system lays down a consistent
% set of units with rules on how these are to be used.  However,
% different countries and publishers have differing conventions on
% the exact appearance of numbers (and units).
%
% The \pkg{siunitx} package provides a set of tools for authors
% to typeset numbers and units in a consistent way.  The package has an
% extended set of configuration options which make it possible to 
% follow varying typographic conventions with the same input syntax.  
% The package includes automated processing of numbers and units, and
% the ability to control tabular alignment of numbers.
%
% A number of \LaTeX\ packages have been developed in the past for
% formatting units: \pkg{SIunits}, \pkg{SIstyle}, 
% \pkg{unitsdef}, \pkg{units}, \pkg{fancyunits} and 
% \pkg{fancynum}. Support for users of all of these packages is 
% available as emulation modules in \pkg{siunitx}. In addition, 
% \pkg{siunitx} can carry out many of the functions of the
% \pkg{dcolumn}, \pkg{rccol} and \pkg{numprint} packages.
%\end{abstract}
%
%\begin{multicols}{2}
%  \tableofcontents
%\end{multicols}
%
%\begin{documentation}
%
%\end{documentation}
%
%\begin{implementation}
%
%\section{Implementation}
%
% Most of the code here is \LaTeX3, but there are some parts which need
% to use \LaTeXe functions to work correctly. 
%
%\subsection{Preliminaries}
%
% The usual preliminaries.  
%    \begin{macrocode}
%<*package>
\ProvidesExplPackage
  {\filename}{\filedate}{\fileversion}{\filedescription}
%    \end{macrocode}
%    
%\begin{macro}{\siunitx_load_check:}
%\begin{macro}[aux]{\siunitx_load_check_aux:n}
% There are also a number of packages that are incompatible with
% \pkg{siunitx}. These are all checked for next.  Some of the 
% incompatible packages will not raise an error if loaded after 
% \pkg{siunitx}.  So a test is made at the beginning of the 
% document as well. The message for this may be needed immediately, so
% it is created here not with the other messages.
%    \begin{macrocode}
\msg_new:nnnn { siunitx } { incompatible-package } {%
  Package `#1' incompatible.%
}{%
  The #1 package and siunitx are incompatible.\\%
  Use the `emulate=#1' package option when loading siunitx.%
}
\cs_new_nopar:Nn \siunitx_load_check: {
  \clist_map_function:nN { SIunits, sistyle, siunits, SIstyle }
    \siunitx_load_check_aux:n
}

\cs_new_nopar:Nn \siunitx_load_check_aux:n {
  \group_begin:
    \@ifpackageloaded {#1} {
      \msg_error:nnx { siunitx } { incompatible-package } {#1}
    } {}
  \group_end:
}
\clist_map_function:nN 
  { SIunits, sistyle, siunits, SIstyle, unitsdef, fancyunits }
  \siunitx_load_check_aux:n
\AtBeginDocument { \siunitx_load_check: } 
%    \end{macrocode}
%\end{macro}
%\end{macro}
%
% Now load the support packages.
%    \begin{macrocode}
\RequirePackage{ 
  amstext,
  array,
  keys3,
  xparse
} 
%    \end{macrocode}
%    
%\begin{macro}{\siunitx_encoding_ts_one:}
% Some of the \texttt{TS1} encoding is needed to provide symbols in 
% text mode. If the user has not loaded the encoding themselves, it is
% done here.
%    \begin{macrocode}
\cs_new_nopar:Nn \siunitx_encoding_ts_one: { 
  \cs_if_free:cT { T@TS1 } {
    \DeclareFontEncoding{TS1}{}{}%
    \DeclareFontSubstitution{TS1}{cmr}{m}{n}%
  }
}
\AtBeginDocument { \siunitx_encoding_ts_one: }
%    \end{macrocode}
%\end{macro}
%
%\subsection{Font detection}
%
% A number of controls and tests are needed to control the font used
% for output.  The detection routines set up various functions which can
% then be used inside the \cs{text} macro from the \pkg{amstext}
% package.  By carrying out the changes outside maths mode, bold can
% reliably be applied to maths output.
% 
%\begin{macro}{\c_siunitx_mathsf_int}
%\begin{macro}{\c_siunitx_mathtt_int}
%\begin{macro}{\siunitx_set_maths_fam:n}
% The detection routine needs some basic data about the sans serif and
% monospaced fonts in use.  This is collected up at the start of the 
% document. The families used can change between here and the start of 
% the document. This is run as a hook onto \cs{document}, rather than
% using \cs{AtBeginDocument} as it has to come after anything that
% \pkg{fontspec} does (nasty errors arise otherwise). 
%    \begin{macrocode}
\cs_new_nopar:Nn \siunitx_set_maths_fam:n {
  \group_begin:
    \hbox_set:Nn \l_tmpa_box {
      \ensuremath {
        \use:c { math #1 } {
          \expandafter \global \expandafter \chardef % Sort out!
            \csname c_siunitx_math #1 _int \endcsname \fam \relax       
        }
      }
    }
  \group_end:
}
\tl_put_right:Nn \document {
  \siunitx_set_maths_fam:n {sf}
  \siunitx_set_maths_fam:n {tt}
}
%    \end{macrocode}
%\end{macro}
%\end{macro}
%\end{macro}
%
%\begin{macro}{\c_siunitx_textsf_tl}
%\begin{macro}{\c_siunitx_texttt_tl}
%\begin{macro}{\siunitx_font_set_text_fam:n}
% The text families are done in a somewhat different manner: there are 
% no issues with \cs{AtBeginDocument}, luckily. The need to define 
% new macros here arises as \cs{sfdefault} and \cs{ttdefault} are 
% \cs{long}, whereas \cs{f@family} is not.
%    \begin{macrocode}
\cs_new_nopar:Nn \siunitx_set_text_fam:n {
  \tl_new:c { c_siunitx_text #1 _tl }
  \tl_set:cx { c_siunitx_text #1 _tl } { \use:c { #1 default } }
}
\AtBeginDocument{
  \siunitx_set_text_fam:n {sf}
  \siunitx_set_text_fam:n {tt}
}
%    \end{macrocode}
%\end{macro}
%\end{macro}
%\end{macro}
%
%\begin{macro}{\l_siunitx_detect_bold_bool}
%\begin{macro}{\l_siunitx_detect_display_maths_bool}
%\begin{macro}{\l_siunitx_detect_family_bool}
%\begin{macro}{\l_siunitx_detect_inline_bold_bool}
%\begin{macro}{\l_siunitx_detect_italic_bool}
%\begin{macro}{\l_siunitx_detect_mode_bool}
% Font detection is all about creating switches, with just a little
% gloss for \texttt{detect-all} and \texttt{detect-none}.
%    \begin{macrocode}
\keys_define:nn { siunitx } {
  detect-all .choice:,
  detect-all .default:n = true,
  detect-all / false .meta:n = {
    detect-bold   = false,
    detect-family = false,
    detect-italic = false,
    detect-mode   = false
  },
  detect-all / true .meta:n = {
    detect-bold   = true,
    detect-family = true,
    detect-italic = true,
    detect-mode   = true
  },
  detect-bold .set_bool:N          = \l_siunitx_detect_bold_bool,
  detect-display-math .set_bool:N  = 
    \l_siunitx_detect_display_maths_bool,
  detect-display-maths .set_bool:N = 
    \l_siunitx_detect_display_maths_bool,
  detect-family .set_bool:N        = \l_siunitx_detect_family_bool,
  detect-inline-bold .choice:,
  detect-inline-bold .value_required:,
  detect-inline-bold / math  .set_bool:N = 
    \l_siunitx_detect_inline_bold_bool,
  detect-inline-bold / maths .set_bool:N = 
    \l_siunitx_detect_inline_bold_bool,
  detect-inline-bold / text .set_bool_inverse:N = 
    \l_siunitx_detect_inline_bold_bool, 
  detect-italic .set_bool:N        = \l_siunitx_detect_italic_bool,
  detect-mode .set_bool:N          = \l_siunitx_detect_mode_bool,
  detect-none .choice:,
  detect-none .default:n = true,
  detect-none / false .meta:n = {
    detect-bold   = true,
    detect-family = true,
    detect-italic = true,
    detect-mode   = true
  },
  detect-none / true .meta:n = {
    detect-bold   = false,
    detect-family = false,
    detect-italic = false,
    detect-mode   = false
  }
}
%    \end{macrocode}
%\end{macro}
%\end{macro}
%\end{macro}
%\end{macro}
%\end{macro}
%\end{macro}
%
%\begin{macro}{\l_siunitx_font_set_bool}
%\begin{macro}{\l_siunitx_font_text_mode_bool}
% The internals of the package need a number of switches to track what
% is happening.
%    \begin{macrocode}
\bool_new:N \l_siunitx_font_set_bool
\bool_new:N \l_siunitx_font_text_mode_bool
%    \end{macrocode}
%\end{macro}
%\end{macro}
%
%\begin{macro}{\l_siunitx_font_bold_tl}
%\begin{macro}{\l_siunitx_font_family_tl}
%\begin{macro}{\l_siunitx_font_italic_tl}
% Also some token storage.
%    \begin{macrocode}
\tl_new:N \l_siunitx_font_bold_tl
\tl_new:N \l_siunitx_font_family_tl
\tl_new:N \l_siunitx_font_italic_tl
%    \end{macrocode}
%\end{macro}
%\end{macro}
%\end{macro}
%
%\begin{macro}{\siunitx_detect_font:}
% The main font-setting macro is a control point for the other
% functions.  Nesting font-control should not happen, so the first
% test is to ensure this is not happening.
%    \begin{macrocode}
\cs_new_nopar:Nn \siunitx_detect_font: {
  \bool_if:NF \l_siunitx_font_set_bool {
    \siunitx_detect_font_init:
    \bool_if:NT \l_siunitx_detect_bold_bool {
      \siunitx_detect_font_bold:
    }
    \bool_if:NT \l_siunitx_detect_family_bool {
      \siunitx_detect_font_family:
    }
    \bool_if:NT \l_siunitx_detect_italic_bool {
      \siunitx_detect_font_italic:
    }
    \bool_if:NT \l_siunitx_detect_mode_bool {
      \siunitx_detect_font_mode:
    }
  }
}
%    \end{macrocode}
%\end{macro}
%
%\begin{macro}{\siunitx_detect_font_init:}
% Set up the basic macros on the assumption nothing happens.
%    \begin{macrocode}
\cs_new_nopar:Nn \siunitx_detect_font_init: {
  \bool_set_true:N \l_siunitx_font_set_bool
  \tl_set:Nn \l_siunitx_font_bold_tl {
    \unboldmath 
    \mdseries
  }
  \tl_set:Nn \l_siunitx_font_family_tl { rm }
  \tl_set:Nn \l_siunitx_font_italic_tl { \upshape }
  \bool_set_false:N \l_siunitx_font_text_mode_bool
}
%    \end{macrocode}
%\end{macro}
%
%\begin{macro}{\siunitx_detect_font_bold:}
% The bold detection code depends on whether the current text is maths
% or text, and also the type of maths (if appropriate).  For inline 
% maths, there is a choice of what test to apply.
%    \begin{macrocode}
\cs_new_nopar:Nn \siunitx_detect_font_bold: {
  \mode_if_math:TF {
    \mode_if_inner:TF {
      \bool_if:NTF \l_siunitx_detect_inline_bold_bool {
        \siunitx_detect_font_bold_maths:
      }{
        \siunitx_detect_font_bold_text:
      }
    }{
      \bool_if:NTF \l_siunitx_detect_display_maths_bool {
        \siunitx_detect_font_bold_maths:
      }{
        \siunitx_detect_font_bold_text:
      }
    }
  }{
    \siunitx_detect_font_bold_text:
  }
}
%    \end{macrocode}
%\end{macro}
%\begin{macro}{\siunitx_detect_font_bold_maths:}
%\begin{macro}{\siunitx_detect_font_bold_text:}
% The tests for bold depend on whether the module is testing text or 
% maths content.
%    \begin{macrocode}
\cs_new_nopar:Nn \siunitx_detect_font_bold_maths: {
  \tl_if_eq:nVT { bold } \math@version {
    \tl_set:Nn \l_siunitx_font_bold_tl {
      \boldmath
      \bfseries
    }
  }
}
\cs_new_nopar:Nn \siunitx_detect_font_bold_text: {
  \tl_if_eq:nxT { b } { \exp_args:NV \tl_head_i:n \f@series } {
    \tl_set:Nn \l_siunitx_font_bold_tl {
      \boldmath
      \bfseries
    }
  }
}
%    \end{macrocode}
%\end{macro}
%\end{macro}
%
%\begin{macro}{\siunitx_detect_font_family:}
% Family detection usually picks up the surrounding text.  However, for
% display maths there are two options, and so a second test may be made.
%    \begin{macrocode}
\cs_new_nopar:Nn \siunitx_detect_font_family: {
  \siunitx_detect_font_family_text:
  \bool_if:NT \l_siunitx_detect_display_maths_bool {  
    \mode_if_math:TF {
      \mode_if_inner:F {
        \siunitx_detect_font_family_maths:
      }
    }
  }
}
%    \end{macrocode}
%\end{macro}
%\begin{macro}{\siunitx_detect_font_family_maths:}
%\begin{macro}{\siunitx_detect_font_family_text:}
% The two detections routines check if the current font is sans serif or
% monospaced.  The method used depends on whether the current maths or
% text font is of interest. The maths detection routine re-applies the 
% default (\texttt{rm}), as the text function is always called and may
% have changed things.
%    \begin{macrocode}
\cs_new_nopar:Nn \siunitx_detect_font_family_maths: {
  \tl_set:Nn \l_siunitx_font_family_tl { rm }
  \intexp_compare:nT { \int:use:N \fam = \c_siunitx_mathsf_int } {
    \tl_set:Nn \l_siunitx_font_family_tl { sf }
  }
  \intexp_compare:nT { \int:use:N \fam = \c_siunitx_mathtt_int } {
    \tl_set:Nn \l_siunitx_font_family_tl { tt }
  }  
}
\cs_new_nopar:Nn \siunitx_detect_font_family_text: {
  \tl_if_eq:NNT \f@family \c_siunitx_textsf_tl {
    \tl_set:Nn \l_siunitx_font_family_tl { sf }
  }
  \tl_if_eq:NNT \f@family \c_siunitx_texttt_tl {
    \tl_set:Nn \l_siunitx_font_family_tl { tt }
  }
}
%    \end{macrocode}
%\end{macro}
%\end{macro}
%
%\begin{macro}{\siunitx_detect_font_italic:}
% The italic test uses \cs{f@shape} as a handy indicator.
%    \begin{macrocode}
\cs_new_nopar:Nn \siunitx_detect_font_italic: {
  \tl_if_eq:nVT { it } \f@shape {
    \tl_set:Nn \l_siunitx_font_italic_tl { \itshape }
  }
}
%    \end{macrocode}
%\end{macro}
%
%\begin{macro}{\siunitx_detect_font_mode:}
% The mode switch works from the fact that maths-mode output it the
% standard.
%    \begin{macrocode}
\cs_new_nopar:Nn \siunitx_detect_font_mode: {
  \mode_if_math:F {
    \bool_set_true:N \l_siunitx_font_text_mode_bool
  }
}
%    \end{macrocode}
%\end{macro}
%
%\subsection{Font macros}
%
%\subsection{Messages}
%
%\subsection{Document macros}
%
% The user document macros are all collected together here for ease: all
% of them are \cs{protected}.
% 
%\begin{macro}{\sisetup}
% The set up macro simply moves to the correct path and executes 
% whatever has been passed.
%    \begin{macrocode}
\DeclareDocumentCommand \sisetup { m } {
  \keys_set:nn { siunitx } {#1}
}
%    \end{macrocode}
%\end{macro}
%    
%    \begin{macrocode}
%</package>
%    \end{macrocode}
%
%\end{implementation}
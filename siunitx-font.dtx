% \iffalse meta-comment
%
% File: siunitx-font.dtx Copyright (C) 2018 Joseph Wright
%
% It may be distributed and/or modified under the conditions of the
% LaTeX Project Public License (LPPL), either version 1.3c of this
% license or (at your option) any later version.  The latest version
% of this license is in the file
%
%    https://www.latex-project.org/lppl.txt
%
% This file is part of the "siunitx bundle" (The Work in LPPL)
% and all files in that bundle must be distributed together.
%
% The released version of this bundle is available from CTAN.
%
% -----------------------------------------------------------------------
%
% The development version of the bundle can be found at
%
%    https://github.com/josephwright/siunitx
%
% for those people who are interested.
%
% -----------------------------------------------------------------------
%
%<*driver>
\documentclass{l3doc}
% The next line is needed so that \GetFileInfo will be able to pick up
% version data
\usepackage{siunitx}
\begin{document}
  \DocInput{\jobname.dtx}
\end{document}
%</driver>
% \fi
%
% \GetFileInfo{siunitx.sty}
%
% \title{^^A
%   \pkg{siunitx-font} -- Font-related settings^^A
%   \thanks{This file describes \fileversion,
%     last revised \filedate.}^^A
% }
%
% \author{^^A
%  Joseph Wright^^A
%  \thanks{^^A
%    E-mail:
%    \href{mailto:joseph.wright@morningstar2.co.uk}
%      {joseph.wright@morningstar2.co.uk}^^A
%   }^^A
% }
%
% \date{Released \filedate}
%
% \maketitle
%
% \begin{documentation}
%
% \end{documentation}
%
% \begin{implementation}
%
% \section{\pkg{siunitx-angle} implementation}
%
% Start the \pkg{DocStrip} guards.
%    \begin{macrocode}
%<*package>
%    \end{macrocode}
%
% Identify the internal prefix (\LaTeX3 \pkg{DocStrip} convention): only
% internal material in this \emph{submodule} should be used directly.
%    \begin{macrocode}
%<@@=siunitx_font>
%    \end{macrocode}
%
% \begin{macro}{\@@_textmu:}
%   Some of the \texttt{TS1} encoding is needed to provide symbols in
%   text mode for 8-bit engines. If the user has not loaded the encoding
%   themselves, it is done here before creating the required commands.
%    \begin{macrocode}
\bool_lazy_or:nnF
  { \sys_if_engine_luatex_p: }
  { \sys_if_engine_xetex_p: }
  {
    \AtBeginDocument
      {
        \cs_if_free:cT { T@TS1 }
          {
            \DeclareFontEncoding { TS1 } { } { }
            \DeclareFontSubstitution { TS1 } { cmr } { m } { n }
          }
        \cs_if_free:NTF \textmu
          {
            \DeclareTextSymbolDefault \@@_textmu: { TS1 }
            \DeclareTextSymbol \@@_textmu: { TS1 } { "00B5 }
          }
          { \cs_new_eq:NN \@@_textmu: \textmu }
      }
  }
%    \end{macrocode}
% \end{macro}
%
% \begin{variable}{\l_@@_tmpa_tl}
%   Scratch space.
%    \begin{macrocode}
\tl_new:N \l_@@_tmpa_tl
\tl_new:N \l_@@_tmpb_tl
%    \end{macrocode}
% \end{variable}
%
% \begin{macro}{\@@_non_latin:n}
% \begin{macro}{\@@_non_latin:nnnn}
%   As in \pkg{siunutx-unit}, but internal in both cases as it's rather
%   specialised.
%    \begin{macrocode}
\bool_lazy_or:nnTF
  { \sys_if_engine_luatex_p: }
  { \sys_if_engine_xetex_p: }
  {
    \cs_new:Npn \@@_non_latin:n #1
      { \char_generate:nn {#1} { \char_value_catcode:n {#1} } }
  }
  {
    \cs_new:Npn \@@_non_latin:n #1
      {
        \exp_last_unbraced:Nf \@@_non_latin:nnnn
          { \char_codepoint_to_bytes:n {#1} }
      }
    \cs_new:Npn \@@_non_latin:nnnn #1#2#3#4
      {
        \exp_after:wN \exp_after:wN \exp_after:wN
          \exp_not:N \char_generate:nn {#1} { 13 }
        \exp_after:wN \exp_after:wN \exp_after:wN
          \exp_not:N \char_generate:nn {#2} { 13 }
      }
  }
%    \end{macrocode}
% \end{macro}
% \end{macro}
%
% \begin{macro}{\@@_if_replace_unitdef:NnT}
%   A test to see if the unit definition which applies is still one we expect:
%   here that means it is just using a (Unicode) codepoint. The comparison
%   is string-based as \pkg{unicode-math} (at least) can alter some of them.
%    \begin{macrocode}
\prg_new_protected_conditional:Npnn \@@_if_replace_unitdef:Nn #1#2 { T , TF }
  {
    \group_begin:
      \tl_set:Nx \l_@@_tmpa_tl { \@@_non_latin:n {#2} }
      \protected@edef \l_@@_tmpa_tl
        { \exp_not:N \mathrm { \l_@@_tmpa_tl } }
      \keys_set:nn { siunitx } { parse-units = false }
      \siunitx_unit_format:nN {#1} \l_@@_tmpb_tl
      \str_if_eq_x:nnTF
        { \tl_to_str:N \l_@@_tmpa_tl }
        { \tl_to_str:N \l_@@_tmpb_tl }
        {
          \group_end:
          \prg_return_true:
        }
        {
          \group_end:
          \prg_return_false:
        }
  }
%    \end{macrocode}
% \end{macro}
%
% \begin{macro}[EXP]{\@@_text_only:n}
%   A short auxiliary to set up text-only symbols.
%    \begin{macrocode}
\cs_new:Npn \@@_text_only:n #1
  {
    \exp_not:N \ifmmode
      \exp_not:N \expandafter \exp_not:N \mbox
    \exp_not:N \else
      \exp_not:N \expandafter \exp_not:N \@firstofone
    \exp_not:N \fi
      {#1}
  }
%    \end{macrocode}
% \end{macro}
%
% At the start of the document, fonts are fixed and the user may have
% altered unit set up. If things are unchanged, we can alter the settings
% such that they use something \enquote{more sensible}.
%    \begin{macrocode}
\AtBeginDocument
  {
    \@@_if_replace_unitdef:NnT \micro { "03BC }
      {
        \exp_args:NNnx \siunitx_declare_prefix:Nnn \micro { -6 }
          {
            \@@_text_only:n
              {
                \bool_lazy_or:nnTF
                  { \sys_if_engine_luatex_p: }
                  { \sys_if_engine_xetex_p: }
                  { \char "00B5 \relax }
                  { \exp_not:N \@@_textmu: }
              }
          }
      }
  }
%    \end{macrocode}
%
%    \begin{macrocode}
%</package>
%    \end{macrocode}
%
% \end{implementation}
%
% \PrintIndex
% \iffalse meta-comment
%
% File: siunitx-cell.dtx Copyright (C) 2016-2017 Joseph Wright
%
% It may be distributed and/or modified under the conditions of the
% LaTeX Project Public License (LPPL), either version 1.3c of this
% license or (at your option) any later version.  The latest version
% of this license is in the file
%
%    http://www.latex-project.org/lppl.txt
%
% This file is part of the "siunitx bundle" (The Work in LPPL)
% and all files in that bundle must be distributed together.
%
% The released version of this bundle is available from CTAN.
%
% -----------------------------------------------------------------------
%
% The development version of the bundle can be found at
%
%    http://github.com/josephwright/siunitx
%
% for those people who are interested.
%
% -----------------------------------------------------------------------
%
%<*driver>
\documentclass{l3doc}
% The next line is needed so that \GetFileInfo will be able to pick up
% version data
\usepackage{siunitx}
\begin{document}
  \DocInput{\jobname.dtx}
\end{document}
%</driver>
% \fi
%
% \GetFileInfo{siunitx.sty}
%
% \title{^^A
%   \pkg{siunitx-cell} -- Formatting table cells^^A
%   \thanks{This file describes \fileversion,
%     last revised \filedate.}^^A
% }
%
% \author{^^A
%  Joseph Wright^^A
%  \thanks{^^A
%    E-mail:
%    \href{mailto:joseph.wright@morningstar2.co.uk}
%      {joseph.wright@morningstar2.co.uk}^^A
%   }^^A
% }
%
% \date{Released \filedate}
%
% \maketitle
%
% \begin{documentation}
%
% \end{documentation}
%
% \begin{implementation}
%
% \section{\pkg{siunitx-cell} implementation}
%
% Start the \pkg{DocStrip} guards.
%    \begin{macrocode}
%<*package>
%    \end{macrocode}
%
% Identify the internal prefix (\LaTeX3 \pkg{DocStrip} convention): only
% internal material in this \emph{submodule} should be used directly.
%    \begin{macrocode}
%<@@=siunitx_cell>
%    \end{macrocode}
%
% \subsection{Collecting tokens}
%
% \begin{variable}{\l_@@_collect_tl}
%   Space for tokens.
%    \begin{macrocode}
\tl_new:N \l_@@_collect_tl
%    \end{macrocode}
% \end{variable}
%
% \begin{macro}{\siunitx_cell_begin:}
%   Collecting a tabular cell means doing a token-by-token collection.
%   In previous versions of \pkg{siunitx} that was done along with picking
%   out the numerical part, but the code flow ends up very tricky. Here,
%   therefore, we just collect up the unchanged tokens first. Some issues
%   with the way that the \texttt{peek} functions are currently set up in
%   \pkg{expl3} mean that at present there is a need to do stuff by hand
%   here. The definition of \cs{cr} is used to allow collection of any tokens
%   inserted after the main content when dealing with the last cell of a row:
%   the \enquote{group} around it is needed to avoid issues with the underlying
%   |\halign|. (The approach is based on that in \pkg{collcell}.) Notice that
%   as each cell forms a group there is no need to reset the definition of
%   \cs{cr}.
%    \begin{macrocode}
\cs_new_protected:Npn \siunitx_cell_begin:
  {
    \tl_clear:N \l_@@_collect_tl
    \cs_set_protected:Npn \__peek_execute_branches:
      {
        \if_catcode:w \exp_not:N \l_peek_token \c_group_begin_token
          \exp_after:wN \@@_collect_group:n
        \else:
          \exp_after:wN \@@_collect_token:N
        \fi:
      }
    \if_false: { \fi:
    \cs_set_protected:Npn \cr
      {
        \@@_collect_loop:
        \tex_cr:D
      }
    \if_false: } \fi:
    \@@_collect_loop:
  }
%    \end{macrocode}
% \end{macro}
%
% \begin{macro}[int]{\@@_collect_loop:}
% \begin{macro}[aux]{\@@_collect_group:n}
% \begin{macro}[aux]{\@@_collect_token:N}
% \begin{macro}[aux]{\@@_collect_search:NnF}
% \begin{macro}[aux]{\@@_collect_search_aux:NNn}
%   Collecting up the cell content needs a loop: this is done using
%   a |peek| approach as it's most natural. (A slower approach is possible
%   using something like the |\tl_lower_case:n| loop code.) The set of
%   possible tokens is somewhat limited compared to an arbitrary cell
%   (\foreign{cf.}~the approach in \pkg{collcell}): the special cases are
%   pulled out for manual handling. The flexible lookup approach is more-or-less
%   the same idea as in the kernel |case| functions. The |\relax| special case
%   covers the case where |\\| has been expanded in an empty cell.
%    \begin{macrocode}
\cs_new_protected:Npn \@@_collect_loop:
  { \peek_after:Nw \__peek_ignore_spaces_execute_branches: }
\cs_new_protected:Npn \@@_collect_group:n #1
  {
    \tl_put_right:Nn \l_@@_collect_tl { {#1} }
    \@@_collect_loop:
  }
\cs_new_protected:Npn \@@_collect_token:N #1
  {
    \@@_collect_search:NnF #1
      {
        \ignorespaces      { \@@_collect_loop: }
        \unskip            { \@@_collect_loop: }
        \end               { \tabularnewline \end }
        \relax             { \relax }
        \tabularnewline    { \tabularnewline }
        \siunitx_cell_end: { \siunitx_cell_end: }
      }
      {
        \tl_put_right:Nn \l_@@_collect_tl {#1}
        \@@_collect_loop:
      }
  }
\AtBeginDocument
  {
    \@ifpackageloaded { mdwtab }
      {
        \cs_set_protected:Npn \@@_collect_token:N #1
          {
            \@@_collect_search:NnF #1
              {
                \@maybe@unskip     { \@@_collect_loop: }
                \ignorespaces      { \@@_collect_loop: }
                \tab@setcr         { \@@_collect_loop: }
                \unskip            { \@@_collect_loop: }
                \end               { \tabularnewline \end }
                \relax             { \relax }
                \tabularnewline    { \tabularnewline }
                \siunitx_cell_end: { \siunitx_cell_end: }
              }
              {
                \tl_put_right:Nn \l_@@_collect_tl {#1}
                \@@_collect_loop:
              }
          }
      }
      { }
  }
\cs_new_protected:Npn \@@_collect_search:NnF #1#2#3
  {
    \@@_collect_search_aux:NNn #1
      #2
      #1 {#3}
    \q_stop
  }
\cs_new_protected:Npn \@@_collect_search_aux:NNn #1#2#3
  {
    \token_if_eq_meaning:NNTF #1 #2
      { \use_i_delimit_by_q_stop:nw {#3} }
      { \@@_collect_search_aux:NNn #1 }
  }
%    \end{macrocode}
% \end{macro}
% \end{macro}
% \end{macro}
% \end{macro}
% \end{macro}
%
% \subsection{Separating collected material}
%
% The input needs to be divided into numerical tokens and those which appear
% before and after them. This needs a second loop and validation.
%
% \begin{variable}{\l_@@_pre_tl, \l_@@_number_tl, \l_@@_post_tl}
%   Space for tokens.
%    \begin{macrocode}
\tl_new:N \l_@@_pre_tl
\tl_new:N \l_@@_number_tl
\tl_new:N \l_@@_post_tl
%    \end{macrocode}
% \end{variable}
%
% \begin{macro}{\siunitx_cell_end:}
%   At the end of the cell, expand all of the content as far as possible then
%   split it up into numerical and non-numerical parts.
%    \begin{macrocode}
\cs_new_protected:Npn \siunitx_cell_end:
  {
    \protected@edef \l_@@_collect_tl { \l_@@_collect_tl }
    \tl_clear:N \l_@@_pre_tl
    \tl_clear:N \l_@@_number_tl
    \tl_clear:N \l_@@_post_tl
    \cs_set_protected:Npn \__peek_execute_branches:
      {
        \if_catcode:w \exp_not:N \l_peek_token \c_group_begin_token
          \exp_after:wN \@@_split_group:n
        \else:
          \exp_after:wN \@@_split_token:N
        \fi:
      }
    \exp_after:wN \@@_split_loop: \l_@@_collect_tl
      \q_recursion_tail \q_recursion_stop
    \@@_split_tidy:N \l_@@_pre_tl
    \@@_split_tidy:N \l_@@_post_tl
    \@@_print:
  }
%    \end{macrocode}
% \end{macro}
%
% \begin{macro}[int]{\@@_split_loop:}
% \begin{macro}[aux]{\@@_split_group:n}
% \begin{macro}[aux]{\@@_split_token:N}
%   Splitting into parts uses the fact that numbers cannot contain groups
%   and that we can track where we are up to based on the content of the
%   token lists.
%    \begin{macrocode}
\cs_new_protected:Npn \@@_split_loop:
  { \peek_after:Nw \__peek_ignore_spaces_execute_branches: }
\cs_new_protected:Npn \@@_split_group:n #1
  {
    \tl_if_empty:NTF \l_@@_number_tl
      { \tl_put_right:Nn \l_@@_pre_tl { {#1} } }
      { \tl_put_right:Nn \l_@@_post_tl { {#1} } }
    \@@_split_loop:
  }
\cs_new_protected:Npn \@@_split_token:N #1
  {
    \quark_if_recursion_tail_stop:N #1
    \tl_if_empty:NTF \l_@@_post_tl
      {
        \siunitx_if_number_token:NTF #1
          { \tl_put_right:Nn \l_@@_number_tl {#1} }
          {
            \tl_if_empty:NTF \l_@@_number_tl
              { \tl_put_right \l_@@_pre_tl {#1} }
              { \tl_put_right \l_@@_post_tl {#1} }
          }
      }
      { \tl_put_right:Nn \l_@@_post_tl {#1} }
    \@@_split_loop:
  }
%    \end{macrocode}
% \end{macro}
% \end{macro}
% \end{macro}
%
% \begin{macro}[int]{\@@_split_tidy:N}
% \begin{macro}[aux]{\@@_split_tidy:Nn, \@@_split_tidy:NV}
%   A quick test for the entire content being surrounded by a set of braces:
%   rather than look explicitly, use the fact that a string comparison can
%   detect the same thing. The auxiliary is needed to avoid having to go
%   \foreign{via} a |:D| function (for the expansion behaviour).
%    \begin{macrocode}
\cs_new_protected:Npn \@@_split_tidy:N #1
  {
    \tl_if_blank:NF #1
      { \@@_split_tidy:NV #1 #2 }
  }
\cs_new_protected:Npn \@@_split_tidy:Nn #1#2
  {
    \str_if_eq_x:nnT
      { \exp_not:n {#2} }
      { { \exp_not:o { \use:n #2 } } }
      { \tl_set:No #1 { \use:n #2 } }
  }
\cs_generate_variant:Nn \@@_split_tidy:Nn { NV }
%    \end{macrocode}
% \end{macro}
% \end{macro}
%
% \subsection{Printing numbers in cells: main functions}
%
% \begin{macro}[int]{\@@_print:}
%    \begin{macrocode}
\cs_new_protected:Npn \@@_print:
  {
  }
%    \end{macrocode}
% \end{macro}
%
% \subsection{Printing numbers in cells: spacing}
%
% Getting the general alignment correct in tables is made more complex than one
% would like by the \pkg{colortbl} package. In the original \LaTeXe{}
% definition, cell material is centred by a construction of the (primitive)
% form
% \begin{verbatim}
%   \hfil
%   #
%   \hfil
% \end{verbatim}
% which only uses \texttt{fil} stretch. That is altered by \pkg{colortbl} to
% broadly
% \begin{verbatim}
%   \hskip 0pt plus 0.5fill
%   \kern 0pt
%   #
%   \hskip 0pt plus 0.5fill
% \end{verbatim}
% which means there is \texttt{fill} stretch to worry about and the kern as
% well.
%
%    \begin{macrocode}
%</package>
%    \end{macrocode}
%
% \end{implementation}
%
% \PrintIndex
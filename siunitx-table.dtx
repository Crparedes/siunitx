% \iffalse meta-comment
%
% File: siunitx-table.dtx Copyright (C) 2016-2018 Joseph Wright
%
% It may be distributed and/or modified under the conditions of the
% LaTeX Project Public License (LPPL), either version 1.3c of this
% license or (at your option) any later version.  The latest version
% of this license is in the file
%
%    https://www.latex-project.org/lppl.txt
%
% This file is part of the "siunitx bundle" (The Work in LPPL)
% and all files in that bundle must be distributed together.
%
% The released version of this bundle is available from CTAN.
%
% -----------------------------------------------------------------------
%
% The development version of the bundle can be found at
%
%    https://github.com/josephwright/siunitx
%
% for those people who are interested.
%
% -----------------------------------------------------------------------
%
%<*driver>
\documentclass{l3doc}
% The next line is needed so that \GetFileInfo will be able to pick up
% version data
\usepackage{siunitx}
\begin{document}
  \DocInput{\jobname.dtx}
\end{document}
%</driver>
% \fi
%
% \GetFileInfo{siunitx.sty}
%
% \title{^^A
%   \pkg{siunitx-table} -- Formatting numbers in tables^^A
%   \thanks{This file describes \fileversion,
%     last revised \filedate.}^^A
% }
%
% \author{^^A
%  Joseph Wright^^A
%  \thanks{^^A
%    E-mail:
%    \href{mailto:joseph.wright@morningstar2.co.uk}
%      {joseph.wright@morningstar2.co.uk}^^A
%   }^^A
% }
%
% \date{Released \filedate}
%
% \maketitle
%
% \begin{documentation}
%
% \end{documentation}
%
% \begin{implementation}
%
% \section{\pkg{siunitx-table} implementation}
%
% Start the \pkg{DocStrip} guards.
%    \begin{macrocode}
%<*package>
%    \end{macrocode}
%
% Identify the internal prefix (\LaTeX3 \pkg{DocStrip} convention): only
% internal material in this \emph{submodule} should be used directly.
%    \begin{macrocode}
%<@@=siunitx_table>
%    \end{macrocode}
%
% \begin{variable}{\l_@@_tmp_tl}
%   Scratch space.
%    \begin{macrocode}
\tl_new:N \l_@@_tmp_tl
%    \end{macrocode}
% \end{variable}
%
% \subsection{Interface functions}
%
% \begin{variable}{\l_@@_parse_bool}
%   One top-level setting applies to all number cells.
%    \begin{macrocode}
\keys_define:nn { siunitx / table }
  {
    parse .bool_set:N = \l_@@_parse_bool
  }
%    \end{macrocode}
% \end{variable}
%
% \begin{macro}{\siunitx_cell_begin:, \siunitx_cell_end:}
%    \begin{macrocode}
\cs_new_protected:Npn \siunitx_cell_begin:
  {
    \bool_if:NTF \l_@@_parse_bool
      { \@@_collect_begin:N }
      { \@@_direct_begin: }
  }
\cs_new_protected:Npn \siunitx_cell_end:
  {
    \@@_collect_end:
  }
%    \end{macrocode}
% \end{macro}
%
% \subsection{Collecting tokens}
%
% \begin{variable}{\l_@@_collect_tl}
%   Space for tokens.
%    \begin{macrocode}
\tl_new:N \l_@@_collect_tl
%    \end{macrocode}
% \end{variable}
%
% \begin{macro}{\@@_collect_begin:N}
%   Collecting a tabular cell means doing a token-by-token collection.
%   In previous versions of \pkg{siunitx} that was done along with picking
%   out the numerical part, but the code flow ends up very tricky. Here,
%   therefore, we just collect up the unchanged tokens first. The |#1| collects
%   the leading |\ignorespaces| from the template: that has to go to
%   avoid issues with the peek-ahead code. The definition of
%   \cs{cr} is used to allow collection of any tokens
%   inserted after the main content when dealing with the last cell of a row:
%   the \enquote{group} around it is needed to avoid issues with the underlying
%   |\halign|. (The approach is based on that in \pkg{collcell}.) Notice that
%   as each cell forms a group there is no need to reset the definition of
%   \cs{cr}.
%    \begin{macrocode}
\cs_new_protected:Npn \@@_collect_begin:N #1
  {
    \tl_clear:N \l_@@_collect_tl
    \if_false: { \fi:
    \cs_set_protected:Npn \cr
      {
        \@@_collect_loop:
        \tex_cr:D
      }
    \if_false: } \fi:
    \@@_collect_loop:
  }
%    \end{macrocode}
% \end{macro}
%
% \begin{macro}{\@@_collect_loop:}
% \begin{macro}{\@@_collect_group:n}
% \begin{macro}{\@@_collect_token:N}
% \begin{macro}{\@@_collect_search:NnF}
% \begin{macro}{\@@_collect_search_aux:NNn}
%   Collecting up the cell content needs a loop: this is done using
%   a |peek| approach as it's most natural. (A slower approach is possible
%   using something like the |\tl_lower_case:n| loop code.) The set of
%   possible tokens is somewhat limited compared to an arbitrary cell
%   (\foreign{cf.}~the approach in \pkg{collcell}): the special cases are
%   pulled out for manual handling. The flexible lookup approach is more-or-less
%   the same idea as in the kernel |case| functions. The |\relax| special case
%   covers the case where |\\| has been expanded in an empty cell.
%    \begin{macrocode}
\cs_new_protected:Npn \@@_collect_loop:
  {
    \peek_catcode_ignore_spaces:NTF \c_group_begin_token
      { \@@_collect_group:n }
      { \@@_collect_token:N }
  }
\cs_new_protected:Npn \@@_collect_group:n #1
  {
    \tl_put_right:Nn \l_@@_collect_tl { {#1} }
    \@@_collect_loop:
  }
\cs_new_protected:Npn \@@_collect_token:N #1
  {
    \@@_collect_search:NnF #1
      {
        \unskip            { \@@_collect_loop: }
        \end               { \tabularnewline \end }
        \relax             { \relax }
        \tabularnewline    { \tabularnewline }
        \siunitx_cell_end: { \siunitx_cell_end: }
      }
      {
        \tl_put_right:Nn \l_@@_collect_tl {#1}
        \@@_collect_loop:
      }
  }
\AtBeginDocument
  {
    \@ifpackageloaded { mdwtab }
      {
        \cs_set_protected:Npn \@@_collect_token:N #1
          {
            \@@_collect_search:NnF #1
              {
                \@maybe@unskip     { \@@_collect_loop: }
                \tab@setcr         { \@@_collect_loop: }
                \unskip            { \@@_collect_loop: }
                \end               { \tabularnewline \end }
                \relax             { \relax }
                \tabularnewline    { \tabularnewline }
                \siunitx_cell_end: { \siunitx_cell_end: }
              }
              {
                \tl_put_right:Nn \l_@@_collect_tl {#1}
                \@@_collect_loop:
              }
          }
      }
      { }
  }
\cs_new_protected:Npn \@@_collect_search:NnF #1#2#3
  {
    \@@_collect_search_aux:NNn #1
      #2
      #1 {#3}
    \q_stop
  }
\cs_new_protected:Npn \@@_collect_search_aux:NNn #1#2#3
  {
    \token_if_eq_meaning:NNTF #1 #2
      { \use_i_delimit_by_q_stop:nw {#3} }
      { \@@_collect_search_aux:NNn #1 }
  }
%    \end{macrocode}
% \end{macro}
% \end{macro}
% \end{macro}
% \end{macro}
% \end{macro}
%
% \subsection{Separating collected material}
%
% The input needs to be divided into numerical tokens and those which appear
% before and after them. This needs a second loop and validation.
%
% \begin{variable}{\l_@@_pre_tl, \l_@@_number_tl, \l_@@_post_tl}
%   Space for tokens.
%    \begin{macrocode}
\tl_new:N \l_@@_pre_tl
\tl_new:N \l_@@_number_tl
\tl_new:N \l_@@_post_tl
%    \end{macrocode}
% \end{variable}
%
% \begin{macro}{\@@_collect_end:}
%   At the end of the cell, expand all of the content as far as possible then
%   split it up into numerical and non-numerical parts.
%    \begin{macrocode}
\cs_new_protected:Npn \@@_collect_end:
  {
    \protected@edef \l_@@_collect_tl
      { \l_@@_collect_tl }
    \tl_clear:N \l_@@_pre_tl
    \tl_clear:N \l_@@_number_tl
    \tl_clear:N \l_@@_post_tl
    \cs_set_protected:Npn \__peek_execute_branches:
      {
        \if_catcode:w \exp_not:N \l_peek_token \c_group_begin_token
          \exp_after:wN \@@_split_group:n
        \else:
          \exp_after:wN \@@_split_token:N
        \fi:
      }
    \exp_after:wN \@@_split_loop: \l_@@_collect_tl
      \q_recursion_tail \q_recursion_stop
    \@@_split_tidy:N \l_@@_pre_tl
    \@@_split_tidy:N \l_@@_post_tl
    \tl_if_empty:NTF \l_@@_number_tl
      { \@@_print_text:V \l_@@_pre_tl }
      {
        \@@_print:VVV
          \l_@@_pre_tl
          \l_@@_number_tl
          \l_@@_post_tl
      }
  }
%    \end{macrocode}
% \end{macro}
%
% \begin{macro}{\@@_split_loop:}
% \begin{macro}{\@@_split_group:n}
% \begin{macro}{\@@_split_token:N}
%   Splitting into parts uses the fact that numbers cannot contain groups
%   and that we can track where we are up to based on the content of the
%   token lists.
%    \begin{macrocode}
\cs_new_protected:Npn \@@_split_loop:
  { \peek_after:Nw \__peek_ignore_spaces_execute_branches: }
\cs_new_protected:Npn \@@_split_group:n #1
  {
    \tl_if_empty:NTF \l_@@_number_tl
      { \tl_put_right:Nn \l_@@_pre_tl { {#1} } }
      { \tl_put_right:Nn \l_@@_post_tl { {#1} } }
    \@@_split_loop:
  }
\cs_new_protected:Npn \@@_split_token:N #1
  {
    \quark_if_recursion_tail_stop:N #1
    \tl_if_empty:NTF \l_@@_post_tl
      {
        \siunitx_if_number_token:NTF #1
          { \tl_put_right:Nn \l_@@_number_tl {#1} }
          {
            \tl_if_empty:NTF \l_@@_number_tl
              { \tl_put_right \l_@@_pre_tl {#1} }
              { \tl_put_right \l_@@_post_tl {#1} }
          }
      }
      { \tl_put_right:Nn \l_@@_post_tl {#1} }
    \@@_split_loop:
  }
%    \end{macrocode}
% \end{macro}
% \end{macro}
% \end{macro}
%
% \begin{macro}{\@@_split_tidy:N}
% \begin{macro}{\@@_split_tidy:Nn, \@@_split_tidy:NV}
%   A quick test for the entire content being surrounded by a set of braces:
%   rather than look explicitly, use the fact that a string comparison can
%   detect the same thing. The auxiliary is needed to avoid having to go
%   \foreign{via} a |:D| function (for the expansion behaviour).
%    \begin{macrocode}
\cs_new_protected:Npn \@@_split_tidy:N #1
  {
    \tl_if_empty:NF #1
      { \@@_split_tidy:NV #1 #1 }
  }
\cs_new_protected:Npn \@@_split_tidy:Nn #1#2
  {
    \str_if_eq_x:nnT
      { \exp_not:n {#2} }
      { { \exp_not:o { \use:n #2 } } }
      { \tl_set:No #1 { \use:n #2 } }
  }
\cs_generate_variant:Nn \@@_split_tidy:Nn { NV }
%    \end{macrocode}
% \end{macro}
% \end{macro}
%
% \subsection{Printing numbers in cells: spacing}
%
% Getting the general alignment correct in tables is made more complex than one
% would like by the \pkg{colortbl} package. In the original \LaTeXe{}
% definition, cell material is centred by a construction of the (primitive)
% form
% \begin{verbatim}
%   \hfil
%   #
%   \hfil
% \end{verbatim}
% which only uses \texttt{fil} stretch. That is altered by \pkg{colortbl} to
% broadly
% \begin{verbatim}
%   \hskip 0pt plus 0.5fill
%   \kern 0pt
%   #
%   \hskip 0pt plus 0.5fill
% \end{verbatim}
% which means there is \texttt{fill} stretch to worry about and the kern as
% well.
%
% \begin{macro}{\@@_skip:n}
%   To prevent combination of skips, a kern is inserted after each one.
%   This is best handled as a short auxiliary.
%    \begin{macrocode}
\cs_new_protected:Npn \@@_skip:n #1
  {
    \skip_horizontal:n {#1}
    \tex_kern:D \c_zero_skip
  }
%    \end{macrocode}
% \end{macro}
%
% \begin{variable}{\l_@@_column_width_dim, \l_@@_fixed_width_bool}
%   Settings which apply to aligned columns in general.
%    \begin{macrocode}
\keys_define:nn { siunitx / table }
  {
    column-width .dim_set:N =
      \l_@@_column_width_dim ,
    fixed-width .bool_set:N =
      \l_@@_fixed_width_bool
  }
%    \end{macrocode}
% \end{variable}
%
% \begin{macro}{\@@_align_center:n, \@@_align_left:n, \@@_align_right:n}
% \begin{macro}{\@@_align_auxi:nn, \@@_align_auxii:nn}
%   The beginning and end of each table cell have to adjust the position of
%   the content using glue. When \pkg{colortbl} is loaded the glue is done in
%   two parts: one for our positioning and one to explicitly override that from
%   the package. Using a two-step auxiliary chain avoids needing to repeat any
%   code and the impact of the extra expansion should be trivial.
%    \begin{macrocode}
\cs_new_protected:Npn \@@_align_center:n #1
  { \@@_align_auxi:nn {#1} { 0pt plus 0.5fill } {#1} }
\cs_new_protected:Npn \@@_align_left:n #1
  { \@@_align_auxi:nn {#1} { 0pt } {#1} }
\cs_new_protected:Npn \@@_align_right:n #1
  { \@@_align_auxi:nn {#1} { 0pt plus 1fill } {#1} }
\cs_new_protected:Npn \@@_align_auxi:nn #1#2
  {
    \bool_if:NTF \l_@@_fixed_width_bool
      { \hbox_to_wd:nn \l_@@_column_width_dim }
      { \use:n }
      {
        \@@_skip:n {#2}
        #1
        \@@_skip:n { 0pt plus 1fill - #2 }
      }
  }
\AtBeginDocument
  {
    \@ifpackageloaded { colortbl }
      {
        \cs_new_eq:NN
          \@@_align_auxii:nn
          \@@_align_auxi:nn
        \cs_set_protected:Npn \@@_align_auxi:nn #1#2
          {
            \@@_skip:n{ 0pt plus -0.5fill }
            \@@_align_auxii:nn {#1} {#2}
            \@@_skip:n { 0pt plus -0.5fill }
          }
      }
      { }
  }
%    \end{macrocode}
% \end{macro}
% \end{macro}
%
% \subsection{Printing just text}
%
% In cases where there is no numerical part, \pkg{siunitx} allows alignment
% of the \enquote{escaped} text independent of the underlying column type.
%
% \begin{variable}{\l_@@_align_text_tl}
%   Alignment is handled using a |tl| as this allows a fast lookup at the
%   point of use.
%    \begin{macrocode}
\keys_define:nn { siunitx / table }
  {
    text-alignment .choices:nn =
      { center , left , right }
      { \tl_set:Nn \l_@@_align_text_tl {#1} } ,
  }
\tl_new:N \l_@@_align_text_tl
%    \end{macrocode}
% \end{variable}
%
% \begin{macro}{\@@_print_text:n, \@@_print_text:V}
%   Printing escaped text is easy: just place it in correctly in the
%   column.
%    \begin{macrocode}
\cs_new_protected:Npn \@@_print_text:n #1
  {
    \use:c { @@_align_ \l_@@_align_text_tl :n } {#1}
  }
\cs_generate_variant:Nn \@@_print_text:n { V }
%    \end{macrocode}
% \end{macro}
%
% \subsection{Reserving space: the table format}
%
% \begin{variable}{\l_@@_format_tl}
%    \begin{macrocode}
\keys_define:nn { siunitx / table }
  {
    format .tl_set:N = \l_@@_format_tl
  }
%    \end{macrocode}
% \end{variable}
%
% \subsection{Directly printing without collection}
%
% Collecting the number allows for various effects but is not as fast
% as simply aligning on the first token that is a decimal marker. The
% strategy here is that used by \pkg{dcolumn}.
%
% \begin{macro}{\@@_direct_begin:}
% \begin{macro}{\@@_direct_begin:w}
% \begin{macro}{\@@_direct_begin_aux:}
%   After removing the |\ignorespaces| at the start of the cell, check to see
%   if there is a |{| and branch as appropriate.
%    \begin{macrocode}
\cs_new_protected:Npn \@@_direct_begin:
  { \@@_direct_begin:w }
\cs_new_protected:Npn \@@_direct_begin:w \ignorespaces
  {
    \cs_set:Npn \__peek_execute_branches:
      {
        \if_catcode:w \exp_not:N \l_peek_token \c_group_begin_token
          \exp_after:wN \@@_print_text:n
        \else:
          \m@th
          \exp_after:wN \@@_direct_begin_aux:
        \fi:
      }
    \peek_after:Nw \__peek_ignore_spaces_execute_branches:
  }
\cs_new_protected:Npn \@@_direct_begin_aux:
  {
  }
%    \end{macrocode}
% \end{macro}
% \end{macro}
% \end{macro}
%
% \subsection{Printing numbers in cells: main functions}
%
% \begin{variable}
%   {
%     \l_@@_align_comparator_bool  ,
%     \l_@@_align_exponent_bool    ,
%     \l_@@_align_uncertainty_bool ,
%     \l_@@_alignment_tl           ,
%     \l_@@_parse_only_bool
%   }
%   Alignment is handled using a |tl| as this allows a fast lookup at the
%   point of use.
%    \begin{macrocode}
\keys_define:nn { siunitx / table }
  {
    align-comparator .bool_set:N =
      \l_@@_align_comparator_bool ,
    align-exponent .bool_set:N =
      \l_@@_align_exponent_bool ,
    align-uncertainty .bool_set:N =
      \l_@@_align_uncertainty_bool ,
    alignment .choices:nn =
      { center , left , right }
      { \tl_set:Nn \l_@@_alignment_tl {#1} } ,
    parse-only .bool_set:N =
      \l_@@_parse_only_bool
  }
\tl_new:N \l_@@_alignment_tl
%    \end{macrocode}
% \end{variable}
%
% \begin{macro}{\@@_print:nnn, \@@_print:VVV}
% \begin{macro}{\@@_print_non_aligned:nnn, \@@_print_aligned:nnn}
%    \begin{macrocode}
\cs_new_protected:Npn \@@_print:nnn #1#2#3
  {
    \use:c { @@_align_ \l_@@_alignment_tl :n }
      {
        \bool_if:NTF \l_@@_parse_only_bool
          { \@@_print_non_aligned:nnn }
          { \@@_print_aligned:nnn }
          {#1} {#2} {#3}
      }
  }
\cs_generate_variant:Nn \@@_print:nnn { VVV }
\cs_new_protected:Npn \@@_print_non_aligned:nnn #1#2#3
  {
    #1
    \siunitx_number_format:nN {#2} \l_@@_tmp_tl
    \siunitx_print:nV { number } \l_@@_tmp_tl
    #3
  }
\cs_new_protected:Npn \@@_print_aligned:nnn #1#2#3
  {
  }
%    \end{macrocode}
% \end{macro}
% \end{macro}
%
% \subsection{Standard settings for module options}
%
% Some of these follow naturally from the point of definition
% (\foreign{e.g.}~boolean variables are always |false| to begin with),
% but for clarity everything is set here.
%    \begin{macrocode}
\keys_set:nn { siunitx / table }
  {
    align-comparator  = false  ,
    align-exponent    = false  ,
    align-uncertainty = false  ,
    alignment         = center ,
    column-width      = 0pt    ,
    fixed-width       = false  ,
    format            =        ,
    parse             = true   ,
    parse-only        = false  ,
    text-alignment    = center
  }
%    \end{macrocode}
%
%
%    \begin{macrocode}
%</package>
%    \end{macrocode}
%
% \end{implementation}
%
% \PrintIndex
% \iffalse meta-comment
%
% File: siunitx-table.dtx Copyright (C) 2016-2019,2021 Joseph Wright
%
% It may be distributed and/or modified under the conditions of the
% LaTeX Project Public License (LPPL), either version 1.3c of this
% license or (at your option) any later version.  The latest version
% of this license is in the file
%
%    https://www.latex-project.org/lppl.txt
%
% This file is part of the "siunitx bundle" (The Work in LPPL)
% and all files in that bundle must be distributed together.
%
% The released version of this bundle is available from CTAN.
%
% -----------------------------------------------------------------------
%
% The development version of the bundle can be found at
%
%    https://github.com/josephwright/siunitx
%
% for those people who are interested.
%
% -----------------------------------------------------------------------
%
%<*driver>
\documentclass{l3doc}
\ProvideDocumentCommand\foreign{m}{\textit{#1}}
% The next line is needed so that \GetFileInfo will be able to pick up
% version data
\usepackage{siunitx}
\begin{document}
  \DocInput{\jobname.dtx}
\end{document}
%</driver>
% \fi
%
% \GetFileInfo{siunitx.sty}
%
% \title{^^A
%   \pkg{siunitx-table} -- Formatting numbers in tables^^A
%   \thanks{This file describes \fileversion,
%     last revised \filedate.}^^A
% }
%
% \author{^^A
%  Joseph Wright^^A
%  \thanks{^^A
%    E-mail:
%    \href{mailto:joseph.wright@morningstar2.co.uk}
%      {joseph.wright@morningstar2.co.uk}^^A
%   }^^A
% }
%
% \date{Released \filedate}
%
% \maketitle
%
% \begin{documentation}
%
% \section{Numbers in tables}
%
% This submodule is concerned with formatting numbers in table cells or
% similar fixed-width contexts. The main function, \cs{siunitx_cell_begin:w},
% is designed to work with the normal \LaTeXe{} tablular cell construct
% featuring \cs{ignorespaces}. Therefore, if used outside of a \LaTeXe{}
% tabular, it is necessary to provide this token.
%
% \begin{function}{\siunitx_cell_begin:w, \siunitx_cell_end:}
%   \begin{syntax}
%     \cs{siunitx_cell_begin:w} \meta{preamble} \cs{ignorespaces}
%       \meta{content}
%     \cs{siunitx_cell_end:}
%   \end{syntax}
%   Collects the \meta{preamble} and \meta{content} tokens, and determines
%   if it is text or a number (as parsed by \cs{siunitx_number_parse:nN}).
%   It produces output of a fixed width suitable for alignment in a table,
%   although it is not \emph{required} that the code is used within a cell.
%   Note that |\ignorespaces| must occur in the \enquote{cell}: it marks
%   the end of the \TeX{} |\halign| template.
% \end{function}
%
% \subsection{Key--value options}
%
% The options defined by this submodule are available within the \pkg{l3keys}
% |siunitx| tree.
%
% \begin{function}{table-align-comparator}
%   \begin{syntax}
%     |table-align-comparator| = |true|\verb"|"|false|
%   \end{syntax}
%   Switch which determines whether alignment of comparators is attempted
%   within table cells. The standard setting is |true|.
% \end{function}
%
% \begin{function}{table-align-exponent}
%   \begin{syntax}
%     |table-align-exponent| = |true|\verb"|"|false|
%   \end{syntax}
%   Switch which determines whether alignment of exponents is attempted
%   within table cells. The standard setting is |true|.
% \end{function}
%
% \begin{function}{table-align-text-after}
%   \begin{syntax}
%     |table-align-text-after| = |true|\verb"|"|false|
%   \end{syntax}
%   Switch which determines whether alignment of text falling after a number
%   is attempted within table cells. The standard setting is |true|.
% \end{function}
%
% \begin{function}{table-align-text-before}
%   \begin{syntax}
%     |table-align-text-before| = |true|\verb"|"|false|
%   \end{syntax}
%   Switch which determines whether alignment of text falling before a number
%   is attempted within table cells. The standard setting is |true|.
% \end{function}
%
% \begin{function}{table-align-uncertainty}
%   \begin{syntax}
%     |table-align-uncertainty| = |true|\verb"|"|false|
%   \end{syntax}
%   Switch which determines whether alignment of separated uncertainty values
%   is attempted within table cells. The standard setting is |true|.
% \end{function}
%
% \begin{function}{table-alignment}
%   \begin{syntax}
%     |table-alignment| = |center|\verb"|"|left|\verb"|"|right|
%   \end{syntax}
%   Selects the alignment of all tabular content with the margins of the table
%   cell (or other boundary). See also |table-number-alignment| and
%   |table-text-alignment|. The standard setting is |center|.
% \end{function}
%
% \begin{function}{table-alignment-mode}
%   \begin{syntax}
%     |table-alignment-mode| = |format|\verb"|"|marker|\verb"|"|none|
%   \end{syntax}
%   Selects the method used to align numbers with the desired position in the
%   cell (set by |table-alignment|). When set to |format|, a dedicated
%   amount of space is calculated from the |table-format|. When |marker|
%   is selected, alignment is carried out symmetrically around the decimal
%   marker. Finally, |none| switches off all alignment: numbers are parsed
%   and formatted but with no attempt at placement within the cell. The
%   standard setting is |marker|.
% \end{function}
%
% \begin{function}{table-auto-round}
%   \begin{syntax}
%     |table-auto-round| = |true|\verb"|"|false|
%   \end{syntax}
%   Switch which determines whether numbers are rounded to fit within
%   the |table-format| specification (if possible). The standard
%   setting is |false|.
% \end{function}
%
% \begin{function}{table-column-width}
%   \begin{syntax}
%     |table-column-width| = \meta{width}
%   \end{syntax}
%   Sets the width of the table column used for numbers. This is only used
%   when |table-fixed-width| is |true|.
% \end{function}
%
% \begin{function}{table-fixed-width}
%   \begin{syntax}
%     |table-fixed-width| = |true|\verb"|"|false|
%   \end{syntax}
%   Switch which determines whether a fixed-width column is used for numbers
%   in tables. When |true|, the width is taken from |table-column-width|. The
%   standard setting is |false|.
% \end{function}
%
% \begin{function}{table-format}
%   \begin{syntax}
%     |table-format| = \meta{format}
%   \end{syntax}
%   Describes the amount of space that should be reserved when
%   |table-alignment-mode| is set to |format|. The \meta{format} takes the
%   same general form as input for a table cell, with the numerical parts
%   describing how many digits to reserve space for. For example, |1.2e3|
%   would allow space for one digit in the integer part, two in the decimal
%   part and three in the exponent part. Signs can be allowed for using any
%   valid input sign, so for example |+1.2 \pm 1.2| would allow for a sign,
%   a number with one integer and two decimal digits and a uncertainty of
%   the same size.
% \end{function}
%
% \begin{function}{table-number-alignment}
%   \begin{syntax}
%     |table-number-alignment| = |center|\verb"|"|left|\verb"|"|right|
%   \end{syntax}
%   Selects the alignment of numerical content with the margins of the table
%   cell (or other boundary). See also |table-alignment| and
%   |table-text-alignment|. The standard setting is |center|.
% \end{function}
%
% \begin{function}{table-text-alignment}
%   \begin{syntax}
%     |table-text-alignment| = |center|\verb"|"|left|\verb"|"|right|
%   \end{syntax}
%   Selects the alignment of non-numerical content with the margins of the
%   table cell (or other boundary). See also |table-alignment| and
%   |table-number-alignment|. The standard setting is |center|.
% \end{function}
%
% \end{documentation}
%
% \begin{implementation}
%
% \section{\pkg{siunitx-table} implementation}
%
% Start the \pkg{DocStrip} guards.
%    \begin{macrocode}
%<*package>
%    \end{macrocode}
%
% Identify the internal prefix (\LaTeX3 \pkg{DocStrip} convention): only
% internal material in this \emph{submodule} should be used directly.
%    \begin{macrocode}
%<@@=siunitx_table>
%    \end{macrocode}
%
% \begin{variable}{\l_@@_tmp_box, \l_@@_tmp_dim, \l_@@_tmp_tl}
%   Scratch space.
%    \begin{macrocode}
\box_new:N \l_@@_tmp_box
\dim_new:N \l_@@_tmp_dim
\tl_new:N \l_@@_tmp_tl
%    \end{macrocode}
% \end{variable}
%
% \subsection{Interface functions}
%
% \begin{variable}{\l_@@_text_bool}
%   Used to track that a cell is purely text.
%    \begin{macrocode}
\bool_new:N \l_@@_text_bool
%    \end{macrocode}
% \end{variable}
%
% \begin{macro}{\siunitx_cell_begin:w}
% \begin{macro}{\siunitx_cell_end:}
%   The start and end of the cell need to deal with the possibility of
%   a cell containing only text.
%    \begin{macrocode}
\cs_new_protected:Npn \siunitx_cell_begin:w
  {
    \bool_set_false:N \l_@@_text_bool
    \bool_if:NTF \l_siunitx_number_parse_bool
      { \@@_collect_begin: }
      { \@@_direct_begin: }
  }
\cs_new_protected:Npn \siunitx_cell_end:
  {
    \bool_if:NF \l_@@_text_bool
      {
        \bool_if:NTF \l_siunitx_number_parse_bool
          { \@@_collect_end: }
          { \@@_direct_end: }
      }
  }
%    \end{macrocode}
% \end{macro}
% \end{macro}
%
% \subsection{Collecting tokens}
%
% \begin{variable}{\l_@@_collect_tl}
%   Space for tokens.
%    \begin{macrocode}
\tl_new:N \l_@@_collect_tl
%    \end{macrocode}
% \end{variable}
%
% \begin{macro}{\@@_collect_begin:}
% \begin{macro}{\@@_collect_begin:w}
%   Collecting a tabular cell means doing a token-by-token collection.
%   In previous versions of \pkg{siunitx} that was done along with picking
%   out the numerical part, but the code flow ends up very tricky. Here,
%   therefore, we just collect up the unchanged tokens first. The definition of
%   \cs{cr} is used to allow collection of any tokens
%   inserted after the main content when dealing with the last cell of a row:
%   the \enquote{group} around it is needed to avoid issues with the underlying
%   |\halign|. (The approach is based on that in \pkg{collcell}.) Whilst the
%   group formed by a cell will normally tidy up |\cr|, we add an extra one as
%   the collected material could be a tabular in itself. We use an auxiliary to
%   fish out the |\ignorespaces| from the template: that has to go to avoid
%   issues with the peek-ahead code (everything before the |#| needs to be
%   read \emph{before} the Appendix~D trick gets applied). Some packages
%   add additional tokens before the |\ignorespaces|, which are dealt with by
%   the delimited argument.
%    \begin{macrocode}
\cs_new_protected:Npn \@@_collect_begin:
  {
    \group_begin:
      \tl_clear:N \l_@@_collect_tl
      \if_false: { \fi:
      \cs_set_protected:Npn \cr
        {
          \@@_collect_loop:
          \tex_cr:D
        }
      \if_false: } \fi:
      \@@_collect_begin:w
  }
\cs_new_protected:Npn \@@_collect_begin:w #1 \ignorespaces
  { \@@_collect_loop: #1 }
%    \end{macrocode}
% \end{macro}
% \end{macro}
%
% \begin{macro}{\@@_collect_loop:}
% \begin{macro}{\@@_collect_group:n}
% \begin{macro}{\@@_collect_token:N}
% \begin{macro}{\@@_collect_relax:N}
% \begin{macro}{\@@_collect_search:NnF}
% \begin{macro}{\@@_collect_search_aux:NNn}
%   Collecting up the cell content needs a loop: this is done using
%   a |peek| approach as it's most natural. (A slower approach is possible
%   using something like the |\text_lowercase:n| loop code.) The set of
%   possible tokens is somewhat limited compared to an arbitrary cell
%   (\foreign{cf.}~the approach in \pkg{collcell}): the special cases are
%   pulled out for manual handling. The flexible lookup approach is more-or-less
%   the same idea as in the kernel |case| functions. The |\relax| special case
%   covers the case where |\\| has been expanded in an empty cell. This has to
%   be an explicit token as we can get the same meaning from |\protect|.
%    \begin{macrocode}
\cs_new_protected:Npn \@@_collect_loop:
  {
    \peek_catcode_ignore_spaces:NTF \c_group_begin_token
      { \@@_collect_group:n }
      { \@@_collect_token:N }
  }
\cs_new_protected:Npn \@@_collect_group:n #1
  {
    \tl_put_right:Nn \l_@@_collect_tl { {#1} }
    \@@_collect_loop:
  }
\cs_new_protected:Npn \@@_collect_token:N #1
  {
    \@@_collect_search:NnF #1
      {
        \unskip            { \@@_collect_loop: }
        \end               { \tabularnewline \end }
        \relax             { \@@_collect_relax:N #1 }
        \tabularnewline    { \tabularnewline }
        \siunitx_cell_end: { \siunitx_cell_end: }
      }
      {
        \tl_put_right:Nn \l_@@_collect_tl {#1}
        \@@_collect_loop:
      }
  }
\cs_new_protected:Npn \@@_collect_relax:N #1
  {
    \str_if_eq:nnTF {#1} { \relax }
      { \relax }
      {
        \tl_put_right:Nn \l_@@_collect_tl {#1}
        \@@_collect_loop:
      }
  }
\AtBeginDocument
  {
    \@ifpackageloaded { mdwtab }
      {
        \cs_set_protected:Npn \@@_collect_token:N #1
          {
            \@@_collect_search:NnF #1
              {
                \@maybe@unskip     { \@@_collect_loop: }
                \tab@setcr         { \@@_collect_loop: }
                \unskip            { \@@_collect_loop: }
                \end               { \tabularnewline \end }
                \relax             { \@@_collect_relax:N #1 }
                \tabularnewline    { \tabularnewline }
                \siunitx_cell_end: { \siunitx_cell_end: }
              }
              {
                \tl_put_right:Nn \l_@@_collect_tl {#1}
                \@@_collect_loop:
              }
          }
      }
      { }
  }
\cs_new_protected:Npn \@@_collect_search:NnF #1#2#3
  {
    \@@_collect_search_aux:NNn #1
      #2
      #1 {#3}
    \q_stop
  }
\cs_new_protected:Npn \@@_collect_search_aux:NNn #1#2#3
  {
    \token_if_eq_meaning:NNTF #1 #2
      { \use_i_delimit_by_q_stop:nw {#3} }
      { \@@_collect_search_aux:NNn #1 }
  }
%    \end{macrocode}
% \end{macro}
% \end{macro}
% \end{macro}
% \end{macro}
% \end{macro}
% \end{macro}
%
% \subsection{Separating collected material}
%
% The input needs to be divided into numerical tokens and those which appear
% before and after them. This needs a second loop and validation.
%
% \begin{variable}{\l_@@_before_tl, \l_@@_number_tl, \l_@@_after_tl}
%   Space for tokens.
%    \begin{macrocode}
\tl_new:N \l_@@_before_tl
\tl_new:N \l_@@_number_tl
\tl_new:N \l_@@_after_tl
%    \end{macrocode}
% \end{variable}
%
% \begin{macro}{\@@_collect_end:}
% \begin{macro}{\@@_collect_end:n}
% \begin{macro}[EXP]{\@@_collect_end_aux:n}
% \begin{macro}[EXP]{\@@_collect_end:w}
%   At the end of the cell, escape the group and check for expansion. We
%   only do that if the entire content is not a brace group: there is more
%   likely to be problematic content in the case of a header.
%    \begin{macrocode}
\cs_new_protected:Npn \@@_collect_end:
  {
    \exp_args:NNV \group_end:
    \@@_collect_end:n \l_@@_collect_tl
    \exp_args:NV \@@_split:nNNN
      \l_@@_collect_tl
      \l_@@_before_tl
      \l_@@_number_tl
      \l_@@_after_tl
    \tl_if_empty:NTF \l_@@_number_tl
      { \@@_print_text:V \l_@@_before_tl }
      {
        \@@_print:VVV
          \l_@@_before_tl
          \l_@@_number_tl
          \l_@@_after_tl
      }
  }
\cs_new_protected:Npn \@@_collect_end:n #1
  {
    \str_if_eq:eeTF { \exp_not:n {#1} }
      { { \@@_collect_end_aux:n {#1} } }
      { \tl_set:Nn }
      { \protected@edef }
        \l_@@_collect_tl {#1}
  }
\cs_new:Npn \@@_collect_end_aux:n #1
  { \exp_after:wN \@@_collect_end:w #1 \q_stop }
\cs_new:Npn \@@_collect_end:w #1 \q_stop
  { \exp_not:n {#1} }
%    \end{macrocode}
% \end{macro}
% \end{macro}
% \end{macro}
% \end{macro}
%
% \begin{macro}{\@@_split:nNNN}
% \begin{macro}{\@@_split_loop:NNN}
% \begin{macro}{\@@_split_group:NNNn}
% \begin{macro}{\@@_split_token:NNNN}
%   Splitting into parts uses the fact that numbers cannot contain groups
%   and that we can track where we are up to based on the content of the
%   token lists.
%    \begin{macrocode}
\cs_new_protected:Npn \@@_split:nNNN #1#2#3#4
  {
    \tl_clear:N #2
    \tl_clear:N #3
    \tl_clear:N #4
    \@@_split_loop:NNN #2#3#4 #1 \q_recursion_tail \q_recursion_stop
    \@@_split_tidy:N #2
    \@@_split_tidy:N #4
  }
\cs_new_protected:Npn \@@_split_loop:NNN #1#2#3
  {
    \peek_catcode_ignore_spaces:NTF \c_group_begin_token
      { \@@_split_group:NNNn #1#2#3 }
      { \@@_split_token:NNNN #1#2#3 }
  }
\cs_new_protected:Npn \@@_split_group:NNNn #1#2#3#4
  {
    \tl_if_empty:NTF #2
      { \tl_put_right:Nn #1 { {#4} } }
      { \tl_put_right:Nn #3 { {#4} } }
    \@@_split_loop:NNN #1#2#3
  }
\cs_new_protected:Npn \@@_split_token:NNNN #1#2#3#4
  {
    \quark_if_recursion_tail_stop:N #4
    \tl_if_empty:NTF \l_@@_after_tl
      {
        \siunitx_if_number_token:NTF #4
          { \tl_put_right:Nn #2 {#4} }
          {
            \tl_if_empty:NTF #2
              { \tl_put_right:Nn #1 {#4} }
              { \tl_put_right:Nn #3 {#4} }
          }
      }
      { \tl_put_right:Nn #3 {#4} }
    \@@_split_loop:NNN #1#2#3
  }
%    \end{macrocode}
% \end{macro}
% \end{macro}
% \end{macro}
% \end{macro}
%
% \begin{macro}{\@@_split_tidy:N}
% \begin{macro}{\@@_split_tidy:Nn, \@@_split_tidy:NV}
%   A quick test for the entire content being surrounded by a set of braces:
%   rather than look explicitly, use the fact that a string comparison can
%   detect the same thing. The auxiliary is needed to avoid having to go
%   \foreign{via} a |:D| function (for the expansion behaviour).
%    \begin{macrocode}
\cs_new_protected:Npn \@@_split_tidy:N #1
  {
    \tl_if_empty:NF #1
      { \@@_split_tidy:NV #1 #1 }
  }
\cs_new_protected:Npn \@@_split_tidy:Nn #1#2
  {
    \str_if_eq:onT { \exp_after:wN { \use:n #2 } } {#2}
      { \tl_set:No #1 { \use:n #2 } }
  }
\cs_generate_variant:Nn \@@_split_tidy:Nn { NV }
%    \end{macrocode}
% \end{macro}
% \end{macro}
%
% \subsection{Printing numbers in cells: spacing}
%
% Getting the general alignment correct in tables is made more complex than one
% would like by the \pkg{colortbl} package. In the original \LaTeXe{}
% definition, cell material is centred by a construction of the (primitive)
% form
% \begin{verbatim}
%   \hfil
%   #
%   \hfil
% \end{verbatim}
% which only uses \texttt{fil} stretch. That is altered by \pkg{colortbl} to
% broadly
% \begin{verbatim}
%   \hskip 0pt plus 0.5fill
%   \kern 0pt
%   #
%   \hskip 0pt plus 0.5fill
% \end{verbatim}
% which means there is \texttt{fill} stretch to worry about and the kern as
% well.
%
% \begin{macro}{\@@_skip:n}
%   To prevent combination of skips, a kern is inserted after each one.
%   This is best handled as a short auxiliary.
%    \begin{macrocode}
\cs_new_protected:Npn \@@_skip:n #1
  {
    \skip_horizontal:n {#1}
    \tex_kern:D \c_zero_skip
  }
%    \end{macrocode}
% \end{macro}
%
% \begin{variable}{\l_@@_column_width_dim, \l_@@_fixed_width_bool}
%   Settings which apply to aligned columns in general.
%    \begin{macrocode}
\keys_define:nn { siunitx }
  {
    table-column-width .dim_set:N =
      \l_@@_column_width_dim ,
    table-fixed-width .bool_set:N =
      \l_@@_fixed_width_bool
  }
%    \end{macrocode}
% \end{variable}
%
% \begin{macro}{\@@_align_center:n, \@@_align_left:n, \@@_align_right:n}
% \begin{macro}{\@@_align_auxi:nn, \@@_align_auxii:nn}
%   The beginning and end of each table cell have to adjust the position of
%   the content using glue. When \pkg{colortbl} is loaded the glue is done in
%   two parts: one for our positioning and one to explicitly override that from
%   the package. Using a two-step auxiliary chain avoids needing to repeat any
%   code and the impact of the extra expansion should be trivial.
%    \begin{macrocode}
\cs_new_protected:Npn \@@_align_center:n #1
  { \@@_align_auxi:nn {#1} { 0pt~plus~0.5fill } }
\cs_new_protected:Npn \@@_align_left:n #1
  { \@@_align_auxi:nn {#1} { 0pt } }
\cs_new_protected:Npn \@@_align_right:n #1
  { \@@_align_auxi:nn {#1} { 0pt~plus~1fill } }
\cs_new_protected:Npn \@@_align_auxi:nn #1#2
  {
    \bool_if:NTF \l_@@_fixed_width_bool
      { \hbox_to_wd:nn \l_@@_column_width_dim }
      { \use:n }
        {
          \@@_skip:n {#2}
          #1
          \@@_skip:n { 0pt~plus~1fill - #2 }
        }
  }
\AtBeginDocument
  {
    \@ifpackageloaded { colortbl }
      {
        \cs_new_eq:NN
          \@@_align_auxii:nn
          \@@_align_auxi:nn
        \cs_set_protected:Npn \@@_align_auxi:nn #1#2
          {
            \@@_skip:n{ 0pt~plus~-0.5fill }
            \@@_align_auxii:nn {#1} {#2}
            \@@_skip:n { 0pt~plus~-0.5fill }
          }
      }
      { }
  }
%    \end{macrocode}
% \end{macro}
% \end{macro}
%
% \subsection{Printing just text}
%
% In cases where there is no numerical part, \pkg{siunitx} allows alignment
% of the \enquote{escaped} text independent of the underlying column type.
%
% \begin{variable}{\l_@@_align_text_tl}
%   Alignment is handled using a |tl| as this allows a fast lookup at the
%   point of use.
%    \begin{macrocode}
\keys_define:nn { siunitx }
  {
    table-text-alignment .choices:nn =
      { center , left , right }
      { \tl_set:Nn \l_@@_align_text_tl {#1} } ,
  }
\tl_new:N \l_@@_align_text_tl
%    \end{macrocode}
% \end{variable}
%
% \begin{macro}{\@@_print_text:n, \@@_print_text:V}
%   Printing escaped text is easy: just place it in correctly in the
%   column.
%    \begin{macrocode}
\cs_new_protected:Npn \@@_print_text:n #1
  {
    \bool_set_true:N \l_@@_text_bool
    \use:c { @@_align_ \l_@@_align_text_tl :n } {#1}
  }
\cs_generate_variant:Nn \@@_print_text:n { V }
%    \end{macrocode}
% \end{macro}
%
% \subsection{Number alignment: core ideas}
%
% \begin{variable}{\l_@@_integer_box, \l_@@_decimal_box}
%   Boxes for the content before and after the decimal marker.
%    \begin{macrocode}
\box_new:N \l_@@_integer_box
\box_new:N \l_@@_decimal_box
%    \end{macrocode}
% \end{variable}
%
% \begin{macro}{\@@_fil:}
%    A primitive renamed.
%    \begin{macrocode}
\cs_new_eq:NN \@@_fil: \tex_hfil:D
%    \end{macrocode}
% \end{macro}
%
% \begin{macro}{\@@_cleanup_decimal:w}
%   To remove the excess marker tokens in a decimal part.
%    \begin{macrocode}
\cs_new:Npn \@@_cleanup_decimal:w
  #1 \q_nil #2 \q_nil #3 \q_nil #4 \q_nil #5 \q_nil #6 \q_nil #7 \q_nil
  { #1#2#3#4#5#6#7 }
%    \end{macrocode}
% \end{macro}
%
% \begin{macro}{\@@_center_marker:}
%   When centering on the decimal marker, the easiest approach is to simply
%   re-box the two parts. That is needed whether or not we are parsing numbers,
%   so is best as a short auxiliary.
%    \begin{macrocode}
\cs_new_protected:Npn \@@_center_marker:
  {
    \dim_compare:nNnTF
      { \box_wd:N \l_@@_integer_box }
        > { \box_wd:N \l_@@_decimal_box }
      {
        \hbox_set_to_wd:Nnn \l_@@_decimal_box
          { \box_wd:N \l_@@_integer_box }
          {
            \hbox_unpack:N \l_@@_decimal_box
            \@@_fil:
          }
      }
      {
        \hbox_set_to_wd:Nnn \l_@@_integer_box
          { \box_wd:N \l_@@_decimal_box }
          {
            \@@_fil:
            \hbox_unpack:N \l_@@_integer_box
          }
      }
  }
%    \end{macrocode}
% \end{macro}
%
% \begin{variable}
%   {\l_@@_auto_round_bool, \l_@@_align_mode_tl, \l_@@_align_number_tl}
%   Options for tables with defined space.
%    \begin{macrocode}
\keys_define:nn { siunitx }
  {
    table-alignment .meta:n =
      { table-number-alignment = #1 , table-text-alignment = #1 },
    table-alignment-mode .choices:nn =
      { none , format , marker }
      { \tl_set_eq:NN \l_@@_align_mode_tl \l_keys_choice_tl } ,
    table-auto-round .bool_set:N =
      \l_@@_auto_round_bool ,
    table-format .code:n =
      {
        \@@_split:nNNN {#1}
          \l_@@_before_model_tl
          \l_@@_model_tl
          \l_@@_after_model_tl
        \exp_args:NV \@@_generate_model:n \l_@@_model_tl
        \tl_set:Nn \l_@@_align_mode_tl { format }
      } ,
    table-number-alignment .choices:nn =
      { center , left , right }
      { \tl_set_eq:NN \l_@@_align_number_tl \l_keys_choice_tl }
  }
\tl_new:N \l_@@_align_mode_tl
\tl_new:N \l_@@_align_number_tl
%    \end{macrocode}
% \end{variable}
%
% \begin{variable}{\l_@@_format_tl, \l_@@_model_tl}
%   The input and output versions of the model entry in a table.
%    \begin{macrocode}
\tl_new:N \l_@@_format_tl
\tl_new:N \l_@@_before_model_tl
\tl_new:N \l_@@_model_tl
\tl_new:N \l_@@_after_model_tl
%    \end{macrocode}
% \end{variable}
%
% \begin{macro}{\@@_generate_model:n}
% \begin{macro}{\@@_generate_model:nnnnnnn}
% \begin{macro}[EXP]{\@@_generate_model_S:nnw}
% \begin{macro}[EXP]{\@@_generate_model_S:nnn}
%   Creating a model for a table at this stage means parsing the format and
%   converting that to an appropriate model. Things are quite straight-forward
%   other than the uncertainty part. At this stage there is no point in
%   formatting the model: that has to happen at point-of-use. Notice that
%   the uncertainty part needs to allow for the case where we cross the
%   decimal.
%    \begin{macrocode}
\cs_new_protected:Npn \@@_generate_model:n #1
  {
    \group_begin:
      \bool_set_true:N \l_siunitx_number_parse_bool
      \keys_set:nn { siunitx } { retain-explicit-plus = true }
      \siunitx_number_parse:nN {#1} \l_@@_format_tl
    \exp_args:NNNV \group_end:
    \tl_set:Nn \l_@@_format_tl \l_@@_format_tl
    \tl_if_empty:NF \l_@@_format_tl
      {
        \exp_after:wN \@@_generate_model:nnnnnnn
          \l_@@_format_tl
      }
  }
\cs_new_protected:Npn \@@_generate_model:nnnnnnn #1#2#3#4#5#6#7
  {
    \tl_set:Nx \l_@@_model_tl
      {
        \exp_not:n { {#1} {#2} }
        { \prg_replicate:nn {#3} { 8 } }
        { \prg_replicate:nn { 0 #4 } { 8 } }
        {
          \tl_if_blank:nF {#5}
            {
              \use:c { @@_generate_model_ \tl_head:n {#5} :nnw }
                {#4} #5
            }
        }
        \exp_not:n { {#6} }
        {
          \int_compare:nNnTF {#7} = 0
            { 0 }
            { \prg_replicate:nn {#7} { 8 } }
        }
      }
  }
\cs_new:Npn \@@_generate_model_S:nnw #1#2#3
  {
    { S }
    {
      \exp_args:Nff \@@_generate_model_S:nnn
        { \tl_count:n {#1} } { \tl_count:n {#3} }
        {#3}
    }
  }
\cs_new:Npn \@@_generate_model_S:nnn #1#2#3
  {
    \prg_replicate:nn
      {
        \int_compare:nNnTF {#2} > {#1}
          {
            \str_range:nnn {#3} { 1 } {#1}
            +
            \str_range:nnn {#3} { 1 + #1 } {#2}
          }
          {#3}
      }
      { 8 }
  }
%    \end{macrocode}
% \end{macro}
% \end{macro}
% \end{macro}
% \end{macro}
%
% \subsection{Directly printing without collection}
%
% Collecting the number allows for various effects but is not as fast
% as simply aligning on the first token that is a decimal marker. The
% strategy here is that used by \pkg{dcolumn}.
%
% \begin{macro}{\@@_direct_begin:}
% \begin{macro}{\@@_direct_begin:w}
% \begin{macro}{\@@_direct_end:}
% \begin{macro}
%   {
%     \@@_direct_marker:        ,
%     \@@_direct_marker_switch: ,
%     \@@_direct_marker_end:
%   }
% \begin{macro}{\@@_direct_format:}
% \begin{macro}[EXP]{\@@_direct_format:nnnnnnn}
% \begin{macro}{\@@_direct_format:w}
% \begin{macro}
%   {
%     \@@_direct_format_switch: ,
%     \@@_direct_format_end:
%   }
% \begin{macro}{\@@_direct_none:, \@@_direct_none_end:}
%   After removing the |\ignorespaces| at the start of the cell (see comments
%   for \cs{@@_collect_begin:N}), check to see  if there is a |{| and branch
%   as appropriate.
%    \begin{macrocode}
\cs_new_protected:Npn \@@_direct_begin:
  { \@@_direct_begin:w }
\cs_new_protected:Npn \@@_direct_begin:w #1 \ignorespaces
  {
    #1
    \peek_catcode_ignore_spaces:NTF \c_group_begin_token
      { \@@_print_text:n }
      {
        \m@th
        \use:c { @@_direct_ \l_@@_align_mode_tl : }
      }
  }
\cs_new_protected:Npn \@@_direct_end:
  { \use:c { @@_direct_ \l_@@_align_mode_tl _end: } }
%    \end{macrocode}
%   When centring the content about a decimal marker, the trick is
%   to collect everything into two boxes and then compare the sizes.
%   As we are always in math mode, we can use a math active token
%   to make the switch. The up-front setting of the |decimal| box deals
%   with the case where there is no decimal part.
%    \begin{macrocode}
\cs_new_protected:Npn \@@_direct_marker:
  {
    \hbox_set:Nn \l_@@_tmp_box
      { \ensuremath { \mathord { \l_siunitx_number_output_decimal_tl } } }
    \hbox_set_to_wd:Nnn \l_@@_decimal_box
      { \box_wd:N \l_@@_tmp_box }
      { \@@_fil: }
    \hbox_set:Nw \l_@@_integer_box
      \c_math_toggle_token
      \tl_map_inline:Nn \l_siunitx_number_input_decimal_tl
        {
          \char_set_active_eq:NN ##1 \@@_direct_marker_switch:
          \char_set_mathcode:nn { `##1 } { "8000 }
        }
  }
\cs_new_protected:Npn \@@_direct_marker_switch:
  {
      \c_math_toggle_token
    \hbox_set_end:
    \hbox_set:Nw \l_@@_decimal_box
      \c_math_toggle_token
      \l_siunitx_number_output_decimal_tl
  }
\cs_new_protected:Npn \@@_direct_marker_end:
  {
      \c_math_toggle_token
    \hbox_set_end:
    \@@_center_marker:
    \box_use_drop:N \l_@@_integer_box
    \box_use_drop:N \l_@@_decimal_box
  }
%    \end{macrocode}
%   For the version where there is space reserved, first format and decompose
%   that, then create appropriately-sized boxes.
%    \begin{macrocode}
\cs_new_protected:Npn \@@_direct_format:
  {
    \tl_set:Nx \l_@@_tmp_tl
      { \siunitx_number_output:NN \l_@@_model_tl \q_nil }
    \exp_after:wN \@@_direct_format_aux:w
      \l_@@_tmp_tl \q_stop
  }
\cs_new_protected:Npn \@@_direct_format_aux:w
  #1 \q_nil #2 \q_nil #3 \q_nil #4 \q_stop
  {
    \hbox_set:Nn \l_@@_tmp_box
      { \ensuremath { \@@_cleanup_decimal:w #4 } }
    \hbox_set_to_wd:Nnn \l_@@_decimal_box
      { \box_wd:N \l_@@_tmp_box }
      { \@@_fil: }
    \hbox_set:Nn \l_@@_tmp_box { \ensuremath { #1#2#3 } }
    \hbox_set_to_wd:Nnw \l_@@_integer_box
      { \box_wd:N \l_@@_tmp_box }
      \c_math_toggle_token
      \tl_map_inline:Nn \l_siunitx_number_input_decimal_tl
        {
          \char_set_active_eq:NN ##1 \@@_direct_format_switch:
          \char_set_mathcode:nn { `##1 } { "8000 }
        }
      \@@_fil:
  }
\cs_new_protected:Npn \@@_direct_format_switch:
  {
      \c_math_toggle_token
    \hbox_set_end:
    \hbox_set_to_wd:Nnw \l_@@_decimal_box
      { \box_wd:N \l_@@_decimal_box }
      \c_math_toggle_token
      \mathord { \l_siunitx_number_output_decimal_tl }
  }
\cs_new_protected:Npn \@@_direct_format_end:
  {
      \c_math_toggle_token
      \@@_fil:
    \hbox_set_end:
    \use:c { @@_align_ \l_@@_align_number_tl :n }
      {
        \box_use_drop:N \l_@@_integer_box
        \box_use_drop:N \l_@@_decimal_box
      }
  }
%    \end{macrocode}
%   No parsing and no alignment is easy.
%    \begin{macrocode}
\cs_new_protected:Npn \@@_direct_none: { \c_math_toggle_token }
\cs_new_protected:Npn \@@_direct_none_end: { \c_math_toggle_token }
%    \end{macrocode}
% \end{macro}
% \end{macro}
% \end{macro}
% \end{macro}
% \end{macro}
% \end{macro}
% \end{macro}
% \end{macro}
% \end{macro}
%
% \subsection{Printing numbers in cells: main functions}
%
% \begin{variable}{\l_@@_before_box, \l_@@_after_box}
%   For alignment of text outside of a number.
%    \begin{macrocode}
\box_new:N \l_@@_before_box
\box_new:N \l_@@_after_box
%    \end{macrocode}
% \end{variable}
%
% \begin{variable}{\l_@@_carry_dim}
%   Used to \enquote{carry forward} the amount of white space which needs to
%   be inserted after the decimal marker.
%    \begin{macrocode}
\dim_new:N \l_@@_carry_dim
%    \end{macrocode}
% \end{variable}
%
% \begin{variable}
%   {
%     \l_@@_align_comparator_bool  ,
%     \l_@@_align_exponent_bool    ,
%     \l_@@_align_after_bool       ,
%     \l_@@_align_before_bool      ,
%     \l_@@_align_uncertainty_bool
%   }
%   Alignment is handled using a |tl| as this allows a fast lookup at the
%   point of use.
%    \begin{macrocode}
\keys_define:nn { siunitx }
  {
    table-align-comparator .bool_set:N =
      \l_@@_align_comparator_bool ,
    table-align-exponent .bool_set:N =
      \l_@@_align_exponent_bool ,
    table-align-text-after .bool_set:N =
      \l_@@_align_after_bool ,
    table-align-text-before .bool_set:N =
      \l_@@_align_before_bool ,
    table-align-uncertainty .bool_set:N =
      \l_@@_align_uncertainty_bool
  }
%    \end{macrocode}
% \end{variable}
%
% \begin{macro}{\@@_print:nnn, \@@_print:VVV}
% \begin{macro}{\@@_print_marker:nnn}
% \begin{macro}{\@@_print_marker:w}
% \begin{macro}[EXP]{\@@_print_marker_aux:w}
% \begin{macro}{\@@_print_format:nnn}
% \begin{macro}
%   {
%     \@@_print_marker_auxi:w,
%     \@@_print_marker_auxii:w,
%     \@@_print_marker_auxiii:w
%   }
% \begin{macro}{\@@_print_format_after:N}
% \begin{macro}{\@@_print_format_box:Nn}
% \begin{macro}{\@@_print_none:nnn}
%    \begin{macrocode}
\cs_new_protected:Npn \@@_print:nnn #1#2#3
  { \use:c { @@_print_ \l_@@_align_mode_tl :nnn } {#1} {#2} {#3} }
\cs_generate_variant:Nn \@@_print:nnn { VVV }
%    \end{macrocode}
%   When centering on the decimal marker, alignment is relatively simple, and
%   close in concept to that used without parsing. First we need to deal with any
%   text before or after the number. For text \emph{before}, there's the case
%   where is has no width and might be a font or color change: that has to be
%   filtered out first. Then we can adjust the size of this material and that
%   after the number such that they are equal. The number itself can then be
%   formatted, splitting at he decimal marker. A bit more size adjustment, then
%   the number itself and any text at the end can be inserted.
%    \begin{macrocode}
\cs_new_protected:Npn \@@_print_marker:nnn #1#2#3
  {
    \hbox_set:Nn \l_@@_before_box {#1}
    \dim_compare:nNnT { \box_wd:N \l_@@_before_box } = { 0pt }
      {
        \box_clear:N \l_@@_before_box
        #1
      }
    \hbox_set:Nn \l_@@_after_box {#3}
    \dim_compare:nNnTF
      { \box_wd:N \l_@@_after_box }
       > { \box_wd:N \l_@@_before_box }
      {
        \hbox_set_to_wd:Nnn \l_@@_before_box
          { \box_wd:N \l_@@_after_box }
          {
            \@@_fil:
            \hbox_unpack:N \l_@@_before_box
          }
      }
      {
        \hbox_set_to_wd:Nnn \l_@@_after_box
          { \box_wd:N \l_@@_before_box }
          {
            \hbox_unpack:N \l_@@_after_box
            \@@_fil:
          }
      }
    \box_use_drop:N \l_@@_before_box
    \siunitx_number_parse:nN {#2} \l_@@_tmp_tl
    \siunitx_number_process:NN \l_@@_tmp_tl \l_@@_tmp_tl
    \tl_set:Nx \l_@@_tmp_tl
      { \siunitx_number_output:NN \l_@@_tmp_tl \q_nil }
    \exp_after:wN \@@_print_marker:w
      \l_@@_tmp_tl \q_stop
    \box_use_drop:N \l_@@_after_box
  }
\cs_new_protected:Npn \@@_print_marker:w
  #1 \q_nil #2 \q_nil #3 \q_nil #4 \q_stop
  {
    \hbox_set:Nn \l_@@_integer_box
      { \siunitx_print_number:n { #1#2#3 } }
    \hbox_set:Nn \l_@@_decimal_box
      {
        \siunitx_print_number:x
          { \@@_print_marker_aux:w #4 }
      }
    \@@_center_marker:
    \box_use_drop:N \l_@@_integer_box
    \box_use_drop:N \l_@@_decimal_box
  }
\cs_new:Npn \@@_print_marker_aux:w
  #1 \q_nil #2 \q_nil #3 \q_nil #4 \q_nil #5 \q_nil #6 \q_nil #7 \q_nil
  {
    \exp_not:n {#1#2#3#4#5}
    \tl_if_blank:nT {#1#2#3#4#5} { { } }
    \exp_not:n {#6#7}
  }
%    \end{macrocode}
%   For positioning based on a format, we have to work part-by-part as there
%   are a number of alignment points to get right. As for the |marker| approach,
%   first we check if the material before the numerical content is of zero
%   width. Next we need to format the model and content numbers, before
%   starting an auxiliary chain to pick out the various parts in order.
%    \begin{macrocode}
\cs_new_protected:Npn \@@_print_format:nnn #1#2#3
  {
    \hbox_set:Nn \l_@@_tmp_box { \l_@@_before_model_tl }
    \hbox_set:Nn \l_@@_before_box {#1}
    \dim_compare:nNnT { \box_wd:N \l_@@_before_box } = { 0pt }
      {
        \box_clear:N \l_@@_before_box
        #1
      }
    \hbox_set_to_wd:Nnn \l_@@_before_box
      { \box_wd:N \l_@@_tmp_box }
      {
        \@@_fil:
        \hbox_unpack:N \l_@@_before_box
      }
    \siunitx_number_parse:nN {#2} \l_@@_tmp_tl
    \group_begin:
      \bool_if:NT \l_@@_auto_round_bool
        {
          \exp_args:Nx \keys_set:nn { siunitx }
            {
              round-mode      = places ,
              round-pad       = true   ,
              round-precision =
                \exp_after:wN \@@_print_format:nnnnnn
                  \l_@@_format_tl
            }
        }
      \siunitx_number_process:NN \l_@@_tmp_tl \l_@@_tmp_tl
    \exp_args:NNNV \group_end:
    \tl_set:Nn \l_@@_tmp_tl \l_@@_tmp_tl
    \tl_set:Nx \l_@@_tmp_tl
      {
        \siunitx_number_output:NN \l_@@_model_tl \q_nil
        \exp_not:N \q_mark
        \siunitx_number_output:NN \l_@@_tmp_tl \q_nil
      }
    \exp_after:wN \@@_print_format_auxi:w
      \l_@@_tmp_tl \q_stop
    \hbox_set:Nn \l_@@_tmp_box { \l_@@_after_model_tl }
    \hbox_set_to_wd:Nnn \l_@@_after_box
      { \box_wd:N \l_@@_tmp_box + \l_@@_carry_dim }
      {
        \bool_if:NT \l_@@_align_after_bool
          { \skip_horizontal:n { \l_@@_carry_dim } }
        #3
        \@@_fil:
      }
    \use:c { @@_align_ \l_@@_align_number_tl :n }
      {
        \box_use_drop:N \l_@@_before_box
        \box_use_drop:N \l_@@_integer_box
        \box_use_drop:N \l_@@_decimal_box
         \box_use_drop:N \l_@@_after_box
      }
  }
\cs_new:Npn \@@_print_format:nnnnnn #1#2#3#4#5#6#7
  { 0 #4 }
%    \end{macrocode}
%   The first numerical part to handle is the comparator. Any white space we
%   need to add goes into the text part \emph{if} alignment is not active
%   (\foreign{i.e.}~we are looking \enquote{backwards} to place this filler).
%    \begin{macrocode}
\cs_new_protected:Npn \@@_print_format_auxi:w
  #1 \q_nil #2 \q_mark #3 \q_nil #4 \q_stop
  {
    \@@_print_format_box:Nn \l_@@_tmp_box {#1}
    \bool_if:NTF \l_@@_align_before_bool
      {
        \hbox_set_to_wd:Nnn \l_@@_integer_box
          { \box_wd:N \l_@@_tmp_box }
          {
            \@@_fil:
            \tl_if_blank:nF {#3}
             { \siunitx_print_number:n {#3} }
          }
      }
      {
        \@@_print_format_box:Nn \l_@@_integer_box {#3}
        \hbox_set_to_wd:Nnn \l_@@_before_box
          {
              \box_wd:N \l_@@_before_box
            + \box_wd:N \l_@@_tmp_box
            - \box_wd:N \l_@@_integer_box
          }
          {
            \@@_fil:
            \hbox_unpack:N \l_@@_before_box
          }
      }
    \@@_print_format_auxii:w #2 \q_mark #4 \q_stop
  }
%    \end{macrocode}
%   The integer part follows much the same pattern, except now it is control
%   of the comparator alignment that determines where the white space goes.
%   As we already have content in the |integer| box, we need to measure how
%   much \emph{extra} material has been added. To avoid using more boxes
%   or re-setting, we do that by recording sizes before and after the change.
%   (In effect, \cs{l_@@_tmp_dim} is here \enquote{l_@@_comparator_dim}.)
%    \begin{macrocode}
\cs_new_protected:Npn \@@_print_format_auxii:w
  #1 \q_nil #2 \q_nil #3 \q_mark #4 \q_nil #5 \q_nil #6 \q_stop
  {
    \@@_print_format_box:Nn \l_@@_tmp_box {#1#2}
    \bool_lazy_and:nnTF
      { \l_@@_align_comparator_bool }
      { \dim_compare_p:nNn { \box_wd:N \l_@@_integer_box } > { 0pt } }
      {
        \hbox_set_to_wd:Nnn \l_@@_integer_box
          {
              \box_wd:N \l_@@_integer_box
            + \box_wd:N \l_@@_tmp_box
          }
          {
            \hbox_unpack:N \l_@@_integer_box
            \@@_fil:
            \siunitx_print_number:n {#4#5}
          }
      }
      {
        \bool_if:NTF \l_@@_align_before_bool
          {
            \hbox_set_to_wd:Nnn \l_@@_integer_box
              {
                  \box_wd:N \l_@@_integer_box
                + \box_wd:N \l_@@_tmp_box
              }
              {
                \@@_fil:
                \hbox_unpack:N \l_@@_integer_box
                \siunitx_print_number:n {#4#5}
              }
          }
          {
            \dim_set:Nn \l_@@_tmp_dim
              { \box_wd:N \l_@@_integer_box }
            \hbox_set:Nn \l_@@_integer_box
              {
                \hbox_unpack:N \l_@@_integer_box
                \siunitx_print_number:n {#4#5}
              }
            \hbox_set_to_wd:Nnn \l_@@_before_box
              {
                  \box_wd:N \l_@@_before_box
                + \box_wd:N \l_@@_tmp_box
                + \l_@@_tmp_dim
                - \box_wd:N \l_@@_integer_box
              }
              {
                \@@_fil:
                \hbox_unpack:N \l_@@_before_box
              }
          }
      }
    \@@_print_format_auxiii:w #3 \q_mark #6 \q_stop
  }
%    \end{macrocode}
%   We now deal with the decimal part: there is nothing already in the
%   |decimal| box, so the basics are easy. We need to \enquote{carry forward}
%   any white space, as where it gets inserted depends on the options for
%   subsequent parts.
%    \begin{macrocode}
\cs_new_protected:Npn \@@_print_format_auxiii:w
  #1 \q_nil #2 \q_nil #3 \q_mark #4 \q_nil #5 \q_nil #6 \q_stop
  {
    \@@_print_format_box:Nn \l_@@_tmp_box {#1#2}
    \@@_print_format_box:Nn \l_@@_decimal_box {#4#5}
    \dim_set:Nn \l_@@_carry_dim
      {
          \box_wd:N \l_@@_tmp_box
        - \box_wd:N \l_@@_decimal_box
      }
    \@@_print_format_auxiv:w #3 \q_mark #6 \q_stop
  }
%    \end{macrocode}
%   Any separated uncertainty is now picked up. That has a number of parts, so
%   the first step is to look for a sign (which will be |#1|). We then split,
%   either simply tidying up the markers if there is no uncertainty, or
%   setting it.
%    \begin{macrocode}
\cs_new_protected:Npn \@@_print_format_auxiv:w
  #1 \q_nil #2 \q_mark #3 \q_nil #4 \q_stop
  {
    \tl_if_blank:nTF {#1}
      { \@@_print_format_auxv:w }
      { \@@_print_format_auxvi:w }
        #1#2 \q_mark #3#4 \q_stop
  }
\cs_new_protected:Npn \@@_print_format_auxv:w
  #1 \q_nil #2 \q_nil #3 \q_nil #4 \q_mark
  #5 \q_nil #6 \q_nil #7 \q_nil #8 \q_stop
  { \@@_print_format_auxvii:w #4 \q_mark #8 \q_stop }
%    \end{macrocode}
%   Sorting out the placement of the uncertainty requires both the model and
%   real data widths, so we store the former to avoiding needing more boxes.
%   It's then just a case of putting the carry-over white space in the right
%   place.
%    \begin{macrocode}
\cs_new_protected:Npn \@@_print_format_auxvi:w
  #1 \q_nil #2 \q_nil #3 \q_nil #4 \q_mark
  #5 \q_nil #6 \q_nil #7 \q_nil #8 \q_stop
  {
    \@@_print_format_box:Nn \l_@@_tmp_box { { } #1#2#3 }
    \dim_set:Nn \l_@@_tmp_dim { \box_wd:N \l_@@_tmp_box }
    \@@_print_format_box:Nn \l_@@_tmp_box { { } #5#6#7 }
    \@@_print_format_after:N \l_@@_align_uncertainty_bool
    \@@_print_format_auxvii:w #4 \q_mark #8 \q_stop
  }
%    \end{macrocode}
%   Finally, we get to the exponent part: the multiplication symbol is
%   |#1| and the number itself is |#2|. The code is almost the same as for
%   uncertainties, which allows a shared auxiliary to be used.
%    \begin{macrocode}
\cs_new_protected:Npn \@@_print_format_auxvii:w
  #1 \q_nil #2 \q_mark #3 \q_nil #4 \q_stop
  {
    \tl_if_blank:nF {#2}
      {
        \@@_print_format_box:Nn \l_@@_tmp_box { { } #1#2 }
        \dim_set:Nn \l_@@_tmp_dim { \box_wd:N \l_@@_tmp_box }
        \@@_print_format_box:Nn \l_@@_tmp_box { { } #3#4 }
        \@@_print_format_after:N \l_@@_align_exponent_bool
      }
  }
%    \end{macrocode}
%   A simple auxiliary to avoid relatively expensive use of the print routine
%   for empty parts.
%    \begin{macrocode}
\cs_new_protected:Npn \@@_print_format_box:Nn #1#2
  {
    \hbox_set:Nn #1
      {
        \tl_if_blank:nF {#2}
          { \siunitx_print_number:n {#2} }
      }
  }
%    \end{macrocode}
%   A common routine for placing material after the decimal marker and
%   \enquote{shuffling}.
%    \begin{macrocode}
\cs_new_protected:Npn \@@_print_format_after:N #1
  {
    \bool_if:NTF #1
      {
        \hbox_set_to_wd:Nnn \l_@@_decimal_box
          {
              \box_wd:N \l_@@_decimal_box
            + \l_@@_carry_dim
            + \box_wd:N \l_@@_tmp_box
          }
          {
            \hbox_unpack:N \l_@@_decimal_box
            \@@_fil:
            \hbox_unpack:N \l_@@_tmp_box
          }
        \dim_set:Nn \l_@@_carry_dim
          {
              \l_@@_tmp_dim
            - \box_wd:N \l_@@_tmp_box
          }
      }
      {
        \hbox_set:Nn \l_@@_decimal_box
          {
            \hbox_unpack:N \l_@@_decimal_box
            \hbox_unpack:N \l_@@_tmp_box
          }
        \dim_add:Nn \l_@@_carry_dim
          {
              \l_@@_tmp_dim
            - \box_wd:N \l_@@_tmp_box
          }
      }
  }
%    \end{macrocode}
%   With no alignment, everything supplied is treated more-or-less the
%   same as \cs{num} (but without the \pkg{xparse} wrapper).
%    \begin{macrocode}
\cs_new_protected:Npn \@@_print_none:nnn #1#2#3
  {
    \use:c { @@_align_ \l_@@_align_number_tl :n }
      {
        #1
        \siunitx_number_format:nN {#2} \l_@@_tmp_tl
        \siunitx_print_number:V \l_@@_tmp_tl
        #3
     }
  }
%    \end{macrocode}
% \end{macro}
% \end{macro}
% \end{macro}
% \end{macro}
% \end{macro}
% \end{macro}
% \end{macro}
% \end{macro}
% \end{macro}
%
% \subsection{Standard settings for module options}
%
% Some of these follow naturally from the point of definition
% (\foreign{e.g.}~boolean variables are always |false| to begin with),
% but for clarity everything is set here.
%    \begin{macrocode}
\keys_set:nn { siunitx }
  {
    table-align-comparator  = true   ,
    table-align-exponent    = true   ,
    table-align-text-after  = true   ,
    table-align-text-before = true   ,
    table-align-uncertainty = true   ,
    table-alignment         = center ,
    table-auto-round        = false  ,
    table-column-width      = 0pt    ,
    table-fixed-width       = false  ,
    table-format            = 2.2    ,
    table-number-alignment  = center ,
    table-text-alignment    = center ,
%    \end{macrocode}
% Out of order as |table-format| sets this implicitly too.
%    \begin{macrocode}
    table-alignment-mode    = marker
  }
%    \end{macrocode}
%
%    \begin{macrocode}
%</package>
%    \end{macrocode}
%
% \end{implementation}
%
% \PrintIndex

% \iffalse meta-comment
%
% File: siunitx-table.dtx Copyright (C) 2016-2018 Joseph Wright
%
% It may be distributed and/or modified under the conditions of the
% LaTeX Project Public License (LPPL), either version 1.3c of this
% license or (at your option) any later version.  The latest version
% of this license is in the file
%
%    https://www.latex-project.org/lppl.txt
%
% This file is part of the "siunitx bundle" (The Work in LPPL)
% and all files in that bundle must be distributed together.
%
% The released version of this bundle is available from CTAN.
%
% -----------------------------------------------------------------------
%
% The development version of the bundle can be found at
%
%    https://github.com/josephwright/siunitx
%
% for those people who are interested.
%
% -----------------------------------------------------------------------
%
%<*driver>
\documentclass{l3doc}
% The next line is needed so that \GetFileInfo will be able to pick up
% version data
\usepackage{siunitx}
\begin{document}
  \DocInput{\jobname.dtx}
\end{document}
%</driver>
% \fi
%
% \GetFileInfo{siunitx.sty}
%
% \title{^^A
%   \pkg{siunitx-table} -- Formatting numbers in tables^^A
%   \thanks{This file describes \fileversion,
%     last revised \filedate.}^^A
% }
%
% \author{^^A
%  Joseph Wright^^A
%  \thanks{^^A
%    E-mail:
%    \href{mailto:joseph.wright@morningstar2.co.uk}
%      {joseph.wright@morningstar2.co.uk}^^A
%   }^^A
% }
%
% \date{Released \filedate}
%
% \maketitle
%
% \begin{documentation}
%
% \end{documentation}
%
% \begin{implementation}
%
% \section{\pkg{siunitx-table} implementation}
%
% Start the \pkg{DocStrip} guards.
%    \begin{macrocode}
%<*package>
%    \end{macrocode}
%
% Identify the internal prefix (\LaTeX3 \pkg{DocStrip} convention): only
% internal material in this \emph{submodule} should be used directly.
%    \begin{macrocode}
%<@@=siunitx_table>
%    \end{macrocode}
%
% \begin{variable}{\l_@@_tmp_box, \l_@@_tmp_tl}
%   Scratch space.
%    \begin{macrocode}
\box_new:N \l_@@_tmp_box
\tl_new:N \l_@@_tmp_tl
%    \end{macrocode}
% \end{variable}
%
% \subsection{Interface functions}
%
% \begin{variable}{\l_@@_text_bool}
%   Used to track that a cell is purely text.
%    \begin{macrocode}
\bool_new:N \l_@@_text_bool
%    \end{macrocode}
% \end{variable}
%
% \begin{macro}{\siunitx_cell_begin:, \siunitx_cell_end:}
%    \begin{macrocode}
\cs_new_protected:Npn \siunitx_cell_begin:
  {
    \bool_set_false:N \l_@@_text_bool
    \bool_if:NTF \l_siunitx_number_parse_bool
      { \@@_collect_begin: }
      { \@@_direct_begin: }
  }
\cs_new_protected:Npn \siunitx_cell_end:
  {
    \bool_if:NF \l_@@_text_bool
      {
        \bool_if:NTF \l_siunitx_number_parse_bool
          { \@@_collect_end: }
          { \@@_direct_end: }
      }
  }
%    \end{macrocode}
% \end{macro}
%
% \subsection{Collecting tokens}
%
% \begin{variable}{\l_@@_collect_tl}
%   Space for tokens.
%    \begin{macrocode}
\tl_new:N \l_@@_collect_tl
%    \end{macrocode}
% \end{variable}
%
% \begin{macro}{\@@_collect_begin:}
% \begin{macro}{\@@_collect_begin:w}
%   Collecting a tabular cell means doing a token-by-token collection.
%   In previous versions of \pkg{siunitx} that was done along with picking
%   out the numerical part, but the code flow ends up very tricky. Here,
%   therefore, we just collect up the unchanged tokens first. The definition of
%   \cs{cr} is used to allow collection of any tokens
%   inserted after the main content when dealing with the last cell of a row:
%   the \enquote{group} around it is needed to avoid issues with the underlying
%   |\halign|. (The approach is based on that in \pkg{collcell}.) Notice that
%   as each cell forms a group there is no need to reset the definition of
%   \cs{cr}. We use an auxiliary to fish out the |\ignorespaces| from the
%   template: that has to go to avoid issues with the peek-ahead code
%   (everything before the |#| needs to be read \emph{before} the Appendix~D
%   trick gets applied). Some packages add additional tokens before the
%   |\ignorespaces|, which are dealt with by the delimited argument.
%    \begin{macrocode}
\cs_new_protected:Npn \@@_collect_begin:
  {
    \tl_clear:N \l_@@_collect_tl
    \if_false: { \fi:
    \cs_set_protected:Npn \cr
      {
        \@@_collect_loop:
        \tex_cr:D
      }
    \if_false: } \fi:
    \@@_collect_begin:w
  }
\cs_new_protected:Npn \@@_collect_begin:w #1 \ignorespaces
  { \@@_collect_loop: #1 }
%    \end{macrocode}
% \end{macro}
% \end{macro}
%
% \begin{macro}{\@@_collect_loop:}
% \begin{macro}{\@@_collect_group:n}
% \begin{macro}{\@@_collect_token:N}
% \begin{macro}{\@@_collect_search:NnF}
% \begin{macro}{\@@_collect_search_aux:NNn}
%   Collecting up the cell content needs a loop: this is done using
%   a |peek| approach as it's most natural. (A slower approach is possible
%   using something like the |\tl_lower_case:n| loop code.) The set of
%   possible tokens is somewhat limited compared to an arbitrary cell
%   (\foreign{cf.}~the approach in \pkg{collcell}): the special cases are
%   pulled out for manual handling. The flexible lookup approach is more-or-less
%   the same idea as in the kernel |case| functions. The |\relax| special case
%   covers the case where |\\| has been expanded in an empty cell.
%    \begin{macrocode}
\cs_new_protected:Npn \@@_collect_loop:
  {
    \peek_catcode_ignore_spaces:NTF \c_group_begin_token
      { \@@_collect_group:n }
      { \@@_collect_token:N }
  }
\cs_new_protected:Npn \@@_collect_group:n #1
  {
    \tl_put_right:Nn \l_@@_collect_tl { {#1} }
    \@@_collect_loop:
  }
\cs_new_protected:Npn \@@_collect_token:N #1
  {
    \@@_collect_search:NnF #1
      {
        \unskip            { \@@_collect_loop: }
        \end               { \tabularnewline \end }
        \relax             { \relax }
        \tabularnewline    { \tabularnewline }
        \siunitx_cell_end: { \siunitx_cell_end: }
      }
      {
        \tl_put_right:Nn \l_@@_collect_tl {#1}
        \@@_collect_loop:
      }
  }
\AtBeginDocument
  {
    \@ifpackageloaded { mdwtab }
      {
        \cs_set_protected:Npn \@@_collect_token:N #1
          {
            \@@_collect_search:NnF #1
              {
                \@maybe@unskip     { \@@_collect_loop: }
                \tab@setcr         { \@@_collect_loop: }
                \unskip            { \@@_collect_loop: }
                \end               { \tabularnewline \end }
                \relax             { \relax }
                \tabularnewline    { \tabularnewline }
                \siunitx_cell_end: { \siunitx_cell_end: }
              }
              {
                \tl_put_right:Nn \l_@@_collect_tl {#1}
                \@@_collect_loop:
              }
          }
      }
      { }
  }
\cs_new_protected:Npn \@@_collect_search:NnF #1#2#3
  {
    \@@_collect_search_aux:NNn #1
      #2
      #1 {#3}
    \q_stop
  }
\cs_new_protected:Npn \@@_collect_search_aux:NNn #1#2#3
  {
    \token_if_eq_meaning:NNTF #1 #2
      { \use_i_delimit_by_q_stop:nw {#3} }
      { \@@_collect_search_aux:NNn #1 }
  }
%    \end{macrocode}
% \end{macro}
% \end{macro}
% \end{macro}
% \end{macro}
% \end{macro}
%
% \subsection{Separating collected material}
%
% The input needs to be divided into numerical tokens and those which appear
% before and after them. This needs a second loop and validation.
%
% \begin{variable}{\l_@@_pre_tl, \l_@@_number_tl, \l_@@_post_tl}
%   Space for tokens.
%    \begin{macrocode}
\tl_new:N \l_@@_pre_tl
\tl_new:N \l_@@_number_tl
\tl_new:N \l_@@_post_tl
%    \end{macrocode}
% \end{variable}
%
% \begin{macro}{\@@_collect_end:}
%   At the end of the cell, expand all of the content as far as possible then
%   split it up into numerical and non-numerical parts.
%    \begin{macrocode}
\cs_new_protected:Npn \@@_collect_end:
  {
    \protected@edef \l_@@_collect_tl
      { \l_@@_collect_tl }
    \tl_clear:N \l_@@_pre_tl
    \tl_clear:N \l_@@_number_tl
    \tl_clear:N \l_@@_post_tl
    \exp_after:wN \@@_split_loop: \l_@@_collect_tl
      \q_recursion_tail \q_recursion_stop
    \@@_split_tidy:N \l_@@_pre_tl
    \@@_split_tidy:N \l_@@_post_tl
    \tl_if_empty:NTF \l_@@_number_tl
      { \@@_print_text:V \l_@@_pre_tl }
      {
        \@@_print:VVV
          \l_@@_pre_tl
          \l_@@_number_tl
          \l_@@_post_tl
      }
  }
%    \end{macrocode}
% \end{macro}
%
% \begin{macro}{\@@_split_loop:}
% \begin{macro}{\@@_split_group:n}
% \begin{macro}{\@@_split_token:N}
%   Splitting into parts uses the fact that numbers cannot contain groups
%   and that we can track where we are up to based on the content of the
%   token lists.
%    \begin{macrocode}
\cs_new_protected:Npn \@@_split_loop:
  {
    \peek_catcode_ignore_spaces:NTF \c_group_begin_token
      { \@@_split_group:n }
      { \@@_split_token:N }
  }
\cs_new_protected:Npn \@@_split_group:n #1
  {
    \tl_if_empty:NTF \l_@@_number_tl
      { \tl_put_right:Nn \l_@@_pre_tl { {#1} } }
      { \tl_put_right:Nn \l_@@_post_tl { {#1} } }
    \@@_split_loop:
  }
\cs_new_protected:Npn \@@_split_token:N #1
  {
    \quark_if_recursion_tail_stop:N #1
    \tl_if_empty:NTF \l_@@_post_tl
      {
        \siunitx_if_number_token:NTF #1
          { \tl_put_right:Nn \l_@@_number_tl {#1} }
          {
            \tl_if_empty:NTF \l_@@_number_tl
              { \tl_put_right:Nn \l_@@_pre_tl {#1} }
              { \tl_put_right:Nn \l_@@_post_tl {#1} }
          }
      }
      { \tl_put_right:Nn \l_@@_post_tl {#1} }
    \@@_split_loop:
  }
%    \end{macrocode}
% \end{macro}
% \end{macro}
% \end{macro}
%
% \begin{macro}{\@@_split_tidy:N}
% \begin{macro}{\@@_split_tidy:Nn, \@@_split_tidy:NV}
%   A quick test for the entire content being surrounded by a set of braces:
%   rather than look explicitly, use the fact that a string comparison can
%   detect the same thing. The auxiliary is needed to avoid having to go
%   \foreign{via} a |:D| function (for the expansion behaviour).
%    \begin{macrocode}
\cs_new_protected:Npn \@@_split_tidy:N #1
  {
    \tl_if_empty:NF #1
      { \@@_split_tidy:NV #1 #1 }
  }
\cs_new_protected:Npn \@@_split_tidy:Nn #1#2
  {
    \str_if_eq:onT { \exp_after:wN { \use:n #2 } } {#2}
      { \tl_set:No #1 { \use:n #2 } }
  }
\cs_generate_variant:Nn \@@_split_tidy:Nn { NV }
%    \end{macrocode}
% \end{macro}
% \end{macro}
%
% \subsection{Printing numbers in cells: spacing}
%
% Getting the general alignment correct in tables is made more complex than one
% would like by the \pkg{colortbl} package. In the original \LaTeXe{}
% definition, cell material is centred by a construction of the (primitive)
% form
% \begin{verbatim}
%   \hfil
%   #
%   \hfil
% \end{verbatim}
% which only uses \texttt{fil} stretch. That is altered by \pkg{colortbl} to
% broadly
% \begin{verbatim}
%   \hskip 0pt plus 0.5fill
%   \kern 0pt
%   #
%   \hskip 0pt plus 0.5fill
% \end{verbatim}
% which means there is \texttt{fill} stretch to worry about and the kern as
% well.
%
% \begin{macro}{\@@_skip:n}
%   To prevent combination of skips, a kern is inserted after each one.
%   This is best handled as a short auxiliary.
%    \begin{macrocode}
\cs_new_protected:Npn \@@_skip:n #1
  {
    \skip_horizontal:n {#1}
    \tex_kern:D \c_zero_skip
  }
%    \end{macrocode}
% \end{macro}
%
% \begin{variable}{\l_@@_column_width_dim, \l_@@_fixed_width_bool}
%   Settings which apply to aligned columns in general.
%    \begin{macrocode}
\keys_define:nn { siunitx }
  {
    table-column-width .dim_set:N =
      \l_@@_column_width_dim ,
    table-fixed-width .bool_set:N =
      \l_@@_fixed_width_bool
  }
%    \end{macrocode}
% \end{variable}
%
% \begin{macro}{\@@_align_center:n, \@@_align_left:n, \@@_align_right:n}
% \begin{macro}{\@@_align_auxi:nn, \@@_align_auxii:nn}
%   The beginning and end of each table cell have to adjust the position of
%   the content using glue. When \pkg{colortbl} is loaded the glue is done in
%   two parts: one for our positioning and one to explicitly override that from
%   the package. Using a two-step auxiliary chain avoids needing to repeat any
%   code and the impact of the extra expansion should be trivial.
%    \begin{macrocode}
\cs_new_protected:Npn \@@_align_center:n #1
  { \@@_align_auxi:nn {#1} { 0pt~plus~0.5fill } }
\cs_new_protected:Npn \@@_align_left:n #1
  { \@@_align_auxi:nn {#1} { 0pt } }
\cs_new_protected:Npn \@@_align_right:n #1
  { \@@_align_auxi:nn {#1} { 0pt~plus~1fill } }
\cs_new_protected:Npn \@@_align_auxi:nn #1#2
  {
    \bool_if:NTF \l_@@_fixed_width_bool
      { \hbox_to_wd:nn \l_@@_column_width_dim }
      { \use:n }
        {
          \@@_skip:n {#2}
          #1
          \@@_skip:n { 0pt~plus~1fill - #2 }
        }
  }
\AtBeginDocument
  {
    \@ifpackageloaded { colortbl }
      {
        \cs_new_eq:NN
          \@@_align_auxii:nn
          \@@_align_auxi:nn
        \cs_set_protected:Npn \@@_align_auxi:nn #1#2
          {
            \@@_skip:n{ 0pt~plus~-0.5fill }
            \@@_align_auxii:nn {#1} {#2}
            \@@_skip:n { 0pt~plus~-0.5fill }
          }
      }
      { }
  }
%    \end{macrocode}
% \end{macro}
% \end{macro}
%
% \subsection{Printing just text}
%
% In cases where there is no numerical part, \pkg{siunitx} allows alignment
% of the \enquote{escaped} text independent of the underlying column type.
%
% \begin{variable}{\l_@@_align_text_tl}
%   Alignment is handled using a |tl| as this allows a fast lookup at the
%   point of use.
%    \begin{macrocode}
\keys_define:nn { siunitx }
  {
    table-text-alignment .choices:nn =
      { center , left , right }
      { \tl_set:Nn \l_@@_align_text_tl {#1} } ,
  }
\tl_new:N \l_@@_align_text_tl
%    \end{macrocode}
% \end{variable}
%
% \begin{macro}{\@@_print_text:n, \@@_print_text:V}
%   Printing escaped text is easy: just place it in correctly in the
%   column.
%    \begin{macrocode}
\cs_new_protected:Npn \@@_print_text:n #1
  {
    \bool_set_true:N \l_@@_text_bool
    \use:c { @@_align_ \l_@@_align_text_tl :n } {#1}
  }
\cs_generate_variant:Nn \@@_print_text:n { V }
%    \end{macrocode}
% \end{macro}
%
% \subsection{Number alignment: core ideas}
%
% \begin{variable}{\l_@@_align_mode_tl, \l_@@_align_number_tl, \l_@@_format_tl}
%    \begin{macrocode}
\keys_define:nn { siunitx }
  {
    table-alignment-mode .choices:nn =
      { none , format , marker }
      { \tl_set_eq:NN \l_@@_align_mode_tl \l_keys_choice_tl } ,
    table-format .code:n =
      {
        \@@_generate_model:n {#1}
        \tl_set:Nn \l_@@_align_mode_tl { format }
      } ,
    table-number-alignment .choices:nn =
      { center , left , right }
      { \tl_set_eq:NN \l_@@_align_number_tl \l_keys_choice_tl } ,
  }
\tl_new:N \l_@@_align_mode_tl
\tl_new:N \l_@@_align_number_tl
%    \end{macrocode}
% \end{variable}
%
% \begin{variable}{\l_@@_format_tl, \l_@@_model_tl}
%   The input and output versions of the model entry in a table.
%    \begin{macrocode}
\tl_new:N \l_@@_format_tl
\tl_new:N \l_@@_model_tl
%    \end{macrocode}
% \end{variable}
%
% \begin{macro}{\@@_generate_model:n}
% \begin{macro}{\@@_generate_model:nNnnnNn}
% \begin{macro}{\@@_generate_model_S:nw}
%   Creating a model for a table at this stage means parsing the format and
%   converting that to an appropriate model. Things are quite straight-forward
%   other than the uncertainty part. At this stage there is no point in
%   formatting the model: that has to happen at point-of-use.
%    \begin{macrocode}
\cs_new_protected:Npn \@@_generate_model:n #1
  {
    \group_begin:
      \bool_set_true:N \l_siunitx_number_parse_bool
      \siunitx_number_parse:nN {#1} \l_@@_format_tl
    \exp_args:NNNV \group_end:
    \tl_set:Nn \l_@@_format_tl \l_@@_format_tl
    \tl_if_empty:NF \l_@@_format_tl
      {
        \exp_after:wN \@@_generate_model:nNnnnNn
          \l_@@_format_tl
      }
  }
\cs_new_protected:Npn \@@_generate_model:nNnnnNn #1#2#3#4#5#6#7
  {
    \tl_set:Nx \l_@@_model_tl
      {
        \exp_not:n { {#1} #2 }
        { \prg_replicate:nn {#3} { 8 } }
        { \prg_replicate:nn { 0 #4 } { 8 } }
        {
          \tl_if_blank:nF {#5}
            {
              \use:c { @@_generate_model_ \tl_head:n {#5} :nw }
                #5
            }
        }
        \exp_not:N #6
        {
          \int_compare:nNnTF {#7} = 0
            { 0 }
            { \prg_replicate:nn {#7} { 8 } }
        }
      }
  }
\cs_new:Npn \@@_generate_model_S:nw #1#2
  { { S } { \prg_replicate:nn {#2} { 8 } } }
%    \end{macrocode}
% \end{macro}
% \end{macro}
% \end{macro}
%
% \subsection{Directly printing without collection}
%
% Collecting the number allows for various effects but is not as fast
% as simply aligning on the first token that is a decimal marker. The
% strategy here is that used by \pkg{dcolumn}.
%
% \begin{variable}{\l_@@_integer_box, \l_@@_decimal_box}
%   Boxes for the content before and after the decimal marker.
%    \begin{macrocode}
\box_new:N \l_@@_integer_box
\box_new:N \l_@@_decimal_box
%    \end{macrocode}
% \end{variable}
%
% \begin{macro}{\@@_fil:}
%    A primitive renamed.
%    \begin{macrocode}
\cs_new_eq:NN \@@_fil: \tex_hfil:D
%    \end{macrocode}
% \end{macro}
%
% \begin{macro}{\@@_direct_begin:}
% \begin{macro}{\@@_direct_begin:w}
% \begin{macro}{\@@_direct_end:}
% \begin{macro}
%   {
%     \@@_direct_marker:        ,
%     \@@_direct_marker_switch: ,
%     \@@_direct_marker_end:
%   }
% \begin{macro}{\@@_direct_format:}
% \begin{macro}{\@@_direct_format:w}
% \begin{macro}
%   {
%     \@@_direct_format_switch: ,
%     \@@_direct_format_end:
%   }
% \begin{macro}{\@@_direct_none:, \@@_direct_none_end:}
%   After removing the |\ignorespaces| at the start of the cell (see comments
%   for \cs{@@_collect_begin:N}), check to see  if there is a |{| and branch
%   as appropriate.
%    \begin{macrocode}
\cs_new_protected:Npn \@@_direct_begin:
  { \@@_direct_begin:w }
\cs_new_protected:Npn \@@_direct_begin:w #1 \ignorespaces
  {
    #1
    \peek_catcode_ignore_spaces:NTF \c_group_begin_token
      { \@@_print_text:n }
      {
        \m@th
        \use:c { @@_direct_ \l_@@_align_mode_tl : }
      }
  }
\cs_new_protected:Npn \@@_direct_end:
  { \use:c { @@_direct_ \l_@@_align_mode_tl _end: } }
%    \end{macrocode}
%   When centring the content about a decimal marker, the trick is
%   to collect everything into two boxes and then compare the sizes.
%   As we are always in math mode, we can use a math active token
%   to make the switch. The up-front setting of the |decimal| box deals
%   with the case where there is no decimal part.
%    \begin{macrocode}
\cs_new_protected:Npn \@@_direct_marker:
  {
    \hbox_set:Nn \l_@@_tmp_box
      { \ensuremath { \mathord { \l_siunitx_number_output_decimal_tl } } }
    \hbox_set_to_wd:Nnn \l_@@_decimal_box
      { \box_wd:N \l_@@_tmp_box }
      { \@@_fil: }
    \hbox_set:Nw \l_@@_integer_box
      \c_math_toggle_token
      \tl_map_inline:Nn \l_siunitx_number_input_decimal_tl
        {
          \char_set_active_eq:NN ##1 \@@_direct_marker_switch:
          \char_set_mathcode:nn { `##1 } { "8000 }
        }
  }
\cs_new_protected:Npn \@@_direct_marker_switch:
  {
      \c_math_toggle_token
    \hbox_set_end:
    \hbox_set:Nw \l_@@_decimal_box
      \c_math_toggle_token
      \l_@@_output_align_tl
  }
\cs_new_protected:Npn \@@_direct_marker_end:
  {
      \c_math_toggle_token
    \hbox_set_end:
    \dim_compare:nNnTF
      { \box_wd:N \l_@@_integer_box }
        > { \box_wd:N \l_@@_decimal_box }
      {
        \hbox_set_to_wd:Nnn \l_@@_decimal_box
          { \box_wd:N \l_@@_integer_box }
          {
            \hbox_unpack:N \l_@@_decimal_box
            \@@_fil:
          }
      }
      {
        \hbox_set_to_wd:Nnn \l_@@_integer_box
          { \box_wd:N \l_@@_decimal_box }
          {
            \@@_fil:
            \hbox_unpack:N \l_@@_integer_box
          }
      }
    \box_use_drop:N \l_@@_integer_box
    \box_use_drop:N \l_@@_decimal_box
  }
%    \end{macrocode}
%   For the version where there is space reserved, first format and decompose
%   that, then create appropriately-sized boxes. The token used there is
%   |\relax| as it can be passed through to the output: only have to
%   find \enquote{just enough} of the number to do the work.
%    \begin{macrocode}
\cs_new_protected:Npn \@@_direct_format:
  {
    \tl_set:Nx \l_@@_tmp_tl
      { \siunitx_number_format:NN \l_@@_model_tl \relax }
    \exp_after:wN \@@_direct_format_aux:w
      \l_@@_tmp_tl \q_stop
  }
\cs_new_protected:Npn \@@_direct_format_aux:w
  #1 \relax #2 \relax #3 \q_stop
  {
    \hbox_set:Nn \l_@@_tmp_box { \ensuremath {#3} }
    \hbox_set_to_wd:Nnn \l_@@_decimal_box
      { \box_wd:N \l_@@_tmp_box }
      { \@@_fil: }
    \hbox_set:Nn \l_@@_tmp_box { \ensuremath { #1#2 } }
    \hbox_set_to_wd:Nnw \l_@@_integer_box
      { \box_wd:N \l_@@_tmp_box }
      \c_math_toggle_token
      \tl_map_inline:Nn \l_siunitx_number_input_decimal_tl
        {
          \char_set_active_eq:NN ##1 \@@_direct_format_switch:
          \char_set_mathcode:nn { `##1 } { "8000 }
        }
      \@@_fil:
  }
\cs_new_protected:Npn \@@_direct_format_switch:
  {
      \c_math_toggle_token
    \hbox_set_end:
    \hbox_set_to_wd:Nnw \l_@@_decimal_box
      { \box_wd:N \l_@@_decimal_box }
      \c_math_toggle_token
      \mathord { \l_siunitx_number_output_decimal_tl }
  }
\cs_new_protected:Npn \@@_direct_format_end:
  {
      \c_math_toggle_token
      \@@_fil:
    \hbox_set_end:
    \use:c { @@_align_ \l_@@_align_number_tl :n }
      {
        \box_use_drop:N \l_@@_integer_box
        \box_use_drop:N \l_@@_decimal_box
      }
  }
%    \end{macrocode}
%   No parsing and no alignment is easy.
%    \begin{macrocode}
\cs_new_protected:Npn \@@_direct_none: { \c_math_toggle_token }
\cs_new_protected:Npn \@@_direct_none_end: { \c_math_toggle_token }
%    \end{macrocode}
% \end{macro}
% \end{macro}
% \end{macro}
% \end{macro}
% \end{macro}
% \end{macro}
% \end{macro}
% \end{macro}
%
% \subsection{Printing numbers in cells: main functions}
%
% \begin{variable}
%   {
%     \l_@@_align_comparator_bool  ,
%     \l_@@_align_exponent_bool    ,
%     \l_@@_align_uncertainty_bool
%   }
%   Alignment is handled using a |tl| as this allows a fast lookup at the
%   point of use.
%    \begin{macrocode}
\keys_define:nn { siunitx }
  {
    table-align-comparator .bool_set:N =
      \l_@@_align_comparator_bool ,
    table-align-exponent .bool_set:N =
      \l_@@_align_exponent_bool ,
    table-align-uncertainty .bool_set:N =
      \l_@@_align_uncertainty_bool
  }
%    \end{macrocode}
% \end{variable}
%
% \begin{macro}{\@@_print:nnn, \@@_print:VVV}
% \begin{macro}{\@@_print_format:nnn, \@@_print_marker:nnn, \@@_print_none:nnn}
%    \begin{macrocode}
\cs_new_protected:Npn \@@_print:nnn #1#2#3
  {
    \use:c { @@_align_ \l_@@_align_number_tl :n }
      {
        \use:c { @@_print_ \l_@@_align_mode_tl :nnn }
          {#1} {#2} {#3}
      }
  }
\cs_generate_variant:Nn \@@_print:nnn { VVV }
\cs_new_protected:Npn \@@_print_marker:nnn #1#2#3
  {
  }
\cs_new_protected:Npn \@@_print_format:nnn #1#2#3
  {
  }
\cs_new_protected:Npn \@@_print_none:nnn #1#2#3
  {
    #1
    \siunitx_number_format:nN {#2} \l_@@_tmp_tl
    \siunitx_print:nV { number } \l_@@_tmp_tl
    #3
  }
%    \end{macrocode}
% \end{macro}
% \end{macro}
%
% \subsection{Standard settings for module options}
%
% Some of these follow naturally from the point of definition
% (\foreign{e.g.}~boolean variables are always |false| to begin with),
% but for clarity everything is set here.
%    \begin{macrocode}
\keys_set:nn { siunitx  }
  {
    table-align-comparator  = false  ,
    table-align-exponent    = false  ,
    table-align-uncertainty = false  ,
    table-alignment-mode    = marker ,
    table-column-width      = 0pt    ,
    table-fixed-width       = false  ,
    table-format            = 2.2    ,
    table-number-alignment  = center ,
    table-text-alignment    = center
  }
%    \end{macrocode}
%
%
%    \begin{macrocode}
%</package>
%    \end{macrocode}
%
% \end{implementation}
%
% \PrintIndex
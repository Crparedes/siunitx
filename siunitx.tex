\iffalse meta-comment

File: siunitx.tex Copyright (C) 2014-2021 Joseph Wright

It may be distributed and/or modified under the conditions of the
LaTeX Project Public License (LPPL), either version 1.3c of this
license or (at your option) any later version.  The latest version
of this license is in the file

   https://www.latex-project.org/lppl.txt

This file is part of the "siunitx bundle" (The Work in LPPL)
and all files in that bundle must be distributed together.

The released version of this bundle is available from CTAN.

-----------------------------------------------------------------------

The development version of the bundle can be found at

   https://github.com/josephwright/siunitx

for those people who are interested.

-----------------------------------------------------------------------

\fi

\documentclass{l3doc}

% The next line is needed so that \GetFileInfo will be able to pick up
% version data (quite apart from making the demos work).
\usepackage{siunitx}
\DeclareSIUnit\noop{\relax} % For printing prefixes
\DeclareSIPower\quartic\tothefourth{4} % For demos
\DeclareSIUnit\KWH{kWh}
\DeclareSIQualifier\polymer{pol}
\DeclareSIQualifier\catalyst{cat}

% Commands for this document
\usepackage{xparse}
\NewDocumentCommand\acro{m}{\textsc{#1}}
\NewDocumentCommand\email{m}{\href{mailto:#1}{\nolinkurl{#1}}}
\NewDocumentCommand\ext{m}{\texttt{.#1}}
\NewDocumentCommand\foreign{m}{\textit{#1}}
\NewDocumentCommand\opt{m}{\texttt{#1}}
% Tidy up the above in bookmarks
\makeatletter
\pdfstringdefDisableCommands{%
  \let\acro\@firstofone
  \let\ext\@firstofone
  \let\foreign\@firstofone
  \let\opt\@firstofone
}
\makeatother

\NewDocumentCommand\DescribePrefix{m}{%
  #1 &
  \cs{#1} &
  \unit{\csname #1\endcsname\noop}
}
\NewDocumentCommand\DescribeUnit{O{#2}m}{%
  #1 &
  \cs{#2} &
  \unit{\csname #2\endcsname}
}

% For creating code demonstrations
% This needs access to some code-level interfaces in listings
\usepackage{listings}
\makeatletter
\lst@RequireAspects{writefile}
\newsavebox\LaTeXdemo@box
\lstnewenvironment{LaTeXdemo}[1][code and example]
  {%
    \global\let\lst@intname\@empty
    \edef\LaTeXdemo@end{%
      \expandafter\noexpand\csname LaTeXdemo@@#1@end\endcsname
    }%
    \@nameuse{LaTeXdemo@@#1}%
  }
  {\LaTeXdemo@end}
\newcommand\LaTeXdemo@new[3]{%
  \@namedef{LaTeXdemo@@#1}{#2}%
  \@namedef{LaTeXdemo@@#1@end}{#3}%
}
\newcommand*\LaTeXdemo@common{%
  \setkeys{lst}
    {%
       basicstyle   = \small\ttfamily,
       basewidth    = 0.51em,
       gobble       = 2,
       language     = [LaTeX]{TeX},
    }%
}
\newcount\LaTeXdemo@count
\newcommand*\LaTeXdemo@input{%
  \catcode`\^^M = 10\relax
  \input{\jobname-\number\LaTeXdemo@count.tmp}%
}
\LaTeXdemo@new{code and example}{%
  \setbox\LaTeXdemo@box=\hbox\bgroup
    \global\advance\LaTeXdemo@count by 1 %
    \lst@BeginAlsoWriteFile{\jobname-\number\LaTeXdemo@count.tmp}%
    \LaTeXdemo@common
}{%
    \lst@EndWriteFile
  \egroup
  \begin{center}
    \ifdim\wd\LaTeXdemo@box > 0.48\linewidth
      \begin{minipage}{\linewidth}
        \usebox\LaTeXdemo@box
      \end{minipage}%
      \par
      \begin{minipage}{\linewidth}
        \LaTeXdemo@input
      \end{minipage}
    \else
      \begin{minipage}{0.48\linewidth}
        \LaTeXdemo@input
      \end{minipage}%
      \hfil
      \begin{minipage}{0.48\linewidth}
        \usebox\LaTeXdemo@box
      \end{minipage}%
    \fi
  \end{center}
}
\LaTeXdemo@new{code and float}{%
  \global\advance\LaTeXdemo@count by 1 %
  \lst@BeginAlsoWriteFile{\jobname-\number\LaTeXdemo@count.tmp}%
  \LaTeXdemo@common
}{%
  \lst@EndWriteFile
  \LaTeXdemo@input
}
\LaTeXdemo@new{code only}{\LaTeXdemo@common}{}
\makeatother

% For demos
\usepackage[french,german,spanish,UKenglish]{babel}
\AtBeginDocument{\shorthandoff{:<>}}
\usepackage{translations}
\usepackage{cancel}
\usepackage{collcell}
\usepackage{sansmath}
\newlength{\mylength}

% For table demos
\usepackage{datatool}
\usepackage{multirow}
\usepackage[table]{xcolor}
% Has to come after xcolor
\usepackage{pgfplots}
\pgfplotsset{compat = 1.16, compat/show suggested version = false}
\usepackage{xfp}

% For the extra-long tables
\usepackage{xtab}

% Design changes
\usepackage{caption}
\makeatletter
\edef\@floatboxreset{%
  \unexpanded\expandafter{\@floatboxreset}%
  \noexpand\centering
}
\makeatother
\usepackage[osf]{mathpazo}

\hypersetup{hidelinks}

\begin{document}

\GetFileInfo{siunitx.sty}

\title{%
  \pkg{siunitx} -- A comprehensive (SI) units package%
  \thanks{This file describes \fileversion,
    last revised \filedate.}%
}

\author{%
  Joseph Wright%
  \thanks{%
    E-mail:
    \href{mailto:joseph.wright@morningstar2.co.uk}
      {joseph.wright@morningstar2.co.uk}%
  }%
}

\date{Released \filedate}

\maketitle

\tableofcontents

\begin{abstract}
  Physical quantities have both numbers and units, and each physical quantity
  should be expressed as the product of a number and a unit. Typesetting
  physical quantities requires care to ensure that the combined mathematical
  meaning of the number--unit combination is clear. In particular, the SI units
  system lays down a consistent set of units with rules on how these are to be
  used. However, different countries and publishers have differing conventions
  on the exact appearance of numbers (and units). The \pkg{siunitx} package
  provides a set of tools for authors to typeset quantities in a consistent
  way. The package has an extended set of configuration options which make it
  possible to follow varying typographic conventions with the same input
  syntax. The package includes automated processing of numbers and units, and
  the ability to control tabular alignment of numbers.
\end{abstract}

\begin{documentation}

\section{Introduction}

The correct application of units of measurement is very important in technical
applications. For this reason, carefully-crafted definitions of a coherent
units system have been laid down by the \foreign{Conférence Génrale des
Poids et Mesures} (CGPM): this has resulted in the \foreign{Système
International d'Unités}~(SI). At the same time, typographic conventions for
correctly displaying both numbers and units exist to ensure that no loss of
meaning occurs in printed matter.

The \pkg{siunitx} package aims to provide a unified method for \LaTeX{} users
to typeset numbers and units correctly and easily. The design philosophy of
\pkg{siunitx} is to follow the agreed rules by default, but to allow variation
through option settings. In this way, users can use \pkg{siunitx} to follow the
requirements of publishers, co-authors, universities, \foreign{etc}.\ without
needing to alter the input at all.

\section{Using the \pkg{siunitx} package}

\subsection{Numbers}

\begin{function}{\num}
  \begin{syntax}
    \cs{num}\oarg{options}\marg{number}
  \end{syntax}
\end{function}
Numbers are automatically formatted by the \cs{num} macro. This takes one
optional argument, \meta{options}, and one mandatory one, \meta{number}. The
contents of \meta{number} are automatically formatted. The formatter removes
both \enquote{soft} (\verb*| |) and \enquote{hard} spaces (|\,| and |~|),
automatically identifies exponents (as standard marked using |e|, |E|, |d| or
|D|) and adds the appropriate spacing of large numbers. With the standard
settings a leading zero is added before a decimal marker, if needed: both |.|
and "," are recognised as decimal markers.
\begin{LaTeXdemo}
  \num{123}     \\
  \num{1234}    \\
  \num{12345}   \\
  \num{0.123}   \\
  \num{0,1234}  \\
  \num{.12345}  \\
  \num{3.45d-4} \\
  \num{-e10}
\end{LaTeXdemo}

Note that numbers are parsed before typesetting, which does have a performance
overhead (only obvious with very large amounts of numerical input). The parser
understands a range of input syntaxes, as demonstrated above.

\begin{function}{\numlist}
  \begin{syntax}
    \cs{numlist}\oarg{options}\marg{numbers}
  \end{syntax}
\end{function}
Lists of numbers may be processed using the \cs{numlist} function. Each
\meta{number} is given within the list of \meta{numbers} within a brace pair,
as the list can have a flexible length.
\begin{LaTeXdemo}
  \numlist{10;30;50;70}
\end{LaTeXdemo}

\begin{function}{\numproduct}
  \begin{syntax}
    \cs{numproduct}\oarg{options}\marg{numbers}
  \end{syntax}
\end{function}
Runs of products of numbers may be inserted using the \cs{numproduct} function.
This acts in the same way as \cs{num}, but inserts either a symbol or phrase
between the entries. The latter should be separated by |x| tokens.
\begin{LaTeXdemo}
  \numproduct{10 x 30}
\end{LaTeXdemo}

\begin{function}{\numrange}
  \begin{syntax}
    \cs{numrange}\oarg{options}\marg{number1}\marg{number2}
  \end{syntax}
\end{function}
Simple ranges of numbers can be handled using the \cs{numrange} function. This
acts in the same way as \cs{num}, but inserts a phrase or other text between
the two entries.
\begin{LaTeXdemo}
  \numrange{10}{30}
\end{LaTeXdemo}

\subsection{Angles}

\begin{function}{\ang}
  \begin{syntax}
    \cs{ang}\oarg{options}\marg{angle}
  \end{syntax}
\end{function}
Angles can be typeset using the \cs{ang} command. The \meta{angle} can be given
either as a decimal number or as a semi-colon separated list of degrees,
minutes and seconds, which is called \enquote{arc format} in this document. The
numbers which make up an angle are processed using the same system as other
numbers.
\begin{LaTeXdemo}
  \ang{10}    \\
  \ang{12.3}  \\
  \ang{4,5}   \\
  \ang{1;2;3} \\
  \ang{;;1}   \\
  \ang{+10;;} \\
  \ang{-0;1;}
\end{LaTeXdemo}

\subsection{Units}

\begin{function}{\unit}
  \begin{syntax}
    \cs{unit}\oarg{options}\marg{unit}
  \end{syntax}
\end{function}
The symbol for a unit can be typeset using the \cs{unit} macro: this provides
full control over output format for the unit. Like the \cs{num} macro,
\cs{unit} takes one optional and one mandatory argument. The unit formatting
system can accept two types of input. When the \meta{unit} contains literal
items (for example letters or numbers) then \pkg{siunitx} converts |.| and |~|
into inter-unit product and correctly positions sub- and superscripts specified
using |_| and |^|. The formatting methods will work with both math and text
mode.
\begin{LaTeXdemo}
  \unit{kg.m/s^2} \\
  \unit{g_{polymer}~mol_{cat}.s^{-1}}
\end{LaTeXdemo}

The second operation mode for the \cs{unit} macro is an \enquote{interpreted}
system, Here, each unit, SI multiple prefix and power is given a macro name.
These are entered in a method very similar to the reading of the unit name in
English.
\begin{LaTeXdemo}
  \unit{\kilo\gram\metre\per\square\second} \\
  \unit{\gram\per\cubic\centi\metre}        \\
  \unit{\square\volt\cubic\lumen\per\farad} \\
  \unit{\metre\squared\per\gray\cubic\lux}  \\
  \unit{\henry\second}
\end{LaTeXdemo}

On its own, this is less convenient than the direct method, although it does
use meaning rather than appearance for input. However, the package allows you
to define new unit macros; a large number of pre-defined abbreviations are also
supplied. More importantly, by defining macros for units, instead of literal
input, new functionality is made available. By altering the settings used by
the package, the same input can yield a variety of different output formats.
For example, the \cs{per} macro can give reciprocal powers, slashes or be used
to construct units as fractions.

\begin{function}{\qty}
  \begin{syntax}
    \cs{qty}\oarg{options}\marg{number}\marg{unit}
  \end{syntax}
\end{function}
Very often, numbers and units are given together. Formally, the value of a
quantity is the product of the number and the unit, the space being regarded as
a multiplication sign~\cite{SI:5.3.3}. The \cs{qty} macro combines the
functionality of \cs{num} and \cs{unit}, and makes this both possible and easy.
The \meta{number} and \meta{unit} arguments work exactly like those for the
\cs{num} and \cs{unit} macros, respectively.
\begin{LaTeXdemo}
  \qty[mode = text]{1.23}{J.mol^{-1}.K^{-1}}          \\
  \qty{.23e7}{\candela}                               \\
  \qty[per-mode = symbol]{1.99}{\per\kilogram}        \\
  \qty[per-mode = fraction]{1,345}{\coulomb\per\mole}
\end{LaTeXdemo}

It is possible to set up the unit macros to be available outside of the
\cs{qty} and \cs{unit} functions. This is not the standard behaviour as there
is the risk of name clashes (for example, \cs{bar} is used by other packages,
and several packages define \cs{degree}). Full details of using \enquote{stand
alone} units are found in \ref{sec:units:creating}.

\subsection{The unit macros}

The package always defines the basic set of SI units with macro names. This
includes the base SI units, the derived units with special names and the
prefixes. A small number of powers are also given pre-defined names. Full
details of units in the SI are available on-line~\cite{BIPM}.

The seven base SI units are always defined (Table~\ref{tab:unit:base}). In
addition, the macro \cs{meter} is available as an alias for \cs{metre}, for
users of US spellings. The full details of the base units are given in the SI
Brochure~\cite{SI:2.1}.
\begin{table}
  \caption{SI base units.}
  \label{tab:unit:base}
  \begin{tabular}{@{}lll@{}}
    \toprule
      Unit & Command & Symbol \\
    \midrule
      \DescribeUnit{ampere}   \\
      \DescribeUnit{candela}  \\
      \DescribeUnit{kelvin}   \\
      \DescribeUnit{kilogram} \\
      \DescribeUnit{metre}    \\
      \DescribeUnit{mole}     \\
      \DescribeUnit{second}   \\
    \bottomrule
  \end{tabular}
\end{table}

The SI also lists a number of units which have special names and
symbols~\cite{SI:2.2.2}: these are listed in Table~\ref{tab:unit:derived}.
\begin{table}
  \caption{Coherent derived units in the SI with special names and
    symbols.}
  \label{tab:unit:derived}
  \begin{tabular}{@{}lll>{\qquad}lll@{}}
    \toprule
      Unit & Command & Symbol & Unit & Command & Symbol \\
    \midrule
      \DescribeUnit{becquerel} &
      \DescribeUnit{newton}    \\
      \DescribeUnit[degree Celsius]{degreeCelsius} &
      \DescribeUnit{ohm}       \\
      \DescribeUnit{coulomb}   &
      \DescribeUnit{pascal}    \\
      \DescribeUnit{farad}     &
      \DescribeUnit{radian}    \\
      \DescribeUnit{gray}      &
      \DescribeUnit{siemens}   \\
      \DescribeUnit{hertz}     &
      \DescribeUnit{sievert}   \\
      \DescribeUnit{henry}     &
      \DescribeUnit{steradian} \\
      \DescribeUnit{joule}     &
      \DescribeUnit{tesla}     \\
      \DescribeUnit{lumen}     &
      \DescribeUnit{volt}      \\
      \DescribeUnit{katal}     &
      \DescribeUnit{watt}      \\
      \DescribeUnit{lux}       &
      \DescribeUnit{weber}     \\
    \bottomrule
  \end{tabular}
\end{table}

In addition to the official SI units, \pkg{siunitx} also provides macros for a
number of units which are accepted for use in the SI although they are not SI
units. Table~\ref{tab:unit:accepted} lists the \enquote{accepted}
units~\cite{SI:T6}. Some units are fundamental physical quantities, and these
are non-SI but can be used within the SI
(Table~\ref{tab:unit:physical},~\cite{SI:T7}). There are also a set of non-SI
units which are used in certain defined circumstances
(Table~\ref{tab:unit:other}), although they are not necessarily officially
sanctioned~\cite{SI:T8}. Finally, for specialist applications, a range of
\acro{CGS} units are sanctioned (Table~\ref{tab:unit:CGS})~\cite{SI:T9}.
The command \cs{percent} is also provided for use in units: this is accepted
with the SI as detailed in Section~5.3.7 of the Brochure~\cite{SI:5.3.7}.
\begin{table}
  \caption{Non-SI units accepted for use with the International System of
    Units.}
  \label{tab:unit:accepted}
  \begin{tabular}{@{}lll@{}}
    \toprule
      Unit & Command & Symbol \\
    \midrule
      \DescribeUnit{day}     \\
      \DescribeUnit{degree}  \\
      \DescribeUnit{hectare} \\
      \DescribeUnit{hour}    \\
      \DescribeUnit{litre}   \\
       & \cs{liter} & \unit{\liter} \\
      \DescribeUnit[minute (plane angle)]{arcminute} \\
      \DescribeUnit[minute (time)]{minute} \\
      \DescribeUnit[second (plane angle)]{arcsecond} \\
      \DescribeUnit{tonne}   \\
    \bottomrule
  \end{tabular}
\end{table}
\begin{table}
  \caption{Non-SI units whose values in SI units must be obtained
    experimentally.}
  \label{tab:unit:physical}
  \begin{tabular}{@{}lll@{}}
    \toprule
      Unit & Command & Symbol \\
    \midrule
      \DescribeUnit[astronomical unit]{astronomicalunit}\\
      \DescribeUnit[atomic mass unit]{atomicmassunit} \\
      \DescribeUnit[a.u.~of action]{auaction} \\
      \DescribeUnit[a.u.~of charge]{aucharge} \\
      \DescribeUnit[a.u.~of energy]{auenergy} \\
      \DescribeUnit[a.u.~of length]{aulength} \\
      \DescribeUnit[a.u.~of mass]{aumass} \\
      \DescribeUnit[a.u.~of time]{autime} \\
      \DescribeUnit{bohr} \\
      \DescribeUnit{dalton} \\
      \DescribeUnit{electronvolt} \\
      \DescribeUnit{hartree} \\
      \DescribeUnit[n.u.~of action]{nuaction} \\
      \DescribeUnit[n.u.~of mass]{numass} \\
      \DescribeUnit[n.u.~of speed]{nuspeed} \\
      \DescribeUnit[n.u.~of time]{nutime} \\
    \bottomrule
  \end{tabular}
\end{table}
\begin{table}
  \caption{Other non-SI units.}
  \label{tab:unit:other}
  \begin{tabular}{@{}lll@{}}
    \toprule
      Unit & Command & Symbol \\
    \midrule
      \DescribeUnit[ångström]{angstrom} \\
      \DescribeUnit{bar} \\
      \DescribeUnit{barn} \\
      \DescribeUnit{bel} \\
      \DescribeUnit{decibel} \\
      \DescribeUnit{knot} \\
      \DescribeUnit[millimetre of mercury]{millimetremercury} \\
      \DescribeUnit[nautical mile]{nauticalmile} \\
      \DescribeUnit{neper} \\
    \bottomrule
  \end{tabular}
\end{table}
\begin{table}
  \caption{Non-SI units associated with the \acro{CGS}.}
  \label{tab:unit:CGS}
  \begin{tabular}{@{}lll@{}}
    \toprule
      Unit & Command & Symbol \\
    \midrule
      \DescribeUnit{dyne}    \\
      \DescribeUnit{erg}     \\
      \DescribeUnit{gal}     \\
      \DescribeUnit{gauss}   \\
      \DescribeUnit{maxwell} \\
      \DescribeUnit{oersted} \\
      \DescribeUnit{phot}    \\
      \DescribeUnit{poise}   \\
      \DescribeUnit{stilb}   \\
      \DescribeUnit{stokes}  \\
    \bottomrule
  \end{tabular}
\end{table}

In addition to the units themselves, \pkg{siunitx} provides pre-defined macros
for all of the SI prefixes (Table~\ref{tab:unit:prefix})~\cite{SI:3.1}. The
spelling \enquote{\cs{deka}} is provided for US users as an alternative to
\cs{deca}.
\begin{table}
  \caption{SI prefixes.}
  \label{tab:unit:prefix}
  \sisetup{table-number-alignment = right, table-format = 2}
  \begin{tabular}{@{}llcS[table-format = -2]>{\qquad}llcS@{}}
    \toprule
      Prefix & Command & Symbol & \multicolumn{1}{l}{Power} &
      Prefix & Command & Symbol & \multicolumn{1}{l@{}}{Power} \\
    \midrule
      \DescribePrefix{yocto} & -24 &
      \DescribePrefix{deca}  &   1 \\
      \DescribePrefix{zepto} & -21 &
      \DescribePrefix{hecto} &   2 \\
      \DescribePrefix{atto}  & -18 &
      \DescribePrefix{kilo}  &   3 \\
      \DescribePrefix{femto} & -15 &
      \DescribePrefix{mega}  &   6 \\
      \DescribePrefix{pico}  & -12 &
      \DescribePrefix{giga}  &  9 \\
      \DescribePrefix{nano}  &  -9 &
      \DescribePrefix{tera}  &  12 \\
      \DescribePrefix{micro} & -6 &
      \DescribePrefix{peta}  &  15 \\
      \DescribePrefix{milli} &  -3 &
      \DescribePrefix{exa}   &  18 \\
      \DescribePrefix{centi} &  -2 &
      \DescribePrefix{zetta} &  21 \\
      \DescribePrefix{deci}  &  -1 &
      \DescribePrefix{yotta} &  24 \\
    \bottomrule
  \end{tabular}
\end{table}

A small number of pre-defined powers are provided as macros. \cs{square} and
\cs{cubic} are intended for use before units, with \cs{squared} and \cs{cubed}
going after the unit.
\begin{LaTeXdemo}
  \unit{\square\becquerel} \\
  \unit{\joule\squared\per\lumen} \\
  \unit{\cubic\lux\volt\tesla\cubed}
\end{LaTeXdemo}
Generic powers can be inserted on a one-off basis using the \cs{tothe} and
\cs{raiseto} macros. These are the only macros for units which take an
argument:
\begin{LaTeXdemo}
  \unit{\henry\tothe{5}} \\
  \unit{\raiseto{4.5}\radian}
\end{LaTeXdemo}
Reciprocal powers are indicated using the \cs{per} macro. This applies to the
next unit only, unless the \opt{sticky-per} option is turned on.
\begin{LaTeXdemo}
  \unit{\joule\per\mole\per\kelvin} \\
  \unit{\joule\per\mole\kelvin} \\
  \unit{\per\henry\tothe{5}} \\
  \unit{\per\square\becquerel}
\end{LaTeXdemo}
As for generic powers, generic qualifiers are also available using the \cs{of}
function:
\begin{LaTeXdemo}
  \unit{\kilogram\of{metal}} \\
  \qty[qualifier-mode = bracket]
    {0.1}{\milli\mole\of{cat}\per\kilogram\of{prod}}
\end{LaTeXdemo}

When the \pkg{cancel} package is loaded, it is possible to \enquote{cancel out}
units using the \cs{cancel} macro. This applies to the next unit, in a similar
manner to a prefix. The \cs{highlight} macro is also available to selectively
color units. Both \cs{cancel} and \cs{highlight} are outside of the normal
semantic meaning of units, but are provided as they may be useful in some
cases.
\begin{LaTeXdemo}
  \unit[per-mode = fraction]
    {\cancel\kilogram\metre\per\cancel\kilogram\per\second} \\
  \unit{\highlight{red}\kilogram\metre\per\second} \\
  \unit[unit-color = purple]
    {\highlight{blue}\kilogram\metre\per\second}
\end{LaTeXdemo}

\subsection{Unit abbreviations}

In addition to the \enquote{full} names, \pkg{siunitx} loads a set of
abbreviated versions of the SI units (Table~\ref{tab:unit:abbr}). The standard
\pkg{siunitx} settings only create these abbreviations within the scope of the
\cs{unit} and \cs{qty} functions, meaning that no clashes should occur (for
example with the standard \cs{pm} symbol).
\begin{center}
  \label{tab:unit:abbr}
  \tablefirsthead{%
    \toprule
      \multicolumn{1}{@{}l}{Unit}      &
      \multicolumn{1}{l}{Abbreviation} &
      \multicolumn{1}{l@{}}{Symbol} \\
    \midrule
  }
  \tablehead{%
    \multicolumn{3}{@{}l@{}}{\emph{Continued from previous page}} \\
    \toprule
      \multicolumn{1}{@{}l}{Unit}      &
      \multicolumn{1}{l}{Abbreviation} &
      \multicolumn{1}{l@{}}{Symbol} \\
    \midrule
  }
  \tabletail{%
    \bottomrule
    \multicolumn{3}{@{}r@{}}{\emph{Continued on next page}} \\
  }
  \tablelasttail{\bottomrule}
  \begin{xtabular}{@{}lcc@{}}
      \DescribeUnit[femtogram]{fg} \\
      \DescribeUnit[picogram]{pg}  \\
      \DescribeUnit[nanogram]{ng}  \\
      \DescribeUnit[microgram]{ug} \\
      \DescribeUnit[milligram]{mg} \\
      \DescribeUnit[gram]{g}       \\
      \DescribeUnit[kilogram]{kg}  \\
      \DescribeUnit[atomic mass unit]{amu} \\

      \midrule

      \DescribeUnit[picometre]{pm}  \\
      \DescribeUnit[nanometre]{nm}  \\
      \DescribeUnit[micrometre]{um} \\
      \DescribeUnit[millimetre]{mm} \\
      \DescribeUnit[centimetre]{cm} \\
      \DescribeUnit[decimetre]{dm}  \\
      \DescribeUnit[metre]{m}       \\
      \DescribeUnit[kilometre]{km}  \\

      \midrule

      \DescribeUnit[attosecond]{as}  \\
      \DescribeUnit[femtosecond]{fs} \\
      \DescribeUnit[picosecond]{ps}  \\
      \DescribeUnit[nanosecond]{ns}  \\
      \DescribeUnit[microsecond]{us} \\
      \DescribeUnit[millisecond]{ms} \\
      \DescribeUnit[second]{s}       \\

      \midrule

      \DescribeUnit[femtomole]{fmol} \\
      \DescribeUnit[picomole]{pmol}  \\
      \DescribeUnit[nanomole]{nmol}  \\
      \DescribeUnit[micromole]{umol} \\
      \DescribeUnit[millimole]{mmol} \\
      \DescribeUnit[mole]{mol}       \\
      \DescribeUnit[kilomole]{kmol}  \\

      \midrule

      \DescribeUnit[picoampere]{pA}  \\
      \DescribeUnit[nanoampere]{nA}  \\
      \DescribeUnit[microampere]{uA} \\
      \DescribeUnit[milliampere]{mA} \\
      \DescribeUnit[ampere]{A}       \\
      \DescribeUnit[kiloampere]{kA}  \\

      \midrule

      \DescribeUnit[microlitre]{ul} \\
      \DescribeUnit[millilitre]{ml} \\
      \DescribeUnit[litre]{l}       \\
      \DescribeUnit[hectolitre]{hl} \\
      \DescribeUnit[microliter]{uL} \\
      \DescribeUnit[milliliter]{mL} \\
      \DescribeUnit[liter]{L}       \\
      \DescribeUnit[hectoliter]{hL} \\

      \midrule

      \DescribeUnit[millihertz]{mHz} \\
      \DescribeUnit[hertz]{Hz}       \\
      \DescribeUnit[kilohertz]{kHz}  \\
      \DescribeUnit[megahertz]{MHz}  \\
      \DescribeUnit[gigahertz]{GHz}  \\
      \DescribeUnit[terahertz]{THz}  \\

      \midrule

      \DescribeUnit[millinewton]{mN} \\
      \DescribeUnit[newton]{N}       \\
      \DescribeUnit[kilonewton]{kN}  \\
      \DescribeUnit[meganewton]{MN}  \\

      \midrule

      \DescribeUnit[pascal]{Pa}      \\
      \DescribeUnit[kilopascal]{kPa} \\
      \DescribeUnit[megapacal]{MPa}  \\
      \DescribeUnit[gigapascal]{GPa} \\

      \midrule

      \DescribeUnit[milliohm]{mohm} \\
      \DescribeUnit[kilohm]{kohm}   \\
      \DescribeUnit[megohm]{Mohm}   \\

      \midrule

      \DescribeUnit[picovolt]{pV}  \\
      \DescribeUnit[nanovolt]{nV}  \\
      \DescribeUnit[microvolt]{uV} \\
      \DescribeUnit[millivolt]{mV} \\
      \DescribeUnit[volt]{V}       \\
      \DescribeUnit[kilovolt]{kV}  \\

      \midrule

      \DescribeUnit[watt]{W}                \\
      \DescribeUnit[microwatt]{uW}          \\
      \DescribeUnit[milliwatt]{mW}          \\
      \DescribeUnit[kilowatt]{kW}           \\
      \DescribeUnit[megawatt]{MW}           \\
      \DescribeUnit[gigawatt]{GW}           \\
      \DescribeUnit[joule]{J}               \\
      \DescribeUnit[microjoule]{uJ}         \\
      \DescribeUnit[millijoule]{mJ}         \\
      \DescribeUnit[kilojoule]{kJ}          \\
      \DescribeUnit[electronvolt]{eV}       \\
      \DescribeUnit[millielectronvolt]{meV} \\
      \DescribeUnit[kiloelectronvolt]{keV}  \\
      \DescribeUnit[megaelectronvolt]{MeV}  \\
      \DescribeUnit[gigaelectronvolt]{GeV}  \\
      \DescribeUnit[teraelectronvolt]{TeV}  \\
      \DescribeUnit[kilowatt hour]{kWh}     \\

      \midrule

      \DescribeUnit[farad]{F}       \\
      \DescribeUnit[femtofarad]{fF} \\
      \DescribeUnit[picofarad]{pF}  \\

      \midrule

      \DescribeUnit[kelvin]{K} \\

      \midrule

      \DescribeUnit[decibel]{dB} \\

  \end{xtabular}
\end{center}

\subsection{Creating new macros}

The various macro components of a unit have to be defined before they can be
used. The package supplies a number of common definitions, but new definitions
are also possible. As the definition of a logical unit should remain the same
in a single document, these creation functions are all preamble-only.

\begin{function}{\DeclareSIUnit}
  \begin{syntax}
    \cs{DeclareSIUnit}\oarg{options}\marg{unit}\marg{symbol}
  \end{syntax}
\end{function}
New units are produced using the \cs{DeclareSIUnit} macro. The \meta{symbol}
can contain literal input, other units, multiple prefixes, powers and \cs{per},
although literal text should not be intermixed with unit macros. Units can be
created with \meta{options} from the usual list understood by \pkg{siunitx},
and apply the specific unit macro only. The (first) optional argument to
\cs{qty} and \cs{unit} can be used to override the settings for the unit: an
example is the \cs{degree} unit.
\begin{LaTeXdemo}
  \qty{3.1415}{\degree}
\end{LaTeXdemo}
 This is declared in the package (effectively) as
\begin{LaTeXdemo}[code only]
  \DeclareSIUnit[quantity-product = {}]
    \degree{\text{\textdegree}}
\end{LaTeXdemo}
 The spacing can still be altered at point of use:
\begin{LaTeXdemo}
  \qty{67890}{\degree} \\
  \qty[quantity-product = \,]{67890}{\degree}
\end{LaTeXdemo}
The meaning of a pre-defined unit can be altered by using \cs{DeclareSIUnit}
after loading \pkg{siunitx}. This will overwrite the original definition with
the newer version.

\begin{function}{\DeclareSIPrefix}
  \begin{syntax}
    \cs{DeclareSIPrefix}\marg{prefix}\marg{symbol}\marg{powers-ten}
  \end{syntax}
\end{function}
 The standard SI powers of ten are defined by the package, and are
 described above.  However, the user can define new prefixes with
 \cs{DeclareSIPrefix}. For example, \cs{kilo} is defined
\begin{LaTeXdemo}[code only]
  \DeclareSIPrefix\kilo{k}{3}
\end{LaTeXdemo}

\begin{function}{\DeclareSIPower}
  \begin{syntax}
    \cs{DeclareSIPower}\marg{symbol-before}\marg{symbol-after}\marg{power}
  \end{syntax}
\end{function}
This function creates two symbols, one for use before a unit, the second
for use after a unit, both of which are equivalent to the \meta{power}.
For example, one might use
\begin{LaTeXdemo}[code only]
  \DeclareSIPower\quartic\tothefourth{4}
\end{LaTeXdemo}
 with the functions then used in the document as
\begin{LaTeXdemo}
  \unit{\kilogram\tothefourth}\\
  \unit{\quartic\metre}
\end{LaTeXdemo}

\begin{function}{\DeclareSIQualifier}
  \begin{syntax}
    \cs{DeclareSIQualifier}\marg{qualifier}\marg{symbol}
  \end{syntax}
\end{function}
Following the syntax of the other macros, qualifiers are created with the
\cs{DeclareSIQualifier} command. In contrast to the other parts of a unit,
there are no pre-defined qualifiers. It is therefore entirely up to the user to
create these. For example, to identify the mass of a product created when using
a particular catalyst, the preamble could contain:
\begin{LaTeXdemo}[code only]
  \DeclareSIQualifier\polymer{pol}
  \DeclareSIQualifier\catalyst{cat}
\end{LaTeXdemo}
 and then in the body the document could read
\begin{LaTeXdemo}
  \qty{1.234}{\gram\polymer\per\mole\catalyst\per\hour}
\end{LaTeXdemo}

\subsection{Tabular material}

Aligning numbers in tabular content is handled by a new column type, the
\texttt{S} column. This new column type can align material using a number of
different strategies, with the aim of flexibility of output without needing to
alter the input. The method used as standard is to place the decimal marker in
the number at the centre of the cell and to align the material appropriately
(Table~\ref{tab:S:standard}).
\begin{LaTeXdemo}[code and float]
  \begin{table}
    \caption{Standard behaviour of the \texttt{S} column type.}
    \label{tab:S:standard}
    \begin{tabular}{S}
    \toprule
      {Some Values} \\
    \midrule
         2.3456 \\
        34.2345 \\
        -6.7835 \\
        90.473  \\
      5642.5    \\
          1.2e3 \\
            e4  \\
    \bottomrule
    \end{tabular}
  \end{table}
\end{LaTeXdemo}

The \texttt{S} column will attempt to automatically detect material which
should be placed before or after a number, and will maintain the alignment of
the numerical data (Table~\ref{tab:S:extras}). If the material could be
mistaken for part of a number, it should be protected by braces. The use of
\cs{color} in a table cell will also be detected and will override any general
color applied by \pkg{siunitx}.
\begin{LaTeXdemo}[code and float]
  \begin{table}
    \caption{Detection of surrounding material in an \texttt{S}
      column.}
    \label{tab:S:extras}
    \begin{tabular}{S[color = orange]}
    \toprule
      {Some Values} \\
    \midrule
      12.34 \\
      \color{purple} 975,31 \\
      44.268 \textsuperscript{\emph{a}} \\
    \bottomrule
    \end{tabular}
  \end{table}
\end{LaTeXdemo}

\begin{function}{\tablenum}
  \begin{syntax}
    \cs{tablenum}\oarg{options}\marg{number}
  \end{syntax}
\end{function}
Within more complex tables, aligned numbers may be desirable within the
argument of \cs{multicolumn} or \cs{multirow}.\footnote{Provided by the
\pkg{multirow} package} The \cs{tablenum} function is available to achieve
alignment in these situations: this is, in effect, a macro version of the
\texttt{S} column (Table~\ref{tab:tablenum}).
\begin{LaTeXdemo}[code and float]
  \begin{table}
    \caption{Controlling complex alignment with the \cs{tablenum}
      macro.}
    \label{tab:tablenum}
    \begin{tabular}{@{}lr@{}}
      \toprule
      Heading & Heading \\
      \midrule
      Info & More info \\
      Info & More info \\
      \multicolumn{2}{c}{\tablenum[table-format = 4.4]{12,34}}    \\
      \multicolumn{2}{c}{\tablenum[table-format = 4.4]{333.5567}} \\
      \multicolumn{2}{c}{\tablenum[table-format = 4.4]{4563.21}}  \\
      \bottomrule
    \end{tabular}
    \hfil
    \begin{tabular}{@{}lr@{}}
      \toprule
      Heading & Heading \\
      \midrule
      \multirow{2}*{\tablenum{88,999}} & aaa \\
                                       & bbb \\
      \multirow{2}*{\tablenum{33,435}} & ccc \\
                                       & ddd \\
      \bottomrule
    \end{tabular}
  \end{table}
\end{LaTeXdemo}

\section{Package control options}

\subsection{The key--value control system}

 The package uses a range of different key types:
\begin{description}
  \item[\texttt{Choice}] Takes a limited number of choices, which are described
    separately for each key.
  \item[\texttt{Integer}] Requires a number as the argument.
  \item[\texttt{Length}] Requires a length, either as a literal
    value such as \texttt{2.0cm}, or stored as a \LaTeX{} length.
  \item[\texttt{Literal}] A key which uses the value(s) given directly,
    either to check input or in output.
  \item[\texttt{Macro}] Requires a macro, which may need a single argument.
  \item[\texttt{Math}] Similar to a \texttt{literal} option, but the input is
    always used in math mode, irrespective of other \pkg{siunitx} settings.
    Thus to text-mode only input must be placed inside the argument of a
    \cs{text} macro.
  \item[\texttt{Meta}] These are options which actually apply a number of
    other options.
  \item[\texttt{Switch}] These are on--off switches, and recognise \opt{true}
    and \opt{false}.  Giving just the key name also turns the key on.
\end{description}
The tables of option names use these descriptions to indicate how the keys
should be used.

\subsection{Printing}
\label{sec:print}

The \pkg{siunitx} package can control the font used to print output
independently of the surrounding material. Which aspects of the font follow
those of the surroundings is influenced by a range of setting as detailed in
Section~\ref{tab:opt:print}.
\begin{table}
  \caption{Print options.}
  \label{tab:opt:print}
  \begin{tabular}{@{}>{\ttfamily}ll>{\ttfamily}l@{}}
    \toprule
      \multicolumn{1}{@{}l}{Option name} &
      Type &
      \multicolumn{1}{l@{}}{Default} \\
    \midrule
      color               & Literal & \meta{none} \\
      mode                & Choice  & math        \\
      number-color        & Literal & \meta{none} \\
      number-mode         & Choice  & math        \\
      propagate-math-font & Switch  & false       \\
      reset-math-version  & Switch  & true        \\
      reset-text-family   & Switch  & true        \\
      reset-text-series   & Switch  & true        \\
      reset-text-shape    & Switch  & true        \\
      text-family-to-math & Switch  & false       \\
      text-weight-to-math & Switch  & false       \\
      unit-color          & Literal & \meta{none} \\
      unit-mode           & Choice  & math        \\
    \bottomrule
  \end{tabular}
\end{table}

\DescribeOption{mode}
\DescribeOption{number-mode}
\DescribeOption{unit-mode}
The \opt{mode} option determines whether \pkg{siunitx} uses math or text mode
when printing output. The choices are \opt{match}, \opt{math}, \opt{text}. The
\opt{match} setting means that printing uses the prevailing mode unchanged
whereas \opt{math} and \opt{text} select the appropriate \TeX{} mode. It is
possible to have different fonts in math and text modes, which will highlight
the difference. The font settings which apply are also different depending on
the mode. As well as the overall setting, it is possible to apply mode
to numbers and units separately using the \opt{number-mode} and \opt{unit-mode}
options.

\DescribeOption{reset-text-family}
\DescribeOption{reset-text-series}
\DescribeOption{reset-text-shape}
When printing in text mode, the options \opt{reset-text-family},
\opt{reset-text-series} and \opt{reset-text-shape} apply. When these are
active, \pkg{siunitx} resets the relevant font selection axis property
when printing: the standard font setting is upright mid-weight roman
(|\upshape| |\mdseries| |\rmfamily|).
\begin{LaTeXdemo}
  \sisetup{mode = text}
  {\itshape  \num{1234}}\\
  {\bfseries \num{1234}}\\
  {\sffamily \num{1234}}\\
  \sisetup{
    reset-text-family = false ,
    reset-text-series = false ,
    reset-text-shape  = false
  }
  {\itshape  \num{1234}}\\
  {\bfseries \num{1234}}\\
  {\sffamily \num{1234}}\\
\end{LaTeXdemo}

\DescribeOption{propagate-math-font}
\DescribeOption{reset-math-version}
In math mode, the font used by \LaTeX{} is \enquote{invariant}, and this is
reflected in the options available. With the standard settings, in math mode
printing uses the standard math font and version (weight). The option
\opt{propagate-math-font} may be used to apply the a prevailing math font
to the printed material. The setting \opt{reset-math-version} controls
whether the math version is reset or not. Note that math version is typically
used to set \enquote{bold math} but may also be used for other effects, for
example all sanserif math.
\begin{LaTeXdemo}
  {\boldmath \unit{\kilogram}}\\
  {\sansmath $\unit{\kilogram}$}\\
  {$\mathsf{\unit{\kilogram}}$}\\
  \sisetup{
    propagate-math-font = true  ,
    reset-math-version  = false
  }
  {\boldmath \unit{\kilogram}}\\
  {\sansmath $\unit{\kilogram}$}\\
  {$\mathsf{\unit{\kilogram}}$}
\end{LaTeXdemo}

\DescribeOption{text-family-to-math}
\DescribeOption{text-weight-to-math}
The options \opt{text-family-to-math} and \opt{text-family-to-math} can be
used to match (as far as possible) math mode output to the surrounding text.
These options work by detecting the current text settings and making the
appropriate choice in math mode.
\begin{LaTeXdemo}
  {\sffamily \unit{\kilogram}}\\
  {\bfseries $\unit{\kilogram}$}\\
  \sisetup{
    text-family-to-math = true  ,
    text-weight-to-math = true
  }
  {\sffamily \unit{\kilogram}}\\
  {\bfseries $\unit{\kilogram}$}
\end{LaTeXdemo}

\DescribeOption{color}
\DescribeOption{number-color}
\DescribeOption{unit-color}
The color of printed output can be set using the \opt{color} option. When no
color is given, printing follows the surrounding text. In contrast, when a
specific color is given, it is used irrespective of the surroundings. As with
\opt{mode}, the \opt{color} setting may also be applied to numbers and units
independently.
\begin{LaTeXdemo}
  \color{red}%
  Some text \\
  \qty{4}{\kilogram} \\
  More text \\
  \qty[color = blue]{4}{\kilogram} \\
  Still red here!
\end{LaTeXdemo}

\subsection{Parsing numbers}

The package uses a sophisticated parsing system to understand numbers. This
allows \pkg{siunitx} to carry out a range of formatting, as described later.
All of the input options take lists of literal tokens, and are summarised in
Table~\ref{tab:opt:num:in}.
\begin{table}
  \caption{Options for number parsing.}
  \label{tab:opt:num:in}
  \begin{tabular}{@{}>{\ttfamily}ll>{\ttfamily}l@{}}
    \toprule
      \multicolumn{1}{@{}l}{Option name} &
      Type &
      \multicolumn{1}{l@{}}{Default} \\
    \midrule
      evaluate-expression     & Switch  & false            \\
      expression              & Literal & |#1|             \\^^A (
      input-close-uncertainty & Literal & )                \\
      input-comparators       & Literal &
        <=>\cs{approx}\cs{ge}\cs{geq} \\
        & & \cs{gg}\cs{le}\cs{leq}\cs{ll} \cs{sim} \\
      input-decimal-markers   & Literal & .,               \\
      input-digits            & Literal & 0123456789       \\
      input-exponent-markers  & Literal & dDeE             \\
      input-ignore            & Literal & \meta{none}      \\
      input-open-uncertainty  & Literal & (                \\ ^^A )
      input-signs             & Literal & +-\cs{pm}\cs{mp} \\
      input-uncertainty-signs & Literal & \cs{pm}          \\
      parse-numbers           & Switch  & true             \\
    \bottomrule
  \end{tabular}
\end{table}

\DescribeOption{input-digits}
\DescribeOption{input-decimal-markers}
\DescribeOption{input-signs}
\DescribeOption{input-exponent-markers}
The basic parts of a number are the digits, any sign and a separator between
the integer and decimal parts. These are stored in the input options
\opt{input-digits}, \opt{input-decimal-markers} and \opt{input-signs},
respectively. More than one input decimal marker can be used: it will be
converted by the package to the appropriate output marker. Numbers which
include an exponent part also require a marker for the exponent: this again is
taken from the range of tokens in the \opt{input-exponent-markers} option.

\DescribeOption{input-ignore}
Tokens given in the \opt{input-ignore} list are totally passed over by
\pkg{siunitx}: they will be removed from the input with no further processing.

\DescribeOption{input-comparators}
In addition to signs, \pkg{siunitx} can recognise comparators, such as |<|. The
package will automatically carry out conversions for |<<|, |>>|, |<=| and |>=|
to |\ll|, |\gg|, |\le| and |\ge|, respectively.
\begin{LaTeXdemo}
  \num{< 10} \\
  \qty{>> 5}{\metre} \\
  \num{\le 0.12}
\end{LaTeXdemo}

\DescribeOption{input-open-uncertainty}
\DescribeOption{input-close-uncertainty}
\DescribeOption{input-uncertainty-signs}
In some fields, it is common to give the uncertainty in a number in brackets
after the main part of the number, for example \enquote{\num{1.234(5)}}. The
opening and closing symbols used for this type of input are set as
\opt{input-open-uncertainty} and \opt{input-close-uncertainty}. Alternatively,
the uncertainty may be given as a separate part following a sign. Which signs
are valid for this operation is determined by the \opt{input-uncertainty-signs}
option. As with other signs, the combination |+-| will automatically be
converted to |\pm| internally.
\begin{LaTeXdemo}
  \num{9.99(9)}       \\
  \num{9.99 +- 0.09}  \\
  \num{9.99 \pm 0.09} \\
  \num{123 +- 4.5}    \\
  \num{12.3 +- 6}
\end{LaTeXdemo}

\DescribeOption{parse-numbers}
The \opt{parse-numbers} option turns the entire parsing system on and off. The
option is made available for two reasons. First, if all of the numbers in a
document are to be reproduced \enquote{as given}, turning off the parser will
represent a significant saving in processing required. Second, it allows the
use of arbitrary \TeX{} code in numbers. If the parser is turned off, the input
will be printed in math mode (requiring |\text| to protect any text in the
number).
\begin{LaTeXdemo}
  \num[parse-numbers = false]{\sqrt{2}}        \\
  \qty[parse-numbers = false]{\sqrt{3}}{\metre}
\end{LaTeXdemo}

\DescribeOption{evaluate-expression}
\DescribeOption{expression}
With the standard settings, numerical input is parsed \enquote{as is} with no
attempt to interpret it mathematically. By enabling the
\opt{evaluate-expression} option, the input can be processed by the standard
\LaTeX3 FPU (see package \pkg{xfp} for more). The nature of the expression
itself can be adjusted using the \opt{expression} setting: as standard, the
entire input is simply parsed with no change, but this setting may be used to
add additional steps. The \emph{input} in such an expression is represented
by |#1|. Note that the FPU uses its own syntax for numbers, most notably
in that a decimal marker must be |.|.
\begin{LaTeXdemo}
  \sisetup{evaluate-expression}%
  \qty{2 + 4 * 3}{\joule} \\
  \qty[expression = 10 * (#1)]{2 + 4 * 3}{\joule}
\end{LaTeXdemo}

\subsection{Post-processing numbers}

Before typesetting numbers, various post-processing steps can be carried out.
These involve adding or removing information from the number in a systematic
way; the options are summarised in Table~\ref{tab:opt:num:post}.
\begin{table}
  \caption{Number post-processing options.}
  \label{tab:opt:num:post}
  \begin{tabular}{@{}>{\ttfamily}ll>{\ttfamily}l@{}}
    \toprule
      \multicolumn{1}{@{}l}{Option name} &
      Type &
      \multicolumn{1}{l@{}}{Default} \\
    \midrule
      drop-exponent          & Switch  & false       \\
      drop-uncertainty       & Switch  & false       \\
      drop-zero-decimal      & Switch  & false       \\
      exponent-mode          & Switch  & input       \\
      fixed-exponent         & Integer & 0           \\
      minimum-integer-digits & Integer & 0           \\
      minimum-decimal-digits & Integer & 0           \\
      round-half             & Choice  & up          \\
      round-minimum          & Literal & 0           \\
      round-mode             & Choice  & none        \\
      round-pad              & Switch  & true        \\
      round-precision        & Integer & 2           \\
    \bottomrule
  \end{tabular}
\end{table}

\DescribeOption{exponent-mode}
\DescribeOption{fixed-exponent}
Numbers can be converted to scientific notation by the package. This is
controlled by the \opt{exponent-mode} option, which takes choices \opt{input},
\opt{fixed}, \opt{engineering} and \opt{scientific}. The \opt{fixed} setting
will use the exponent value by the \opt{fixed-exponent} option. When
\opt{engineering} is set, the exponent is always a power of three.
\begin{LaTeXdemo}
  \num{0.001}   \\
  \num{0.0100} \\
  \num{1200}    \\
  \sisetup{exponent-mode = scientific}%
  \num{0.001}   \\
  \num{0.0100} \\
  \num{1200}    \\
  \sisetup{exponent-mode = engineering}%
  \num{0.001}   \\
  \num{0.0100} \\
  \num{1200}    \\
  \sisetup{
    exponent-mode  = fixed,
    fixed-exponent = 2,
  }%
  \num{0.001}   \\
  \num{0.0100} \\
  \num{1200}
\end{LaTeXdemo}
When used with a \opt{fixed-exponent} of zero, this may be used to remove
scientific notation from the input
\begin{LaTeXdemo}
  \num{1.23e4} \\
  \num[exponent-mode = fixed, fixed-exponent = 0]{1.23e4}
\end{LaTeXdemo}
Exponent mode applies \emph{after} rounding, such that the number of decimal
places for rounding is those which appear in the output.

\DescribeOption{drop-exponent}
\DescribeOption{drop-uncertainty}
The use of an uncertainty can be suppressed entirely using the
\opt{drop-uncertainty} option: this applies \emph{before} rounding is
attempted. Similarly, exponents can be dropped using \opt{drop-exponent} can be
used to suppress the exponent part (\emph{after} conversion to a fixed
exponent).
\begin{LaTeXdemo}
  \num{0.01(2)} \\
  \num[drop-uncertainty]{0.01(2)} \\
   \num{0.01e3} \\
  \num[drop-exponent]{0.01e3}
\end{LaTeXdemo}

\DescribeOption{round-mode}
\DescribeOption{round-precision}
\DescribeOption{round-pad}
The package can round numerical input to a fixed number of significant figures
or decimal places. This is controlled by the \opt{round-mode} option, which
takes the choices \opt{none}, \opt{figures}, \opt{places} and
\opt{uncertainty}. When rounding is turned on, the number of digits used
(either decimal places or significant figures in the mantissa) is set using the
\opt{round-precision} option. Rounding numbers with uncertainties may be
carried out using the \opt{uncertainty} setting to \opt{round-mode}. In this
case the precision is used first to round the uncertainty itself (to a number
of figures), before rounding the main value to follow.
\begin{LaTeXdemo}
  \num{1.23456} \\
  \num{14.23} \\
  \num{0.12345(9)} \\
  \sisetup{
    round-mode      = places,
    round-precision = 3
  }%
  \num{1.23456} \\
  \num{14.23} \\
  \num{0.12345(9)} \\
  \sisetup{
    round-mode      = figures,
    round-precision = 3
  }%
  \num{1.23456} \\
  \num{14.23} \\
  \num{0.12345(9)}
  \sisetup{
    round-mode      = uncertainty,
    round-precision = 1
  }%
  \num{0.12345(9)} \\
  \num{0.12345(23)} \\
  \num{0.12345(234)}
\end{LaTeXdemo}
Rounding my \enquote{extend} a short number to more digits (or figures): this
is controlled by the switch \opt{round-pad}, which is \opt{true} as standard.
\begin{LaTeXdemo}
  \sisetup{round-mode = figures, round-precision = 4}%
  \num{12.3} \\
  \num[round-pad = false]{12.3}
\end{LaTeXdemo}

\DescribeOption{round-half}
In cases where the rounded part of a number is exactly half, there are two
common methods for \enquote{breaking the tie}. The choice of method is
determined by the option \opt{round-half}, which recognises the choices
\opt{up} and \opt{even}.
\begin{LaTeXdemo}
  \sisetup{
    round-mode      = figures,
    round-precision = 1,
    round-half      = up
  }%
  \num{0.055} \\
  \num{0.045} \\
  \sisetup{round-half = even}%
  \num{0.055} \\
  \num{0.045}
\end{LaTeXdemo}

\DescribeOption{round-minimum}
There are cases in which rounding will result in the number reaching zero. It
may be desirable to show such results as below a threshold value. This can be
achieved by setting \opt{round-minimum} to the threshold value. There will be
no effect when rounding to a number of significant figures as it is not
possible to obtain the value zero in these cases.
\begin{LaTeXdemo}
  \sisetup{round-mode = places}%
  \num{0.0055} \\
  \num{0.0045} \\
  \sisetup{round-minimum = 0.01}% TODO
  \num{0.0055} \\
  \num{0.0045}
\end{LaTeXdemo}

\DescribeOption{drop-zero-decimal}
It may be desirable to convert decimals to integers when the decimal part is
zero. This is set up using the \opt{drop-zero-decimal} option, which applies
after rounding but before setting minimum numbers of digits.
\begin{LaTeXdemo}
  \num{2.0} \\
  \num{2.1} \\
  \sisetup{drop-zero-decimal}%
  \num{2.0} \\
  \num{2.1}
\end{LaTeXdemo}

\DescribeOption{minimum-decimal-digits}
\DescribeOption{minimum-integer-digits}
The \opt{minimum-decimal-digits} and \opt{minimum-integer-digits} option may be
used to pad numbers to a given size. This applies independent of any rounding.
\begin{LaTeXdemo}
  \num{123} \\
  \num[minimum-integer-digits = 2]{123} \\
  \num[minimum-integer-digits = 4]{123} \\
  \num{0.123} \\
  \num[minimum-decimal-digits = 2]{0.123} \\
  \num[minimum-decimal-digits = 4]{0.123} \\
\end{LaTeXdemo}

\subsection{Printing numbers}

Actually printing numbers is controlled by a number of settings, which apply
ideas such as differing decimal markers, digit grouping and so on. All of these
options are concerned with the appearance of output, rather than the data it
conveys. The options are summarised in Table~\ref{tab:opt:num:out}.
\begin{table}
  \caption{Output options for numbers.}
  \label{tab:opt:num:out}
  \begin{tabular}{>{\ttfamily}ll>{\ttfamily}l}
    \toprule
      \multicolumn{1}{l}{Option name} &
      Type &
      \multicolumn{1}{l}{Default} \\
    \midrule
      bracket-negative-numbers & Switch  & false             \\ ^^A (
      close-bracket            & Literal & )                 \\
      exponent-base            & Literal & 10                \\
      exponent-product         & Math    & \verb=\times=     \\
      group-digits             & Choice  & all               \\
      group-minimum-digits     & integer & 5                 \\
      group-separator          & Literal & \cs{,}            \\
      negative-color           & Literal & \meta{none}       \\
      open-bracket             & Literal & (                 \\ ^^A ) (
      output-close-uncertainty & Literal & )                 \\
      output-decimal-marker    & Literal & .                 \\
      output-open-uncertainty  & Literal & (                 \\ ^^A )
      separate-uncertainty     & Switch  & false             \\
      tight-spacing            & Switch  & false             \\
      uncertainty-separator    & Literal & \meta{none}       \\
    \bottomrule
  \end{tabular}
\end{table}

\DescribeOption{group-digits}
\DescribeOption{group-four-digits}
\DescribeOption{group-separator}
Grouping digits into blocks of three is a common method to increase the ease of
reading of numbers. The \opt{group-digits} choice controls whether this
behaviour applies, and takes the values \opt{all}, \opt{none}, \opt{decimal}
and \opt{integer}. Grouping can be activated separately for the integer and
decimal parts of a number using the appropriately-named values.
\begin{LaTeXdemo}
  \num{12345.67890} \\
  \num[group-digits = none]{12345.67890}   \\
  \num[group-digits = decimal]{12345.67890} \\
  \num[group-digits = integer]{12345.67890}
\end{LaTeXdemo}
The separator used between groups of digits is stored by the
\opt{group-separator} option. This takes literal input and may be used in math
mode: for a text-mode full space use \verb*|\ |.
\begin{LaTeXdemo}
  \num{12345} \\
  \num[group-separator = {,}]{12345} \\
  \num[group-separator = \ ]{12345}
\end{LaTeXdemo}

\DescribeOption{group-minimum-digits}
Grouping is not always applied to smaller numbers, and the option
\opt{group-minimum-digits} is available to specify how many digits must be
present before grouping is applied. The number of digits is considered
separately for the integer and decimal parts of the number: grouping does not
\enquote{cross the boundary}.
\begin{LaTeXdemo}
  \num{1234} \\
  \num{12345} \\
  \num[group-minimum-digits = 5]{1234} \\
  \num[group-minimum-digits = 5]{12345} \\
  \num{1234.5678} \\
  \num{12345.67890} \\
  \num[group-minimum-digits = 5]{1234.5678} \\
  \num[group-minimum-digits = 5]{12345.67890}
\end{LaTeXdemo}

\DescribeOption{output-decimal-marker}
The decimal marker used in output is set using the \opt{output-decimal-marker}
option; this can differ from the input marker.
\begin{LaTeXdemo}
  \num{1.23} \\
  \num[output-decimal-marker = {,}]{1.23} \\
\end{LaTeXdemo}

\DescribeOption{exponent-base}
\DescribeOption{exponent-product}
When exponents are present in the input, the \opt{exponent-base} and
\opt{exponent-product} options set the obvious parts of the output.
\begin{LaTeXdemo}
  \num[exponent-product = \times]{1e2} \\
  \num[exponent-product = \cdot]{1e2} \\
  \num[exponent-base = 2]{1e2}
\end{LaTeXdemo}

\DescribeOption{separate-uncertainty}
\DescribeOption{uncertainty-separator}
\DescribeOption{output-open-uncertainty}
\DescribeOption{output-close-uncertainty}
When input is given including an uncertainty in a number, it can be printed
either with the uncertainty in brackets or as a separate number. This behaviour
is controlled by the \opt{separate-uncertainty} choice. If the uncertainty is
given in brackets, a space may be added between the main number and the
uncertainty: this is stored using the \opt{uncertainty-separator} option. The
opening and closing brackets used are stored in \opt{output-open-uncertainty}
and \opt{output-close-uncertainty}, respectively.
\begin{LaTeXdemo}
  \num{1.234(5)} \\
  \num[separate-uncertainty = true]{1.234(5)} \\
  \sisetup{
    output-open-uncertainty  = [,
    output-close-uncertainty = ],
    uncertainty-separator    = \,
  }%
  \num{1.234(5)}
\end{LaTeXdemo}
Notice that \pkg{siunitx} correctly interprets uncertainties which cross the
decimal marker position whether these are separated out or not.
\begin{LaTeXdemo}
  \num{8.2(13)} \\
  \num[separate-uncertainty]{8.2(13)}
\end{LaTeXdemo}

\DescribeOption{negative-color}
The package can detect negative mantissa values and alter print color
accordingly. This is disabled by setting the option to an empty value.
\begin{LaTeXdemo}
  \num{-15673} \\
  \num[negative-color = red]{-15673}
\end{LaTeXdemo}

\DescribeOption{bracket-negative-numbers}
A common means to display negative numbers in financial situations is to place
them in brackets. This can be carried out automatically using the
\opt{bracket-negative-numbers} option.
\begin{LaTeXdemo}
  \num{-15673} \\
  \num[bracket-negative-numbers]{-15673} \\
  \qty{-10}{\metre} \\
  \qty[bracket-negative-numbers]{-10}{\metre}
\end{LaTeXdemo}

\DescribeOption{tight-spacing}
Under some circumstances is may be desirable to \enquote{squeeze} the output
spacing. This is turned on using the \opt{tight-spacing} switch, which
compresses spacing where possible.
\begin{LaTeXdemo}
  \num{2e3} \\
  \num[tight-spacing = true]{2e3}
\end{LaTeXdemo}

\subsection{Lists, products and ranges of numbers}

Lists, products and ranges of numbers have a small number of specialised
options, which apply to these more unusual input forms
(Table~\ref{tab:opt:num:list}).
\begin{table}
  \caption{Output options for lists, products ranges of numbers.}
  \label{tab:opt:num:list}
  \begin{tabular}{@{}>{\ttfamily}ll>{\ttfamily}l@{}}
    \toprule
      \multicolumn{1}{l}{Option name} &
      Type &
      \multicolumn{1}{l}{Default} \\
    \midrule
      list-final-separator & Literal & \verb*=\text{ and }= \\
      list-pair-separator  & Literal & \verb*=\text{ and }= \\
      list-separator       & Literal & \verb*=\text{, }=    \\
      product-mode         & Choice  & symbol               \\
      product-phrase       & Literal & \verb*=\text{ by }=  \\
      product-symbol       & Literal & \cs{times}           \\
      range-phrase         & Literal & \verb*=\text{ to }=  \\
    \bottomrule
  \end{tabular}
\end{table}

\DescribeOption{list-final-separator}
\DescribeOption{list-pair-separator}
\DescribeOption{list-separator}
Lists of numbers are printed with a separator between each item, which is
stored using the \opt{list-separator} option. The separator before the last
item of a list may be different, and is therefore set using the
\opt{list-final-separator} option. The separator used for exactly two items is
set using the \opt{list-pair-separator} option. Any spaces needed should be
included in the option settings: none are added within the code.
\begin{LaTeXdemo}
  \numlist{0.1;0.2;0.3}                              \\
  \numlist[list-separator = {; }]{0.1;0.2;0.3}       \\
  \numlist[list-final-separator = {, }]{0.1;0.2;0.3} \\
  \numlist[
    list-separator       = { and },
    list-final-separator = { and finally }
  ]{0.1;0.2;0.3} \\
  \numlist{0.1;0.2} \\
  \numlist[list-pair-separator = {, and }]{0.1;0.2}
\end{LaTeXdemo}

\DescribeOption{product-mode}
\DescribeOption{product-phrase}
\DescribeOption{product-symbol}
Products of numbers can be output using either a product symbol or phrase:
this is controlled by the \opt{product-mode} setting. When \opt{symbol} is
set, the appropriate symbol is stored in \opt{product-symbol}. When using
\opt{phrase} mode, the information is stored in \opt{product-phrase}.
\begin{LaTeXdemo}
  \numproduct{5 x 100 x 2} \\
  \numproduct[product-symbol = \ensuremath{\cdot}]{5 x 100 x 2} \\
  \sisetup{product-mode = phrase}%
  \numproduct{5 x 100 x 2}\\
  \numproduct[product-phrase = { BY }]{5 x 100 x 2} \\
\end{LaTeXdemo}

\DescribeOption{range-phrase}
Ranges of numbers can be given as input. These will have an appropriate word or
symbol inserted between the two entries: this is stored using the
\opt{range-phrase} option. The phrase should include any necessary spaces: no
extra space is added.
\begin{LaTeXdemo}
  \numrange{5}{100} \\
  \numrange[range-phrase = --]{5}{100}
\end{LaTeXdemo}

\subsection{Angles}

Angle processing provided by the \cs{ang} function has a set of options which
apply in addition to the general ones set up for number processing.
\begin{table}
  \caption{Angle options.}
  \label{tab:opt:ang}
  \begin{tabular}{@{}>{\ttfamily}ll>{\ttfamily}l@{}}
    \toprule
      \multicolumn{1}{@{}l}{Option name} &
      Type &
      \multicolumn{1}{l@{}}{Default} \\
    \midrule
      angle-mode                & Choice  & input        \\
      angle-symbol-over-decimal & Switch  & false        \\
      arc-separator             & Literal & \meta{empty} \\
      fill-arc-degrees          & Switch  & false        \\
      fill-arc-minutes          & Switch  & false        \\
      fill-arc-seconds          & Switch  & false        \\
      number-angle-product      & Literal & \meta{empty} \\
    \bottomrule
  \end{tabular}
\end{table}

\DescribeOption{angle-mode}
The format in which angles are printed can be set using the \opt{angle-mode}
option. With the standard setting (\opt{input}), the angle is printed as-given.
By setting the option to \opt{arc} or \opt{decimal}, the output format can
be set to an arc (degrees/minutes/seconds) or decimal value. Conversion uses
the \LaTeX3 floating-point unit, so is limited to $16$ decimal places.
\begin{LaTeXdemo}
  \ang{2.67} \\
  \ang{2;3;4} \\
  \ang[angle-mode = arc]{2.67} \\
  \ang[angle-mode = arc]{2;3;4} \\
  \ang[angle-mode = decimal]{2.67} \\
  \ang[angle-mode = decimal]{2;3;4} \\
\end{LaTeXdemo}

\DescribeOption{number-angle-product}
The separator between the number and angle symbol (degrees, minutes or seconds)
can be set using the \opt{number-angle-product} option, independent of the
related \opt{number-unit-product} option used by the \cs{qty} command.
\begin{LaTeXdemo}
  \ang{2.67} \\
  \ang[number-angle-product = \,]{2.67}
\end{LaTeXdemo}

\DescribeOption{arc-separator}
When angles are printed in arc format, the separation of the different parts is
set up using the \opt{arc-separator} option.
\begin{LaTeXdemo}
  \ang{6;7;6.5} \\
  \ang[arc-separator = \,]{6;7;6.5}
\end{LaTeXdemo}

\DescribeOption{fill-arc-degrees}
\DescribeOption{fill-arc-minutes}
\DescribeOption{fill-arc-seconds}
Zero-filling for the degree, minute or second parts of an arc is controlled
using the \opt{fill-arc-degrees}, \opt{fill-arc-minutes} and
\opt{fill-arc-seconds} options. All are off as standard.
\begin{LaTeXdemo}
  \ang{-1;;} \\
  \ang{;-2;} \\
  \ang{;;-3} \\
  {
    \sisetup{fill-arc-degrees}
    \ang{-1;;} \\
    \ang{;-2;} \\
    \ang{;;-3} \\
  }
  {
    \sisetup{fill-arc-minutes}
    \ang{-1;;} \\
    \ang{;-2;} \\
    \ang{;;-3} \\
  }
  {
    \sisetup{fill-arc-seconds}
    \ang{-1;;} \\
    \ang{;-2;} \\
    \ang{;;-3}
  }
\end{LaTeXdemo}

\DescribeOption{angle-symbol-over-decimal}
In some subject areas, most notably astronomy, the angle symbols are given over
the decimal marker, rather than at the end of the number. This behaviour is
available using the \opt{angle-symbol-over-decimal} option.
\begin{LaTeXdemo}
  \ang{45.697}  \\
  \ang{6;7;6.5} \\
  \ang[angle-symbol-over-decimal]{45.697} \\
  \ang[angle-symbol-over-decimal]{6;7;6.5}
\end{LaTeXdemo}

\subsection{Creating units}
\label{sec:units:creating}

The various macro units are created at the start of the document. \pkg{siunitx}
can define these such that they are only available for use within the \cs{unit}
and \cs{qty} functions, or can make the unit macros available throughout the
document body. There are a number of settings which control this creation
process (Table~\ref{tab:opt:units:def}). As a result, these options all apply
in the preamble only.
\begin{table}
  \caption{Unit creation options.}
  \label{tab:opt:units:def}
  \begin{tabular}{@{}>{\ttfamily}ll>{\ttfamily}l@{}}
    \toprule
      \multicolumn{1}{@{}l}{Option name} &
      Type &
      \multicolumn{1}{l@{}}{Default} \\
    \midrule
      free-standing-units    & Switch & false \\
      overwrite-command      & Switch & false \\
      space-before-unit      & Switch & false \\
      unit-optional-argument & Switch & false \\
      use-xspace             & Switch & false \\
    \bottomrule
  \end{tabular}
\end{table}

\DescribeOption{free-standing-units}
\DescribeOption{overwrite-functions}
The \opt{free-standing-units} option controls whether the unit macros exist
outside of the \cs{unit} and \cs{qty} arguments. When this option is
\opt{true}, \pkg{siunitx} creates the macros for general use. The standard
method to achieve this does not overwrite any existing macros: this behaviour
can be altered using the \opt{overwrite-commands} switch.

\DescribeOption{space-before-unit}
\DescribeOption{unit-optional-argument}
\DescribeOption{use-xspace}
When \enquote{free standing} unit macros are created, their behaviour can be
adjusted by a number of options. These are mainly intended for emulating the
input syntax of older packages. The option \opt{unit-optional-argument} gives
the same behaviour for the inputs
\begin{LaTeXdemo}[code only]
  \qty{10}{\metre}
\end{LaTeXdemo}
and
\begin{LaTeXdemo}[code only]
  \metre[10].
\end{LaTeXdemo}
The \opt{space-before-unit} and \opt{use-xspace} options control the behaviour
at the \enquote{ends} of the unit macros. Activating \opt{space-before-unit}
inserts the number--unit space before the unit is printed. This is suitable for
the input syntax
\begin{LaTeXdemo}[code only]
  30\metre
\end{LaTeXdemo}
but does mean that the unit macros are incorrectly spaced in running text. On
the other hand, the \opt{use-xspace} option attempts to correctly space input
such as
\begin{LaTeXdemo}[code only]
  \metre is the symbol for metres.
\end{LaTeXdemo}

\subsection{Using units}

Part of the power of \pkg{siunitx} is the ability to alter the output format
for units without changing the input. The behaviour of units is therefore
controlled by a number of options which alter either the processing of units or
the output directly (Table~\ref{tab:opt:units:out}).
\begin{table}
  \centering
  \caption{Unit output options.}
  \label{tab:opt:units:out}
  \begin{tabular}{@{}>{\ttfamily}ll>{\ttfamily}l@{}}
    \toprule
      \multicolumn{1}{@{}l}{Option name} &
      Type &
      \multicolumn{1}{l@{}}{Default} \\
    \midrule
      bracket-unit-denominator & Switch  & true         \\
      font-command             & Literal & \cs{mathrm}  \\
      forbid-literal-units     & Switch  & false        \\
      inter-unit-product       & Literal & \cs{,}       \\
      parse-units              & Switch  & true         \\
      per-mode                 & Choice  & power        \\
      per-symbol               & Literal & /            \\
      qualifier-mode           & Choice  & subscript    \\
      qualifier-phrase         & Literal & \meta{empty} \\
      sticky-per               & Switch  & false        \\
    \bottomrule
  \end{tabular}
\end{table}

\DescribeOption{inter-unit-product}
The separator between each unit is stored using the \opt{inter-unit-product}
option. The standard setting is a thin space: another common choice is a
centred dot. To get the correct spacing it is necessary to use
|\ensuremath{{}\cdot{}}| in the latter case.
\begin{LaTeXdemo}
  \unit{\farad\squared\lumen\candela} \\
  \unit[inter-unit-product = \ensuremath{{}\cdot{}}]
    {\farad\squared\lumen\candela}
\end{LaTeXdemo}

\DescribeOption{per-mode}
\DescribeOption{per-symbol}
\DescribeOption{bracket-unit-denominator}
The handling of \cs{per} is altered using the \opt{per-mode} choice option. The
standard setting is \opt{power}, meaning that \cs{per} generates reciprocal
powers for units. Setting the option to \opt{fraction} uses the |\frac|
function to typeset the positive and negative powers of a unit separately.
\begin{LaTeXdemo}
  \unit{\joule\per\mole\per\kelvin} \\
  \unit{\metre\per\second\squared} \\
  \unit[per-mode = fraction]{\joule\per\mole\per\kelvin} \\
  \unit[per-mode = fraction]{\metre\per\second\squared}
\end{LaTeXdemo}
The closely-related \opt{power-positive-first} setting acts in the same way but
places all of the positive powers before any negative ones.
\begin{LaTeXdemo}
  \unit{\ampere\per\mole\second} \\
  \unit[per-mode = power-positive-first]
    {\ampere\per\mole\second}
\end{LaTeXdemo}
It is possible to use a symbol (usually |/|) to separate the two parts of a
unit by setting \opt{per-mode} to \opt{symbol}; the symbol used is stored using
the setting \opt{per-symbol}. This method for displaying units can be
ambiguous, and so brackets are added unless \opt{bracket-unit-denominator} is
set to \opt{false}. Notice that \opt{bracket-unit-denominator} only applies
when \opt{per-mode} is set to \opt{symbol} or \opt{symbol-or-fraction}.
\begin{LaTeXdemo}
  \sisetup{per-mode = symbol}%
  \unit{\joule\per\mole\per\kelvin} \\
  \unit{\metre\per\second\squared} \\
  \unit[per-symbol = \ \text{div}\ ]{\joule\per\mole\per\kelvin} \\
  \unit[bracket-unit-denominator = false]{\joule\per\mole\per\kelvin}
\end{LaTeXdemo}
The often-requested (but mathematically invalid) \opt{repeated-symbol} option
is also available to repeat the symbol for each \cs{per}.
\begin{LaTeXdemo}
  \unit[per-mode = repeated-symbol]{\joule\per\mole\per\kelvin}
\end{LaTeXdemo}
Finally, it is possible for the behaviour of the \cs{per} function to depend on
the prevailing math style. Setting \opt{per-mode} to \opt{symbol-or-fraction}
will use the \opt{symbol} setting for in line math, and the \opt{fraction}
setting when used in display math.
\begin{LaTeXdemo}
  \sisetup{per-mode = symbol-or-fraction}%
  $ \unit{\joule\per\mole\per\kelvin} $
  \[ \unit{\joule\per\mole\per\kelvin} \]
     \unit{\joule\per\mole\per\kelvin} \\
  $
    \displaystyle
    \unit{\joule\per\mole\per\kelvin}
  $
  \[
    \textstyle
    \unit{\joule\per\mole\per\kelvin}
  \]
\end{LaTeXdemo}

\DescribeOption{sticky-per}
By default, \cs{per} applies only to the next unit given.\footnote{This is the
standard method of reading units in English: for example,
\unit{\joule\per\mole\per\kelvin} is pronounced \enquote{joules per mole per
kelvin}.} By setting the \opt{sticky-per} flag, this behaviour is changed so
that \cs{per} applies to all subsequent units.
\begin{LaTeXdemo}
  \unit{\pascal\per\gray\henry} \\
  \unit[sticky-per]{\pascal\per\gray\henry}
\end{LaTeXdemo}

\DescribeOption{qualifier-mode}
\DescribeOption{qualifier-phrase}
Unit qualifiers can be printed in three different formats, set by the
\opt{qualifier-mode} option. The standard setting is \opt{subscript}, while the
options \opt{bracket}, \opt{combine} and \opt{phrase} are also possible. With
the last settings, powers can lead to ambiguity and are automatically detected
and brackets added as appropriate.
\begin{LaTeXdemo}
  \unit{\kilogram\of{pol}\squared\per\mole\of{cat}\per\hour} \\
  \unit[qualifier-mode = bracket]
    {\kilogram\of{pol}\squared\per\mole\of{cat}\per\hour} \\
  \unit[qualifier-mode = combine]
    {\deci\bel\of{i}}
\end{LaTeXdemo}
The \opt{phrase} option is used with \opt{qualifier-phrase}, which allows for
example a space or other linking text to be inserted.
\begin{LaTeXdemo}
  \sisetup{qualifier-mode = phrase, qualifier-phrase = \ }%
  \unit{\kilogram\of{pol}\squared\per\mole\of{cat}\per\hour} \\
  \sisetup{qualifier-phrase = \ \mbox{of}\ }%
  \unit{\kilogram\of{pol}\squared\per\mole\of{cat}\per\hour}
\end{LaTeXdemo}

\DescribeOption{parse-units}
Normally, \pkg{siunitx} is used with the unit parse enabled, and only prints
units directly if there is literal input. However, if the input is known to be
essentially consistent and high performance is desired, then the parser can be
turned off using the \opt{parse-units} switch.
\begin{LaTeXdemo}
  \qty{300}{\MHz} \\
  \qty[parse-units = false]{300}{\MHz}
\end{LaTeXdemo}

\DescribeOption{forbid-literal-units}
Some users may prefer to completely disable the use of literal input in units,
for example to enforce consistency. This can be accomplished by setting the
\opt{forbid-literal-units} switch. With this option enabled, only macro-based
units can be used in a document. This only applies when \opt{parse-units} is
\opt{true}.

\DescribeOption{font-command}
The command used to set unit themselves may be adjusted using the
\opt{font-command} option. This is typically set to |\mathrm|.
\begin{LaTeXdemo}
  \unit{\lumen} \\
  \unit[font-command = \mathit]{\lumen}
\end{LaTeXdemo}

\subsection{Tabular material}

Processing of material in tables obeys the same settings as described for the
functions already described. However, there are some settings which apply only
to the layout of tabular material (Table~\ref{tab:opt:tables}).
\begin{table}
  \caption{Options for tabular material.}
  \label{tab:opt:tables}
  \begin{tabular}{@{}>{\ttfamily}ll>{\ttfamily}l@{}}
    \toprule
      \multicolumn{1}{@{}l}{Option name} &
      Type &
      \multicolumn{1}{l@{}}{Default} \\
    \midrule
      table-align-comparator    & Switch  & true   \\
      table-align-exponent      & Switch  & true   \\
      table-align-text-after    & Switch  & true   \\
      table-align-text-before   & Switch  & true   \\
      table-align-uncertainty   & Switch  & true   \\
      table-alignment           & Meta    & center \\
      table-alignment-mode      & Choice  & marker \\
      table-auto-round          & Switch  & false  \\
      table-column-width        & Length  & 0pt    \\
      table-fixed-width         & Switch  & false  \\
      table-format              & Special & 2.2    \\
      table-number-alignment    & Choice  & center \\
      table-text-alignment      & Choice  & center \\
    \bottomrule
  \end{tabular}
\end{table}

\DescribeOption{table-mode}
The method used by \pkg{siunitx} to align numbers is selected using the
\opt{table-mode} option, which may be one of \opt{marker}, \opt{format} or
\meta{none}. With the standard setting, \meta{marker}, the package centers the
decimal marker in a tabular column, potentially leaving white space at the
shorter end of a number. The \meta{format} mode uses information from the
\meta{table-format} key to construct a model: this is then used to define the
space available to a number. For asymmetrical numbers, this method is strongly
preferable. Finally, \opt{none} disables alignment entirely: numbers are simply
parsed.

\DescribeOption{table-number-alignment}
When \opt{table-mode} is set to \meta{format} or \meta{none}, the placement of
the number \enquote{block} within the cell as a whole is set by the
\meta{table-number-alignment} option, which may be one of \meta{left},
\meta{center} or \meta{right}. (When \opt{table-mode} is set to \meta{marker},
the decimal marker is always centered in the cell.) The different alignment
choices are illustrated in Table~\ref{tab:S:align}, which uses somewhat
exaggerated column headings to show the relative position of the cell contents.
\begin{LaTeXdemo}[code and float]
  \begin{table}
    \caption{Aligning the \texttt{S} column.}
    \label{tab:S:align}
    \centering
    \sisetup{table-format = 2.4, table-alignment-mode = format}
    \begin{tabular}{@{}
      S[table-alignment-mode = marker]
      S[table-number-alignment = center]
      S[table-number-alignment = left]
      S[table-number-alignment = right]
    @{}}
    \toprule
    {Some Values} & {Some Values} & {Some Values} & {Some Values} \\
    \midrule
       2.3456  &  2.3456  &  2.3456 &  2.3456 \\
      34.2345  & 34.2345  & 34.2345 & 34.2345 \\
      56.7835  & 56.7835  & 56.7835 & 56.7835 \\
      90.473   & 90.473   & 90.473  & 90.473  \\
    \bottomrule
    \end{tabular}
  \end{table}
\end{LaTeXdemo}
When the alignment mode is set to \opt{none}, number are simply collected and
parsed without any further processing, as illustrated in
Table~\ref{tab:S:parse}.
\begin{LaTeXdemo}[code and float]
  \begin{table}
    \caption{Parsing without aligning in an \texttt{S} column.}
    \label{tab:S:parse}
    \begin{tabular}
      {@{}
        S
        S[table-alignment-mode = none]
      @{}}
      \toprule
        {Decimal-centred} &
        {Simple centring} \\
      \midrule
        12.345   & 12.345   \\
        6,78     & 6,78     \\
        -88.8(9) & -88.8(9) \\
        4.5e3    & 4.5e3    \\
      \bottomrule
    \end{tabular}
  \end{table}
\end{LaTeXdemo}

\DescribeOption{table-format}
When the \opt{table-alignment-mode} is set to \opt{format}, \pkg{siunitx} uses
the information set in \meta{table-format} to construct a \enquote{model} which
defines the space to reserve for a number. The \opt{table-format} key is
interpreted in much the same way as a table cell. The numerical part should
consist of a number showing how many figures to reserve in each part of the
input, plus any comparators, signs, \foreign{etc.} A variety of examples are
given in Table~\ref{tab:S:format}.
\begin{LaTeXdemo}[code and float]
  \begin{table}
    \caption{Reserving space in \texttt{S} columns.}
    \label{tab:S:format}
    \sisetup{
      table-alignment-mode   = format,
      table-number-alignment = center,
    }
    \begin{tabular}{@{}
      S[table-format = 2.2]
      S[table-format = 2.2, table-number-alignment = right]
      S[table-format = 2.2(1)]
      S[table-format = 2.2(1), separate-uncertainty]
      S[table-format = +2.2]
      S[table-format = 2.2e1]
     @{}}
    \toprule
        {Values}
      & {Values}
      & {Values}
      & {Values}
      & {Values}
      & {Values} \\
    \midrule
       2.3  &  2.3  &  2.3(5)  &  2.3(5)  &   2.3  &  2.3e8  \\
      34.23 & 34.23 & 34.23(4) & 34.23(4) &  34.23 & 34.23   \\
      56.78 & 56.78 & 56.78(3) & 56.78(3) & -56.78 & 56.78e3 \\
       3,76 &  3,76 &  3,76(2) &  3.76(2) & +-3.76 &      e6 \\
    \bottomrule
    \end{tabular}
  \end{table}
\end{LaTeXdemo}
It is important to note that any parts of a number \emph{not} specified in the
table format argument are set to be absent (the number of figures is set to
zero). Setting the \opt{table-format} option also resets
\opt{table-alignment-mode} to \opt{format}.

Space for material before and after the \texttt{S} column can be reserved by
giving model text as part of the \opt{table-format} key. This is then used to
provide the necessary gap while maintaining alignment (Table~\ref{tab:S:ends}).
\begin{LaTeXdemo}[code and float]
  \begin{table}
    \caption{Text before and after numbers.}
    \label{tab:S:ends}
    \sisetup{table-format = {now }2.4{\textsuperscript{\emph{a}}}}
    \begin{tabular}{@{}S@{}}
    \toprule
    {Values} \\
    \midrule
            2.3456  \\
           34.2345  \textsuperscript{\emph{a}}\\
           56.7835  \\
      now~ 90.473   \\
    \bottomrule
    \end{tabular}
  \end{table}
\end{LaTeXdemo}

\DescribeOption{table-align-exponent}
\DescribeOption{table-align-uncertainty}
When printing exponents in tables, there is a choice of aligning the exponent
parts or having these close up to the mantissa. This is controlled by the
\opt{table-align-exponent} option (Table~\ref{tab:align:exp}). Similarly,
uncertainty parts which are printed separately from the mantissa can be aligned
or closed up. This is set by the \opt{table-align-uncertainty} option
(Table~\ref{tab:align:uncert}). Finally, the same approach is available for the
comparator with the \opt{table-align-comparator} option
(Table~\ref{tab:align:comp}).
\begin{LaTeXdemo}[code and float]
  \begin{table}
    \caption{The \opt{table-align-exponent} option}
    \label{tab:align:exp}
    \sisetup{table-format = 1.3e2}
    \begin{tabular}{@{}SS[table-align-exponent = false]@{}}
      \toprule
        {Header} & {Header} \\
      \midrule
        1.2e3    & 1.2e3    \\
        1.234e56 & 1.234e56 \\
      \bottomrule
    \end{tabular}
  \end{table}
\end{LaTeXdemo}
\begin{LaTeXdemo}[code and float]
  \begin{table}
    \caption{The \opt{table-align-uncertainty} option}
    \label{tab:align:uncert}
    \sisetup{
      separate-uncertainty,
      table-format = 1.3(1),
    }
    \begin{tabular}{@{}SS[table-align-uncertainty = false]@{}}
      \toprule
        {Header} & {Header} \\
      \midrule
        1.2(1)   & 1.2(3)    \\
        1.234(5) & 1.234(5) \\
      \bottomrule
    \end{tabular}
  \end{table}
\end{LaTeXdemo}
\begin{LaTeXdemo}[code and float]
  \begin{table}
    \caption{The \opt{table-align-comparator} option}
    \label{tab:align:comp}
    \sisetup{table-format = >2.2}
    \begin{tabular}{@{}SS[table-align-comparator = false]@{}}
      \toprule
        {Header} & {Header} \\
      \midrule
        >  1.2  & >  1.2  \\
        < 12.34 & < 12.34 \\
      \bottomrule
    \end{tabular}
  \end{table}
\end{LaTeXdemo}

\DescribeOption{table-align-text-before}
\DescribeOption{table-align-text-after}
Note markers are often given in tables after the numerical content. It may be
desirable for these to close up to the numbers. Whether this takes place is
controlled by the \opt{table-align-text-before} and \opt{\ldots-after}
 option (Table~\ref{tab:S:notes}).
\begin{LaTeXdemo}[code and float]
  \begin{table}
    \caption{Closing notes up to text.}
    \label{tab:S:notes}
    \newrobustcmd\NoteMark[1]{%
      \textsuperscript{\emph{#1}}%
    }
    \sisetup{table-format = {\NoteMark{a}}2.4}
    \begin{tabular}{@{}
      S
      S[table-align-text-before = false]
    @{}}
    \toprule
      {Values}             & {Values} \\
    \midrule
                    2.3456 &               2.3456 \\
      \NoteMark{a}  4.234  & \NoteMark{a}  4.234 \\
      \NoteMark{b}   .78   & \NoteMark{b}   .78  \\
      \NoteMark{d} 88      & \NoteMark{d} 88     \\
    \bottomrule
    \end{tabular}
    \hfil
    \sisetup{table-format = 2.4\NoteMark{a}}
    \begin{tabular}{
      S
      S[table-align-text-after = false]
    }
    \toprule
      {Values}             & {Values} \\
    \midrule
       2.3456              & 2.3456 \\
      34.234  \NoteMark{a} & 34.234 \NoteMark{a} \\
      56.78   \NoteMark{b} & 56.78  \NoteMark{b} \\
      90.4    \NoteMark{c} & 90.4   \NoteMark{c} \\
      88      \NoteMark{d} & 88     \NoteMark{d} \\
    \bottomrule
    \end{tabular}
  \end{table}
\end{LaTeXdemo}

\DescribeOption{table-auto-round}
The contents of table cells can automatically be rounded or zero-filled to the
number of decimal digits given for the decimal part of the \opt{table-format}
option. This mode is activated using the \opt{table-auto-round} switch, as
illustrated in Table~\ref{tab:S:auto}.
\begin{LaTeXdemo}[code and float]
  \begin{table}
    \centering
    \caption{The \opt{table-auto-round} option.}
    \label{tab:S:auto}
    \sisetup{table-format = 1.3}
    \begin{tabular}{@{}SS[table-auto-round]@{}}
      \toprule
        {Header} & {Header} \\
      \midrule
        1.2    & 1.2    \\
        1.2345 & 1.2345 \\
      \bottomrule
    \end{tabular}
  \end{table}
\end{LaTeXdemo}

\DescribeOption{parse-numbers}
When the \opt{parse-numbers} option is set to \opt{false}, then the alignment
code for tables takes a different approach. The output is always set in math
mode, and alignment takes place at the first decimal marker. This is achieved
by making it active in math mode. When reserving space for content only the
integer and decimal values for the mantissa are considered
(Table~\ref{tab:S:nonparsed}).
\begin{LaTeXdemo}[code and float]
  \begin{table}
    \caption{Aligning without parsing.}
    \label{tab:S:nonparsed}
    \sisetup{
      parse-numbers = false,
      table-format  = 3.3
    }
    \centering
    \begin{tabular}{@{}
      S
      S[table-number-alignment = center]
      S[table-number-alignment = right]
      S[table-number-alignment = left]
      @{}}
    \toprule
        {Some values}
      & {Some values}
      & {Some values}
      & {Some values} \\
    \midrule
       2.35  &    2.35  &    2.35  &   2.35   \\
      34.234 &   34.234 &   34.234 &  34.234  \\
      56.783 &   56.783 &   56.783 &  56.783  \\
       3,762 &    3,762 &    3,762 &   3.762  \\
    \sqrt{2} & \sqrt{2} & \sqrt{2} & \sqrt{2} \\
    \bottomrule
    \end{tabular}
  \end{table}
\end{LaTeXdemo}

\DescribeOption{drop-exponent}
In cases where data cover a range of values, printing using a fixed exponent in
a table may make presentation clearer. In these cases, dropping the exponent
value from the table is useful. The general numerical options \opt{drop-exponent}
combined with \opt{exponent-mode = fixed} can be used to achieve this
(Table~\ref{tab:exp:omit}).
\begin{LaTeXdemo}[code and float]
  \begin{table}
    \caption{The \opt{table-omit-exponent} option}
    \label{tab:exp:omit}
    \begin{tabular}{@{}
      S[table-format = 1.1e1]
      S[
        drop-exponent  = true,
        exponent-mode  = fixed,
        fixed-exponent = 3,
        table-format   = 2.1,
      ]
    @{}}
      \toprule
        {Header} & \multicolumn{1}{c@{}}{Header / \num{e3}} \\
      \midrule
        1.2e3 & 1.2e3 \\
         3e2  &   3e2 \\
       1.0e4  & 1.0e4 \\
      \bottomrule
    \end{tabular}
  \end{table}
\end{LaTeXdemo}

\DescribeOption{table-column-width}
\DescribeOption{table-fixed-width}
Usually, the width of the numerical column is allowed to vary depending on the
content. However, there are cases where a strictly fixed width is desirable.
For these cases, the \opt{table-fixed-width} and \opt{table-column-width}
options are available. The \opt{table-fixed-width} option activates fixed-width
columns, whilst \opt{table-column-width} defines the target width
(Table~\ref{tab:width:fixed}).
\begin{LaTeXdemo}[code and float]
  \begin{table}
    \caption{Fixed-width columns.}
    \label{tab:width:fixed}
    \begin{tabular}
      {@{}
        S
        S[table-column-width = 2cm]
      @{}}
      \toprule
        {Flexible} &
        {Fixed}    \\
      \midrule
        1.23 & 1.23 \\
        45.6 & 45.6 \\
      \bottomrule
    \end{tabular}
  \end{table}
\end{LaTeXdemo}

The \opt{table-column-width} option can also be used to achieve special
effects. One example is centring a column of numbers under a wide heading, with
the numbers themselves right-aligned (Table~\ref{tab:width:special}).
\begin{LaTeXdemo}[code and float]
  \begin{table}
    \centering
    \caption{Right-aligning under a heading.}
    \label{tab:width:special}
    \settowidth{\mylength}{Long header}
    \sisetup{
      table-alignment-mode   = none      ,
      table-column-width     = \mylength ,
      table-number-alignment = right
    }
    \begin{tabular}{@{}S@{}}
    \toprule
    {Long header} \\
    \midrule
        12.33 \\
         2    \\
      1234    \\
     \bottomrule
    \end{tabular}
  \end{table}
\end{LaTeXdemo}

\DescribeOption{table-text-alignment}
\DescribeOption{table-alignment}
Cell contents which are not part of a number can be protected using braces, as
illustrated. Cells which contain no numerical data at all are aligned using the
setting specified by the \opt{table-text-alignment} option, which recognises
the values \opt{center}, \opt{left} and \opt{right} (Table~\ref{tab:S:text}).
\begin{LaTeXdemo}[code and float]
  \begin{table}
    \caption{Aligning text in \texttt{S} columns.}
    \label{tab:S:text}
    \sisetup{table-format = 4.4}
    \centering
    \begin{tabular}{@{}
      S
      S[table-text-alignment = left]
      S[table-text-alignment = right]
    @{}}
    \toprule
        {Values}
      & {Values}
      & {Values} \\
    \midrule
      992.435  &  992.435  &  992.435  \\
     7734.2344 & 7734.2344 & 7734.2344 \\
       56.7834 &   56.7834 &   56.7834 \\
        3,7462 &    3,7462 &    3,7462 \\
    \bottomrule
    \end{tabular}
  \end{table}
\end{LaTeXdemo}

\DescribeOption{table-alignment}
The table alignment options \opt{table-number-alignment} and
\opt{table-text-alignment} can be set to the same value using the
\opt{table-alignment} option. This will set all three alignment options to the
same value (one of \opt{center}, \opt{right} or \opt{left}).

\DescribeOption{locale}
\pkg{siunitx} allows the user to switch between the typographic conventions of
different (geographical) areas by using locales. Currently, the package is
supplied with configurations for locales \opt{UK}, \opt{US}, \opt{DE}
(Germany), \opt{FR} (French) and \opt{ZA} (South Africa). The \opt{locale}
option is used to switch to a particular locale.
\begin{LaTeXdemo}
  \qty{1.234}{\metre}\\
  \qty[locale = DE]{6.789}{\metre}
\end{LaTeXdemo}

\section{Upgrading}

\subsection{Upgrading from version~$2$}

The package has been largely re-written internally between versions $2$ and
$3$. A significant number of key--value setting s have new, more descriptive,
names. Where possible, older names are mapped to newer ones internally: you
will be warned in the log if this is the case.

The font control system has been completely re-written for version~$3$. The
method used in entirely different from version~$2$. Emulation is therefore not
provided: if you need non-standard font settings, you will need to adjust your
source. See Section~\label{sec:print} for more details on the options available
in this area.

\section{Localisation}

The \pkg{translations} package provides a structured framework for localisation
of words and phrases, and is part of the larger \pkg{beamer} bundle. The
\pkg{translations} package provides the \cs{translate} macro, which will provide
appropriate translations based on the current \pkg{babel} or \pkg{polyglossia}
language setting.

If \pkg{translations} is available, \pkg{siunitx} will load it and alter the
standard settings for the \opt{list-final-separator} and \opt{range-phrase}
options to read:
\begin{LaTeXdemo}[code only]
  \sisetup{
    list-final-separator = { \GetTranslation{and} },
    list-pair-separator  = { \GetTranslation{and} },
    range-phrase         = { \GetTranslation{to (numerical range)} },
  }
\end{LaTeXdemo}
If the current language is known to the \pkg{translator} package then the
result will be localised text. The preamble for this manual loads English,
French, German and Spanish as options, and also loads the \pkg{babel} package:
\begin{LaTeXdemo}
  % In English by default
  \numlist{1;2;3} \\
  \numrange{1}{10} \\
  \selectlanguage{french}%
  \numlist{1;2;3} \\
  \numrange{1}{10} \\
  \selectlanguage{german}%
  \numlist{1;2;3} \\
  \numrange{1}{10} \\
  \selectlanguage{spanish}%
  \numlist{1;2;3} \\
  \numrange{1}{10}
\end{LaTeXdemo}

\section{Hints for using \pkg{siunitx}}

\subsection{Ensuring text or math output}
\label{sec:hint:text-math}

The macros \cs{ensuremath} and \cs{text} should be used to ensure that a
particular item is always printed in the desired mode. Some mathematical output
does not work well in \cs{mathrm} (the font setting used by \pkg{siunitx} for
printing units). The easiest way to solve this is to use the construction
"\text{\ensuremath{...}}", which will print the material in the standard
mathematics font without affecting the rest of the output. In some cases,
simply forcing \cs{mathnormal} will suffice, but this is less reliable with
non-Latin characters.

\subsection{Expanding content in tables}
\label{sec:hint:expanding}

When processing tables, \pkg{siunitx} will expand anything stored inside a
macro, unless it is long or protected. \LaTeXe{} robust commands are also
detected and are not expanded (Table~\ref{tab:xmpl:macro}). Values which would
otherwise be expanded can be protected by wrapping them in a set of braces. As
\TeX{} itself will expand the first token in a table cell before \pkg{siunitx}
can act on it, using the \eTeX{} protected mechanism is the recommended course
of action to prevent expansion of macros in table cells. (As is shown in the
table, \TeX's expansion of \LaTeXe{} robust commands can lead to unexpected
results.)
\begin{LaTeXdemo}[code and float]
  \begin{table}
    \caption{Values as macros in \texttt{S} columns.}
    \label{tab:xmpl:macro}
    \newcommand*\myvaluea{1234}
    \newcommand\myvalueb{1234}
    \DeclareRobustCommand*\myvaluec{1234}
    \protected\def\myvalued{1234}
    \begin{tabular}{@{}S@{}}
    \toprule
      {Some Values} \\
    \midrule
       \myvaluea  8.8 \myvaluea  \\ % Both expanded
       \myvalueb  8.8 \myvalueb  \\ % First expanded by TeX
                                    % to numbers
       \myvaluec  8.8 \myvaluec  \\ % First expanded by TeX
                                    % but not to numbers!
       \myvalued  8.8 \myvalued  \\ % Neither expanded
      {\myvaluea\ 8.8 \myvaluea} \\ % Neither expanded
    \bottomrule
    \end{tabular}
  \end{table}
\end{LaTeXdemo}

It is possible to use calculated values in tables. For this to work, the
calculation must take place before attempting to parse the number (the parser
cannot \enquote{know} all of the possible values inside an expression). This is
most conveniently handled using the \pkg{xfp} package, which is distributed as
part of the required support for \pkg{siunitx}. The general approach is
illustrated in Table~\ref{tab:xmpl:calc}
\begin{LaTeXdemo}[code and float]
  \begin{table}
    \caption{Calculated values.}
    \label{tab:xmpl:calc}
    \newcommand{\valuea}{66.7012}
    \newcommand{\valueb}{66.0212}
    \newcommand{\valuec}{64.9026}
    \begin{tabular}{
      @{}
      S[table-format = 2.4]
      S[table-format = 3.4]
      @{}
    }
      \toprule
        {Value} & {Doubled} \\
      \midrule
        \valuea & \fpeval{\valuea * 2} \\
        \valueb & \fpeval{\valueb * 2} \\
        \valuec & \fpeval{\valuec * 2} \\
      \bottomrule
    \end{tabular}
  \end{table}
\end{LaTeXdemo}

A more sophisticated approach is to generate the rows themselves from a
database: this is illustrated in Section~\ref{sec:hint:datatool}.

\subsection{Using \pkg{siunitx} with \pkg{datatool}}
\label{sec:hint:datatool}

As illustrated in Table~\ref{tab:xmpl:calc}, \pkg{siunitx} can be used to
typeset data stored using \pkg{datatool}. For quickly displaying the contents
of tables, \pkg{datatool} offers the \cs{DTLshowtable} macro. This will only
work with \texttt{S} columns if number parsing is turned off
(Table~\ref{tab:xmpl:datatool}).
\begin{LaTeXdemo}[code and float]
  \DTLnewdb{moredata}
  \DTLnewrow{moredata}\DTLnewdbentry{moredata}{value}{ 6.7012}
  \DTLnewrow{moredata}\DTLnewdbentry{moredata}{value}{66.0212}
  \DTLnewrow{moredata}\DTLnewdbentry{moredata}{value}{64.902 }
  \begin{table}
    \caption{Displaying a \textsf{datatool} table.}
    \label{tab:xmpl:datatool}
    \sisetup{parse-numbers= false, table-format = 2.4}
    \renewcommand*\dtlrealalign{S}
    \DTLdisplaydb{moredata}
  \end{table}
\end{LaTeXdemo}

The \pkg{datatool} package may also be used to create on-the-fly tables using
calculations. For example, the demonstration in Table~\ref{tab:xmpl:calc} may
be achieved without needing to write out each row, as shown in
Table~\ref{tab:xmpl:datatool-calc}. An extra column is used to contain the
calculations in this case.
\begin{LaTeXdemo}[code and float]
  \begin{table}
    \caption{Calculated values using \pkg{datatool}.}
    \label{tab:xmpl:datatool-calc}
    \DTLnewdb{data}
    \DTLnewrow{data}\DTLnewdbentry{data}{value}{66.7012}
    \DTLnewrow{data}\DTLnewdbentry{data}{value}{66.0212}
    \DTLnewrow{data}\DTLnewdbentry{data}{value}{64.9026}
    \begin{tabular}{
      @{}
      S[table-format = 2.4]
      S[table-format = 3.4]
      @{}l
      @{}
    }
      \toprule
        {Value} & {Doubled} &
      \DTLforeach{data}{\myvalue=value}{%
        \DTLiffirstrow {\\ \midrule}{\\}%
        \myvalue &            % First column
        \fpeval{\myvalue * 2} % second column
        & }\\
      \bottomrule
    \end{tabular}
  \end{table}
\end{LaTeXdemo}

\subsection{Using units in section headings and bookmarks}
\label{sec:hint:bookmarks}

The \pkg{siunitx} code is designed to work correctly with functions in
headings. They will print correctly in headings and in the table of contents.
As illustrated here, the standard behaviour is to ignore font changes. When the
\pkg{hyperref} package is loaded, the functions automatically \enquote{degrade
gracefully} to produce useful information in \acro{pdf} bookmarks. If you want
more control over the bookmark text, use the \cs{texorpdfstring} function from
\pkg{hyperref}, for example:
\begin{LaTeXdemo}[code only]
  \section{Some text
    \texorpdfstring
      {\unit{\joule\per\mole\per\kelvin}}
      {J mol-1 K-1}%
    }
\end{LaTeXdemo}

You may find it useful to load \pkg{hyperref} with the \opt{unicode} option,
as this will allow \unit{\ohm} to appear in bookmarks. Without the option,
the encoding used by \pkg{hyperref} does not support this symbol.

\subsection{A left-aligned column visually centred under a heading}
\label{sec:hint:left-column}

When you have a column of non-related numbers, the usual advice is to make
these left-aligned and then centre the resulting column under the heading. With
the \pkg{dcolumn} package, that would be done with something like
|D{x}{}{5.0}|. This is something of an abuse of the nature of a number, but can
also be done using \pkg{siunitx} (Table~\ref{tbl:xmpl:unrel}).
\begin{LaTeXdemo}[code and float]
  \begin{table}
    \caption{Formatting unrelated numbers}
    \label{tbl:xmpl:unrel}
    \centering
    \begin{tabular}
      {
        @{}
        S[
          table-format = 5.0,
          parse-numbers = false,
          input-digits = .0123456789,
          input-decimal-markers = x
        ]
        @{}
      }
      \toprule
      \multicolumn{1}{@{}c@{}}{Header} \\
      \midrule
      120   \\
      12.3  \\
      12340 \\
      12.02 \\
      123   \\
      1     \\
      \bottomrule
    \end{tabular}
  \end{table}
\end{LaTeXdemo}

\subsection{Maximising performance}
\label{sec:hint:performance}

Both the number and unit parsers require significant effort in terms of \TeX{}
programming. For input that does not require such processing, the maximum
performance for \pkg{siunitx} can therefore be obtained by turning off both
systems:
\begin{LaTeXdemo}
  \qty{7.3}{\Hz} \\
  \qty[parse-units = false]{7.3}{\Hz} \\
  \qty[
    parse-numbers = false,
    parse-units   = false
  ]{7.3}{\Hz}
\end{LaTeXdemo}

\subsection{Special considerations for the \cs{kWh} unit}
\label{sec:hint:kWh}

The standard settings provide the \cs{kWh} unit set up with no spacing between
the \enquote{\unit{\kilo\watt}} and the \enquote{\unit{\hour}} unit to give
\enquote{\unit{\kWh}}. However, this only applies when the unit is given on its
own: combinations will follow the normal rules
\begin{LaTeXdemo}
  \unit{\kWh} \\
  \unit{\kWh\per\metre}
\end{LaTeXdemo}
This is because the unit \cs{kWh} is defined so that it can still be varied by
altering \cs{kilo}, \cs{watt} and \cs{hour}, and so that the prefix can still
be turned into a number. However, some users may prefer to have a non-flexible
macro which never adds a space. This can be achieved by redefining \cs{kWh}
with \cs{DeclareSIUnit}, by added an alternative definition
\begin{LaTeXdemo}[code only]
  \DeclareSIUnit\kWh{kWh}
  \DeclareSIUnit\KWH{kWh}
\end{LaTeXdemo}
 or of course by using literal unit input.
\begin{LaTeXdemo}
  \unit{\KWH\per\metre}\\
  \unit{kWh.m^{-1}}
\end{LaTeXdemo}

Another point to notice is that the \cs{per} macro applies to the next unit,
and not an entire unit combination. Thus in
\begin{LaTeXdemo}
  \unit{\candela\per\kWh}
\end{LaTeXdemo}
\cs{per} applies to the watts but not to the hours. In this case, the units
need to be written out in full or the \opt{sticky-per} option should be used.
\begin{LaTeXdemo}
  \unit{\candela\per\kilo\watt\per\hour} \\
  \unit[sticky-per]{\candela\per\kWh}
\end{LaTeXdemo}

\subsection{Creating a column with numbers and units}
\label{sec:hint:mixed}

Usually, numbers in a table should be given with the units in the column
heading. However, there are cases where a series of data are best presented in
a table but have different units. There are two ways to do this
(Table~\ref{tab:xmpl:mixed}). The first is to place the units in the first
column of the table, which makes sense if there are several related items in
the table. The second method is to generate two columns, one for numbers and a
second for units, and then to format these to give the visual effect of a
single column. The later effect is most appropriate when only one set of
numbers are presented in a table. This method requires cell content is
collected, easiest to do using the \pkg{collcell} package.
\begin{LaTeXdemo}[code and float]
  \begin{table}
    \caption{Tables where numbers have different units}
    \label{tab:xmpl:mixed}
    \hfil
    \begin{tabular}
      {
        @{}
        >{$}l<{$}
        S[table-format = 2.3(1)]
        S[table-format = 3.3(1)]
        @{}
      }
      \toprule
        & {One} & {Two} \\
      \midrule
      a / \unit{\angstrom}   &  1.234(2) &   5.678(4) \\
      \beta / \unit{\degree} & 90.34(4)  & 104.45(5)  \\
      \mu / \unit{\per\mm}   &  0.532    &   0.894    \\
      \bottomrule
    \end{tabular}
    \hfil
    \begin{tabular}
      {
        @{}
        S[table-format=1.3]@{\,}
        >{\collectcell\unit}l<{\endcollectcell}
        @{}
      }
      \toprule
      \multicolumn{2}{@{}c}{Heading} \\
      \midrule
      1.234 & \metre   \\
      0.835 & \candela \\
      4.23  & \joule\per\mole \\
      \bottomrule
    \end{tabular}
    \hfil
  \end{table}
\end{LaTeXdemo}

\subsection{Tables with heading rows}
\label{sec:hint:heading}

A common format for tables is to make the heading row visually distinct using a
background color and bold text. If numbers appear in such a heading row within
an \texttt{S} column then getting the appearance right can be challenging. The
best approach is to make the \cs{bfseries} macro \enquote{robust} (as
demonstrated in Section~\ref{sec:hint:expanding}), then to use this macro to
make the heading cells bold. This approach is illustrated in
Table~\ref{tab:xmpl:headers}, along with the use of \cs{rowcolor} to provide a
background color.
\begin{LaTeXdemo}[code and float]
  \begin{table}
    \caption{Header row in a table}
    \label{tab:xmpl:headers}
    \robustify\bfseries
    \centering
    \begin{tabular}
      {S[text-weight-to-math, table-format = 3.3]}
      \rowcolor[gray]{0.9}
      \bfseries 123.456 \\
       23.45  \\
      123.4   \\
        3.456 \\
    \end{tabular}
  \end{table}
\end{LaTeXdemo}

\subsection{Associating a locale with a \pkg{babel} language}
\label{sec:hint:babel}

It is possible to instruct the \pkg{babel} package to switch to a particular
\pkg{siunitx} locale when changing language. This can be done using the
\pkg{babel} \cs{extras\meta{language}} system. For example, to associate the
\texttt{DE} locale with the \texttt{german} \pkg{babel} language, the
appropriate code would be
\begin{LaTeXdemo}[code only]
  \addto\extrasgerman{\sisetup{locale = DE}}
\end{LaTeXdemo}

\section{Using (\acro{SI}) units}

Consistent and logical units are a necessity for scientific work, and have
applicability everywhere. Historically, a number of systems have been used for
physical units. SI units were introduced by the \foreign{Conférence Générale
des Poids et Mesures} (\acro{CGPM}) in 1960. SI units are a coherent system
based on seven base units, from which all other units may be derived.

At the same time, physical quantities with units are mathematical entities, and
as such way that they are typeset is important. In mathematics, changes of type
(such as using bold, italic, sans serif typeface and so on) convey information.
This means that rules exist not only for the type of units to be used under the
SI system, but also the way they should appear in print. Advice on best
practice has been made available by the \emph{National Institute of Standards
and Technology} (\acro{NIST})~\cite{NIST}.

As befits an agreed international standard, the full rules are detailed. It is
not appropriate to reproduce these in totality here; instead, a useful summary
of the key points is provided. The full details are available from the
\emph{Bureau International des Poids et Mesures}~\cite{BIPM}.

The \pkg{siunitx} package takes account of the information given here, so far
as is possible. Thus the package defaults follow the recommendations made for
typesetting numbers and units. Spacing and so forth is handled in such a way as
to make implementing the rules (relatively) easy.

\subsection{Units}

There are seven base \acro{SI} units, listed in Section~\ref{tab:unit:base}.%
\footnote{Some base units need others defined first; there is therefore a
required order of definition.} The base units have been chosen such that all
physical quantities can be expressed using an appropriative combination of
these units, needing no others and with no redundancy.

All other units within the \acro{SI} system are regarded as \enquote{derived}
from the seven base units. At the most basic, all other SI units can be
expressed as combinations of the base units. However, many units (listed in
Section~\ref{tab:unit:derived,tab:unit:accepted}) have a special name and
symbol. Most of these units are simple combinations of one or more base units
(raised to powers as appropriate). A small number of units derived from
experimental data are allowed as \acro{SI} units
(Section~\ref{tab:unit:physical}).

A series of \acro{SI} prefixes for decimal multiples and sub-multiples are
provided, and can be used as modifiers for any SI unit (either base or derived
units) with the exception of the kilogram. The prefixes are listed in
Section~\ref{tab:unit:prefix}. No space should be used between a prefix and the
unit, and only a single prefix should be used. Even the degree Celsius can be
given a prefix, for example \qty{1}{\milli\degreeCelsius}.

It is important to note that the kilogram is the only \acro{SI} unit with a
prefix as part of its name and symbol. Only single prefix may be used, and so
in the case of the kilogram prefix names are used with the unit name
\enquote{gram} and the prefix symbols are used with the unit symbol
\unit{\gram}. For example $\qty[prefix-mode = power]{1}{\milli\gram} =
\qty{1}{\milli\gram}$.

The application of SI units is meant to provide a single set of units which
ensure consistency and clarity across all areas. However, other units are
common is many areas, and are not without merit. The units provided by
\pkg{siunitx} by default do not include any of these; only units which are part
of the SI set or are accepted for use with SI units are defined. However,
several other sets of units can be loaded as optional modules. The binary
prefixes and units (Section~\ref{tab:unit:binary}) are the most obvious
example. These are \emph{not} part of the \acro{SI} specifications, but the
prefix names are derived from those in Section~\ref{tab:unit:prefix}.

Other units are normally to be avoided where possible. SI units should, in the
main, be preferred due to the advantages of clear definition and
self-consistency this brings. However, there will probably always be a place
for specialist or non-standard units. This is particularly true of units
derived from basic physical constants.

There are also many areas where non-standard units are used so commonly that to
do otherwise is difficult or impossible. For example, most synthetic chemists
measure the pressure inside vacuum apparatus in \unit{\millimetremercury},
partly because the most common gauge for the task still uses a column of
mercury metal. For these reasons, \pkg{siunitx} does define non-\acro{SI}
units.

\subsection{Mathematical meaning}

As explained earlier, a quantity combination is a single mathematical entity.
This has implications for how both the number and the unit should be printed.
Firstly, the two parts should not be separated: a quantity is a product of the
number and the unit. With the exception of the symbols for plane angles
(\unit{\degree}, \unit{\arcminute} and \unit{\arcsecond}), the \acro{BIPM}
specifies either a space or half-height (centred) dot should be
used~\cite{BIPM}.
\begin{LaTeXdemo}
  A space for \qty{10}{\percent}\\
  and also for \qty{100}{\degreeCelsius}\\
  but not for \ang{1.23}.
\end{LaTeXdemo}

The mathematical meaning of units also means that the shape, weight and family
are important. Units are supposed to be typeset in an upright, medium weight
font. Italic, bold and sans serif are all used mathematically to convey other
meanings. (In an all sanserif document, using sans serif for units is
reasonable.) The \pkg{siunitx} package defaults again follow this convention:
any local settings are ignored, and uses the current upright math font.
However, there are occasions where this may not be the most desirable
behaviour. A classic example would be in an all-bold section heading. As the
surrounding text is bold, some people feel that any units should follow this.
\begin{LaTeXdemo}
  Units should \textbf{not be bold: \qty{54}{\farad}}\\
  \textbf{But perhaps in a running block,\\
  it might look better:
    \qty[text-weight-to-math]{54}{\farad}}
\end{LaTeXdemo}

Symbols for units formed from other units by multiplication are indicated by
means of either a half-height (that is, centred) dot or a (thin) space.
\begin{LaTeXdemo}
  $\unit{\metre\second} = \text{metre second}$ \\
  $\unit{\milli\second} = \text{millisecond}$ \\
  \sisetup{inter-unit-product = \ensuremath { { } \cdot { } } }
  $\unit{\metre\second} = \text{metre second}$ \\
  $\unit{\milli\second} = \text{millisecond}$
\end{LaTeXdemo}
There are some circumstances under which it is common practice to omit any
spaces. The classic example is \unit{\kWh}, where
\enquote{\unit[inter-unit-product = \,]{\kWh}} does not add any useful
information. If using such a unit repeatedly, users of \pkg{siunitx} are
advised to create a custom unit to ensure consistency. It is important to note
that while this is common practice, it is \emph{not} allowed by the
\acro{BIPM}~\cite{BIPM}.

Symbols for units formed from other units by division are indicated by means of
a virgule (oblique stroke, slash, "/"), a horizontal line, or negative
exponents.\footnote{Notice that a virgule and a solidus are not the same
symbol.} However, to avoid ambiguity, the virgule must not be repeated on the
same line unless parentheses are used. This is ensured when using named unit
macros in \pkg{siunitx}, which will \enquote{trap} repeated division and format
it correctly. In complicated cases, negative exponents are to be preferred over
other formats.
\begin{LaTeXdemo}
  \unit{\joule\per\mole\per\kelvin} \\
  \unit[per-mode = fraction]{\joule\per\mole\per\kelvin} \\
  \unit[per-mode = symbol]{\joule\per\mole\per\kelvin}
\end{LaTeXdemo}

Products and errors should show what unit applies to each number given. Thus
\qtyproduct[product-mode = brackets]{2x3}{\metre} is an ordered set of lengths
of a geometric area, whereas \qtyproduct[product-units = single]{2x3}{\metre}
is a length (and equal to \qty{6}{\metre}). Thus, $\times$ is not a product but
is a mathematical operator; in the same way, a $2 \times 3$ matrix is not a $6$
matrix! In some areas, areas and volumes are given with separated units but a
unit raised to the appropriate power: \qtyproduct[product-units = power]{2 x
3}{\metre}. Although this does display the correct overall units, it is
potentially-confusing and is not encouraged.

Care must be taken when writing ranges of numbers. For purely numerical values,
it is common to use a en-dash to show a range, for example \enquote{see pages
1--5}. On the other hand, physical quantities could be misinterpret as negative
values if written in this way. As the quantity is a single mathematical entity,
writing the values with an en-dash followed by a single unit is also incorrect.
As a result, using the word \enquote{to} is strongly recommended.
\begin{LaTeXdemo}
  \qtyrange{1}{5}{\metre} long.
\end{LaTeXdemo}

\subsection{Graphs and tables}

In graphs and tables, repetition of the units following each entry or axis mark
is confusing and repetitive. It is therefore best to place the unit in the
label part of the information. Placing the unit in square brackets is common
but mathematically poor.\footnote{For example, for an acceleration \(a\), the
expression $[a]$ is the dimensions of $a$, \foreign{i.e.}~length per time
squared in this case.} Much better is to show division of all quantities by the
unit, which leaves the entries as unitless ratios. This is illustrated in
Tables~\ref{tab:xmpl:unitless} and~\ref{fig:xmpl:unitless}.
\begin{LaTeXdemo}[code and float]
  \begin{table}
    \caption{An example of table labelling.}
    \label{tab:xmpl:unitless}
    \sisetup{
      table-number-alignment = center,
      table-format = 1.4
    }
    \begin{tabular}{@{}cS@{}}
      \toprule
        Entry & {Length/\unit{\metre}} \\
      \midrule
        1 & 1.1234 \\
        2 & 1.1425 \\
        3 & 1.7578 \\
        4 & 1.9560 \\
      \bottomrule
    \end{tabular}
  \end{table}
\end{LaTeXdemo}
\begin{LaTeXdemo}[code and float]
  \begin{figure}
    \begin{tikzpicture}
      \begin{axis}[
        xlabel = $t/\unit{\second}$,
        xmax   = 6,
        xmin   = 0,
        ylabel = $d/\unit{\metre}$,
        ymin   = 0
      ]
        \addplot[smooth,mark=*]
          plot coordinates {
            (0,0)
            (1,5)
            (2,8)
            (3,9)
            (4,8)
            (5,5)
            (6,0)
           };
      \end{axis}
    \end{tikzpicture}
    \caption{An example of graph labelling.}
    \label{fig:xmpl:unitless}
  \end{figure}
\end{LaTeXdemo}

 In most cases, adding exponent values in the body of a table is
 less desirable than adding a fixed exponent to column headers.  An
 example is shown in Section~\ref{tab:good}.  The use of \cs{multicolumn} is
 needed here due to the \enquote{\texttt{<}}; without \cs{multicolumn},
 the titles are followed by \enquote{\unit{\kilo\gram}}!
\begin{LaTeXdemo}[code and float]
  \begin{table}
    \caption{Good and bad columns.}
    \label{tab:good}
    \sisetup{table-number-alignment = center}
    \begin{tabular}{
      @{}
      c
      S[table-format = 1.3e1]
      @{\,\unit{\kilogram}}
      S[table-format = 2.2]
      @{}
    }
      \toprule
        Entry & \multicolumn{1}{c}{Mass} &
          {Mass/\qty{e3}{\kilogram}} \\
      \midrule
        1 & 4.56e3  & 4.56 \\
        2 & 2.40e3  & 2.40 \\
        3 & 1.345e4 & 13.45 \\
        4 & 4.5e2   & 0.45 \\
      \bottomrule
    \end{tabular}
  \end{table}
\end{LaTeXdemo}

\section{Installation}

For most users, there will be no need to explicitly install \pkg{siunitx}:
it is available from the package management system in current \TeX{} Live
and MiK\TeX{} systems.

For manual installation, the package is available from
\href{http://ctan.org/pkg/siunitx}{\acro{CTAN}}. As well as the raw source
files, \acro{CTAN} hold the package as a pre-extracted zip file,
\file{siunitx.tds.zip}. The later is most convenient for most users: simply
unzip this in your local \path{texmf} directory.

The package requires \LaTeX3 support as provided in the \pkg{l3kernel} and
\pkg{l3packages} bundles. Both of these are included in \TeX{} Live and
MiK\TeX{}, or are again available in ready-to-install form from \acro{CTAN}.

\section{Thanks}

Many users have provided feedback, bug reports and ideas for new features for
\pkg{siunitx}: thanks to all of them. Particular thanks to Stefan Pinnow, who
has taken the lead role as beta tester for \pkg{siunitx}, finding incorrect
output, bad documentation and the odd spelling mistake in the documentation.
Thanks also to Enrico Gregorio for encouraging me to complete a fully
\pkg{expl3}-compliant version of the package. Thanks also to Danie Els and
Marcel Heldoorn for the \pkg{SIstyle} and \pkg{SIunits} packages, respectively,
which provided the starting point for the development of \pkg{siunitx}.

\section{Making suggestions and reporting bugs}

Feedback on \pkg{siunitx} is always welcome, either to make suggestions or to
report problems. When sending feedback, it is always useful if a small example
file is included, showing the bug being reported or illustrating the desired
output. It is helpful if a \enquote{reference rendering} is included, showing
what the output should look like. A typical example file might read
\begin{verbatim}
  \listfiles
  % Use the article class unless the problem is class-dependent
  \documentclass{article}
  \usepackage{siunitx}
  % Other packages loaded as required
  \begin{document}
  Reference output: $1.23\,\mathrm{m}$

  siunitx output: \qty{1.23}{\metre}
  \end{document}
\end{verbatim}
As illustrated, it is usually best to use the \cls{article} class and to only
load packages which are needed to show the issue. It is also useful to include
a copy of the log file generate by \LaTeX{} when reporting a bug (as the
versions of packages can be important to solving the issue).

Feedback can be sent in a range of ways. The development code and issue tracker
are hosted on \href{https://github.com/josephwright/siunitx/}{GitHub}. Issues
opened there are visible to other users and makes sure that they cannot be
forgotten.

\end{documentation}

\begin{thebibliography}{10}
  \bibitem{BIPM}
    \emph{The International System of Units (SI)},
    \url{https://www.bipm.org/en/measurement-units/}.
  \bibitem{NIST}
    \emph{International System of Units from \acro{NIST}},
    \url{http://physics.nist.gov/cuu/Units/index.html}.
  \bibitem{SI:2.1}
    \emph{SI base units},
    \url{https://www.bipm.org/en/publications/si-brochure/section2-1.html}.
  \bibitem{SI:2.2.2}
    \emph{Units with special names and symbols; units that
      incorporate special names and symbols},
    \url{https://www.bipm.org/en/publications/si-brochure/section2-2-2.html}.
  \bibitem{SI:3.1}
    \emph{SI Prefixes},
    \url{https://www.bipm.org/en/publications/si-brochure/chapter3.html}.
  \bibitem{SI:5.3.3}
    \emph{Formatting the value of a quantity},
    \url{https://www.bipm.org/en/publications/si-brochure/section5-3-3.html}.
  \bibitem{SI:5.3.7}
    \emph{Stating values of dimensionless quantities, or quantities of
      dimension one},
    \url{https://www.bipm.org/en/publications/si-brochure/section5-3-7.html}.
  \bibitem{SI:T6}
    \emph{Non-SI units accepted for use with the International
      System of Units},
    \url{https://www.bipm.org/en/publications/si-brochure/table6.html}.
  \bibitem{SI:T7}
    \emph{Non-SI units whose values in SI units must be obtained
      experimentally},
    \url{https://www.bipm.org/en/publications/si-brochure/table7.html}.
  \bibitem{SI:T8}
    \emph{Other non-SI units},
    \url{https://www.bipm.org/en/publications/si-brochure/table8.html}.
  \bibitem{SI:T9}
    \emph{Non-SI units associated with the CGS and the CGS-Gaussian
      system of units},
    \url{https://www.bipm.org/en/publications/si-brochure/table9.html}.
\end{thebibliography}

\PrintIndex

\end{document}

% \iffalse meta-comment
%
% File: siunitx-number.dtx Copyright (C) 2014-2019,2021 Joseph Wright
%
% It may be distributed and/or modified under the conditions of the
% LaTeX Project Public License (LPPL), either version 1.3c of this
% license or (at your option) any later version.  The latest version
% of this license is in the file
%
%    https://www.latex-project.org/lppl.txt
%
% This file is part of the "siunitx bundle" (The Work in LPPL)
% and all files in that bundle must be distributed together.
%
% The released version of this bundle is available from CTAN.
%
% -----------------------------------------------------------------------
%
% The development version of the bundle can be found at
%
%    https://github.com/josephwright/siunitx
%
% for those people who are interested.
%
% -----------------------------------------------------------------------
%
%<*driver>
\documentclass{l3doc}
\ProvideDocumentCommand\foreign{m}{\textit{#1}}
% The next line is needed so that \GetFileInfo will be able to pick up
% version data
\usepackage{siunitx}
\begin{document}
  \DocInput{\jobname.dtx}
\end{document}
%</driver>
% \fi
%
% \GetFileInfo{siunitx.sty}
%
% \title{^^A
%   \pkg{siunitx-number} -- Parsing and formatting numbers^^A
%   \thanks{This file describes \fileversion,
%     last revised \filedate.}^^A
% }
%
% \author{^^A
%  Joseph Wright^^A
%  \thanks{^^A
%    E-mail:
%    \href{mailto:joseph.wright@morningstar2.co.uk}
%      {joseph.wright@morningstar2.co.uk}^^A
%   }^^A
% }
%
% \date{Released \filedate}
%
% \maketitle
%
% \begin{documentation}
%
% This submodule is dedicated to parsing and formatting numbers. A small number
% of \LaTeXe{} math mode commands are assumed to be available as part of the
% formatted output. The sign commands \cs{mp}, \cs{pm}, \cs{ll}, \cs{le},
% \cs{gg} and \cs{ge} are used to replace two-character input; \cs{pm}
% is also required for the output of uncertainties. The standard settings
% require \cs{times}. For the display of colored negative numbers, the command
% \cs{color} is assumed to be available. Where the latter may apply, numbers
% should be printed inside a group: note that \TeX{} grouping is not added
% \emph{within} formatted numbers as they may need to be decomposed into parts
% (see \cs{siunitx_number_output:NN}). Such a color will be the \emph{first}
% part of the result, meaning that a test for an initial |\color| and following
% brace group may be used to detect/remove/adjust this part.
%
% \section{Formatting numbers}
%
% \begin{function}{\siunitx_number_parse:nN, \siunitx_number_parse:VN}
%   \begin{syntax}
%     \cs{siunitx_number_parse:nN} \Arg{number} \meta{tl~var}
%   \end{syntax}
%   Parses the \emph{number} and stores the resulting internal representation
%   in the \meta{tl~var}. The parsing is influenced by the various key--value
%   settings for numerical input. The \meta{number} should comprise a single
%   real value, possibly with comparator, uncertainty and exponent parts.
%   If the number is invalid, or if number parsing is disabled, the result will
%   be an entirely empty \meta{tl~var}.
%
%   The structure of a valid number is:
%   \begin{quote}
%     \marg{comparator}\marg{sign}\marg{integer}\marg{decimal}
%       \marg{uncertainty}\\
%       \marg{exponent sign}\marg{exponent}
%   \end{quote}
%   where the two sign parts must be single tokens if present,
%   and all other components
%   must be given in braces. The number will have at least one digit for both the
%   \meta{integer} and \meta{exponent} parts: these are required. The
%   \meta{uncertainty} part should either be blank or contain an
%   \meta{identifier} (as a brace group), followed by one or more data entries.
%   Valid \meta{identifiers} currently are
%   \begin{itemize}
%     \item[\texttt{S}] A single symmetrical uncertainty (\foreign{e.g.}~a
%       statistical standard uncertainty)
%   \end{itemize}
% \end{function}
%
% \begin{function}{\siunitx_number_process:NN}
%   \begin{syntax}
%     \cs{siunitx_number_process:N} \meta{tl~var1} \meta{tl~var2}
%   \end{syntax}
%   Applies a set of number processing operations to the \meta{internal
%   number} stored in the \meta{tl~var1}, \foreign{viz.}~in order
%   \begin{enumerate}
%     \item Dropping uncertainty
%     \item Converting to scientific mode (or similar)
%     \item Rounding
%     \item Dropping zero decimal part
%     \item Forcing a minimum number of digits
%   \end{enumerate}
%   with the result stored in \meta{tl~var2}.
% \end{function}
%
% \begin{function}[rEXP]
%   {
%     \siunitx_number_output:N,  \siunitx_number_output:n,
%     \siunitx_number_output:NN, \siunitx_number_output:nN
%   }
%   \begin{syntax}
%     \cs{siunitx_number_output:N} \meta{number}
%     \cs{siunitx_number_output:NN} \meta{number} \meta{marker}
%   \end{syntax}
%   Formats the \meta{number} (in the \pkg{siunitx} internal format),
%   producing the result in a form suitable for typesetting in math mode.
%   The details for the formatting are controlled by a number of key--value
%   options. Note that \emph{formatting} does not apply any manipulation
%   (processing) to the number. This function is usable in an \texttt{e}-
%   or \texttt{x}-type expansion, and further uncontrolled expansion is
%   prevented by appropriate use of |\exp_not:n| internally.
%
%   In the \texttt{NN} version, the \meta{marker} token is inserted at each
%   possible alignment position in the output, \foreign{viz.}
%   \begin{itemize}
%     \item Between the comparator and the integer (\emph{before} any
%       sign for the integer)
%     \item Between the sign and the first digit of the integer
%     \item Both sides of the decimal marker
%     \item Both sides of the separated uncertainty sign (\foreign{i.e.}~after
%       the decimal part and before any integer uncertainty part)
%     \item Both sides of the decimal marker for a separated uncertainty
%     \item Both sides of the multiplication symbol for the exponent part.
%    \end{itemize}
%
%   The \texttt{n} and \texttt{nN} version take a token list, which should
%   be in the internal \pkg{siunitx} format.
% \end{function}
%
% \begin{function}{\siunitx_number_format:nN}
%   \begin{syntax}
%     \cs{siunitx_number_format:nN} \Arg{number} \meta{tl~var}
%   \end{syntax}
%    Carries out a combination of \cs{siunitx_number_parse:nN},
%    \cs{siunitx_number_process:NN} and \cs{siunitx_number_output:N} using
%    \texttt{x}-type expansion to place the result in the \meta{tl~var}. If
%    \cs{l_siunitx_number_parse_bool} if \texttt{false}, the input is simply
%    stored inside the \meta{tl~var} inside \cs{ensuremath}.
% \end{function}
%
% \begin{function}[EXP]
%   {
%     \siunitx_number_adjust_exponent:Nn ,
%     \siunitx_number_adjust_exponent:nn
%   }
%   \begin{syntax}
%     \cs{siunitx_number_adjust_exponent:Nn} \meta{number} \Arg{fp~expr}
%   \end{syntax}
%    Adjusts the exponent of the \meta{number} (in internal format) by the
%    \meta{fp~expr} and leaves the result in the input stream.
% \end{function}
%
% \begin{function}{\siunitx_number_normalize_symbols:N}
%   \begin{syntax}
%     \cs{siunitx_number_normalize_symbols:N} \meta{tl~var}
%   \end{syntax}
%    Replaces all multi-token signs and comparators in the \meta{tl~var}
%    with their single-token equivalents. Replaces any active hyphen tokens
%    with non-active versions.
% \end{function}
%
% \begin{function}[pTF, EXP]{\siunitx_if_number:n}
%   \begin{syntax}
%     \cs{siunitx_if_number_token:NTF} \Arg{tokens}
%       \Arg{true code} \Arg{false code}
%   \end{syntax}
%   Determines if the \meta{tokens} form a valid number which can be fully
%   parsed by \pkg{siunitx}.
% \end{function}
%
% \begin{function}[TF]{\siunitx_if_number_token:N}
%   \begin{syntax}
%     \cs{siunitx_if_number_token:NTF} \Arg{token}
%       \Arg{true code} \Arg{false code}
%   \end{syntax}
%   Determines if the \meta{token} is valid in a number based on those
%   tokens currently set up for detection in a number.
% \end{function}
%
% \begin{variable}{\l_siunitx_bracket_ambiguous_bool}
%   A switch to control whether ambiguous numbers are bracketed: this can
%   also be covered in quantity formatting by a setting there.
% \end{variable}
%
% \begin{variable}{\l_siunitx_number_parse_bool}
%   A switch to control whether any parsing is attempted for numbers.
% \end{variable}
%
% \begin{variable}
%   {
%     \l_siunitx_number_comparator_tl ,
%     \l_siunitx_number_exponent_tl   ,
%     \l_siunitx_number_sign_tl
%   }
%   The list of  possible input comparators, exponent markers and signs.
% \end{variable}
%
% \begin{variable}
%   {\l_siunitx_number_input_decimal_tl, \l_siunitx_number_output_decimal_tl}
%   The list of  possible input decimal marker(s), and the output marker.
% \end{variable}
%
% \subsection{Key--value options}
%
% The options defined by this submodule are available within the \pkg{l3keys}
% |siunitx| tree.
%
% \begin{function}{bracket-ambiguous-numbers}
%   \begin{syntax}
%     |bracket-ambiguous-numbers| = |true|\verb"|"|false|
%   \end{syntax}
% \end{function}
%
% \begin{function}{bracket-negative-numbers}
%   \begin{syntax}
%     |bracket-negative-numbers| = |true|\verb"|"|false|
%   \end{syntax}
% \end{function}
%
% \begin{function}{drop-exponent}
%   \begin{syntax}
%     |drop-exponent| = |true|\verb"|"|false|
%   \end{syntax}
% \end{function}
%
% \begin{function}{drop-uncertainty}
%   \begin{syntax}
%     |drop-uncertainty| = |true|\verb"|"|false|
%   \end{syntax}
% \end{function}
%
% \begin{function}{drop-zero-decimal}
%   \begin{syntax}
%     |drop-zero-decimal| = |true|\verb"|"|false|
%   \end{syntax}
% \end{function}
%
% \begin{function}{evaluate-expression}
%   \begin{syntax}
%     |evaluate-expression| = |true|\verb"|"|false|
%   \end{syntax}
% \end{function}
%
% \begin{function}{exponent-base}
%   \begin{syntax}
%     |exponent-base| = \meta{base}
%   \end{syntax}
% \end{function}
%
% \begin{function}{exponent-mode}
%   \begin{syntax}
%     |exponent-mode| = |engineering|\verb"|"|fixed|\verb"|"|input|\verb"|"|scientific|
%   \end{syntax}
% \end{function}
%
% \begin{function}{exponent-product}
%   \begin{syntax}
%     |exponent-product| = \meta{symbol}
%   \end{syntax}
% \end{function}
%
% \begin{function}{expression}
%   \begin{syntax}
%     |expression| = \meta{expression}
%   \end{syntax}
% \end{function}
%
% \begin{function}{fixed-exponent}
%   \begin{syntax}
%     |fixed-exponent| = \meta{exponent}
%   \end{syntax}
% \end{function}
%
% \begin{function}{group-digits}
%   \begin{syntax}
%     |group-digits| = |all|\verb"|"|decimal|\verb"|"|integer|\verb"|"|none|
%   \end{syntax}
% \end{function}
%
% \begin{function}{group-minimum-digits}
%   \begin{syntax}
%     |group-minimum-digits| = \meta{value}
%   \end{syntax}
% \end{function}
%
% \begin{function}{group-separator}
%   \begin{syntax}
%     |group-separator| = \meta{symbol}
%   \end{syntax}
% \end{function}
%
% \begin{function}{input-close-uncertainty}
%   \begin{syntax}
%     |input-close-uncertainty| = \meta{tokens}
%   \end{syntax}
% \end{function}
%
% \begin{function}{input-comparators}
%   \begin{syntax}
%     |input-comparators| = \meta{tokens}
%   \end{syntax}
% \end{function}
%
% \begin{function}{input-close-uncertainty}
%   \begin{syntax}
%     |input-close-uncertainty| = \meta{tokens}
%   \end{syntax}
% \end{function}
%
% \begin{function}{input-decimal-markers}
%   \begin{syntax}
%     |input-decimal-markers| = \meta{tokens}
%   \end{syntax}
% \end{function}
%
% \begin{function}{input-digits}
%   \begin{syntax}
%     |input-digits| = \meta{tokens}
%   \end{syntax}
% \end{function}
%
% \begin{function}{input-exponent-markers}
%   \begin{syntax}
%     |input-exponent-markers| = \meta{tokens}
%   \end{syntax}
% \end{function}
%
% \begin{function}{input-open-uncertainty}
%   \begin{syntax}
%     |input-open-uncertainty| = \meta{tokens}
%   \end{syntax}
% \end{function}
%
% \begin{function}{input-signs}
%   \begin{syntax}
%     |input-signs| = \meta{tokens}
%   \end{syntax}
% \end{function}
%
% \begin{function}{input-uncertainty-signs}
%   \begin{syntax}
%     |input-uncertainty-signs| = \meta{tokens}
%   \end{syntax}
% \end{function}
%
% \begin{function}{minimum-decimal-digits}
%   \begin{syntax}
%     |minimum-decimal-digits| = \meta{min}
%   \end{syntax}
% \end{function}
%
% \begin{function}{minimum-integer-digits}
%   \begin{syntax}
%     |minimum-integer-digits| = \meta{min}
%   \end{syntax}
% \end{function}
%
% \begin{function}{negative-color}
%   \begin{syntax}
%     |negative-color| = \meta{color}
%   \end{syntax}
% \end{function}
%
% \begin{function}{output-close-uncertainty}
%   \begin{syntax}
%     |output-close-uncertainty| = \meta{symbol}
%   \end{syntax}
% \end{function}
%
% \begin{function}{output-decimal-marker}
%   \begin{syntax}
%     |output-decimal-marker| = \meta{symbol}
%   \end{syntax}
% \end{function}
%
% \begin{function}{output-open-uncertainty}
%   \begin{syntax}
%     |output-open-uncertainty| = \meta{symbol}
%   \end{syntax}
% \end{function}
%
% \begin{function}{parse-numbers}
%   \begin{syntax}
%     |parse-numbers| = |true|\verb"|"|false|
%   \end{syntax}
% \end{function}
%
% \begin{function}{print-implicit-plus}
%   \begin{syntax}
%     |print-implicit-plus| = |true|\verb"|"|false|
%   \end{syntax}
% \end{function}
%
% \begin{function}{print-unity-mantissa}
%   \begin{syntax}
%     |print-unity-mantissa| = |true|\verb"|"|false|
%   \end{syntax}
% \end{function}
%
% \begin{function}{print-zero-exponent}
%   \begin{syntax}
%     |print-zero-exponent| = |true|\verb"|"|false|
%   \end{syntax}
% \end{function}
%
% \begin{function}{retain-explicit-plus}
%   \begin{syntax}
%     |retain-explicit-plus| = |true|\verb"|"|false|
%   \end{syntax}
% \end{function}
%
% \begin{function}{retain-zero-uncertainty}
%   \begin{syntax}
%     |retain-zero-uncertainty| = |true|\verb"|"|false|
%   \end{syntax}
% \end{function}
%
% \begin{function}{round-half}
%   \begin{syntax}
%     |round-half| = |even|\verb"|"|up|
%   \end{syntax}
% \end{function}
%
% \begin{function}{round-minimum}
%   \begin{syntax}
%     |round-minimum| = \meta{min}
%   \end{syntax}
% \end{function}
%
% \begin{function}{round-mode}
%   \begin{syntax}
%     |round-mode| = |figures|\verb"|"|none|\verb"|"|places|\verb"|"|uncertainty|
%   \end{syntax}
% \end{function}
%
% \begin{function}{round-pad}
%   \begin{syntax}
%     |round-pad| = |true|\verb"|"|false|
%   \end{syntax}
% \end{function}
%
% \begin{function}{round-precision}
%   \begin{syntax}
%     |round-precision| = \meta{precision}
%   \end{syntax}
% \end{function}
%
% \begin{function}{tight-spacing}
%   \begin{syntax}
%     |tight-spacing| = |true|\verb"|"|false|
%   \end{syntax}
% \end{function}
%
% \begin{function}{uncertainty-mode}
%   \begin{syntax}
%     |uncertainty-mode| = |compact|\verb"|"|compact-marker|\verb"|"|full|\verb"|"|separate|
%   \end{syntax}
% \end{function}
%
% \begin{function}{uncertainty-separator}
%   \begin{syntax}
%     |uncertainty-separator| = \meta{separator}
%   \end{syntax}
% \end{function}
%
% \end{documentation}
%
% \begin{implementation}
%
% \section{\pkg{siunitx-number} implementation}
%
% Start the \pkg{DocStrip} guards.
%    \begin{macrocode}
%<*package>
%    \end{macrocode}
%
% Identify the internal prefix (\LaTeX3 \pkg{DocStrip} convention): only
% internal material in this \emph{submodule} should be used directly.
%    \begin{macrocode}
%<@@=siunitx_number>
%    \end{macrocode}
%
% \subsection{Initial set-up}
%
%   Variants not provided by \pkg{expl3}.
%    \begin{macrocode}
\cs_generate_variant:Nn \tl_if_blank:nTF { f }
\cs_generate_variant:Nn \tl_if_blank_p:n { f }
\cs_generate_variant:Nn \tl_if_in:NnTF { NV }
\cs_generate_variant:Nn \tl_replace_all:Nnn { NnV }
%    \end{macrocode}
%
% \begin{variable}{\l_@@_tmp_tl}
%   Scratch space.
%    \begin{macrocode}
\tl_new:N \l_@@_tmp_tl
%    \end{macrocode}
% \end{variable}
%
% \subsection{Main formatting routine}
%
% \begin{variable}{\l_@@_outputted_tl}
%   A token list for the final formatted result: may or may not be generated
%   by the parser, depending on settings which are active.
%    \begin{macrocode}
\tl_new:N \l_@@_outputted_tl
%    \end{macrocode}
% \end{variable}
%
% \begin{variable}{\l_siunitx_number_parse_bool}
%   Tracks whether to parse numbers: public as this may affect other
%   behaviors.
%    \begin{macrocode}
\tl_new:N \l_siunitx_number_parse_bool
%    \end{macrocode}
% \end{variable}
%
% \begin{variable}{\l_siunitx_number_parse_bool}
%   Top-level options.
%    \begin{macrocode}
\keys_define:nn { siunitx }
  {
    parse-numbers .bool_set:N = \l_siunitx_number_parse_bool
  }
%    \end{macrocode}
% \end{variable}
%
% \begin{macro}{\siunitx_number_format:nN}
%    \begin{macrocode}
\cs_new_protected:Npn \siunitx_number_format:nN #1#2
  {
    \group_begin:
      \bool_if:NTF \l_siunitx_number_parse_bool
        {
          \siunitx_number_parse:nN {#1} \l_@@_parsed_tl
          \siunitx_number_process:NN \l_@@_parsed_tl \l_@@_parsed_tl
          \tl_set:Nx \l_@@_outputted_tl
            { \siunitx_number_output:N \l_@@_parsed_tl }
        }
        { \tl_set:Nn \l_@@_outputted_tl { \ensuremath {#1} } }
    \exp_args:NNNV \group_end:
    \tl_set:Nn #2 \l_@@_outputted_tl
  }
%    \end{macrocode}
% \end{macro}
%
% \subsection{Parsing numbers}
%
% Before numbers can be manipulated or formatted they need to be parsed into
% an internal form. In particular, if multiple code paths are to be avoided,
% it is necessary to do such parsing even for relatively simple cases such
% as converting |1e10| to |1 \times 10^{10}|.
%
% Storing the result of such parsing can be done in a number of ways. In the
% first version of \pkg{siunitx} a series of separate data stores were used.
% This is potentially quite fast (as recovery of items relies only on \TeX{}'s
% hash table) but makes managing the various data entries somewhat tedious and
% error-prone. For version two of the package, a single data structure
% (property list) was used for each part of the parsed number. Whilst this is
% easy to manage and extend, it is somewhat slower as at a \TeX{} level there
% are repeated pack--unpack steps. In particular, the fact that there are a
% limited number of items to track for a \enquote{number} means that a more
% efficient approach is desirable (contrast parsing units, which is open-ended
% and therefore fits well with using a property list).
%
% In this release, the structure of a valid number is:
% \begin{quote}
%   \marg{comparator}\meta{sign}\marg{integer}\marg{decimal}
%     \marg{uncertainty}\\
%     \meta{exponent sign}\marg{exponent}
% \end{quote}
% where all components must be given in braces. \emph{All} of the components
% must be present in a stored number (\foreign{i.e.}~at the end of parsing).
% The number must have at least one digit for both the \meta{integer} and
% \meta{exponent} parts.
% 
% A non-empty \meta{uncertainty} must contain one leading brace group
% containing an identifier, then zero or more brace groups which contain
% the uncertainty data. In this release, the known uncertainty types are
% \begin{itemize}
%   \item \texttt{S}: A symmetrical statistical uncertainty made up of
%     a single value. These are stored as uncertainty in significant digits,
%     with no radix point in the stored value.
% \end{itemize}
%
% \begin{variable}{\l_siunitx_number_input_decimal_tl}
%   The input decimal markers(s).
%    \begin{macrocode}
\tl_new:N \l_siunitx_number_input_decimal_tl
%    \end{macrocode}
% \end{variable}
%
% \begin{variable}
%   {
%      \l_@@_expression_bool                 ,
%      \l_@@_input_uncert_close_tl           ,
%      \l_siunitx_number_input_comparator_tl ,
%      \l_@@_input_digit_tl                  ,
%      \l_siunitx_number_input_exponent_tl   ,
%      \l_@@_input_ignore_tl                 ,
%      \l_@@_input_uncert_open_tl            ,
%      \l_siunitx_number_input_sign_tl       ,
%      \l_@@_input_uncert_sign_tl            ,
%      \l_@@_explicit_plus_bool              ,
%      \l_@@_zero_uncert_bool
%   }
%  \begin{macro}[EXP]{\@@_expression:n}
%   Options which determine the various valid parts of a parsed number.
%    \begin{macrocode}
\keys_define:nn { siunitx }
  {
    evaluate-expression .bool_set:N =
      \l_@@_expression_bool ,
    expression .code:n =
      \cs_set:Npn \@@_expression:n ##1 {#1} ,
    input-close-uncertainty .tl_set:N =
      \l_@@_input_uncert_close_tl ,
    input-comparators .tl_set:N =
      \l_siunitx_number_input_comparator_tl ,
    input-decimal-markers .tl_set:N =
      \l_siunitx_number_input_decimal_tl ,
    input-digits .tl_set:N =
      \l_@@_input_digit_tl ,
    input-exponent-markers .tl_set:N =
      \l_siunitx_number_input_exponent_tl ,
    input-ignore .tl_set:N =
      \l_@@_input_ignore_tl ,
    input-open-uncertainty .tl_set:N =
      \l_@@_input_uncert_open_tl ,
    input-signs .tl_set:N =
      \l_siunitx_number_input_sign_tl ,
    input-uncertainty-signs .code:n =
      {
        \tl_set:Nn \l_@@_input_uncert_sign_tl {#1}
        \tl_map_inline:nn {#1}
          {
            \tl_if_in:NnF \l_siunitx_number_input_sign_tl {##1}
              { \tl_put_right:Nn \l_siunitx_number_input_sign_tl {##1} }
          }
      } ,
    parse-numbers .bool_set:N =
      \l_siunitx_number_parse_bool ,
    retain-explicit-plus .bool_set:N =
      \l_@@_explicit_plus_bool ,
    retain-zero-uncertainty .bool_set:N =
      \l_@@_zero_uncert_bool
  }
\cs_new:Npn \@@_expression:n #1 { }
\tl_new:N \l_@@_input_uncert_sign_tl
%    \end{macrocode}
% \end{macro}
% \end{variable}
%
% \begin{variable}{\l_@@_arg_tl}
%   The input argument or a part thereof, depending on the position in
%   the parsing routine.
%    \begin{macrocode}
\tl_new:N \l_@@_arg_tl
%    \end{macrocode}
% \end{variable}
%
% \begin{variable}{\l_@@_comparator_tl}
%   A comparator, if found, is held here.
%    \begin{macrocode}
\tl_new:N \l_@@_comparator_tl
%    \end{macrocode}
% \end{variable}
%
% \begin{variable}{\l_@@_exponent_tl}
%   The exponent part of a parsed number. It is easiest to find this
%   relatively early in the parsing process, but as it needs to go at
%   the end of the internal format is held separately until required.
%    \begin{macrocode}
\tl_new:N \l_@@_exponent_tl
%    \end{macrocode}
% \end{variable}
%
% \begin{variable}{\l_@@_flex_tl}
%   In a number with an uncertainty, the exact meaning of a second part is
%   not fully resolved until parsing is complete. That is handled using
%   this \enquote{flexible} store.
%    \begin{macrocode}
\tl_new:N \l_@@_flex_tl
%    \end{macrocode}
% \end{variable}
%
% \begin{variable}{\l_@@_parsed_tl}
%   The number parsed into internal format.
%    \begin{macrocode}
\tl_new:N \l_@@_parsed_tl
%    \end{macrocode}
% \end{variable}
%
% \begin{variable}{\l_@@_input_tl}
%   The numerical input exactly as given by the user.
%    \begin{macrocode}
\tl_new:N \l_@@_input_tl
%    \end{macrocode}
% \end{variable}
%
% \begin{variable}{\l_@@_partial_tl}
%   To avoid needing to worry about the fact that the final data stores are
%   somewhat tricky to add to token-by-token, a simple store is used to build
%   up the parsed part of a number before transferring in one go.
%    \begin{macrocode}
\tl_new:N \l_@@_partial_tl
%    \end{macrocode}
% \end{variable}
%
% \begin{variable}{\l_@@_validate_bool}
%   Used to set up for validation with no error production.
%    \begin{macrocode}
\bool_new:N \l_@@_validate_bool
%    \end{macrocode}
% \end{variable}
%
% \begin{macro}{\siunitx_number_normalize_symbols:N}
% \begin{macro}{\@@_normalize_aux:nN}
% \begin{macro}{\@@_normalize_sign:N}
% \begin{variable}{\c_@@_normalize_tl}
%   There are two parts to the replacement code. First, any active
%   hyphens signs are normalised: these can come up with some packages and
%   cause issues. Multi-token signs then are converted to the single token
%   equivalents so that everything else can work on a one token basis.
%    \begin{macrocode}
\cs_new_protected:Npn \siunitx_number_normalize_symbols:N #1
  {
    \@@_normalize_minus:N #1
    \exp_after:wN \@@_normalize_aux:NnN \exp_after:wN #1
      \c_@@_normalize_tl
      { ? } \q_recursion_tail
        \q_recursion_stop
  }
\cs_set_protected:Npn \@@_normalize_aux:NnN #1#2#3
  {
    \quark_if_recursion_tail_stop:N #3
    \tl_replace_all:Nnn #1 {#2} {#3}
    \@@_normalize_aux:NnN #1
  }
\tl_const:Nn \c_@@_normalize_tl
  {
    { -+ } \mp
    { +- } \pm
    { << } \ll
    { <= } \le
    { >> } \gg
    { >= } \ge
  }
\group_begin:
  \char_set_catcode_active:N \-
  \cs_new_protected:Npx \@@_normalize_minus:N #1
    {
      \tl_replace_all:Nnn #1
        { \exp_not:N - }  { \token_to_str:N - }
    }
\group_end:
%    \end{macrocode}
% \end{variable}
% \end{macro}
% \end{macro}
% \end{macro}
%
% \begin{macro}{\siunitx_number_parse:nN, \siunitx_number_parse:VN}
% \begin{macro}{\@@_parse:nN}
%   After some initial set up, the parser expands the input and then replaces
%   as far as possible tricky tokens with ones that can be handled using
%   delimited arguments. To avoid multiple conditionals here, the parser is
%   set up as a chain of commands initially, with a loop only later. This
%   avoids more conditionals than are necessary.
%    \begin{macrocode}
\cs_new_protected:Npn \siunitx_number_parse:nN #1#2
  {
    \bool_if:NTF \l_siunitx_number_parse_bool
      { \@@_parse:nN {#1} #2 }
      { \tl_clear:N #2 }
  }
\cs_generate_variant:Nn \siunitx_number_parse:nN { V }
\cs_new_protected:Npn \@@_parse:nN #1#2
  {
    \group_begin:
      \tl_clear:N \l_@@_parsed_tl
      \protected@edef \l_@@_arg_tl
        {
          \bool_if:NTF \l_@@_expression_bool
            { \fp_eval:n { \@@_expression:n {#1} } }
            {#1}
        }
      \tl_set_eq:NN \l_@@_input_tl \l_@@_arg_tl
      \siunitx_number_normalize_symbols:N \l_@@_arg_tl
      \tl_if_empty:NF \l_@@_arg_tl
        { \@@_parse_comparator: }
      \@@_parse_check:
    \exp_args:NNNV \group_end:
    \tl_set:Nn #2 \l_@@_parsed_tl
  }
%    \end{macrocode}
% \end{macro}
% \end{macro}
%
% \begin{macro}{\@@_parse_check:}
%   After the loop there is one case that might need tidying up. If a
%   separated uncertainty was found it will be currently in \cs{l_@@_flex_tl}
%   and needs moving. A series of tests pick up that case, then the check is
%   made that some content was found
%    \begin{macrocode}
\cs_new_protected:Npn \@@_parse_check:
  {
    \tl_if_empty:NF \l_@@_flex_tl
      {
        \bool_lazy_and:nnTF
          {
            \tl_if_blank_p:f
              { \exp_after:wN \use_iv:nnnn \l_@@_parsed_tl }
          }
          {
            \tl_if_blank_p:f
              { \exp_after:wN \use_iv:nnnn \l_@@_flex_tl }
          }
          {
            \tl_set:Nx \l_@@_tmp_tl
              { \exp_after:wN \use_i:nnnn \l_@@_flex_tl }
            \tl_if_in:NVTF \l_@@_input_uncert_sign_tl
              \l_@@_tmp_tl
              { \@@_parse_combine_uncert: }
              { \tl_clear:N \l_@@_parsed_tl }
          }
          { \tl_clear:N \l_@@_parsed_tl }
      }
    \tl_if_empty:NTF \l_@@_parsed_tl
      {
        \bool_if:NF \l_@@_validate_bool
          {
            \msg_error:nnx { siunitx } { invalid-number }
              { \exp_not:V \l_@@_input_tl }
          }
      }
      { \@@_parse_finalise: }
  }
%    \end{macrocode}
% \end{macro}
%
% \begin{macro}{\@@_parse_combine_uncert:}
% \begin{macro}{\@@_parse_combine_uncert_auxi:nnnnnnnn}
% \begin{macro}
%   {
%     \@@_parse_combine_uncert_auxii:nnnnn,
%     \@@_parse_combine_uncert_auxii:fnnnn
%   }
% \begin{macro}
%   {
%     \@@_parse_combine_uncert_auxiii:nnnnnn,
%     \@@_parse_combine_uncert_auxiii:fnnnnn
%   }
% \begin{macro}{\@@_parse_combine_uncert_auxiv:nnnn}
% \begin{macro}[EXP]{\@@_parse_combine_uncert_auxv:w}
% \begin{macro}[EXP]{\@@_parse_combine_uncert_auxvi:w}
%   Conversion of a second numerical part to an uncertainty needs a bit of
%   work. The first step is to extract the useful information from the two
%   stores: the sign, integer and decimal parts from the real number and the
%   integer and decimal parts from the second number. That is done using the
%   input stack to avoid lots of assignments.
%    \begin{macrocode}
\cs_new_protected:Npn \@@_parse_combine_uncert:
  {
    \exp_after:wN \exp_after:wN \exp_after:wN
    \@@_parse_combine_uncert_auxi:nnnnnnnn
      \exp_after:wN \l_@@_parsed_tl \l_@@_flex_tl
  }
%    \end{macrocode}
%   Here, |#4|, |#5| and |#8| are all junk arguments simply there to mop up
%   tokens, while |#1| will be recovered later from \cs{l_@@_parsed_tl} so does
%   not need to be passed about. The difference in places between the two
%   decimal parts is now found: this is done just once to avoid having to
%   parse token lists twice. The value is then used to generate a number of
%   filler |0| tokens, and these are added to the appropriate part of the
%   number. Finally, everything is recombined: the integer part only needs
%   a test to avoid an empty main number.
%    \begin{macrocode}
\cs_new_protected:Npn
  \@@_parse_combine_uncert_auxi:nnnnnnnn #1#2#3#4#5#6#7#8
  {
    \int_compare:nNnTF { \tl_count:n {#6} } > { \tl_count:n {#2} }
      {
        \tl_clear:N \l_@@_parsed_tl
        \tl_clear:N \l_@@_flex_tl
      }
      {
        \@@_parse_combine_uncert_auxii:fnnnn
          { \int_eval:n { \tl_count:n {#3} - \tl_count:n {#7} } }
          {#2} {#3} {#6} {#7}
      }
  }
\cs_new_protected:Npn
  \@@_parse_combine_uncert_auxii:nnnnn #1
  {
    \@@_parse_combine_uncert_auxiii:fnnnnn
      { \prg_replicate:nn { \int_abs:n {#1} } { 0 } }
      {#1}
  }
\cs_generate_variant:Nn \@@_parse_combine_uncert_auxii:nnnnn { f }
\cs_new_protected:Npn
  \@@_parse_combine_uncert_auxiii:nnnnnn #1#2#3#4#5#6
  {
    \int_compare:nNnTF {#2} > 0
      {
        \@@_parse_combine_uncert_auxiv:nnnn
          {#3} {#4} {#5} { #6 #1 }
      }
      {
        \@@_parse_combine_uncert_auxiv:nnnn
          {#3} { #4 #1 } {#5} {#6}
      }
  }
\cs_generate_variant:Nn
  \@@_parse_combine_uncert_auxiii:nnnnnn { f }
\cs_new_protected:Npn
  \@@_parse_combine_uncert_auxiv:nnnn #1#2#3#4
  {
    \tl_set:Nx \l_@@_parsed_tl
      {
        { \tl_head:V \l_@@_parsed_tl }
        { \exp_not:n {#1} }
        {
          \bool_lazy_and:nnTF
            { \tl_if_blank_p:n {#2} }
            { ! \tl_if_blank_p:n {#4} }
            { 0 }
            { \exp_not:n {#2} }
        }
        {
          \@@_parse_combine_uncert_auxv:w #3#4
            \q_recursion_tail \q_recursion_stop
        }
      }
  }
%    \end{macrocode}
%   A short routine to remove any leading zeros in the uncertainty part,
%   which are not needed for the compact representation used by the module.
%    \begin{macrocode}
\cs_new:Npn \@@_parse_combine_uncert_auxv:w #1
  {
    \quark_if_recursion_tail_stop_do:Nn #1
      {
        \bool_if:NT \l_@@_zero_uncert_bool
          { { S } { 0 } }
      }
    \str_if_eq:nnTF {#1} { 0 }
      { \@@_parse_combine_uncert_auxv:w }
      { \@@_parse_combine_uncert_auxvi:w #1 }
  }
\cs_new:Npn \@@_parse_combine_uncert_auxvi:w
  #1 \q_recursion_tail \q_recursion_stop
  { { S } { \exp_not:n {#1} } }
%    \end{macrocode}
% \end{macro}
% \end{macro}
% \end{macro}
% \end{macro}
% \end{macro}
% \end{macro}
% \end{macro}
%
% \begin{macro}{\@@_parse_comparator:}
% \begin{macro}{\@@_parse_comparator_aux:Nw}
%   A comparator has to be the very first token in the input. A such, the
%   test for this can be very fast: grab the first token, do a check and
%   if appropriate store the result.
%    \begin{macrocode}
\cs_new_protected:Npn \@@_parse_comparator:
  {
    \exp_after:wN \@@_parse_comparator_aux:Nw
      \l_@@_arg_tl \q_stop
  }
\cs_new_protected:Npn \@@_parse_comparator_aux:Nw #1#2 \q_stop
  {
    \tl_if_in:NnTF \l_siunitx_number_input_comparator_tl {#1}
      {
        \tl_set:Nn \l_@@_comparator_tl {#1}
        \tl_set:Nn \l_@@_arg_tl {#2}
      }
      { \tl_clear:N \l_@@_comparator_tl }
    \tl_if_empty:NF \l_@@_arg_tl
      { \@@_parse_sign: }
  }
%    \end{macrocode}
% \end{macro}
% \end{macro}
%
% \begin{macro}{\@@_parse_exponent:}
% \begin{macro}{\@@_parse_exponent_auxi:w}
% \begin{macro}{\@@_parse_exponent_auxii:nn}
% \begin{macro}{\@@_parse_exponent_auxiii:Nw}
% \begin{macro}{\@@_parse_exponent_auxiv:nn}
% \begin{macro}
%   {\@@_parse_exponent_zero_test:N, \@@_parse_exponent_check:N}
% \begin{macro}{\@@_parse_exponent_cleanup:N}
%   An exponent part of a number has to come at the end and can only occur
%   once. Thus it is relatively easy to parse. First, there is a check that
%   an exponent part is allowed, and if so a split is made (the previous
%   part of the chain checks that there is some content in \cs{l_@@_arg_tl}
%   before calling this function). After splitting, if there is no exponent
%   then simply save a default. Otherwise, check for a sign and then store
%   either this or an implicit plus, and the digits after a check that nothing
%   else is present after the~|e|. The only slight complication to all of
%   this is allowing an arbitrary token in the input to represent the exponent:
%   this is done by setting any exponent tokens to the first of the allowed
%   list, then using that in a delimited argument set up. Once an exponent
%   part is found, there is a loop to check that each of the tokens is a digit
%   then a tidy up step to remove any leading zeros.
%    \begin{macrocode}
\cs_new_protected:Npn \@@_parse_exponent:
  {
    \tl_if_empty:NTF \l_siunitx_number_input_exponent_tl
      {
        \tl_set:Nn \l_@@_exponent_tl { { } 0 }
        \tl_if_empty:NF \l_@@_parsed_tl
          { \@@_parse_loop: }
      }
      {
        \tl_set:Nx \l_@@_tmp_tl
          { \tl_head:V \l_siunitx_number_input_exponent_tl }
        \tl_map_inline:Nn \l_siunitx_number_input_exponent_tl
          {
            \tl_replace_all:NnV \l_@@_arg_tl
              {##1} \l_@@_tmp_tl
          }
        \use:x
          {
            \cs_set_protected:Npn
              \exp_not:N \@@_parse_exponent_auxi:w
              ####1 \exp_not:V \l_@@_tmp_tl
              ####2 \exp_not:V \l_@@_tmp_tl
              ####3 \exp_not:N \q_stop
          }
            { \@@_parse_exponent_auxii:nn {##1} {##2} }
        \use:x
          {
            \@@_parse_exponent_auxi:w
              \exp_not:V \l_@@_arg_tl
              \exp_not:V \l_@@_tmp_tl \exp_not:N \q_nil
              \exp_not:V \l_@@_tmp_tl \exp_not:N \q_stop
          }
      }
  }
\cs_new_protected:Npn \@@_parse_exponent_auxi:w  { }
\cs_new_protected:Npn \@@_parse_exponent_auxii:nn #1#2
  {
    \quark_if_nil:nTF {#2}
      { \tl_set:Nn \l_@@_exponent_tl { { } 0 } }
      {
        \tl_set:Nn \l_@@_arg_tl {#1}
        \tl_if_blank:nTF {#2}
          { \tl_clear:N \l_@@_parsed_tl }
          { \@@_parse_exponent_auxiii:Nw #2 \q_stop }
      }
    \tl_if_empty:NF \l_@@_parsed_tl
      { \@@_parse_loop: }
  }
\cs_new_protected:Npn \@@_parse_exponent_auxiii:Nw #1#2 \q_stop
  {
    \tl_if_in:NnTF \l_siunitx_number_input_sign_tl {#1}
      { \@@_parse_exponent_auxiv:nn {#1} {#2} }
      { \@@_parse_exponent_auxiv:nn { } {#1#2} }
    \tl_if_empty:NT \l_@@_exponent_tl
      { \tl_clear:N \l_@@_parsed_tl }
  }
\cs_new_protected:Npn \@@_parse_exponent_auxiv:nn #1#2
  {
    \bool_lazy_or:nnTF
      { \l_@@_explicit_plus_bool }
      { ! \str_if_eq_p:nn {#1} { + } }
      { \tl_set:Nn \l_@@_exponent_tl { {#1} } }
      { \tl_set:Nn \l_@@_exponent_tl { { } } }
    \tl_if_blank:nTF {#2}
      { \tl_clear:N \l_@@_parsed_tl }
      {
        \@@_parse_exponent_zero_test:N #2
          \q_recursion_tail \q_recursion_stop
      }
  }
\cs_new_protected:Npn \@@_parse_exponent_zero_test:N #1
  {
    \quark_if_recursion_tail_stop_do:Nn #1
      { \tl_set:Nn \l_@@_exponent_tl { { } 0 } }
    \str_if_eq:nnTF {#1} { 0 }
      { \@@_parse_exponent_zero_test:N }
      { \@@_parse_exponent_check:N #1 }
  }
\cs_new_protected:Npn \@@_parse_exponent_check:N #1
  {
    \quark_if_recursion_tail_stop:N #1
    \tl_if_in:NnTF \l_@@_input_digit_tl {#1}
      {
        \tl_put_right:Nn \l_@@_exponent_tl {#1}
        \@@_parse_exponent_check:N
      }
      { \@@_parse_exponent_cleanup:wN }
  }
\cs_new_protected:Npn \@@_parse_exponent_cleanup:wN
  #1 \q_recursion_stop
  { \tl_clear:N \l_@@_parsed_tl }
%    \end{macrocode}
% \end{macro}
% \end{macro}
% \end{macro}
% \end{macro}
% \end{macro}
% \end{macro}
% \end{macro}
%
% \begin{macro}{\@@_parse_finalise:}
% \begin{macro}{\@@_parse_finalise:nw}
%   Combine all of the bits of a number together: both the real and
%   imaginary parts contain all of the data.
%    \begin{macrocode}
\cs_new_protected:Npn \@@_parse_finalise:
  {
    \tl_if_empty:NF \l_@@_parsed_tl
      {
        \tl_set:Nx \l_@@_parsed_tl
          {
            { \exp_not:V \l_@@_comparator_tl }
            \exp_not:V \l_@@_parsed_tl
            \exp_after:wN \@@_parse_finalise:nw
              \l_@@_exponent_tl \q_stop
          }
      }
  }
\cs_new:Npn \@@_parse_finalise:nw #1#2 \q_stop
  {
    { \exp_not:n {#1} }
    { \exp_not:n {#2} }
  }
%    \end{macrocode}
% \end{macro}
% \end{macro}
%
% \begin{macro}{\@@_parse_loop:}
% \begin{macro}{\@@_parse_loop_first:N}
% \begin{macro}{\@@_parse_loop_main:NNNNN}
% \begin{macro}{\@@_parse_loop_main_end:NN}
% \begin{macro}{\@@_parse_loop_main_digit:NNNNN}
% \begin{macro}{\@@_parse_loop_main_decimal:NN}
% \begin{macro}{\@@_parse_loop_main_uncert:NNN}
% \begin{macro}{\@@_parse_loop_main_sign:NNN}
% \begin{macro}{\@@_parse_loop_main_store:NNN}
% \begin{macro}{\@@_parse_loop_after_decimal:NNN}
% \begin{macro}{\@@_parse_loop_root_swap:NNwNN}
% \begin{macro}{\@@_parse_loop_break:wN}
%   At this stage, the partial input \cs{l_@@_arg_tl} will contain any
%   mantissa, which may contain an uncertainty or complex part. Parsing this
%   and allowing for all of the different formats possible is best done using
%   a token-by-token approach. However, as at each stage only a subset of
%   tokens are valid, the approach take is to use a set of semi-dedicated
%   functions to parse different components along with switches to allow a
%   sensible amount of code sharing.
%    \begin{macrocode}
\cs_new_protected:Npn \@@_parse_loop:
  {
    \tl_clear:N \l_@@_partial_tl
    \exp_after:wN \@@_parse_loop_first:NNN
      \exp_after:wN \l_@@_parsed_tl \exp_after:wN \c_true_bool
        \l_@@_arg_tl
        \q_recursion_tail \q_recursion_stop
  }
%    \end{macrocode}
%   The very first token of the input is handled with a dedicated function.
%   Valid cases here are
%   \begin{itemize}
%     \item Entirely blank if the original input was for example |+e10|:
%       simply clean up if in the integer part of issue an error if in
%       a second part (complex number, \foreign{etc.}).
%     \item An integer part digit: pass through to the main collection
%       routine.
%     \item A decimal marker: store an empty integer part and move to
%       the main collection routine for a decimal part.
%   \end{itemize}
%   Anything else is invalid and sends the code to the abort function.
%    \begin{macrocode}
\cs_new_protected:Npn \@@_parse_loop_first:NNN #1#2#3
  {
    \quark_if_recursion_tail_stop_do:Nn #3
      {
        \bool_if:NTF #2
          { \tl_put_right:Nn #1 { { 1 } { } { } } }
          { \@@_parse_loop_break:wN \q_recursion_stop }
      }
    \tl_if_in:NnTF \l_@@_input_digit_tl {#3}
      {
        \@@_parse_loop_main:NNNNN
          #1 \c_true_bool \c_false_bool #2 #3
      }
      {
        \tl_if_in:NnTF \l_siunitx_number_input_decimal_tl {#3}
          {
            \tl_put_right:Nn #1 { { 0 } }
            \@@_parse_loop_after_decimal:NNN #1 #2
          }
          { \@@_parse_loop_break:wN }
      }
  }
%    \end{macrocode}
%   A single function is used to cover the \enquote{main} part of numbers:
%   finding real, complex or separated uncertainty parts and covering both
%   the integer and decimal components. This works because these elements
%   share a lot of concepts: a small number of switches can be used to
%   differentiate between them. To keep the code at least somewhat readable,
%   this main function deals with the validity testing but hands off other
%   tasks to dedicated auxiliaries for each case.
%
%   The possibilities are
%   \begin{itemize}
%     \item The number terminates, meaning that some digits were collected
%       and everything is simply tidied up (as far as the loop is concerned).
%     \item A digit is found: this is the common case and leads to a storage
%       auxiliary (which handles non-significant zeros).
%     \item A decimal marker is found: only valid in the integer part and
%       there leading to a store-and-switch situation.
%     \item An open-uncertainty token: switch to the dedicated collector
%       for uncertainties.
%     \item A sign token (if allowed): stop collecting this number and
%       restart collection for the second part.
%   \end{itemize}
%    \begin{macrocode}
\cs_new_protected:Npn \@@_parse_loop_main:NNNNN #1#2#3#4#5
  {
    \quark_if_recursion_tail_stop_do:Nn #5
      { \@@_parse_loop_main_end:NN #1#2 }
    \tl_if_in:NnTF \l_@@_input_digit_tl {#5}
      { \@@_parse_loop_main_digit:NNNNN #1#2#3#4#5 }
      {
        \tl_if_in:NnTF \l_siunitx_number_input_decimal_tl {#5}
          {
            \bool_if:NTF #2
              { \@@_parse_loop_main_decimal:NN #1 #4 }
              { \@@_parse_loop_break:wN }
          }
          {
            \tl_if_in:NnTF \l_@@_input_uncert_open_tl {#5}
              { \@@_parse_loop_main_uncert:NNN #1#2 #4 }
              {
                \bool_if:NTF #4
                  {
                    \tl_if_in:NnTF \l_siunitx_number_input_sign_tl {#5}
                      {
                        \@@_parse_loop_main_sign:NNN
                          #1#2 #5
                      }
                      { \@@_parse_loop_break:wN }
                  }
                  { \@@_parse_loop_break:wN }
              }
          }
      }
  }
%    \end{macrocode}
%   If the main loop finds the end marker then there is a tidy up phase.
%   The current partial number is stored either as the integer or decimal,
%   depending on the setting for the indicator switch. For the integer
%   part, if no number has been collected then one or more non-significant
%   zeros have been dropped. Exactly one zero is therefore needed to make
%   sure the parsed result is correct.
%    \begin{macrocode}
\cs_new_protected:Npn \@@_parse_loop_main_end:NN #1#2
  {
    \bool_lazy_and:nnT
      {#2} { \tl_if_empty_p:N \l_@@_partial_tl }
      { \tl_set:Nn \l_@@_partial_tl { 0 } }
    \tl_put_right:Nx #1
      {
        { \exp_not:V \l_@@_partial_tl }
        \bool_if:NT #2 { { } }
        { }
      }
  }
%    \end{macrocode}
%   The most common case for the main loop collector is to find a digit.
%   Here, in the integer part it is possible that zeros are non-significant:
%   that is handled using a combination of a switch and a string test. Other
%   than that, the situation here is simple: store the input and loop.
%    \begin{macrocode}
\cs_new_protected:Npn \@@_parse_loop_main_digit:NNNNN #1#2#3#4#5
  {
    \bool_lazy_or:nnTF
      {#3} { ! \str_if_eq_p:nn {#5} { 0 } }
      {
        \tl_put_right:Nn \l_@@_partial_tl {#5}
        \@@_parse_loop_main:NNNNN #1 #2 \c_true_bool #4
      }
      { \@@_parse_loop_main:NNNNN #1 #2 \c_false_bool #4 }
  }
%    \end{macrocode}
%   When a decimal marker was found, move the integer part to the
%   store and then go back to the loop with the flags set correctly.
%   There is the case of non-significant zeros to cover before that, of course.
%    \begin{macrocode}
\cs_new_protected:Npn \@@_parse_loop_main_decimal:NN #1#2
  {
    \@@_parse_loop_main_store:NNN #1 \c_false_bool \c_false_bool
    \@@_parse_loop_after_decimal:NNN #1 #2
  }
%    \end{macrocode}
%   Starting an uncertainty part means storing the number to date as in other
%   cases, with the possibility of a blank decimal part allowed for. The
%   uncertainty itself is collected by a dedicated function as it is extremely
%   restricted.
%    \begin{macrocode}
\cs_new_protected:Npn \@@_parse_loop_main_uncert:NNN #1#2#3
  {
    \@@_parse_loop_main_store:NNN #1 #2 \c_false_bool
    \@@_parse_uncert:NN #1
  }
%    \end{macrocode}
%   If a sign is found, terminate the current number, store the sign as the
%   first token of the second part and go back to do the dedicated first-token
%   function.
%    \begin{macrocode}
\cs_new_protected:Npn \@@_parse_loop_main_sign:NNN #1#2#3
  {
    \@@_parse_loop_main_store:NNN #1 #2 \c_true_bool
    \tl_set:Nn \l_@@_flex_tl { {#3} }
    \@@_parse_loop_first:NNN
      \l_@@_flex_tl \c_false_bool
  }
%    \end{macrocode}
%   A common auxiliary for the various non-digit token functions: tidy up the
%   integer and decimal parts of a number. Here, the two flags are used to
%   indicate if empty decimal and uncertainty parts should be included in
%   the storage cycle.
%    \begin{macrocode}
\cs_new_protected:Npn \@@_parse_loop_main_store:NNN #1#2#3
  {
    \tl_if_empty:NT \l_@@_partial_tl
      { \tl_set:Nn \l_@@_partial_tl { 0 } }
    \tl_put_right:Nx #1
      {
        { \exp_not:V \l_@@_partial_tl }
        \bool_if:NT #2 { { } }
        \bool_if:NT #3 { { } }
      }
    \tl_clear:N \l_@@_partial_tl
  }
%    \end{macrocode}
%   After a decimal marker there has to be a digit if there wasn't one before
%   it. That is handled by using a dedicated function, which checks for
%   an empty integer part first then either simply hands off or looks for
%   a digit.
%    \begin{macrocode}
\cs_new_protected:Npn \@@_parse_loop_after_decimal:NNN #1#2#3
  {
    \tl_if_blank:fTF { \exp_after:wN \use_none:n #1 }
      {
        \quark_if_recursion_tail_stop_do:Nn #3
          { \@@_parse_loop_break:wN \q_recursion_stop }
        \tl_if_in:NnTF \l_@@_input_digit_tl {#1}
          {
            \tl_put_right:Nn \l_@@_partial_tl {#3}
            \@@_parse_loop_main:NNNNN
              #1 \c_false_bool \c_true_bool #2
          }
          { \@@_parse_loop_break:wN }
      }
      {
        \@@_parse_loop_main:NNNNN
          #1 \c_false_bool \c_true_bool #2 #3
      }
  }
%    \end{macrocode}
%   Something is not right: remove all of the remaining tokens from the
%   number and clear the storage areas as a signal for the next part of the
%   code.
%    \begin{macrocode}
\cs_new_protected:Npn \@@_parse_loop_break:wN
  #1 \q_recursion_stop
  {
    \tl_clear:N \l_@@_flex_tl
    \tl_clear:N \l_@@_parsed_tl
  }
%    \end{macrocode}
% \end{macro}
% \end{macro}
% \end{macro}
% \end{macro}
% \end{macro}
% \end{macro}
% \end{macro}
% \end{macro}
% \end{macro}
% \end{macro}
% \end{macro}
% \end{macro}
%
% \begin{macro}{\@@_parse_sign:}
% \begin{macro}{\@@_parse_sign_aux:Nw}
%   The first token of a number after a comparator could be a sign. A quick
%   check is made and if found stored. For the number to be valid it has to be
%   more than just a sign, so the next part of the chain is only called if that
%   is the case.
%    \begin{macrocode}
\cs_new_protected:Npn \@@_parse_sign:
  {
    \exp_after:wN \@@_parse_sign_aux:Nw
      \l_@@_arg_tl \q_stop
  }
\cs_new_protected:Npn \@@_parse_sign_aux:Nw #1#2 \q_stop
  {
    \tl_if_in:NnTF \l_siunitx_number_input_sign_tl {#1}
      {
        \tl_set:Nn \l_@@_arg_tl {#2}
        \bool_lazy_and:nnTF
          { \token_if_eq_charcode_p:NN #1 + }
          { ! \l_@@_explicit_plus_bool }
          { \tl_set:Nn \l_@@_parsed_tl { { } } }
          { \tl_set:Nn \l_@@_parsed_tl { {#1} } }
      }
      { \tl_set:Nn \l_@@_parsed_tl { { } } }
    \tl_if_empty:NTF \l_@@_arg_tl
      { \tl_clear:N \l_@@_parsed_tl }
      { \@@_parse_exponent: }
  }
%    \end{macrocode}
% \end{macro}
% \end{macro}
%
% \begin{macro}{\@@_parse_uncert:NN}
% \begin{macro}{\@@_parse_uncert:NNNN}
% \begin{macro}{\@@_parse_uncert_auxi:NN, \@@_parse_uncert_auxii:NN}
% \begin{macro}
%   {
%     \@@_parse_uncert_auxii:N  ,
%     \@@_parse_uncert_marker:N ,
%     \@@_parse_uncert_after:N
%   }
%   Parsing a combined uncertainty has a very restricted range of allowed
%   tokens. A closing uncertainty token in the first place is an error,
%   so we filter that out explicitly. After that, we check for digits,
%   which require checking for significant digits. The non-digit function
%   is separate to make the flow clearer.
%    \begin{macrocode}
\cs_new_protected:Npn \@@_parse_uncert:NN #1#2
  {
    \quark_if_recursion_tail_stop_do:Nn #2
      { \@@_parse_loop_break:wN \q_recursion_stop }
    \tl_if_in:NnTF \l_@@_input_uncert_close_tl {#2}
      { \@@_parse_loop_break:wN }
      {
        \@@_parse_uncert:NNNN
          #1 \c_false_bool \@@_parse_uncert_auxi:NN #2
      }
  }
%    \end{macrocode}
%   Deal with digits: a simple question of whether they are significant.
%    \begin{macrocode}
\cs_new_protected:Npn \@@_parse_uncert:NNNN #1#2#3#4
  {
    \quark_if_recursion_tail_stop_do:Nn #4
      { \@@_parse_loop_break:wN \q_recursion_stop }
    \tl_if_in:NnTF \l_@@_input_digit_tl {#4}
      {
        \bool_lazy_or:nnTF
          {#2} { ! \str_if_eq_p:nn {#4} { 0 } }
          {
            \tl_put_right:Nn \l_@@_partial_tl {#4}
            \@@_parse_uncert:NNNN #1 \c_true_bool #3
          }
          { \@@_parse_uncert:NNNN #1 \c_false_bool #3 }
      }
      { #3 #1#4 }
  }
%    \end{macrocode}
%   For the two auxiliaries, the difference is the handling of a
%   decimal marker: one may be present, but only exactly one.
%    \begin{macrocode}
\cs_new_protected:Npn \@@_parse_uncert_auxi:NN #1#2
  {
    \tl_if_in:NnTF \l_@@_input_uncert_close_tl {#2}
      {
        \@@_parse_uncert_auxiii:N #1
        \@@_parse_uncert_after:N
      }
      {
        \tl_if_in:NnTF \l_siunitx_number_input_decimal_tl {#2}
          { \@@_parse_uncert_marker:N #1 }
          { \@@_parse_loop_break:wN }
      }
  }
\cs_new_protected:Npn \@@_parse_uncert_auxii:NN #1#2
  {
    \tl_if_in:NnTF \l_@@_input_uncert_close_tl {#2}
      {
        \@@_parse_uncert_auxiii:N #1
        \@@_parse_uncert_after:N
      }
      { \@@_parse_loop_break:wN }
  }
%    \end{macrocode}
%   Deal with the closing bracket, which might leave us with nothing if there
%   were no significant digits.
%    \begin{macrocode}
\cs_new_protected:Npn \@@_parse_uncert_auxiii:N #1
  {
    \tl_if_empty:NTF \l_@@_partial_tl
      {
        \tl_put_right:Nx #1
          {
            {
              \bool_if:NT \l_@@_zero_uncert_bool
                { { S } { 0 } }
            }
         }
      }
      {
        \tl_set:Nx \l_@@_partial_tl
          { { S } { \exp_not:V \l_@@_partial_tl } }
         \@@_parse_loop_main_store:NNN #1
           \c_false_bool \c_false_bool
      }
  }
%    \end{macrocode}
%   Handling a decimal marker in the uncertainty is a bit tricky: we need to make
%   sure it's valid. First, we need to be sure that the integer part of the captured
%   uncertainty is not too long. Then we need to check that the decimal part is
%   not too long. Both of these require data from the collected partial number,
%   so we extract that first. Checking the decimal part needs the length of the
%   not-yet-collected uncertainty. Handily, we know that it should be a set of
%   digits then a closing marker. So we can use that as a length: if it's
%   too long we can stop.
%    \begin{macrocode}
\cs_new_protected:Npn \@@_parse_uncert_marker:N #1
  { \exp_after:wN \@@_parse_uncert_marker:nnnN #1 #1 }
\cs_new_protected:Npn \@@_parse_uncert_marker:nnnN #1#2#3#4
  {
    \int_compare:nNnTF
      { \tl_count:N \l_@@_partial_tl } > { \tl_count:n {#2} }
      { \@@_parse_loop_break:wN }
      { \@@_parse_uncert_marker:nNw {#3} #4 }
  }
\cs_new_protected:Npn \@@_parse_uncert_marker:nNw
  #1#2#3 \q_recursion_tail \q_recursion_stop
  {
    \int_compare:nNnTF
      { \tl_count:n {#3} - 1 } = { \tl_count:n {#1} }
      {
        \str_if_eq:eeTF
          { \exp_not:V \l_@@_partial_tl }
          { \prg_replicate:nn { \tl_count:N \l_@@_partial_tl } { 0 } }
          {
            \@@_parse_uncert:NNNN
              #2 \c_false_bool
          }
          {
            \@@_parse_uncert:NNNN
              #2 \c_true_bool
          }
            \@@_parse_uncert_auxii:NN
      }
      { \@@_parse_loop_break:wN }
    #3 \q_recursion_tail \q_recursion_stop
  }
%    \end{macrocode}
%   No further tokens are allowed after an uncertainty in parenthesis.
%    \begin{macrocode}
\cs_new_protected:Npn \@@_parse_uncert_after:N #1
  {
    \quark_if_recursion_tail_stop:N #1
    \@@_parse_loop_break:wN
  }
%    \end{macrocode}
% \end{macro}
% \end{macro}
% \end{macro}
% \end{macro}
%
% \subsection{Processing numbers}
%
% \begin{variable}
%   {
%     \l_@@_drop_exponent_bool     ,
%     \l_@@_drop_uncertainty_bool  ,
%     \l_@@_drop_zero_decimal_bool ,
%     \l_@@_exponent_mode_tl       ,
%     \l_@@_exponent_fixed_int     ,
%     \l_@@_min_decimal_int        ,
%     \l_@@_min_integer_int        ,
%     \l_@@_round_half_even_bool   ,
%     \l_@@_round_mode_tl          ,
%     \l_@@_round_pad_bool         ,
%     \l_@@_round_precision_int
%   }
%    \begin{macrocode}
\keys_define:nn { siunitx }
  {
    drop-exponent .bool_set:N =
      \l_@@_drop_exponent_bool ,
    drop-uncertainty .bool_set:N =
      \l_@@_drop_uncertainty_bool ,
    drop-zero-decimal .bool_set:N =
      \l_@@_drop_zero_decimal_bool ,
    exponent-mode .choices:nn =
      { engineering , fixed , input , scientific }
      { \tl_set_eq:NN \l_@@_exponent_mode_tl \l_keys_choice_tl } ,
    fixed-exponent .int_set:N =
      \l_@@_exponent_fixed_int ,
    minimum-decimal-digits .int_set:N =
      \l_@@_min_decimal_int ,
    minimum-integer-digits .int_set:N =
      \l_@@_min_integer_int ,
    round-half .choice: ,
    round-half / even .code:n =
      { \bool_set_true:N \l_@@_round_half_even_bool } ,
    round-half / up .code:n =
      { \bool_set_false:N \l_@@_round_half_even_bool } ,
    round-minimum .code:n =
      { \@@_set_round_min:n {#1} } ,
    round-mode .choices:nn =
      { figures , none , places, uncertainty }
      { \tl_set_eq:NN \l_@@_round_mode_tl \l_keys_choice_tl } ,
    round-pad .bool_set:N =
      \l_@@_round_pad_bool ,
    round-precision .int_set:N =
      \l_@@_round_precision_int ,
  }
\bool_new:N \l_@@_round_half_even_bool
\tl_new:N \l_@@_exponent_mode_tl
\tl_new:N \l_@@_round_mode_tl
%    \end{macrocode}
% \end{variable}
%
% \begin{variable}{\l_@@_round_min_tl}
%   For storing the minimum for rounding.
%   \begin{macrocode}
\tl_new:N \l_@@_round_min_tl
%    \end{macrocode}
% \end{variable}
%
% \begin{macro}{\@@_set_round_min:n}
% \begin{macro}[EXP]{\@@_set_round_min:nnnnnnn}
%  For setting the rounding minimum, the aim is to do as much of the work
%  now as possible. That's mainly a question of checking if there are any
%  significant digits in the mantissa given.
%   \begin{macrocode}
\cs_new_protected:Npn \@@_set_round_min:n #1
  {
    \siunitx_number_parse:nN {#1} \l_@@_tmp_tl
    \exp_after:wN \@@_set_round_min:nnnnnnn \l_@@_tmp_tl
  }
\cs_new:Npn \@@_set_round_min:nnnnnnn #1#2#3#4#5#6#7
  {
    \tl_set:Nx \l_@@_round_min_tl
      {
        \bool_lazy_and:nnF
          { \str_if_eq_p:nn {#3} { 0 } }
          {
            \str_if_eq_p:ee
              { \exp_not:n {#4} }
              { \prg_replicate:nn { \tl_count:n {#4} } { 0 } }
          }
          { \exp_not:n { {#3} {#4} } }
      }
  }
%    \end{macrocode}
% \end{macro}
% \end{macro}
%
% \begin{macro}{\siunitx_number_process:NN}
% \begin{macro}{\@@_process:nnnnnnnNN}
%   A top-level interface for the processing tools.
%    \begin{macrocode}
\cs_new_protected:Npn \siunitx_number_process:NN #1#2
  {
    \tl_if_empty:NTF #1
      { \tl_clear:N #2 }
      {
        \@@_drop_uncertainty:NN #1 #2
        \exp_after:wN \@@_process:nnnnnnnNN #2 #2 #2
        \@@_drop_exponent:NN #2 #2
        \@@_zero_decimal:NN #2 #2
        \@@_digits:NN #2 #2
      }
  }
\cs_new_protected:Npn \@@_process:nnnnnnnNN #1#2#3#4#5#6#7#8#9
  {
    \bool_lazy_and:nnF
      { \str_if_eq_p:nn {#3} { 0 } }
      {
        \str_if_eq_p:ee
          { \exp_not:n {#4} } { \prg_replicate:nn { \tl_count:n {#4} } { 0 } }
      }
      {
        \@@_exponent:NN #8 #9
        \@@_round:NN #9 #9
      }
  }
%    \end{macrocode}
% \end{macro}
% \end{macro}
%
% \begin{macro}{\@@_exponent:NN}
% \begin{macro}[EXP]
%   {
%     \@@_exponent_engineering:nnnnnnn ,
%     \@@_exponent_fixed:nnnnnnn       ,
%     \@@_exponent_input:nnnnnnn       ,
%     \@@_exponent_scientific:nnnnnnn
%   }
% \begin{macro}[EXP]
%   {
%     \@@_exponent_fixed:nnnnnnnn       ,
%     \@@_exponent_scientific:nnnnnnnn
%   }
% \begin{macro}[EXP]{\@@_exponent_scientific:nnnw}
% \begin{macro}[EXP]{\@@_exponent_shift:nnn,\@@_exponent_shift:nnf}
% \begin{macro}[EXP]{\@@_exponent_shift_down:nnnw}
% \begin{macro}[EXP]{\@@_exponent_shift_down:nnn}
% \begin{macro}[EXP]{\@@_exponent_shift_down:nw}
% \begin{macro}[EXP]{\@@_exponent_shift_up:nnn}
% \begin{macro}[EXP]{\@@_exponent_shift_up:nnw}
% \begin{macro}[EXP]
%   {
%     \@@_exponent_shift_up_aux:nnn ,
%     \@@_exponent_shift_up_aux:fnn ,
%     \@@_exponent_shift_up_aux:ffn
%   }
% \begin{macro}[EXP]{\@@_exponent_shift_uncert:nw}
% \begin{macro}[EXP]
%   {\@@_exponent_shift_uncert_S:nnnn, \@@_exponent_shift_uncert_S:fnnn}
% \begin{macro}[EXP]{\@@_exponent_uncert:n}
% \begin{macro}[EXP]{\@@_exponent_finalise:n}
% \begin{macro}[EXP]{\@@_exponent_engineering_aux:nnnnnnn}
% \begin{macro}[EXP]
%   {
%     \@@_exponent_engineering_0:nnnn ,
%     \@@_exponent_engineering_1:nnnn ,
%     \@@_exponent_engineering_2:nnnn
%   }
%  \begin{macro}[EXP]{\@@_exponent_engineering:nnNw}
%  \begin{macro}[EXP]{\@@_exponent_engineering_uncert:nn}
%  \begin{macro}[EXP]{\@@_exponent_engineering_uncert_S:nnn}
%   Manipulating an exponent is done using a single expansion function
%   \emph{unless} dealing with engineering-style output. The latter is easier
%   to handle by first converting to scientific output, then post-processing.
%   (Once \texttt{e}-type expansion is generally available, this will be
%   handling using a single \cs{tl_set:Nx}.)
%    \begin{macrocode}
\cs_new_protected:Npn \@@_exponent:NN #1#2
  {
    \tl_set:Nx #2
      {
        \cs:w
          @@_exponent_ \l_@@_exponent_mode_tl :nnnnnnn
          \exp_after:wN
        \cs_end: #1
      }
    \str_if_eq:VnT \l_@@_exponent_mode_tl { engineering }
      {
        \tl_set:Nx #2
          { \exp_after:wN \@@_exponent_engineering_aux:nnnnnnn #2 }
      }
  }
\cs_new:Npn \@@_exponent_fixed:nnnnnnn #1#2#3#4#5#6#7
  {
    \exp_args:Nf \@@_exponent_fixed:nnnnnnnn
      { \int_eval:n { \l_@@_exponent_fixed_int - (#6#7) } }
      {#1} {#2} {#3} {#4} {#5} {#6} {#7}
  }
\cs_new:Npn \@@_exponent_fixed:nnnnnnnn #1#2#3#4#5#6#7#8
  {
    \exp_not:n { {#2} {#3} }
    \@@_exponent_shift:nnn {#1} {#4} {#5}
    \@@_exponent_uncert:n {#6}
    \exp_not:n { {#7} } { \int_use:N \l_@@_exponent_fixed_int }
  }
\cs_new:Npn \@@_exponent_input:nnnnnnn #1#2#3#4#5#6#7
  { \exp_not:n { {#1} {#2} {#3} {#4} {#5} {#6} {#7} } }
%    \end{macrocode}
%   To convert to scientific notation, the key question is to find the number
%   of significant places. That is easy enough if the number has a non-zero
%   integer component. For a pure decimal, we have to trim off leading
%   zeros in a loop.
%    \begin{macrocode}
\cs_new:Npn \@@_exponent_scientific:nnnnnnn #1#2#3#4#5#6#7
  {
    \exp_args:Nf \@@_exponent_scientific:nnnnnnnn
      { \int_eval:n { \tl_count:n {#3} } }
      {#1} {#2} {#3} {#4} {#5} {#6} {#7}
  }
\cs_new:Npn \@@_exponent_scientific:nnnnnnnn #1#2#3#4#5#6#7#8
  {
    \exp_not:n { {#2} {#3} }
    \int_compare:nNnTF {#1} = 1
      {
        \str_if_eq:nnTF {#4} { 0 }
          {
            \@@_exponent_scientific:nnnw
              { 0 } {#6} { #7#8 } #5 \q_stop
          }
          { \exp_not:n { {#4} {#5} {#6} {#7} {#8} } }
      }
      {
        \@@_exponent_shift:nnn { #1 - 1 } {#4} {#5}
        \@@_exponent_uncert:n {#6}
        \@@_exponent_finalise:n { #1 + #7#8 - 1 }
      }
  }
\cs_new_eq:NN \@@_exponent_engineering:nnnnnnn
  \@@_exponent_scientific:nnnnnnn
\cs_new:Npn \@@_exponent_scientific:nnnw #1#2#3#4#5 \q_stop
  {
    \str_if_eq:nnTF {#4} { 0 }
      {
        \@@_exponent_scientific:nnnw
          { #1 - 1 } {#2} {#3} #5 \q_stop
      }
      {
        \exp_not:n { {#4} {#5} {#2} }
        \@@_exponent_finalise:n { #1 + #3 - 1 }
      }
  }
%    \end{macrocode}
%   When adjusting the exponent position, there are two paths depending on
%   which way the shift takes place.
%    \begin{macrocode}
\cs_new:Npn \@@_exponent_shift:nnn #1#2#3
  {
    \int_compare:nNnTF {#1} > 0
      { \@@_exponent_shift_down:nnnw {#1} {#3} { } #2 \q_stop }
      {
        \int_compare:nNnTF {#1} < 0
          { \@@_exponent_shift_up:nnn {#1} {#2} {#3} }
          { {#2} {#3} }
      }
  }
\cs_generate_variant:Nn \@@_exponent_shift:nnn { nnf }
%    \end{macrocode}
%   For shifting the exponent down, there is first a loop to reserve the
%   integer part before doing the work: that of course has to be undone
%   for any remainder at he end of the process.
%    \begin{macrocode}
\cs_new:Npn \@@_exponent_shift_down:nnnw #1#2#3#4#5 \q_stop
  {
    \tl_if_blank:nTF {#5}
      { \@@_exponent_shift_down:nnn {#1} { #4 #3 } {#2} }
      { \@@_exponent_shift_down:nnnw {#1} {#2} { #4 #3 } #5 \q_stop }
  }
\cs_new:Npn \@@_exponent_shift_down:nnn #1#2#3
  {
    \int_compare:nNnTF {#1} = 0
      { { \tl_reverse:n {#2} } \exp_not:n { {#3} } }
      { \@@_exponent_shift_down:nw {#1} #2 \q_stop {#3} }
  }
\cs_new:Npn \@@_exponent_shift_down:nw #1#2#3 \q_stop #4
  {
    \tl_if_blank:nTF {#3}
      { \@@_exponent_shift_down:nnn { #1 - 1 } { 0 } { #2#4 } }
      { \@@_exponent_shift_down:nnn { #1 - 1 } {#3} { #2#4 } }
  }
%    \end{macrocode}
%   For shifting the exponent up, we can run out of decimal digits, at which
%   point filling is easy. Other than that a simple loop as we are picking
%   input off the front of the decimal part. We also need to deal with leading
%   zeros: these cannot accumulate.
%    \begin{macrocode}
\cs_new:Npn \@@_exponent_shift_up:nnn #1#2#3
  {
    \tl_if_blank:nTF {#3}
      {
        \@@_exponent_shift_up_aux:ffn
          { \int_eval:n { #1 + 1 } }
          { \str_if_eq:nnF {#2} { 0 } {#2} 0 }
          { }
        \@@_exponent_shift_uncert:nw { 1 }
      }
      {  \@@_exponent_shift_up:nnw {#1} {#2} #3 \q_stop }
  }
\cs_new:Npn \@@_exponent_shift_up:nnw #1#2#3#4 \q_stop
  {
    \@@_exponent_shift_up_aux:ffn
      { \int_eval:n { #1 + 1 } }
      { \str_if_eq:nnF {#2} { 0 } {#2} #3 }
      {#4}
  }
\cs_new:Npn \@@_exponent_shift_up_aux:nnn #1#2#3
  {
    \int_compare:nNnTF {#1} = 0
      { \exp_not:n { {#2} {#3} } }
      {
        \tl_if_blank:nTF {#3}
          {
            {
              \exp_not:n {#2}
              \prg_replicate:nn { \int_abs:n {#1} } { 0 }
            }
            { }
            \@@_exponent_shift_uncert:nw { \int_abs:n {#1} }
         }
         { \@@_exponent_shift_up:nnn {#1} {#2} {#3} }
      }
  }
\cs_generate_variant:Nn \@@_exponent_shift_up_aux:nnn { f , ff }
%    \end{macrocode}
%   If the shift has put digits into the integer part, we have to adjust the
%   uncertainty accordingly. First, we grab the data, then adjust by the
%   number of places that have been transferred.
%    \begin{macrocode}
\cs_new:Npn \@@_exponent_shift_uncert:nw
  #1#2 \@@_exponent_uncert:n #3
  {
    \tl_if_blank:nTF {#3}
      {
        #2
        \@@_exponent_uncert:n { }
      }
      {
        \str_if_eq:nnTF {#3} { 0 }
          {
            #2
            \@@_exponent_uncert:n { { S } { 0 } }
          }
          {      
            \use:c { @@_exponent_shift_uncert_ \use_i:nn #3 :fnnn }
              { \prg_replicate:nn {#1} { 0 } }
              {#2}
              #3
          }
      }
  }
\cs_new:Npn \@@_exponent_shift_uncert_S:nnnn #1#2#3#4
  {
    #2
    \@@_exponent_uncert:n { { S } { #4#1 } }
  }
\cs_generate_variant:Nn \@@_exponent_shift_uncert_S:nnnn { f }
\cs_new:Npn \@@_exponent_uncert:n #1 { { \exp_not:n {#1} } }
%    \end{macrocode}
%   Tidy up the exponent to put the sign in the right place.
%    \begin{macrocode}
\cs_new:Npn \@@_exponent_finalise:n #1
  {
    \int_compare:nNnTF {#1} < 0
      { { - } }
      { { } }
      { \int_abs:n {#1} }
  }
%    \end{macrocode}
%   This could (and eventually will) be combined with the main function above:
%   that will need \texttt{e}-type expansion. The input has already been
%   normalised such that the integer part is in the range $1 \le n < 10$.
%   Thus there are only three cases to deal with, depending on the required
%   adjustment to the exponent.
%    \begin{macrocode}
\cs_new:Npn \@@_exponent_engineering_aux:nnnnnnn #1#2#3#4#5#6#7
  {
    \exp_not:n { {#1} {#2} }
    \use:c
      {
        @@_exponent_engineering_
	     \int_compare:nNnTF {#6#7} < 0
	       {
	         \int_case:nnF { \int_mod:nn { #7 } { 3 } }
	           {
	             { 1 } { 2 }
	             { 2 } { 1 }
	           }
	           { 0 }
	       }
	       { \int_mod:nn {#7} { 3 } }
	       :nnnn
      }
        {#3} {#4} {#5} {#6#7}
  }
\cs_new:cpn { @@_exponent_engineering_0:nnnn } #1#2#3#4
  {
    \exp_not:n { {#1} {#2} {#3} }
    \@@_exponent_finalise:n {#4}
  }
\cs_new:cpn { @@_exponent_engineering_1:nnnn } #1#2#3#4
  {
    \tl_if_blank:nTF {#2}
      {
        { \exp_not:n { #1 0 } } { }
        { \@@_exponent_engineering_uncert:nn {#3} { 0 } }
      }
      {
        { \exp_not:n {#1} \exp_not:o { \tl_head:w #2 \q_stop } }
        { \exp_not:f { \tl_tail:n {#2} } }
        { \exp_not:n {#3} }
      }
    \@@_exponent_finalise:n { #4 - 1 }
  }
\cs_new:cpn { @@_exponent_engineering_2:nnnn } #1#2#3#4
  {
    \tl_if_blank:nTF {#2}
      {
        { \exp_not:n { #1 00 } } { }
        { \@@_exponent_engineering_uncert:nn {#3} { 00 } }
      }
      {\@@_exponent_engineering:nnNw {#1} {#3} #2 \q_stop }
    \@@_exponent_finalise:n { #4 - 2 }
  }
\cs_new:Npn \@@_exponent_engineering:nnNw #1#2#3#4 \q_stop
  {
    \tl_if_blank:nTF {#4}
      {
        { \exp_not:n { #1#3 0 } } { }
        { { \@@_exponent_engineering_uncert:nn {#2} { 0 } } }
      }
      {
        { \exp_not:n {#1#3} \exp_not:o { \tl_head:w #4 \q_stop } }
        { \exp_not:f { \tl_tail:n {#4} } }
        { \exp_not:n {#2} }
      }
  }
\cs_new:Npn \@@_exponent_engineering_uncert:nn #1#2
  {
    \tl_if_blank:nF {#1}
      {
        \use:c { @@_exponent_engineering_uncert_ \use_i:nn #1 :nnn }
          #1 {#2}
      }
  }
\cs_new:Npn \@@_exponent_engineering_uncert_S:nnn #1#2#3
  {
    { S }
    {
      \exp_not:n {#2}
      \str_if_eq:nnF {#2} { 0 } {#3}
    }
  }
%    \end{macrocode}
% \end{macro}
% \end{macro}
% \end{macro}
% \end{macro}
% \end{macro}
% \end{macro}
% \end{macro}
% \end{macro}
% \end{macro}
% \end{macro}
% \end{macro}
% \end{macro}
% \end{macro}
% \end{macro}
% \end{macro}
% \end{macro}
% \end{macro}
% \end{macro}
% \end{macro}
% \end{macro}
%
% \begin{macro}{\@@_digits:NN}
% \begin{macro}[EXP]{\@@_digits:nnnnnnn}
% \begin{macro}[EXP]{\@@_digits:Nn}
% \begin{macro}[EXP]{\@@_digits:nn}
% \begin{macro}[EXP]{\@@_digits_S:n}
%   Forcing a minimum number of digits in each part is quite easy. As
%   the common case is that we don't do anything here, there is no real need
%   to optimise the calculation (normally also numbers have only a few digits).
%    \begin{macrocode}
\cs_new_protected:Npn \@@_digits:NN #1#2
  {
    \tl_set:Nx #2
      { \exp_after:wN \@@_digits:nnnnnnn #1 }
  }
\cs_new:Npn \@@_digits:nnnnnnn #1#2#3#4#5#6#7
  {
    \exp_not:n { {#1} {#2} }
    {
      \@@_digits:Nn \l_@@_min_integer_int {#3}
      \exp_not:n {#3}
    }
    {
      \exp_not:n {#4}
      \@@_digits:Nn \l_@@_min_decimal_int {#4}
    }
    { \tl_if_blank:nF {#5} { \@@_digits_uncert:nn #5 } }
    \exp_not:n { {#6} {#7} }
  }
\cs_new:Npn \@@_digits:Nn #1#2
  {
    \int_compare:nNnT
      { #1 - \tl_count:n {#2} } > 0
      { \prg_replicate:nn { #1 - \tl_count:n {#2}  } { 0 } }
  }
\cs_new:Npn \@@_digits_uncert:nn #1#2
  {
    { #1 }
    { \use:c { @@_digits_uncert_ #1 :n } {#2} }
  }
\cs_new:Npn \@@_digits_uncert_S:n #1
  {
    \exp_not:n {#1}
    \@@_digits:Nn \l_@@_min_decimal_int {#1}
  }
%    \end{macrocode}
% \end{macro}
% \end{macro}
% \end{macro}
% \end{macro}
% \end{macro}
%
% \begin{macro}{\@@_drop_exponent:NN}
% \begin{macro}[EXP]{\@@_drop_exponent:nnnnnnn}
%   Simple stripping of the exponent.
%    \begin{macrocode}
\cs_new_protected:Npn \@@_drop_exponent:NN #1#2
  {
    \bool_if:NT \l_@@_drop_exponent_bool
      {
        \tl_set:Nx #2
          { \exp_after:wN \@@_drop_exponent:nnnnnnn #1 }
      }
  }
\cs_new:Npn \@@_drop_exponent:nnnnnnn #1#2#3#4#5#6#7
  { \exp_not:n { {#1} {#2} {#3} {#4} {#5} { } { 0 } } }
%    \end{macrocode}
% \end{macro}
% \end{macro}
%
% \begin{macro}{\@@_drop_uncertainty:NN}
% \begin{macro}[EXP]{\@@_drop_uncertainty:nnnnnnn}
%   Simple stripping of the uncertainty.
%    \begin{macrocode}
\cs_new_protected:Npn \@@_drop_uncertainty:NN #1#2
  {
    \bool_if:NTF \l_@@_drop_uncertainty_bool
      {
        \tl_set:Nx #2
          { \exp_after:wN \@@_drop_uncertainty:nnnnnnn #1 }
      }
      { \tl_set_eq:NN #2 #1 }

  }
\cs_new:Npn \@@_drop_uncertainty:nnnnnnn #1#2#3#4#5#6#7
  { \exp_not:n { {#1} {#2} {#3} {#4} { } {#6} {#7} } }
%    \end{macrocode}
% \end{macro}
% \end{macro}
%
% \begin{macro}{\@@_round:NN}
% \begin{macro}[EXP]{\@@_round_none:nnnnnnn}
%   Rounding is at the top level simple enough: fire off the expandable
%   set up which does the work.
%    \begin{macrocode}
\cs_new_protected:Npn \@@_round:NN #1#2
  {
    \tl_set:Nx #2
      {
        \cs:w
          @@_round_ \l_@@_round_mode_tl :nnnnnnn
          \exp_after:wN
        \cs_end: #1
      }
  }
\cs_new:Npn \@@_round_none:nnnnnnn #1#2#3#4#5#6#7
  { \exp_not:n { {#1} {#2} {#3} {#4} {#5} {#6} {#7} } }
%    \end{macrocode}
% \end{macro}
% \end{macro}
%
% \begin{macro}[EXP]{\@@_round:nnn, \@@_round:fnn}
% \begin{macro}[EXP]
%   {
%     \@@_round_auxi:nnnN  ,
%     \@@_round_auxii:nnnN ,
%     \@@_round_auxiii:nnnN
%   }
%  \begin{macro}[EXP]{\@@_round_auxiv:nnN, \@@_round_auxv:nnN}
%  \begin{macro}[EXP]{\@@_round_auxvi:nN}
%  \begin{macro}[EXP]{\@@_round_auxvii:nnN, \@@_round_auxviii:nnN}
%  \begin{macro}[EXP]{\@@_round_final_integer:nnw, \@@_round_final_decimal:nnw}
%  \begin{macro}[EXP]{\@@_round_final_output:nn, \@@_round_final_output:ff}
%  \begin{macro}[EXP]{\@@_round_final:nn, \@@_round_final:fn}
%  \begin{macro}[EXP]{\@@_round_final_shift:nn, \@@_round_final_shift:ff}
%  \begin{macro}[EXP]{\@@_round_final_shift:Nw}
%  \begin{macro}[EXP]
%    {
%      \@@_round_engineering:nn ,
%      \@@_round_fixed:nn       ,
%      \@@_round_input:nn       ,
%      \@@_round_scientifitc:nn
%    }
%  \begin{macro}[EXP]{\@@_round_engineering:NNNNn}
%  \begin{macro}[EXP]{\@@_round_engineering:nnN}
%  \begin{macro}[EXP]{\@@_round_truncate:n, \@@_round_truncate_direct:n}
%  \begin{macro}[EXP]{\@@_round_truncate:nnN}
%   Actually doing the rounding needs us to work from the least significant
%   digit, so we start by reversing the input. We \emph{could} also drop
%   digits in this phase, but tracking everything would be horrible, so
%   we go slightly slower but clearer and split the steps. First we reverse
%   the decimal part, then the integer.
%    \begin{macrocode}
\cs_new:Npn \@@_round:nnn #1#2#3
  {
    \@@_round_auxi:nnnN {#1} {#2} { }
      #3 \q_recursion_tail \q_recursion_stop
  }
\cs_generate_variant:Nn \@@_round:nnn { f }
\cs_new:Npn \@@_round_auxi:nnnN #1#2#3#4
  {
    \quark_if_recursion_tail_stop_do:Nn #4
      {
        \@@_round_auxii:nnnN {#1} {#3} { } #2
          \q_recursion_tail \q_recursion_stop
      }
    \@@_round_auxi:nnnN {#1} {#2} {#4#3}
  }
\cs_new:Npn \@@_round_auxii:nnnN #1#2#3#4
  {
    \quark_if_recursion_tail_stop_do:Nn #4
      {
        \tl_if_blank:nTF {#2}
          {
            \@@_round_auxiv:nnnN {#1} { } { } #3
              \q_recursion_tail \q_recursion_stop
          }
          {
            \@@_round_auxiii:nnnN {#1} {#3} { } #2
              \q_recursion_tail \q_recursion_stop
          }
      }
    \@@_round_auxii:nnnN {#1} {#2} {#4#3}
  }
%    \end{macrocode}
%   We now have the input reversed plus how many digits we need to discard
%   (|#1|).  We have two functions, one which deals with the decimal part,
%   one of which deals with the integer. In the latter, we should never hit
%   the end before we've dropped all the digits: the fixed-zero is a
%   fall-back in case something weird happens. For the integer case, we need
%   to collect up zeros to pad the length back out correctly later.
%    \begin{macrocode}
\cs_new:Npn \@@_round_auxiii:nnnN #1#2#3#4
  {
    \quark_if_recursion_tail_stop_do:Nn #4
      {
        \@@_round_auxiv:nnnN {#1} { } {#3} #2
          \q_recursion_tail \q_recursion_stop
      }
    \int_compare:nNnTF {#1} > 0
      {
        \exp_args:Nf \@@_round_auxiii:nnnN
          { \int_eval:n { #1 - 1 } } {#2} { #4#3 }
      }
      { \@@_round_auxv:nnN {#3} {#2} #4 }
  }
\cs_new:Npn \@@_round_auxiv:nnnN #1#2#3#4
  {
    \quark_if_recursion_tail_stop_do:Nn #4
      { { 0 } { } }
    \int_compare:nNnTF {#1} > 0
      {
        \exp_args:Nf \@@_round_auxiv:nnnN
          { \int_eval:n { #1 - 1 } } { #2 0 } { #4#3 }
      }
      { \@@_round_auxvi:nnnN {#3} {#2} #4 }
  }
%    \end{macrocode}
%   The lead off to rounding proper needs to deal with the half-even rule:
%   it can only apply at this stage, when the \emph{discarded} value can
%   be exactly half.
%    \begin{macrocode}
\cs_new:Npn \@@_round_auxv:nnN #1#2#3
  {
    \quark_if_recursion_tail_stop_do:Nn #3
      {
        \@@_round_auxvi:nnN
          {#1} { } #2 \q_recursion_tail \q_recursion_stop
      }
    \bool_lazy_or:nnTF
      { \int_compare_p:nNn { 0 \tl_head:n {#1} } < 5 }
      {
        \bool_lazy_all_p:n
          {
            { \l_@@_round_half_even_bool }
            { \int_if_odd_p:n {#3} }
            { \@@_round_if_half_p:n {#1} }
          }
      }
      { \@@_round_final_decimal:nnw }
      { \@@_round_auxvii:nnN }
        {#2} { } #3
  }
\cs_new:Npn \@@_round_auxvi:nnnN #1#2#3
  {
    \quark_if_recursion_tail_stop_do:Nn #3
      { { 0 } { } }
    \bool_lazy_or:nnTF
      { \int_compare_p:nNn { 0 \tl_head:n {#1} } < 5 }
      {
        \bool_lazy_all_p:n
          {
            { \l_@@_round_half_even_bool }
            { \int_if_odd_p:n {#3} }
            { \@@_round_if_half_p:n {#1} }
          }
      }
      { \@@_round_final_integer:nnw }
      { \@@_round_auxviii:nnN }
        { } {#2} #3
  }
%    \end{macrocode}
%   The main rounding routines. These are only every called when there is
%   rounding to do, so there is no need to carry a flag forward. Thus the
%   question to ask is simple: is the next value a $9$ or not (as that
%   continues the sequence). There is a general need to handle the case
%   where a zero is rounded up: that automatically means a need to trim
%   the other end.
%    \begin{macrocode}
\cs_new:Npn \@@_round_auxvii:nnN #1#2#3
  {
    \quark_if_recursion_tail_stop_do:Nn #3
      {
        \str_if_eq:nnTF {#1} { 0 }
          {
            \@@_round_final_output:ff
              { 1 }
              { \@@_round_truncate:n {#2} }
          }
          {
            \@@_round_auxviii:nnN {#2} { } #1
              \q_recursion_tail \q_recursion_stop
          }
      }
    \int_compare:nNnTF {#3} = 9
      { \@@_round_auxvii:nnN {#1} { 0 #2 } }
      {
        \int_compare:nNnTF {#3} = 0
          {
            \@@_round_final_decimal:nnw
              {#1} { 1 \@@_round_truncate:n {#2} }
          }
          {
            \@@_round_final:fn
              { \int_eval:n { #3 + 1 } }
              { \@@_round_final_decimal:nnw {#1} {#2} }
          }
      }
  }
\cs_new:Npn \@@_round_auxviii:nnN #1#2#3
  {
    \quark_if_recursion_tail_stop_do:Nn #3
      {
        \tl_if_blank:nTF {#1}
          {
            \@@_round_final_shift:ff
              {
                \exp_last_unbraced:Nf 1
                  { \@@_round_truncate_direct:n {#2} } 0
              }
              { }
          }
          {
            \@@_round_final_shift:ff
              { 1 #2 }
              { \@@_round_truncate:n {#1} }
          }
      }
    \int_compare:nNnTF {#3} = 9
      { \@@_round_auxviii:nnN {#1} { 0 #2 } }
      {
        \@@_round_final:fn
          { \int_eval:n { #3 + 1 } }
          { \@@_round_final_integer:nnw {#1} {#2} }
      }
  }
%    \end{macrocode}
%   Tidying up means grabbing the remaining digits and undoing the reversal.
%    \begin{macrocode}
\cs_new:Npn \@@_round_final_decimal:nnw
  #1#2#3 \q_recursion_tail \q_recursion_stop
  {
    \@@_round_final_output:ff
      { \tl_reverse:n {#1} }
      { \tl_reverse:n {#3} #2 }
  }
\cs_new:Npn \@@_round_final_integer:nnw
  #1#2#3 \q_recursion_tail \q_recursion_stop
  {
    \@@_round_final_output:ff
      { \tl_reverse:n {#3} #2 }
      {#1}
  }
\cs_new:Npn \@@_round_final_output:nn #1#2 { {#1} {#2} }
\cs_generate_variant:Nn \@@_round_final_output:nn { ff }
\cs_new:Npn \@@_round_final:nn #1#2
  { #2 #1 }
\cs_generate_variant:Nn \@@_round_final:nn { f }
%    \end{macrocode}
%   Here we deal with the case where rounding applies along with an
%   exponent set based on number of places. We can only get here if an
%   additional integer digit has been added, so there is no need to test for
%   that. There are two cases for action: when using |scientific| mode, where
%   we always need to shift by one, and when using |engineering| mode if
%   we now have four digits. The latter is a bit more work: we need to trim
%   digits off as required.
%    \begin{macrocode}
\cs_new:Npn \@@_round_final_shift:nn #1#2
  {
    \str_if_eq:VnTF \l_@@_round_mode_tl { places }
      {
        \use:c
          { @@_round_ \l_@@_exponent_mode_tl :nn }
          {#1} {#2}
      }
      { {#1} {#2} }
  }
\cs_generate_variant:Nn \@@_round_final_shift:nn { ff }
\cs_new:Npn \@@_round_engineering:nn #1#2
  {
    \int_compare:nNnTF { \tl_count:n {#1} } = 4
      {
        \@@_round_engineering:NNNNn #1 {#2}
        { }
        \@@_round_final_shift:Nw 3
      }
      { {#1} {#2} }
  }
\cs_new:Npn \@@_round_engineering:NNNNn #1#2#3#4#5
  {
    {#1}
    \exp_args:NV \@@_round_engineering:nnN
      { \l_@@_round_precision_int } { }
      #2#3#4#5 \q_recursion_tail \q_recursion_stop
  }
\cs_new:Npn \@@_round_engineering:nnN #1#2#3
  {
    \quark_if_recursion_tail_stop_do:Nn #3 { {#2} }
    \int_compare:nNnTF {#1} = { 0 }
      { \use_i_delimit_by_q_recursion_stop:nw { {#2} } }
      { \@@_round_engineering:nnN { #1 - 1 } { #2#3 } }
  }
\cs_new:Npn \@@_round_fixed:nn #1#2 { {#1} {#2} }
\cs_new:Npn \@@_round_input:nn #1#2 { {#1} {#2} }
\cs_new:Npn \@@_round_scientific:nn #1#2
  {
    \@@_exponent_shift:nnf
      { 1 } {#1} { \@@_round_truncate_direct:n {#2} }
    { }
    \@@_round_final_shift:Nw 1
  }
\cs_new:Npn \@@_round_final_shift:Nw #1#2 \@@_round_places_end:nn #3#4
  { \@@_exponent_finalise:n { #3#4 + #1 } }
%    \end{macrocode}
%   When we have rounded up to the next power of ten, we need to go back and
%   remove one more digit. That only happens when rounding to a number of
%   figures or when dealing with an integer part.
%    \begin{macrocode}
\cs_new:Npn \@@_round_truncate:n #1
  {
    \str_if_eq:VnTF \l_@@_round_mode_tl { figures }
      { \@@_round_truncate_direct:n {#1} }
      {#1}
  }
\cs_new:Npn \@@_round_truncate_direct:n #1
  {
    \@@_round_truncate:nnN { } { }
      #1 \q_recursion_tail \q_recursion_stop
  }
\cs_new:Npn \@@_round_truncate:nnN #1#2#3
  {
    \quark_if_recursion_tail_stop_do:Nn #3 { #1 }
    \@@_round_truncate:nnN {#1#2} {#3}
  }
%    \end{macrocode}
% \end{macro}
% \end{macro}
% \end{macro}
% \end{macro}
% \end{macro}
% \end{macro}
% \end{macro}
% \end{macro}
% \end{macro}
% \end{macro}
% \end{macro}
% \end{macro}
% \end{macro}
% \end{macro}
% \end{macro}
%
% \begin{macro}[EXP]{\@@_round_if_half_p:n}
% \begin{macro}[EXP]{\@@_round_if_half:N}
%   A simple test for a valuing being exactly half: we can only test
%   digit-by-digit as there is no limit on the size of the value given.
%    \begin{macrocode}
\prg_new_conditional:Npnn \@@_round_if_half:n #1 { p }
  {
    \int_compare:nNnTF { \tl_head:n { #1 0 } } = 5
      {
        \exp_after:wN \@@_round_if_half:N \use_none:n #1 0
          \q_recursion_tail \q_recursion_stop
      }
      { \prg_return_false: }
  }
\cs_new:Npn \@@_round_if_half:N #1
  {
    \quark_if_recursion_tail_stop_do:Nn #1
      { \prg_return_true: }
    \int_compare:nNnTF {#1} = 0
      { \@@_round_if_half:N }
      { \use_i_delimit_by_q_recursion_stop:nw { \prg_return_false: } }
  }
%    \end{macrocode}
% \end{macro}
% \end{macro}
%
% \begin{macro}[EXP]{\@@_round_pad:nnn}
%   The case where we are short of digits is easy enough to handle:
%   generate zeros to pad it out.
%    \begin{macrocode}
\cs_new:Npn \@@_round_pad:nnn #1#2#3
  {
    {#2}
    {
      #3
      \bool_if:NT \l_@@_round_pad_bool
        { \prg_replicate:nn {#1} { 0 } }
    }
  }
%    \end{macrocode}
% \end{macro}
%
% \begin{macro}[EXP]{\@@_round_figures:nnnnnnn}
% \begin{macro}[EXP]{\@@_round_figures_count:nnN}
% \begin{macro}[EXP]{\@@_round_figures_count:nnnN}
%   Rounding to a fixed number of significant figures starts by checking that
%   there is no uncertainty, and that the number of figures requested is
%   positive: if not, the result is always fixed at zero.
%    \begin{macrocode}
\cs_new:Npn \@@_round_figures:nnnnnnn #1#2#3#4#5#6#7
  {
    \tl_if_blank:nTF {#5}
      {
        \int_compare:nNnTF \l_@@_round_precision_int > 0
          {
            \exp_not:n { {#1} {#2} }
            \@@_round_figures_count:nnN {#3} {#4} #3#4
              \q_recursion_tail \q_recursion_stop
            \exp_not:n { { } {#6} {#7} }
          }
          { { } { } { 0 } { } { } { } { 0 } }
      }
      { \exp_not:n { {#1} {#2} {#3} {#4} {#5} {#6} {#7} } }
  }
%    \end{macrocode}
%   The first real step is to count up the number of significant figures.
%   The only tricky issue here is dealing with leading zeros.
%    \begin{macrocode}
\cs_new:Npn \@@_round_figures_count:nnN #1#2#3
  {
    \quark_if_recursion_tail_stop_do:Nn #3
      { { } { } { 0 } { } { } { } { 0 } }
    \int_compare:nNnTF {#3} = 0
      { \@@_round_figures_count:nnN {#1} {#2} }
      { \@@_round_figures_count:nnnN { 1 } {#1} {#2} }
  }
\cs_new:Npn \@@_round_figures_count:nnnN #1#2#3#4
  {
    \quark_if_recursion_tail_stop_do:Nn #4
      {
        \int_compare:nNnTF {#1} > \l_@@_round_precision_int
          {
            \@@_round:fnn
              { \int_eval:n { #1 - \l_@@_round_precision_int } }
              {#2} {#3}
          }
          {
            \@@_round_pad:nnn
              { \l_@@_round_precision_int - (#1) } {#2} {#3}
          }
      }
    \exp_args:Nf \@@_round_figures_count:nnnN
      { \int_eval:n { #1 + 1 } } {#2} {#3}
  }
%    \end{macrocode}
% \end{macro}
% \end{macro}
% \end{macro}
%
% \begin{macro}[EXP]{\@@_round_places:nnnnnnn}
% \begin{macro}[EXP]{\@@_round_places_end:nn}
% \begin{macro}[EXP]{\@@_round_places_decimal:nn, \@@_round_places_integer:nn}
% \begin{macro}[EXP]{\@@_round_places_finalise:n}
% \begin{macro}[EXP]{\@@_round_places_finalise:nnnnnnn}
% \begin{macro}[EXP]{\@@_round_places_finalise:nnnnn}
%   The first step when rounding to a fixed number of places is to establish
%   if this is in the decimal or integer parts. The two require different
%   calculations for how many digits to drop from the input. The no-op end
%   function here is to allow tidying up in some cases: see the finalisation
%   of rounding.
%    \begin{macrocode}
\cs_new:Npn \@@_round_places:nnnnnnn #1#2#3#4#5#6#7
  {
    \tl_if_blank:nTF {#5}
      {
        \exp_args:Ne \@@_round_places_finalise:n
          {
            \exp_not:n { {#1} {#2} }
            \int_compare:nNnTF \l_@@_round_precision_int > 0
              { \@@_round_places_decimal:nn }
              { \@@_round_places_integer:nn }
                {#3} {#4}
            \@@_round_places_end:nn {#6} {#7}
          }
      }
      { \exp_not:n { {#1} {#2} {#3} {#4} {#5} {#6} {#7} } }
  }
\cs_new:Npn \@@_round_places_end:nn #1#2 { { } \exp_not:n { {#1} {#2} } }
\cs_new:Npn \@@_round_places_decimal:nn #1#2
  {
    \int_compare:nNnTF
      { \l_@@_round_precision_int - 0 \tl_count:n {#2} } > 0
      {
        \@@_round_pad:nnn
          { \l_@@_round_precision_int - 0 \tl_count:n {#2} }
          {#1} {#2}
      }
      {
        \@@_round:fnn
           {
             \int_eval:n
               { 0 \tl_count:n {#2} - \l_@@_round_precision_int }
           }
           {#1} {#2}
      }
  }
\cs_new:Npn \@@_round_places_integer:nn #1#2
  {
    \@@_round:fnn
       {
         \int_eval:n
           { 0 \tl_count:n {#2} - \l_@@_round_precision_int }
       }
       {#1} {#2}
  }
%    \end{macrocode}
%   To finalise rounding to places, we have to worry about a minimum value:
%   that is basically a case of looking for value of zero and rearranging. We
%   also need to worry about a \enquote{negative zero} arising.
%    \begin{macrocode}
\cs_new:Npn \@@_round_places_finalise:n #1
  { \@@_round_places_finalise:nnnnnnn #1 }
\cs_new:Npn \@@_round_places_finalise:nnnnnnn #1#2#3#4#5#6#7
  {
    \bool_lazy_and:nnTF
      { \str_if_eq_p:nn {#3} { 0 } }
      {
        \str_if_eq_p:ee
          { \exp_not:n {#4} } { \prg_replicate:nn { \tl_count:n {#4} } { 0 } }
      }
      {
        \tl_if_empty:NTF \l_@@_round_min_tl
          {
            \exp_not:n { {#1} }
            { \str_if_eq:nnF {#2} { - } { \exp_not:n {#2} } }
            \exp_not:n { {#3} {#4} {#5} {#6} {#7} }
          }
          {
            \exp_after:wN \@@_round_places_finalise:nnnnn
              \l_@@_round_min_tl {#2} {#6} {#7}
          }
      }
      { \exp_not:n { {#1} {#2} {#3} {#4} {#5} {#6} {#7} } }
  }
\cs_new:Npn \@@_round_places_finalise:nnnnn #1#2#3#4#5
  {
    {
      \str_if_eq:nnTF {#3} { - }
        { > }
        { < }
    }
    \exp_not:n { {#3} {#1} {#2} { } {#4} {#5} }
  }
%    \end{macrocode}
% \end{macro}
% \end{macro}
% \end{macro}
% \end{macro}
% \end{macro}
% \end{macro}
%
% \begin{macro}[EXP]{\@@_round_uncertainty:nnnnnnn}
% \begin{macro}[EXP]{\@@_round_uncertainty:nnn}
% \begin{macro}[EXP]{\@@_round_uncertainty:nnnnn}
%   Rounding to an uncertainty can only happen where the result will have some
%   uncertainty left: otherwise we simply drop the uncertainty entirely. Only
%   |S|-type uncertainties can be used for rounding.
%    \begin{macrocode}
\cs_new:Npn \@@_round_uncertainty:nnnnnnn #1#2#3#4#5#6#7
  {
    \bool_lazy_or:nnTF
      { \tl_if_blank_p:n {#5} }
      { ! \int_compare_p:nNn \l_@@_round_precision_int > 0 }
      { \exp_not:n { {#1} #2 {#3} {#4} { } #6 {#7} } }
      {
        \str_if_eq:eeTF { \tl_head:n {#5} } { S }
          {
            \exp_not:n { {#1} {#2} }
            \exp_args:Nnno \@@_round_uncertainty:nnn
              {#3} {#4} { \use_ii:nn #5 }
            \exp_not:n { {#6} {#7} }
          }
          { \exp_not:n { {#1} {#2} {#3} {#4} {#5} {#6} {#7} } }
      }
  }
%    \end{macrocode}
%   Round the uncertainty first: this is needed to get the number of places
%   correct (for the case where the uncertainty rounds up to |1...|). Once that
%   is done, it's just a question of working out the digits in the main part.
%    \begin{macrocode}
\cs_new:Npn \@@_round_uncertainty:nnn #1#2#3
  {
    \exp_last_unbraced:Nf \@@_round_uncertainty:nnnnn
      {
        \@@_round:fnn
          { \tl_count:n {#3} - \l_@@_round_precision_int } { } {#3}
      }
      {#1} {#2} {#3}
  }
\cs_new:Npn \@@_round_uncertainty:nnnnn #1#2#3#4#5
  {
    \tl_if_blank:nTF {#1}
      {
        \@@_round:fnn
          { \tl_count:n {#5} - \tl_count:n {#2} } {#3} {#4}
        { { S } {#2} }
      }
      {
        \@@_round:fnn
          { \tl_count:n {#5} - \tl_count:n {#2} + 1 } {#3} {#4}
        { { S } { #1 \@@_round_truncate_direct:n {#2} } }
      }
  }
%    \end{macrocode}
% \end{macro}
% \end{macro}
% \end{macro}
%
% \begin{macro}{\@@_zero_decimal:NN}
% \begin{macro}[EXP]{\@@_zero_decimal:nnnnnnn}
%   Simple stripping of the decimal part if zero.
%    \begin{macrocode}
\cs_new_protected:Npn \@@_zero_decimal:NN #1#2
  {
    \bool_if:NT \l_@@_drop_zero_decimal_bool
      {
        \tl_set:Nx #2
          { \exp_after:wN \@@_zero_decimal:nnnnnnn #1 }
      }
  }
\cs_new:Npn \@@_zero_decimal:nnnnnnn #1#2#3#4#5#6#7
  {
    \exp_not:n { {#1} {#2} {#3} }
    \str_if_eq:eeTF
      { \exp_not:n {#4} }
      { \prg_replicate:nn { \tl_count:n {#4} } { 0 } }
      { { } }
      { \exp_not:n { {#4} } }
    \exp_not:n { {#5} {#6} {#7} }
  }
%    \end{macrocode}
% \end{macro}
% \end{macro}
%
% \subsection{Number modification}
%
% \begin{macro}[rEXP]{\siunitx_number_adjust_exponent:nn}
% \begin{macro}[rEXP]{\siunitx_number_adjust_exponent:Nn}
% \begin{macro}[rEXP]{\@@_adjust_exp:nnnnnnnn}
% \begin{macro}[rEXP]{\@@_adjust_exp:nn}
% \begin{macro}[rEXP]{\@@_adjust_exp:nNw}
%   A simply case of breaking down and rebuilding the number.
%    \begin{macrocode}
\cs_new:Npn \siunitx_number_adjust_exponent:nn #1#2
  { \@@_adjust_exp:nnnnnnnn #1 {#2} }
\cs_new:Npn \siunitx_number_adjust_exponent:Nn #1#2
  {
    \tl_if_empty:NF #1
      { \exp_args:NV \siunitx_number_adjust_exponent:nn #1 {#2} }
  }
\cs_new:Npn \@@_adjust_exp:nnnnnnnn #1#2#3#4#5#6#7#8
  {
    \exp_not:n { {#1} {#2} {#3} {#4} {#5} }
    \exp_args:Ne \@@_adjust_exp:nn { \fp_eval:n { #6#7 + #8 } } {#6}
  }
\cs_new:Npn \@@_adjust_exp:nn #1#2
  { \@@_adjust_exp:nNw {#2} #1 \q_stop }
\cs_new:Npn \@@_adjust_exp:nNw #1#2#3 \q_stop
  {
    \token_if_eq_meaning:NNTF #2 -
      { { - } { \exp_not:n {#3} } }
      { { \str_if_eq:nnT {#1} { + } { + } } { \exp_not:n {#2#3} } }
  }
%    \end{macrocode}
% \end{macro}
% \end{macro}
% \end{macro}
% \end{macro}
% \end{macro}
%
% \subsection{Outputting parsed numbers}
%
% \begin{variable}{\l_@@_bracket_close_tl, \l_@@_bracket_open_tl}
%   Purely internal for the present.
%    \begin{macrocode}
\tl_new:N \l_@@_bracket_close_tl
\tl_new:N \l_@@_bracket_open_tl
\tl_set:Nn \l_@@_bracket_open_tl { ( }
\tl_set:Nn \l_@@_bracket_close_tl { ) }
%    \end{macrocode}
% \end{variable}
%
% \begin{variable}{\l_siunitx_number_bracket_ambiguous_bool}
%    \begin{macrocode}
\bool_new:N \l_siunitx_number_bracket_ambiguous_bool
%    \end{macrocode}
% \end{variable}
%
% \begin{variable}{\l_siunitx_number_output_decimal_tl}
%    \begin{macrocode}
\tl_new:N \l_siunitx_number_output_decimal_tl
%    \end{macrocode}
% \end{variable}
%
% \begin{variable}
%   {
%      \l_@@_bracket_negative_bool  ,
%      \l_@@_implicit_plus_bool     ,
%      \l_@@_exponent_base_tl       ,
%      \l_@@_exponent_product_tl    ,
%      \l_@@_group_decimal_bool     ,
%      \l_@@_group_integer_bool     ,
%      \l_@@_group_minimum_int      ,
%      \l_@@_group_separator_tl     ,
%      \l_@@_negative_color_tl      ,
%      \l_@@_output_exp_marker_tl   ,
%      \l_@@_output_uncert_close_tl ,
%      \l_@@_output_uncert_open_tl  ,
%      \l_@@_uncert_mode_tl         ,
%      \l_@@_uncert_separator_tl    ,
%      \l_@@_tight_bool             ,
%      \l_@@_unity_mantissa_bool    ,
%      \l_@@_zero_exponent_bool
%   }
%   Keys producing tokens in the output.
%    \begin{macrocode}
\keys_define:nn { siunitx }
  {
    bracket-ambiguous-numbers .bool_set:N =
      \l_siunitx_number_bracket_ambiguous_bool ,
    bracket-negative-numbers .bool_set:N =
      \l_@@_bracket_negative_bool ,
    exponent-base .tl_set:N =
      \l_@@_exponent_base_tl ,
    exponent-product .tl_set:N =
      \l_@@_exponent_product_tl ,
    group-digits .choice: ,
    group-digits / all .code:n =
      {
        \bool_set_true:N \l_@@_group_decimal_bool
        \bool_set_true:N \l_@@_group_integer_bool
      } ,
    group-digits / decimal .code:n =
      {
        \bool_set_true:N  \l_@@_group_decimal_bool
        \bool_set_false:N \l_@@_group_integer_bool
      } ,
    group-digits / integer .code:n =
      {
        \bool_set_false:N \l_@@_group_decimal_bool
        \bool_set_true:N  \l_@@_group_integer_bool
      } ,
    group-digits / none .code:n =
      {
        \bool_set_false:N \l_@@_group_decimal_bool
        \bool_set_false:N \l_@@_group_integer_bool
      } ,
    group-digits .default:n  = all ,
    group-minimum-digits .int_set:N  =
      \l_@@_group_minimum_int ,
    group-separator .tl_set:N =
      \l_@@_group_separator_tl ,
    negative-color .tl_set:N =
    \l_@@_negative_color_tl ,
    output-close-uncertainty .tl_set:N =
      \l_@@_output_uncert_close_tl ,
    output-decimal-marker .tl_set:N =
      \l_siunitx_number_output_decimal_tl ,
    output-exponent-marker .tl_set:N =
      \l_@@_output_exp_marker_tl ,
    output-open-uncertainty .tl_set:N =
      \l_@@_output_uncert_open_tl ,
    print-implicit-plus .bool_set:N =
      \l_@@_implicit_plus_bool ,
    print-unity-mantissa .bool_set:N =
      \l_@@_unity_mantissa_bool ,
    print-zero-exponent .bool_set:N =
      \l_@@_zero_exponent_bool ,
    tight-spacing .bool_set:N =
      \l_@@_tight_bool ,
    uncertainty-mode .choices:nn =
      { compact , compact-marker , full , separate }
      { \tl_set_eq:NN \l_@@_uncert_mode_tl \l_keys_choice_tl } ,
    uncertainty-separator .tl_set:N =
      \l_@@_uncert_separator_tl
  }
\bool_new:N \l_@@_group_decimal_bool
\bool_new:N \l_@@_group_integer_bool
\tl_new:N \l_@@_uncert_mode_tl
%    \end{macrocode}
% \end{variable}
%
% \begin{macro}[rEXP]{\siunitx_number_output:N}
% \begin{macro}[rEXP]{\siunitx_number_output:n}
% \begin{macro}[rEXP]{\siunitx_number_output:NN}
% \begin{macro}[rEXP]{\siunitx_number_output:nN}
% \begin{macro}[rEXP]{\@@_output:Nn}
% \begin{macro}[rEXP]{\@@_output:nn}
% \begin{macro}[rEXP]{\@@_output:nnnnnnn}
% \begin{macro}[rEXP]{\@@_output_bracket:nn}
% \begin{macro}[rEXP]{\@@_output_bracket:w}
% \begin{macro}[rEXP]{\@@_output_comparator:nn}
% \begin{macro}[rEXP]{\@@_output_sign:nnn}
% \begin{macro}[rEXP]{\@@_output_sign:nN}
% \begin{macro}[rEXP]{\@@_output_sign:N}
% \begin{macro}[rEXP]
%   {\@@_output_sign_color:w, \@@_output_sign_brackets:w}
% \begin{macro}[rEXP]{\@@_output_integer:nnn}
% \begin{macro}[rEXP]{\@@_output_decimal:nn, \@@_output_decimal:fn}
% \begin{macro}[rEXP]{\@@_output_digits:nn}
% \begin{macro}[rEXP]{\@@_output_integer_aux:n}
% \begin{macro}[rEXP]
%   {
%     \@@_output_integer_aux_0:n,
%     \@@_output_integer_aux_1:n,
%     \@@_output_integer_aux_2:n
%   }
% \begin{macro}[rEXP]{\@@_output_decimal_aux:n}
% \begin{macro}[rEXP]{\@@_output_decimal_loop:NNNN}
% \begin{macro}[rEXP]{\@@_output_integer_first:nnNN}
% \begin{macro}[rEXP]{\@@_output_integer_loop:NNNN}
% \begin{macro}[rEXP]{\@@_output_uncertainty:nnnn}
% \begin{macro}[rEXP]{\@@_output_uncertainty_unaligned:n}
% \begin{macro}[rEXP]{\@@_output_uncert_S:nnnnw}
% \begin{macro}[rEXP]{\@@_output_uncert_S_auxi:nn}
% \begin{macro}[rEXP]{\@@_output_uncert_S_auxii:nnN}
% \begin{macro}[rEXP]{\@@_output_uncert_S_auxiii:nnn}
% \begin{macro}[rEXP]
%   {\@@_output_uncert_S_auxiv:nnn, \@@_output_uncert_S_auxiv:fnn}
% \begin{macro}[rEXP]
%   {\@@_output_uncert_S_auxv:nnnw, \@@_output_uncert_S_auxv:fnnw}
% \begin{macro}[rEXP]{\@@_output_uncert_S_auxvi:nnw}
% \begin{macro}[rEXP]
%   {
%     \@@_output_uncert_S_compact:nn        ,
%     \@@_output_uncert_S_compact-marker:nn ,
%     \@@_output_uncert_S_full:nn
%   }
% \begin{macro}[rEXP]
%   {
%     \@@_output_exponent:nnnn       ,
%     \@@_output_exponent_auxi:nnnn  ,
%     \@@_output_exponent_auxii:nnnn
%   }
% \begin{macro}[rEXP]{\@@_output_exponent_auxiii:nn}
% \begin{macro}[rEXP]{\@@_output_end:}
%   The approach to formatting a single number is to split into
%   the constituent parts. All of the parts are assembled including
%   inserting tabular alignment markers (which may be empty) for each
%   separate unit.
%    \begin{macrocode}
\cs_new:Npn \siunitx_number_output:N #1
  { \@@_output:Nn #1 { } }
\cs_new:Npn \siunitx_number_output:n #1
  { \@@_output:nn #1 { } }
\cs_new:Npn \siunitx_number_output:NN #1#2
  { \@@_output:Nn #1 {#2} }
\cs_new:Npn \siunitx_number_output:nN #1#2
  { \@@_output:nn #1 {#2} }
\cs_new:Npn \@@_output:Nn #1#2
  {
    \tl_if_empty:NF #1
      { \exp_after:wN \@@_output:nnnnnnn #1 {#2} }
  }
\cs_new:Npn \@@_output:nn #1#2
  {
    \tl_if_empty:nF {#1}
      { \@@_output:nnnnnnn #1 {#2} }
  }
\cs_new:Npn \@@_output:nnnnnnn #1#2#3#4#5#6#7#8
  {
    \@@_output_color:n {#2}
    \@@_output_comparator:nn {#1} {#8}
    \@@_output_bracket:nn {#5} {#7}
    \@@_output_sign:nnn {#1} {#2} {#8}
    \@@_output_integer:nnn {#3} {#4} {#7}
    \@@_output_decimal:nn {#4} {#8}
    \@@_output_uncertainty:nnnn {#5} {#3} {#4} {#8}
    \@@_output_exponent:nnnn {#6} {#7} { #3 . #4 } {#8}
    \@@_output_end:
  }
%    \end{macrocode}
%   Adding brackets for the combination of a separate uncertainty with an
%   exponent may need brackets. This needs testing up-front, so has to come
%   before the main formatting routines.
%    \begin{macrocode}
\cs_new:Npn \@@_output_bracket:nn #1#2
  {
    \bool_lazy_all:nT
      {
        { \str_if_eq_p:Vn \l_@@_uncert_mode_tl { separate } }
        { \l_siunitx_number_bracket_ambiguous_bool }
        { ! \tl_if_blank_p:n {#1} }
        {
          \bool_lazy_or_p:nn
            { \l_@@_zero_exponent_bool }
            { ! \str_if_eq_p:nn {#2} { 0 } }
        }
      }
    \@@_output_bracket:w
  }
\cs_new:Npn \@@_output_bracket:w #1 \@@_output_exponent:nnnn
  {
    \exp_not:V \l_@@_bracket_open_tl
    #1
    \exp_not:V \l_@@_bracket_close_tl
    \@@_output_exponent:nnnn
  }
%    \end{macrocode}
%   As color for negative values applies to the \emph{whole} output, we have
%   to deal with it before anything else. 
%    \begin{macrocode}
\cs_new:Npn \@@_output_color:n #1
  {
    \bool_lazy_and:nnT
      { \str_if_eq_p:nn {#1} { - } }
      { ! \tl_if_empty_p:N \l_@@_negative_color_tl }
      { \exp_not:N \color { \exp_not:V \l_@@_negative_color_tl } }
  }
%    \end{macrocode}
%   To get the spacing correct this needs to be an ordinary math character.
%    \begin{macrocode}
\cs_new:Npn \@@_output_comparator:nn #1#2
  {
    \tl_if_blank:nF {#1}
      { \exp_not:n { \mathord {#1} } }
    \exp_not:n {#2}
  }
%    \end{macrocode}
%   Formatting signs has to deal with some additional formatting requirements
%   for negative numbers. Making such numbers by bracketing them needs some
%   rearrangement of the order of tokens, which is set up in the main
%   formatting macro by the dedicated do-nothing end function. We also have
%   the comparator passed here: if it is present, we need to deal with
%   tighter spacing.
%    \begin{macrocode}
\cs_new:Npn \@@_output_sign:nnn #1#2#3
  {
    \tl_if_blank:nTF {#2}
      {
        \bool_if:NT \l_@@_implicit_plus_bool
          { \@@_output_sign:nN {#1} + }
      }
      {
        \str_if_eq:nnTF {#2} { - }
          {
            \bool_if:NTF \l_@@_bracket_negative_bool
              { \@@_output_sign_brackets:w }
              { \@@_output_sign:nN {#1} #2 }
          }
          { \@@_output_sign:nN {#1} #2 }
      }
    \exp_not:n {#3}
  }
\cs_new:Npn \@@_output_sign:nN #1#2
  {
    \tl_if_blank:nTF {#1}
      { \@@_output_sign:N #2 }
      { \exp_not:n { \mathord {#2} } }
  }
\cs_new:Npn \@@_output_sign:N #1
  {
    \bool_if:NTF \l_@@_tight_bool
      { \exp_not:n { \mathord {#1} } }
      { \exp_not:n {#1} }
  }
\cs_new:Npn
  \@@_output_sign_brackets:w #1 \@@_output_end:
  {
    \exp_not:V \l_@@_bracket_open_tl
    #1
    \exp_not:V \l_@@_bracket_close_tl
    \@@_output_end:
  }
%    \end{macrocode}
%   Digit formatting leads off with separate functions to allow for a few
%   \enquote{up front} items before using a common set of tests for some common
%   cases. The code then splits again as the two types of grouping need
%   different strategies.
%    \begin{macrocode}
\cs_new:Npn \@@_output_integer:nnn #1#2#3
  {
    \bool_lazy_or:nnT
      { \l_@@_unity_mantissa_bool }
      { ! \str_if_eq_p:nn { #1 . #2 } { 1. } }
      { \@@_output_digits:nn { integer } {#1} }
  }
\cs_new:Npn \@@_output_decimal:nn #1#2
  {
    \exp_not:n {#2}
    \tl_if_blank:nF {#1}
      {
        \str_if_eq:VnTF \l_siunitx_number_output_decimal_tl { , }
          { \exp_not:N \mathord }
          { \use:n }
            { \exp_not:V \l_siunitx_number_output_decimal_tl }
      }
    \exp_not:n {#2}
    \@@_output_digits:nn { decimal } {#1}
  }
\cs_generate_variant:Nn \@@_output_decimal:nn { f }
\cs_new:Npn \@@_output_digits:nn #1#2
  {
    \bool_if:cTF { l_@@_group_ #1 _ bool }
      {
        \int_compare:nNnTF
          { \tl_count:n {#2} } < \l_@@_group_minimum_int
          { \exp_not:n {#2} }
          { \use:c { @@_output_ #1 _aux:n } {#2} }
      }
      { \exp_not:n {#2} }
  }
%    \end{macrocode}
%   For integers, we need to know how many digits there are to allow for the
%   correct insertion of separators. That is done using a two-part set up such
%   that there is no separator on the first pass.
%    \begin{macrocode}
\cs_new:Npn \@@_output_integer_aux:n #1
  {
     \use:c
       {
         @@_output_integer_aux_
         \int_eval:n { \int_mod:nn { \tl_count:n {#1} } { 3 } }
         :n
       } {#1}
  }
\cs_new:cpn { @@_output_integer_aux_0:n } #1
  { \@@_output_integer_first:nnNN #1 \q_nil }
\cs_new:cpn { @@_output_integer_aux_1:n } #1
  { \@@_output_integer_first:nnNN { } { } #1 \q_nil }
\cs_new:cpn { @@_output_integer_aux_2:n } #1
  { \@@_output_integer_first:nnNN { } #1 \q_nil }
\cs_new:Npn \@@_output_integer_first:nnNN #1#2#3#4
  {
    \exp_not:n {#1#2#3}
    \quark_if_nil:NF #4
      { \@@_output_integer_loop:NNNN #4 }
  }
\cs_new:Npn \@@_output_integer_loop:NNNN #1#2#3#4
  {
    \str_if_eq:VnTF \l_@@_group_separator_tl { , }
      { \exp_not:N \mathord }
      { \use:n }
        { \exp_not:V \l_@@_group_separator_tl }
    \exp_not:n {#1#2#3}
    \quark_if_nil:NF #4
      { \@@_output_integer_loop:NNNN #4 }
  }
%    \end{macrocode}
%   For decimals, no need to do any counting, just loop using enough markers to
%   find the end of the list. By passing the decimal marker, it is possible not
%   to have to use a check on the content of the rest of the number. The
%   |\use_none:n(n)| mop up the remaining |\q_nil| tokens.
%    \begin{macrocode}
\cs_new:Npn \@@_output_decimal_aux:n #1
  {
    \@@_output_decimal_loop:NNNN \c_empty_tl
      #1 \q_nil \q_nil \q_nil
  }
\cs_new:Npn \@@_output_decimal_loop:NNNN #1#2#3#4
  {
    \quark_if_nil:NF #2
      {
        \exp_not:V #1
        \exp_not:n {#2}
        \quark_if_nil:NTF #3
          { \use_none:n }
          {
            \exp_not:n {#3}
            \quark_if_nil:NTF #4
              { \use_none:nn }
              {
                \exp_not:n {#4}
                \@@_output_decimal_loop:NNNN
                  \l_@@_group_separator_tl
              }
          }
      }
  }
%    \end{macrocode}
%   Uncertainties which are directly attached are easy to deal with. For those
%   that are separated, the first step is to find if they are entirely
%   contained within the decimal part, and to pad if they are. For the case
%   where the boundary is crossed to the integer part, the correct number of
%   digit tokens need to be removed from the start of the uncertainty and
%   the split result sent to the appropriate auxiliaries.
%    \begin{macrocode}
\cs_new:Npn \@@_output_uncertainty:nnnn #1#2#3#4
  {
    \tl_if_blank:nTF {#1}
      { \@@_output_uncertainty_unaligned:n {#4} }
      {
        \use:c { @@_output_uncert_ \tl_head:n {#1} :nnnnw }
          {#2} {#3} {#4} #1
      }
  }
\cs_new:Npn \@@_output_uncertainty_unaligned:n #1
  { \exp_not:n { #1 #1 #1 #1 } }
%    \end{macrocode}
%   There is a bit more to do here if the uncertainty is purely in the integer
%   part and printed separately. We need to pad it by the number of
%   non-significant figures, then pass on to allow digit separation.
%    \begin{macrocode}
\cs_new:Npn \@@_output_uncert_S:nnnnw #1#2#3#4#5
  {
    \str_if_eq:VnTF \l_@@_uncert_mode_tl { separate }
      {
        \exp_not:n {#3}
        \@@_output_sign:N \pm
        \exp_not:n {#3}
        \tl_if_blank:nTF {#2}
          { \@@_output_uncert_S_auxi:nnn {#1} {#5} {#3} }
          {
            \@@_output_uncert_S_auxiv:fnn
              { \int_eval:n { \tl_count:n {#5} - \tl_count:n {#2} } }
              {#5} {#3}
          }
      }
      {
        \exp_not:V \l_@@_uncert_separator_tl
        \exp_not:V \l_@@_output_uncert_open_tl
        \use:c { @@_output_uncert_S_ \l_@@_uncert_mode_tl :nn } {#2} {#5}
        \exp_not:V \l_@@_output_uncert_close_tl
        \@@_output_uncertainty_unaligned:n {#3}
      }
  }
\cs_new:Npn \@@_output_uncert_S_auxi:nnn #1#2#3
  {
    \@@_output_uncert_S_auxii:nnN { 0 } {#2} #1
      \q_recursion_tail \q_recursion_stop {#3}
  }
\cs_new:Npn \@@_output_uncert_S_auxii:nnN #1#2#3
  {
    \quark_if_recursion_tail_stop_do:Nn #3
      {
        \exp_args:Nf \@@_output_uncert_S_auxiii:nnn
          { \prg_replicate:nn {#1} { 0 } } {#2}
      }
    \str_if_eq:nnTF {#3} { 0 }
      {
        \exp_args:Nf \@@_output_uncert_S_auxii:nnN
          { \int_eval:n { #1 + 1 } } {#2}
      }
      { \@@_output_uncert_S_auxii:nnN { 0 } {#2} }
  }
\cs_new:Npn \@@_output_uncert_S_auxiii:nnn #1#2#3
  {
    \@@_output_uncert_S_auxiv:fnn
      { \int_eval:n { \tl_count:n {#2#1} } } {#2#1} {#3}
  }
%    \end{macrocode}
%   All separated uncertainties bring us to here.
%    \begin{macrocode} 
\cs_new:Npn \@@_output_uncert_S_auxiv:nnn #1#2#3
  {
    \int_compare:nNnTF {#1} > 0
      {
        \@@_output_uncert_S_auxv:fnnw
          { \int_eval:n { #1 - 1 } }
          {#3}
          { }
          #2 \q_nil
      }
      {
        0
        \@@_output_decimal:fn
          {
            \prg_replicate:nn { \int_abs:n {#1} } { 0 }
            #2
          }
          {#3}
      }
  }
\cs_generate_variant:Nn \@@_output_uncert_S_auxiv:nnn { f }
\cs_new:Npn \@@_output_uncert_S_auxv:nnnw #1#2#3#4
  {
    \quark_if_nil:NF #4
      {
        \int_compare:nNnTF {#1} = 0
          { \@@_output_uncert_S_auxvi:nnw {#3#4} {#2} }
          {
            \@@_output_uncert_S_auxv:fnnw
              { \int_eval:n { #1 - 1 } }
              {#2}
              {#3#4}
          }
      }
  }
\cs_generate_variant:Nn \@@_output_uncert_S_auxv:nnnw { f }
\cs_new:Npn \@@_output_uncert_S_auxvi:nnw #1#2#3 \q_nil
  {
    \@@_output_digits:nn { integer } {#1}
    \@@_output_decimal:nn {#3} {#2}
  }
%    \end{macrocode}
%   Handle the content of brackets: the only complex case is the
%   mixed situation.
%    \begin{macrocode}
\cs_new:Npn \@@_output_uncert_S_compact:nn #1#2
  { \exp_not:n {#2} }
\cs_new:cpn { @@_output_uncert_S_compact-marker:nn } #1#2
  {
    \bool_lazy_or:nnTF
      { \tl_if_blank_p:n {#1} }
      { ! \int_compare_p:nNn { \tl_count:n {#2} } > { \tl_count:n {#1} } }
      { \@@_output_uncert_S_compact:nn }
      { \@@_output_uncert_S_full:nn }
        {#1} {#2}
  }
\cs_new:Npn \@@_output_uncert_S_full:nn #1#2
  {
    \@@_output_uncert_S_auxiv:fnn
      { \int_eval:n { \tl_count:n {#2} - \tl_count:n {#1} } }
      {#2} { }
  }
%    \end{macrocode}
%   Setting the exponent part requires some information about the mantissa:
%   was it there or not. This means that whilst only the sign and value for
%   the exponent are typeset here, there is a need to also have access to the
%   combined mantissa part (with a decimal marker). The rest of the work is
%   about picking up the various options and getting the combinations right.
%   For signs, the auxiliary from the main sign routine can be used, but not
%   the main function: negative exponents don't have special handling.
%    \begin{macrocode}
\cs_new:Npn \@@_output_exponent:nnnn #1#2#3#4
  {
    \exp_not:n {#4}
    \bool_lazy_or:nnTF
      { \l_@@_zero_exponent_bool }
      { ! \str_if_eq_p:nn {#2} { 0 } }
      {
        \tl_if_empty:NTF \l_@@_output_exp_marker_tl
          { \@@_output_exponent_auxi:nnnn }
          { \@@_output_exponent_auxii:nnnn }
            {#1} {#2} {#3} {#4}
      }
      { \exp_not:n {#4} }
  }
\cs_new:Npn \@@_output_exponent_auxi:nnnn #1#2#3#4
  {
    \bool_lazy_or:nnTF
      { \l_@@_unity_mantissa_bool }
      { ! \str_if_eq_p:nn {#3} { 1. } }
      {
        \bool_if:NTF \l_@@_tight_bool
          { \exp_not:N \mathord }
          { \use:n }
            { \exp_not:V \l_@@_exponent_product_tl }
        \exp_not:n {#4}
      }
      { \exp_not:n {#4} }
    \exp_not:V \l_@@_exponent_base_tl
    ^
      { \@@_output_exponent_auxiii:nn {#1} {#2} }
  }
\cs_new:Npn \@@_output_exponent_auxii:nnnn #1#2#3#4
  {
    \exp_not:n {#4}
    \exp_not:V \l_@@_output_exp_marker_tl
    \@@_output_exponent_auxiii:nn {#1} {#2}
  }
\cs_new:Npn \@@_output_exponent_auxiii:nn #1#2
  {
    \tl_if_blank:nTF {#1}
      {
        \bool_if:NT \l_@@_implicit_plus_bool
          { \@@_output_sign:N + }
      }
      { \@@_output_sign:N #1 }
    \@@_output_digits:nn { integer } {#2}
  }
%    \end{macrocode}
%   A do-nothing marker used to allow shuffling of the output and so expandable
%   operations for formatting.
%    \begin{macrocode}
\cs_new:Npn \@@_output_end: { }
%    \end{macrocode}
% \end{macro}
% \end{macro}
% \end{macro}
% \end{macro}
% \end{macro}
% \end{macro}
% \end{macro}
% \end{macro}
% \end{macro}
% \end{macro}
% \end{macro}
% \end{macro}
% \end{macro}
% \end{macro}
% \end{macro}
% \end{macro}
% \end{macro}
% \end{macro}
% \end{macro}
% \end{macro}
% \end{macro}
% \end{macro}
% \end{macro}
% \end{macro}
% \end{macro}
% \end{macro}
% \end{macro}
% \end{macro}
% \end{macro}
% \end{macro}
% \end{macro}
% \end{macro}
% \end{macro}
% \end{macro}
% \end{macro}
% \end{macro}
%
% \subsection{Miscellaneous tools}
%
% \begin{variable}{\l_@@_valid_tl}
%   The list of valid tokens.
%    \begin{macrocode}
\tl_new:N \l_@@_valid_tl
%    \end{macrocode}
% \end{variable}
%
% \begin{macro}[TF]{\siunitx_if_number:n}
%   Test if an entire number is valid: this means parsing the number but not
%   returning anything.
%    \begin{macrocode}
\prg_new_protected_conditional:Npnn \siunitx_if_number:n #1
  { T , F , TF }
  {
    \group_begin:
      \bool_set_true:N \l_@@_validate_bool
      \bool_set_true:N \l_siunitx_number_parse_bool
      \siunitx_number_parse:nN {#1} \l_@@_parsed_tl
      \tl_if_empty:NTF \l_@@_parsed_tl
        {
          \group_end:
          \prg_return_false:
        }
        {
          \group_end:
          \prg_return_true:
        }
  }
%    \end{macrocode}
% \end{macro}
%
% \begin{macro}[pTF, EXP]{\siunitx_if_number_token:N}
% \begin{macro}[EXP]
%   {
%     \@@_if_token_auxi:NN   ,
%     \@@_if_token_auxii:NN  ,
%     \@@_if_token_auxiii:NN
%   }
%   A simple conditional to answer the question of whether a specific token is
%   possibly valid in a number.
%    \begin{macrocode}
\prg_new_conditional:Npnn \siunitx_if_number_token:N #1
  { p , T , F , TF }
  {
    \@@_token_auxi:NN #1
      \l_siunitx_number_input_decimal_tl
      \l_@@_input_uncert_close_tl
      \l_siunitx_number_input_comparator_tl
      \l_@@_input_digit_tl
      \l_siunitx_number_input_exponent_tl
      \l_@@_input_ignore_tl
      \l_@@_input_uncert_open_tl
      \l_siunitx_number_input_sign_tl
      \l_@@_input_uncert_sign_tl
      \q_recursion_tail
      \q_recursion_stop
  }
\cs_new:Npn \@@_token_auxi:NN #1#2
  {
    \quark_if_recursion_tail_stop_do:Nn #2 { \prg_return_false: }
    \@@_token_auxii:NN #1 #2
    \@@_token_auxi:NN #1
  }
\cs_new:Npn \@@_token_auxii:NN #1#2
  {
    \exp_after:wN \@@_token_auxiii:NN \exp_after:wN #1
      #2 \q_recursion_tail \q_recursion_stop
  }
\cs_new:Npn \@@_token_auxiii:NN #1#2
  {
    \quark_if_recursion_tail_stop:N #2
    \str_if_eq:nnT {#1} {#2}
      {
        \use_i_delimit_by_q_recursion_stop:nw
          {
            \use_i_delimit_by_q_recursion_stop:nw
              { \prg_return_true: }
          }
      }
    \@@_token_auxiii:NN #1
  }
%    \end{macrocode}
% \end{macro}
% \end{macro}
%
% \subsection{Messages}
%
%    \begin{macrocode}
\msg_new:nnnn { siunitx } { invalid-number }
  { Invalid~number~'#1'. }
  {
    The~input~'#1'~could~not~be~parsed~as~a~number~following~the~
    format~defined~in~module~documentation.
  }
%    \end{macrocode}
%
% \subsection{Standard settings for module options}
%
% Some of these follow naturally from the point of definition
% (\foreign{e.g.}~boolean variables are always |false| to begin with),
% but for clarity everything is set here.
%    \begin{macrocode}
\keys_set:nn { siunitx }
  {
    bracket-ambiguous-numbers = true                                   ,
    bracket-negative-numbers  = false                                  ,
    drop-exponent             = false                                  ,
    drop-uncertainty          = false                                  ,
    drop-zero-decimal         = false                                  ,
    evaluate-expression       = false                                  ,
    exponent-base             = 10                                     ,
    exponent-mode             = input                                  ,
    exponent-product          = \times                                 ,
    expression                = #1                                     ,
    fixed-exponent            = 0                                      ,
    group-digits              = all                                    ,
    group-minimum-digits      = 5                                      ,
    group-separator           = \,                                     , % (
    input-close-uncertainty   = )                                      ,
    input-comparators         = { <=>\approx\ge\geq\gg\le\leq\ll\sim } ,
    input-decimal-markers     = { ., }                                 ,
    input-digits              = 0123456789                             ,
    input-exponent-markers    = dDeE                                   ,
    input-ignore              = \,                                     ,
    input-open-uncertainty    = (                                      , % )
    input-signs               = +-\mp\pm                               ,
    input-uncertainty-signs   = \pm                                    ,
    minimum-decimal-digits    = 0                                      ,
    minimum-integer-digits    = 0                                      ,
    negative-color            =                                        , % (
    output-close-uncertainty  = )                                      ,
    output-decimal-marker     = .                                      ,
    output-open-uncertainty   = (                                      , % )
    parse-numbers             = true                                   ,
    print-implicit-plus       = false                                  ,
    print-unity-mantissa      = true                                   ,
    print-zero-exponent       = false                                  ,
    retain-explicit-plus      = false                                  ,
    retain-zero-uncertainty   = false                                  ,
    round-half                = up                                     ,
    round-minimum             = 0                                      ,
    round-mode                = none                                   ,
    round-pad                 = true                                   ,
    round-precision           = 2                                      ,
    tight-spacing             = false                                  ,
    uncertainty-mode          = compact                                ,
    uncertainty-separator     =
  }
%    \end{macrocode}
%
%    \begin{macrocode}
%</package>
%    \end{macrocode}
%
% \end{implementation}
%
% \PrintIndex

% \iffalse meta-comment
%
% File: siunitx-number.dtx Copyright (C) 2014-2016 Joseph Wright
%
% It may be distributed and/or modified under the conditions of the
% LaTeX Project Public License (LPPL), either version 1.3c of this
% license or (at your option) any later version.  The latest version
% of this license is in the file
%
%    http://www.latex-project.org/lppl.txt
%
% This file is part of the "siunitx bundle" (The Work in LPPL)
% and all files in that bundle must be distributed together.
%
% The released version of this bundle is available from CTAN.
%
% -----------------------------------------------------------------------
%
% The development version of the bundle can be found at
%
%    http://github.com/josephwright/siunitx
%
% for those people who are interested.
%
% -----------------------------------------------------------------------
%
%<*driver>
\documentclass{l3doc}
% The next line is needed so that \GetFileInfo will be able to pick up
% version data
\usepackage{siunitx}
\begin{document}
  \DocInput{\jobname.dtx}
\end{document}
%</driver>
% \fi
%
% \GetFileInfo{siunitx.sty}
%
% \title{^^A
%   \pkg{siunitx-number} -- Parsing and formatting numbers^^A
%   \thanks{This file describes \fileversion,
%     last revised \filedate.}^^A
% }
%
% \author{^^A
%  Joseph Wright^^A
%  \thanks{^^A
%    E-mail:
%    \href{mailto:joseph.wright@morningstar2.co.uk}
%      {joseph.wright@morningstar2.co.uk}^^A
%   }^^A
% }
%
% \date{Released \filedate}
%
% \maketitle
%
% \begin{documentation}
%
% \end{documentation}
%
% \begin{implementation}
%
% \section{\pkg{siunitx-number} implementation}
%
% Start the DocStrip guards.
%    \begin{macrocode}
%<*package>
%    \end{macrocode}
%
% Identify the internal prefix (\LaTeX3 DocStrip convention): only internal
% material in this \emph{submodule} should be used directly.
%    \begin{macrocode}
%<@@=siunitx_number>
%    \end{macrocode}
%
% \subsection{Initial set-up}
%
%   Variants not provided by \pkg{expl3}.
%    \begin{macrocode}
\cs_generate_variant:Nn \tl_if_blank:nTF { f }
\cs_generate_variant:Nn \tl_if_blank_p:n { f }
\cs_generate_variant:Nn \tl_if_in:NnTF { NV }
%    \end{macrocode}
%
% \begin{variable}[int]{\l_@@_tmp_tl}
%   Scratch space.
%    \begin{macrocode}
\tl_new:N \l_@@_tmp_tl
%    \end{macrocode}
% \end{variable}
%
% \subsection{Main formatting routine}
%
% \begin{variable}[int]{\l_@@_formatted_tl}
%   A token list for the final formatted result: may or may not be generated
%   by the parser, depending on settings which are active.
%    \begin{macrocode}
\tl_new:N \l_@@_formatted_tl
%    \end{macrocode}
% \end{variable}
%
% \begin{macro}{\siunitx_number_format:nN}
% \begin{macro}[aux]{\@@_format:nN}
%    \begin{macrocode}
\cs_new_protected:Npn \siunitx_number_format:nN #1#2
  {
    \@@_format:nN {#1} #2
  }
\cs_new_protected:Npn \@@_format:nN #1#2
  {
    \group_begin:
      \@@_parse:n {#1}
      \@@_format:
    \exp_args:NNNV \group_end:
    \tl_set:Nn #2 \l_@@_formatted_tl
  }
%    \end{macrocode}
% \end{macro}
% \end{macro}
%
% \subsection{Parsing numbers}
%
% Before numbers can be manipulated or formatted they need to be parsed into
% an internal form. In particular, if multiple code paths are to be avoided,
% it is necessary to do such parsing even for relatively simple cases such
% as converting |1e10| to |1 \times 10^{10}|.
%
% Storing the result of such parsing can be done in a number of ways. In the
% first version of \pkg{siunitx} a series of separate data stores were used.
% This is potentially quite fast (as recovery of items relies only on \TeX{}'s
% hash table) but makes managing the various data entries somewhat tedious and
% error-prone. For version two of the package, a single data structure
% (property list) was used for each part of the parsed number. Whilst this is
% easy to manage and extend, it is somewhat slower as at a \TeX{} level there
% are repeated pack--unpack steps. In particular, the fact that there are a
% limited number of items to track for a \enquote{number} means that a more
% efficient approach is desirable (contrast parsing units, which is open-ended
% and therefore fits well with using a property list).
%
% To allow for complex numbers, two parallel data structures are used, one for
% the real part and one for the imaginary part. If the part is entirely absent
% then the data structures are left empty. Within each part, the structure
% is
% \begin{quote}
%   \marg{comparator}\meta{sign}\marg{integer}\marg{decimal}
%     \marg{uncertainty}\\
%     \meta{exponent sign}\marg{exponent}
% \end{quote}
% where the two sign parts must be single tokens and all other components
% must be given in braces. \emph{All} of the components must be present in
% a stored number (\emph{e.g.}~at the end of parsing).
%
% \begin{variable}[int]^^A
%   {
%      \l_@@_input_uncert_close_tl ,
%      \l_@@_input_complex_tl      ,
%      \l_@@_input_comparator_tl   ,
%      \l_@@_input_decimal_tl      ,
%      \l_@@_input_digit_tl        ,
%      \l_@@_input_exponent_tl     ,
%      \l_@@_input_ignore_tl       ,
%      \l_@@_input_uncert_open_tl  ,
%      \l_@@_input_protect_tl      ,
%      \l_@@_input_sign_tl         ,
%      \l_@@_input_symbol_tl       ,
%      \l_@@_input_uncert_sign_tl
%   }
%   Options which determine the various valid parts of a parsed number.
%    \begin{macrocode}
\keys_define:nn { siunitx / number }
  {
    input-close-uncertainty .tl_set:N = \l_@@_input_uncert_close_tl ,
    input-complex-roots     .tl_set:N = \l_@@_input_complex_tl      ,
    input-comparators       .tl_set:N = \l_@@_input_comparator_tl   ,
    input-decimal-markers   .tl_set:N = \l_@@_input_decimal_tl      ,
    input-digits            .tl_set:N = \l_@@_input_digit_tl        ,
    input-exponent-markers  .tl_set:N = \l_@@_input_exponent_tl     ,
    input-ignore            .tl_set:N = \l_@@_input_ignore_tl       ,
    input-open-uncertainty  .tl_set:N = \l_@@_input_uncert_open_tl  ,
    input-signs             .tl_set:N = \l_@@_input_sign_tl         ,
    input-uncertainty-signs .code:n   =
      {
        \tl_set:Nn \l_@@_input_uncert_sign_tl {#1}
        \tl_map_inline:nn {#1}
          {
            \tl_if_in:NnF \l_@@_input_sign_tl {##1}
              { \tl_put_right:Nn \l_@@_input_sign_tl {##1} }
          }
      }
  }
%    \end{macrocode}
% \end{variable}
%
% \begin{variable}[int]{\l_@@_arg_tl}
%   The input argument or a part thereof, depending on the position in
%   the parsing routine.
%    \begin{macrocode}
\tl_new:N \l_@@_arg_tl
%    \end{macrocode}
% \end{variable}
%
% \begin{variable}[int]{\l_@@_comparator_tl}
%   A comparator, if found, is held here.
%    \begin{macrocode}
\tl_new:N \l_@@_comparator_tl
%    \end{macrocode}
% \end{variable}
%
% \begin{variable}[int]{\l_@@_exponent_tl}
%   The exponent part of a parsed number. It is easiest to find this
%   relatively early in the parsing process, but as it needs to go at
%   the end of the internal format is held separately until required.
%    \begin{macrocode}
\tl_new:N \l_@@_exponent_tl
%    \end{macrocode}
% \end{variable}
%
% \begin{variable}[int]{\l_@@_flex_tl}
%   When parsing for a separate uncertainty or complex number, the nature
%   of the grabbed part cannot be determined until the end of the number.
%   To avoid abusing the storage areas, this dedicated one is used for
%   \enquote{flexible} cases.
%    \begin{macrocode}
\tl_new:N \l_@@_flex_tl
%    \end{macrocode}
% \end{variable}
%
% \begin{variable}[int]{\l_@@_imaginary_tl, \l_@@_real_tl}
%   Used to hold the real and imaginary parts of a number in the standardised
%   format.
%    \begin{macrocode}
\tl_new:N \l_@@_imaginary_tl
\tl_new:N \l_@@_real_tl
%    \end{macrocode}
% \end{variable}
%
% \begin{variable}[int]{\l_@@_input_tl}
%   The numerical input exactly as given by the user.
%    \begin{macrocode}
\tl_new:N \l_@@_input_tl
%    \end{macrocode}
% \end{variable}
%
% \begin{variable}[int]{\l_@@_partial_tl}
%   To avoid needing to worry about the fact that the final data stores are
%   somewhat tricky to add to token-by-token, a simple store is used to build
%   up the parsed part of a number before transferring in one go.
%    \begin{macrocode}
\tl_new:N \l_@@_partial_tl
%    \end{macrocode}
% \end{variable}
%
% \begin{macro}[int]{\@@_parse:n}
%   After some initial set up, the parser expands the input and then replaces
%   as far as possible tricky tokens with ones that can be handled using
%   delimited arguments. The parser begins with the assumption that the input
%   is a real number. To avoid multiple conditionals here, the parser is
%   set up as a chain of commands initially, with a loop only later. This
%   avoids more conditionals than are necessary.
%    \begin{macrocode}
\cs_new_protected:Npn \@@_parse:n #1
  {
    \tl_clear:N \l_@@_imaginary_tl
    \tl_clear:N \l_@@_real_tl
    \protected@edef \l_@@_arg_tl {#1}
    \tl_set_eq:NN \l_@@_input_tl \l_@@_arg_tl
    \@@_parse_replace:
    \tl_if_empty:NF \l_@@_arg_tl
      { \@@_parse_comparator: }
  }
%    \end{macrocode}
% \end{macro}
%
% \begin{macro}[int]{\@@_parse_check:}
%   After the loop there is one case that might need tidying up. If a
%   separated uncertainty was found it will be currently in \cs{l_@@_flex_tl}
%   and needs moving. A series of tests pick up that case, then the check is
%   made that some content was found for at least one of the real or imaginary
%   parts of the number.
%    \begin{macrocode}
\cs_new_protected:Npn \@@_parse_check:
  {
    \tl_if_empty:NF \l_@@_flex_tl
      {
        \bool_if:nTF
          {
               \tl_if_blank_p:f { \exp_after:wN \use_iv:nnnn \l_@@_real_tl }
            && \tl_if_blank_p:f { \exp_after:wN \use_iv:nnnn \l_@@_flex_tl }
          }
          {
            \tl_set:Nx \l_@@_tmp_tl
              { \exp_after:wN \use_i:nnnn \l_@@_flex_tl }
            \tl_if_in:NVTF \l_@@_input_uncert_sign_tl
              \l_@@_tmp_tl
              { \@@_parse_combine_uncert: }
              { \tl_clear:N \l_@@_real_tl }
          }
          { \tl_clear:N \l_@@_real_tl }
      }
    \bool_if:nTF
      {
        \tl_if_empty_p:N \l_@@_real_tl  &&
        \tl_if_empty_p:N \l_@@_imaginary_tl
      }
      {
        \msg_error:nnx { siunitx / number } { invalid-input }
          { \exp_not:V \l_@@_input_tl }
      }
      { \@@_parse_finalise: }
  }
%    \end{macrocode}
% \end{macro}
%
% \begin{macro}[int]{\@@_parse_combine_uncert:}
% \begin{macro}[aux]{\@@_parse_combine_uncert_auxi:NnnnNnnn}
% \begin{macro}[aux]
%   {^^A
%     \@@_parse_combine_uncert_auxii:nnnnn,
%     \@@_parse_combine_uncert_auxii:fnnnn
%   }
% \begin{macro}[aux]
%   {^^A
%     \@@_parse_combine_uncert_auxiii:nnnnnn,
%     \@@_parse_combine_uncert_auxiii:fnnnnn
%   }
% \begin{macro}[aux]{\@@_parse_combine_uncert_auxiv:nnnn}
% \begin{macro}[aux, EXP]{\@@_parse_combine_uncert_auxv:w}
% \begin{macro}[aux, EXP]{\@@_parse_combine_uncert_auxvi:w}
%   Conversion of a second numerical part to an uncertainty needs a bit of
%   work. The first step is to extract the useful information from the two
%   stores: the sign, integer and decimal parts from the real number and the
%   integer and decimal parts from the second number. That is done using the
%   input stack to avoid lots of assignments.
%    \begin{macrocode}
\cs_new_protected:Npn \@@_parse_combine_uncert:
  {
    \exp_last_unbraced:NNNV \exp_last_unbraced:NV
    \@@_parse_combine_uncert_auxi:NnnnNnnn
      \l_@@_real_tl \l_@@_flex_tl
  }
%    \end{macrocode}
%  Here, |#4|, |#5| and |#8| are all junk arguments simply there to mop up
%  tokens, while |#1| will be recovered later from \cs{l_@@_real_tl} so does
%  not need to be passed about. The difference in places between the two
%  decimal parts is now found: this is done just once to avoid having to
%  part token lists twice. The value is then used to generate a number of
%  filler |0| tokens, and these are added to the appropriate part of the
%  number. Finally, everything is recombined: the integer part only needs
%  a test to avoid an empty main number.
%    \begin{macrocode}
\cs_new_protected:Npn
  \@@_parse_combine_uncert_auxi:NnnnNnnn #1#2#3#4#5#6#7#8
  {
    \int_compare:nNnTF { \tl_count:n {#6} } > { \tl_count:n {#2} }
      {
        \msg_error:nnx { siunitx / number } { uncertainty-invalid }
          { \exp_not:V \l_@@_input_tl }
      }
      {
        \@@_parse_combine_uncert_auxii:fnnnn
          { \int_eval:n { \tl_count:n {#3} - \tl_count:n {#7} } }
          {#2} {#3} {#6} {#7}
      }
  }
\cs_new_protected:Npn
  \@@_parse_combine_uncert_auxii:nnnnn #1
  {
    \@@_parse_combine_uncert_auxiii:fnnnnn
      { \prg_replicate:nn {#1} { 0 } }
      {#1}
  }
\cs_generate_variant:Nn \@@_parse_combine_uncert_auxii:nnnnn { f }
\cs_new_protected:Npn
  \@@_parse_combine_uncert_auxiii:nnnnnn #1#2#3#4#5#6
  {
    \int_compare:nNnTF {#2} > 0
      {
        \@@_parse_combine_uncert_auxiv:nnnn
          {#3} {#4} {#5} { #6 #1 }
      }
      {
        \@@_parse_combine_uncert_auxiv:nnnn
          {#3} { #4 #1 } {#5} {#6}
      }
  }
\cs_generate_variant:Nn \@@_parse_combine_uncert_auxiii:nnnnnn { f }
\cs_new_protected:Npn
  \@@_parse_combine_uncert_auxiv:nnnn #1#2#3#4
  {
    \tl_set:Nx \l_@@_real_tl
      {
        \tl_head:V \l_@@_real_tl
          { \exp_not:n {#1} }
          {
            \bool_if:nTF
              {
                   \tl_if_blank_p:n {#2}
                && ! ( \tl_if_blank_p:n {#3} )
              }
              { 0 }
              { \exp_not:n {#2} }
          }
          {
            \@@_parse_combine_uncert_auxv:w #3#4
              \q_recursion_tail \q_recursion_stop
          }
      }
  }
%    \end{macrocode}
%   A short routine to remove any leading zeros in the uncertainty part,
%   which are not needed for the compact representation used by the module.
%    \begin{macrocode}
\cs_new:Npn \@@_parse_combine_uncert_auxv:w #1
  {
    \quark_if_recursion_tail_stop:N #1
    \str_if_eq:nnTF {#1} { 0 }
      { \@@_parse_combine_uncert_auxv:w }
      { \@@_parse_combine_uncert_auxvi:w #1 }
  }
\cs_new:Npn \@@_parse_combine_uncert_auxvi:w
  #1 \q_recursion_tail \q_recursion_stop
  { \exp_not:n {#1} }
%    \end{macrocode}
% \end{macro}
% \end{macro}
% \end{macro}
% \end{macro}
% \end{macro}
% \end{macro}
% \end{macro}
%
% \begin{macro}[int]{\@@_parse_comparator:}
% \begin{macro}[aux]{\@@_parse_comparator_aux:Nw}
%   A comparator has to be the very first token in the input. A such, the
%   test for this can be very fast: grab the first token, do a check and
%   if appropriate store the result.
%    \begin{macrocode}
\cs_new_protected:Npn \@@_parse_comparator:
  {
    \exp_after:wN \@@_parse_comparator_aux:Nw
      \l_@@_arg_tl \q_stop
  }
\cs_new_protected:Npn \@@_parse_comparator_aux:Nw #1#2 \q_stop
  {
    \tl_if_in:NnTF \l_@@_input_comparator_tl {#1}
      {
        \tl_set:Nn \l_@@_comparator_tl {#1}
        \tl_set:Nn \l_@@_arg_tl {#2}
      }
      { \tl_clear:N \l_@@_comparator_tl }
    \tl_if_empty:NF \l_@@_arg_tl
      { \@@_parse_sign: }
  }
%    \end{macrocode}
% \end{macro}
% \end{macro}
%
% \begin{macro}[int]{\@@_parse_exponent:}
% \begin{macro}[aux]{\@@_parse_exponent_aux:w}
% \begin{macro}[aux]{\@@_parse_exponent_aux:nn}
% \begin{macro}[aux]{\@@_parse_exponent_aux:Nw}
% \begin{macro}[aux]{\@@_parse_exponent_aux:Nn}
% \begin{macro}[aux]
%   {\@@_parse_exponent_zero_test:N, \@@_parse_exponent_check:N}
% \begin{macro}[aux]{\@@_parse_exponent_cleanup:N}
%   An exponent part of a number has to come at the end and can only occur
%   once. Thus it is relatively easy to parse. First, there is a check that
%   an exponent part is allowed, and if so a split is made (the previous
%   part of the chain checks that there is some content in \cs{l_@@_arg_tl}
%   before calling this function). After splitting, if there is no exponent
%   then simply save a default. Otherwise, check for a sign and then store
%   either this or an assumed |+| and the digits after a check that nothing
%   else is present after the~|e|. The only slight complication to all of
%   this is allowing an arbitrary token in the input to represent the exponent:
%   this is done by setting any exponent tokens to the first of the allowed
%   list, then using that in a delimited argument set up. Once an exponent
%   part is found, there is a loop to check that each of the tokens is a digit
%   then a tidy up step to remove any leading zeros.
%    \begin{macrocode}
\cs_new_protected:Npn \@@_parse_exponent:
  {
    \tl_if_empty:NTF \l_@@_input_exponent_tl
      { \tl_set:Nn \l_@@_exponent_tl { +0 } }
      {
        \tl_set:Nx \l_@@_tmp_tl
          { \tl_head:V \l_@@_input_exponent_tl }
        \tl_map_inline:Nn \l_@@_input_exponent_tl
          {
            \tl_replace_all:NnV \l_@@_arg_tl
              {##1} \l_@@_tmp_tl
          }
        \use:x
          {
            \cs_set_protected:Npn
              \exp_not:N \@@_parse_exponent_aux:w
              ####1 \exp_not:V \l_@@_tmp_tl
              ####2 \exp_not:V \l_@@_tmp_tl
              ####3 \exp_not:N \q_stop
          }
            { \@@_parse_exponent_aux:nn {##1} {##2} }
        \use:x
          {
            \@@_parse_exponent_aux:w
              \exp_not:V \l_@@_arg_tl
              \exp_not:V \l_@@_tmp_tl \exp_not:N \q_nil
              \exp_not:V \l_@@_tmp_tl \exp_not:N \q_stop
          }
      }
  }
\cs_new_protected:Npn \@@_parse_exponent_aux:w  { }
\cs_new_protected:Npn \@@_parse_exponent_aux:nn #1#2
  {
    \quark_if_nil:nTF {#2}
      { \tl_set:Nn \l_@@_exponent_tl { +0 } }
      {
        \tl_set:Nn \l_@@_arg_tl {#1}
        \tl_if_blank:nTF {#2}
          { \tl_clear:N \l_@@_real_tl }
          { \@@_parse_exponent_aux:Nw #2 \q_stop }
      }
    \tl_if_empty:NF \l_@@_real_tl
      { \@@_parse_loop: }
  }
\cs_new_protected:Npn \@@_parse_exponent_aux:Nw #1#2 \q_stop
  {
    \tl_if_in:NnTF \l_@@_input_sign_tl {#1}
      { \@@_parse_exponent_aux:Nn #1 {#2} }
      { \@@_parse_exponent_aux:Nn + {#1#2} }
    \tl_if_empty:NT \l_@@_exponent_tl
      { \tl_clear:N \l_@@_real_tl }
  }
\cs_new_protected:Npn \@@_parse_exponent_aux:Nn #1#2
  {
    \tl_set:Nn \l_@@_exponent_tl { #1 }
    \tl_if_blank:nTF {#2}
      { \tl_clear:N \l_@@_real_tl }
      {
        \@@_parse_exponent_zero_test:N #2
          \q_recursion_tail \q_recursion_stop
      }
  }
\cs_new_protected:Npn \@@_parse_exponent_zero_test:N #1
  {
    \quark_if_recursion_tail_stop_do:Nn #1
      { \tl_set:Nn \l_@@_exponent_tl { +0 } }
    \str_if_eq:nnTF {#1} { 0 }
      { \@@_parse_exponent_zero_test:N }
      { \@@_parse_exponent_check:N #1 }
  }
\cs_new_protected:Npn \@@_parse_exponent_check:N #1
  {
    \quark_if_recursion_tail_stop:N #1
    \tl_if_in:NnTF \l_@@_input_digit_tl {#1}
      {
        \tl_put_right:Nn \l_@@_exponent_tl {#1}
        \@@_parse_exponent_check:N
      }
      { \@@_parse_exponent_cleanup:wN }
  }
\cs_new_protected:Npn \@@_parse_exponent_cleanup:wN
  #1 \q_recursion_stop
  {
    \msg_error:nnx { siunitx / number } { invalid-input }
      { \exp_not:V \l_@@_input_tl }
    \tl_clear:N \l_@@_exponent_tl
  }
%    \end{macrocode}
% \end{macro}
% \end{macro}
% \end{macro}
% \end{macro}
% \end{macro}
% \end{macro}
%
% \begin{macro}[int]{\@@_parse_replace:}
% \begin{macro}[aux]{\@@_parse_replace_aux:nN}
% \begin{macro}[aux]{\@@_parse_replace_sign:}
% \begin{variable}{\c_@@_parse_sign_replacement_tl}
%   There are two parts to the replacement code. First, any active
%   hyphens signs are normalised: these can come up with some packages and
%   cause issues. Multi-token signs then are converted to the single token
%   equivalents so that everything else can work on a one token basis.
%    \begin{macrocode}
\cs_new_protected:Npn \@@_parse_replace:
  {
    \@@_parse_replace_minus:
    \exp_last_unbraced:NV \@@_parse_replace_aux:nN
      \c_@@_parse_sign_replacement_tl
      { ? } \q_recursion_tail
        \q_recursion_stop
  }
\cs_set_protected:Npn \@@_parse_replace_aux:nN #1#2
  {
    \quark_if_recursion_tail_stop:N #2
    \tl_replace_all:Nnn \l_@@_arg_tl {#1} {#2}
    \@@_parse_replace_aux:nN
  }
\tl_const:Nn \c_@@_parse_sign_replacement_tl
  {
    { -+ } \mp
    { +- } \pm
    { << } \ll
    { <= } \le
    { >> } \gg
    { >= } \ge
  }
\group_begin:
  \char_set_catcode_active:N \-
  \cs_new_protected:Npx \@@_parse_replace_minus:
    {
      \tl_replace_all:Nnn \exp_not:N \l_@@_arg_tl
        { \exp_not:N - }  { \token_to_str:N - }
    }
\group_end:
%    \end{macrocode}
% \end{variable}
% \end{macro}
% \end{macro}
% \end{macro}
% \end{macro}
%
% \begin{macro}[int]{\@@_parse_finalise:}
% \begin{macro}[aux]{\@@_parse_finalise_aux:N}
% \begin{macro}[aux]{\@@_parse_finalise_aux:Nw}
%   Combine all of the bits of a number together: both the real and
%   imaginary parts contain all of the data.
%    \begin{macrocode}
\cs_new_protected:Npn \@@_parse_finalise:
  {
    \@@_parse_finalise_aux:N \l_@@_real_tl
    \@@_parse_finalise_aux:N \l_@@_imaginary_tl
  }
\cs_new_protected:Npn \@@_parse_finalise_aux:N #1
  {
    \tl_if_empty:NF #1
      {
        \tl_set:Nx #1
          {
            { \exp_not:V \l_@@_comparator_tl }
            \exp_not:V #1
            \exp_after:wN \@@_parse_finalise_aux:Nw
              \l_@@_exponent_tl \q_stop
          }
      }
  }
\cs_new:Npn \@@_parse_finalise_aux:Nw #1#2 \q_stop
  {
    \exp_not:N #1
    { \exp_not:n {#2} }
  }
%    \end{macrocode}
% \end{macro}
% \end{macro}
% \end{macro}
%
% \begin{macro}[int]{\@@_parse_loop:}
% \begin{macro}[aux]{\@@_parse_loop_first:N}
% \begin{macro}[aux]{\@@_parse_loop_main:NNNNN}
% \begin{macro}[aux]{\@@_parse_loop_main_end:NN}
% \begin{macro}[aux]{\@@_parse_loop_main_digit:NNNNN}
% \begin{macro}[aux]{\@@_parse_loop_main_decimal:NN}
% \begin{macro}[aux]{\@@_parse_loop_main_uncert:NNN}
% \begin{macro}[aux]{\@@_parse_loop_main_complex:N}
% \begin{macro}[aux]{\@@_parse_loop_main_sign:NNN}
% \begin{macro}[aux]{\@@_parse_loop_main_store:NNN}
% \begin{macro}[aux]{\@@_parse_loop_after_decimal:NNN}
% \begin{macro}[aux]{\@@_parse_loop_uncert:NNNNN}
% \begin{macro}[aux]{\@@_parse_loop_after_uncert:NNN}
% \begin{macro}[aux]{\@@_parse_loop_root_swap:NNwNN}
% \begin{macro}[aux]{\@@_parse_loop_complex_cleanup:wN}
% \begin{macro}[aux]{\@@_parse_loop_break:wN}
%   At this stage, the partial input \cs{l_@@_arg_tl} will contain any
%   mantissa, which may contain an uncertainty or complex part. Parsing this
%   and allowing for all of the different formats possible is best done using
%   a token-by-token approach. However, as at each stage only a subset of
%   tokens are valid, the approach take is to use a set of semi-dedicated
%   functions to parse different components along with switches to allow a
%   sensible amount of code sharing.
%    \begin{macrocode}
\cs_new_protected:Npn \@@_parse_loop:
  {
    \tl_clear:N \l_@@_partial_tl
    \exp_after:wN \@@_parse_loop_first:NNN
      \exp_after:wN \l_@@_real_tl \exp_after:wN \c_true_bool
        \l_@@_arg_tl
        \q_recursion_tail \q_recursion_stop
    \@@_parse_check:
  }
%    \end{macrocode}
%   The very first token of the input is handled with a dedicated function.
%   Valid cases here are
%   \begin{itemize}
%     \item Entirely blank if the original input was for example |+e10|:
%       simply clean up if in the integer part of issue an error if in
%       a second part (complex number, \emph{etc.}).
%     \item An integer part digit: pass through to the main collection
%       routine.
%     \item A decimal marker: store an empty integer part and move to
%       the main collection routine for a decimal part.
%     \item A complex root token: shuffle to the end of the input.
%   \end{itemize}
%   Anything else is invalid and sends the code to the abort function.
%    \begin{macrocode}
\cs_new_protected:Npn \@@_parse_loop_first:NNN #1#2#3
  {
    \quark_if_recursion_tail_stop_do:Nn #3
      {
        \bool_if:NTF #2
          { \tl_put_right:Nn #1 { { 1 } { } { } } }
          { \@@_parse_loop_break:wN \q_recursion_stop }
      }
    \tl_if_in:NnTF \l_@@_input_digit_tl {#3}
      {
        \@@_parse_loop_main:NNNNN
          #1 \c_true_bool \c_false_bool #2 #3
      }
      {
        \tl_if_in:NnTF \l_@@_input_decimal_tl {#3}
          {
            \tl_put_right:Nn #1 { { } }
            \@@_parse_loop_after_decimal:NNN #1 #2
          }
          {
            \tl_if_in:NnTF \l_@@_input_complex_tl {#3}
              { \@@_parse_loop_root_swap:NNwNN #1 #3 }
              { \@@_parse_loop_break:wN }
          }
      }
  }
%    \end{macrocode}
%   A single function is used to cover the \enquote{main} part of numbers:
%   finding real, complex or separated uncertainty parts and covering both
%   the integer and decimal components. This works because these elements
%   share a lot of concepts: a small number of switches can be used to
%   differentiate between them. To keep the code at least somewhat readable,
%   this main function deals with the validity testing but hands off other
%   tasks to dedicated auxiliaries for each case.
%
%   The possibilities are
%   \begin{itemize}
%     \item The number terminates, meaning that some digits were collected
%       and everything is simply tidied up (as far as the loop is concerned).
%     \item A digit is found: this is the common case and leads to a storage
%       auxiliary (which handles non-significant zeros).
%     \item A decimal marker is found: only valid in the integer part and
%       there leading to a store-and-switch situation.
%     \item An open-uncertainty token: switch to the dedicated collector
%       for uncertainties.
%     \item A complex root token: store the current number as an imaginary
%       part and terminate the loop.
%     \item A sign token (if allowed): stop collecting this number and
%       restart collection for the second part.
%   \end{itemize}
%    \begin{macrocode}
\cs_new_protected:Npn \@@_parse_loop_main:NNNNN #1#2#3#4#5
  {
    \quark_if_recursion_tail_stop_do:Nn #5
      { \@@_parse_loop_main_end:NN #1#2 }
    \tl_if_in:NnTF \l_@@_input_digit_tl {#5}
      { \@@_parse_loop_main_digit:NNNNN #1#2#3#4#5 }
      {
        \tl_if_in:NnTF \l_@@_input_decimal_tl {#5}
          {
            \bool_if:NTF #2
              { \@@_parse_loop_main_decimal:NN #1 #4 }
              { \@@_parse_loop_break:wN }
          }
          {
            \tl_if_in:NnTF \l_@@_input_uncert_open_tl {#5}
              { \@@_parse_loop_main_uncert:NNN #1#2 #4 }
              {
                \tl_if_in:NnTF \l_@@_input_complex_tl {#5}
                  {
                    \@@_parse_loop_main_store:NNN #1 #2 \c_true_bool
                    \@@_parse_loop_main_complex:N #1
                  }
                  {
                    \bool_if:NTF #4
                      {
                        \tl_if_in:NnTF \l_@@_input_sign_tl {#5}
                          {
                            \@@_parse_loop_main_sign:NNN
                              #1#2 #5
                          }
                          { \@@_parse_loop_break:wN }
                      }
                      { \@@_parse_loop_break:wN }
                  }
              }
          }
      }
  }
%    \end{macrocode}
%   If the main loop finds the end marker then there is a tidy up phase.
%   The current partial number is stored either as the integer or decimal,
%   depending on the setting for the indicator switch. For the integer
%   part, if no number has been collected then one or more non-significant
%   zeros have been dropped. Exactly one zero is therefore needed to make
%   sure the parsed result is correct.
%    \begin{macrocode}
\cs_new_protected:Npn \@@_parse_loop_main_end:NN #1#2
  {
    \bool_if:nT
      { #2 && \tl_if_empty_p:N \l_@@_partial_tl }
      { \tl_set:Nn \l_@@_partial_tl { 0 } }
    \tl_put_right:Nx #1
      {
        { \exp_not:V \l_@@_partial_tl }
        \bool_if:NT #2 { { } }
        { }
      }
  }
%    \end{macrocode}
%   The most common case for the main loop collector is to find a digit.
%   Here, in the integer part it is possible that zeros are non-significant:
%   that is handled using a combination of a switch and a string test. Other
%   than that, the situation here is simple: store the input and loop.
%    \begin{macrocode}
\cs_new_protected:Npn \@@_parse_loop_main_digit:NNNNN #1#2#3#4#5
  {
    \bool_if:nTF
      {
           #3
        || ! ( \str_if_eq_p:nn {#5} { 0 } )
      }
      {
        \tl_put_right:Nn \l_@@_partial_tl {#5}
        \@@_parse_loop_main:NNNNN #1 #2 \c_true_bool #4
      }
      { \@@_parse_loop_main:NNNNN #1 #2 \c_false_bool #4 }
  }
%    \end{macrocode}
%   When a decimal marker was found, move the integer part to the
%   store and then go back to the loop with the flags set correctly.
%   There is the case of non-significant zeros to cover before that, of course.
%    \begin{macrocode}
\cs_new_protected:Npn \@@_parse_loop_main_decimal:NN #1#2
  {
    \@@_parse_loop_main_store:NNN #1 \c_false_bool \c_false_bool
    \@@_parse_loop_after_decimal:NNN #1 #2
  }
%    \end{macrocode}
%   Starting an uncertainty part means storing the number to date as in other
%   cases, with the possibility of a blank decimal part allowed for. The
%   uncertainty itself is collected by a dedicated function as it is extremely
%   restricted.
%    \begin{macrocode}
\cs_new_protected:Npn \@@_parse_loop_main_uncert:NNN #1#2#3
  {
    \@@_parse_loop_main_store:NNN #1 #2 \c_false_bool
    \@@_parse_loop_uncert:NNNNN
      #1 \c_true_bool \c_false_bool #3
  }
%    \end{macrocode}
%   A complex root token has to be at the end of the input (leading ones
%   are dealt with specially). Thus after moving the data to the correct
%   place there is a hand-off to a cleanup function. The case where only the
%   complex root token was given is covered by
%   \cs{@@_parse_loop_root_swap:NNwNN}.
%    \begin{macrocode}
\cs_new_protected:Npn \@@_parse_loop_main_complex:N #1
  {
    \tl_set_eq:NN \l_@@_imaginary_tl #1
    \tl_clear:N #1
    \@@_parse_loop_complex_cleanup:wN
  }
%    \end{macrocode}
%   If a sign is found, terminate the current number, store the sign as the
%   first token of the second part and go back to do the dedicated first-token
%   function.
%    \begin{macrocode}
\cs_new_protected:Npn \@@_parse_loop_main_sign:NNN #1#2#3
  {
    \@@_parse_loop_main_store:NNN #1 #2 \c_true_bool
    \tl_set:Nn \l_@@_flex_tl {#3}
    \@@_parse_loop_first:NNN
      \l_@@_flex_tl \c_false_bool
  }
%    \end{macrocode}
%   A common auxiliary for the various non-digit token functions: tidy up the
%   integer and decimal parts of a number. Here, the two flags are used to
%   indicate if empty decimal and uncertainty parts should be included in
%   the storage cycle.
%    \begin{macrocode}
\cs_new_protected:Npn \@@_parse_loop_main_store:NNN #1#2#3
  {
    \tl_if_empty:NT \l_@@_partial_tl
      { \tl_set:Nn \l_@@_partial_tl { 0 } }
    \tl_put_right:Nx #1
      {
        { \exp_not:V \l_@@_partial_tl }
        \bool_if:NT #2 { { } }
        \bool_if:NT #3 { { } }
      }
    \tl_clear:N \l_@@_partial_tl
  }
%    \end{macrocode}
%   After a decimal marker there has to be a digit if there wasn't one before
%   it. That is handled by using a dedicated function, which checks for
%   an empty integer part first then either simply hands off or looks for
%   a digit.
%    \begin{macrocode}
\cs_new_protected:Npn \@@_parse_loop_after_decimal:NNN #1#2#3
  {
    \tl_if_blank:fTF { \exp_after:wN \use_none:n #1 }
      {
        \quark_if_recursion_tail_stop_do:Nn #3
          { \@@_parse_loop_break:wN \q_recursion_stop }
        \tl_if_in:NnTF \l_@@_input_digit_tl {#1}
          {
            \tl_put_right:Nn \l_@@_partial_tl {#3}
            \@@_parse_loop_main:NNNNN #1 \c_false_bool \c_true_bool #2
          }
          { \@@_parse_loop_break:wN}
      }
      {
        \@@_parse_loop_main:NNNNN
          #1 \c_false_bool \c_true_bool #2 #3
      }
  }
%    \end{macrocode}
%   Inside the brackets for an uncertainty the range of valid choices is
%   very limited. Either the token is a digit, in which case there is a
%   test to look for non-significant zeros, or it is a closing bracket. The
%   latter is not valid for the very first token, which is handled using a
%   switch (it's a simple enough difference).
%    \begin{macrocode}
\cs_new_protected:Npn \@@_parse_loop_uncert:NNNNN #1#2#3#4#5
  {
    \quark_if_recursion_tail_stop_do:Nn #5
      { \@@_parse_loop_break:wN \q_recursion_stop }
    \tl_if_in:NnTF \l_@@_input_digit_tl {#5}
      {
        \bool_if:nTF
          {
               #3
            || ! ( \str_if_eq_p:nn {#5} { 0 } )
          }
          {
            \tl_put_right:Nn \l_@@_partial_tl {#5}
            \@@_parse_loop_uncert:NNNNN
              #1 \c_false_bool \c_true_bool #4
          }
          {
            \@@_parse_loop_uncert:NNNNN
              #1 \c_false_bool \c_false_bool #4
          }
      }
      {
        \tl_if_in:NnTF \l_@@_input_uncert_close_tl {#5}
          {
            \bool_if:NTF #2
              { \@@_parse_loop_break:wN }
              {
                \@@_parse_loop_main_store:NNN #1
                  \c_false_bool \c_false_bool
                \@@_parse_loop_after_uncert:NNN #1 #3
              }
          }
          { \@@_parse_loop_break:wN }
      }
  }
%    \end{macrocode}
%   After a bracketed uncertainty there are only a very small number of
%   valid choices. The number can end, there can be a complex root token
%   or there can be a sign. The latter is only allowed if the part being
%   parsed at the moment was the first part of the number. The case where
%   there is no root symbol but there should have been is cleared up after
%   the loop code, so at this stage there is no check.
%    \begin{macrocode}
\cs_new_protected:Npn \@@_parse_loop_after_uncert:NNN #1#2#3
  {
    \quark_if_recursion_tail_stop:N #3
    \tl_if_in:NnTF \l_@@_input_complex_tl {#3}
      { \@@_parse_loop_main_complex:N #1 }
      {
        \bool_if:NTF #2
          {
            \tl_if_in:NnTF \l_@@_input_sign_tl {#3}
              {
                \tl_set:Nn \l_@@_flex_tl {#3}
                \@@_parse_loop_first:NNN
                  \l_@@_flex_tl \c_false_bool
              }
              { \@@_parse_loop_break:wN }
          }
          { \@@_parse_loop_break:wN }
      }
  }
%    \end{macrocode}
%   When the complex root symbol comes at the start of the number rather than
%   at the end, the easiest approach is to shuffle it to the \enquote{normal}
%   position. As the exponent has already been removed, this must be the last
%   token of the input and any duplication will be picked up. The case where
%   just a complex root token has to be covered: in that situation, there is
%   an implicit |1| to store after which the loop stops.
%    \begin{macrocode}
\cs_new_protected:Npn \@@_parse_loop_root_swap:NNwNN #1#2#3
  \q_recursion_tail \q_recursion_stop
  {
    \tl_if_blank:nTF {#3}
      {
        \tl_set:Nx \l_@@_imaginary_tl
          {
            \exp_not:V #1
            { 1 } { } { }
          }
        \tl_clear:N #1
      }
      {
        \use:x
          {
            \tl_clear:N \exp_not:N #1
            \tl_set:Nn \exp_not:N \l_@@_flex_tl { \exp_not:V #1 }
          }
        \@@_parse_loop_first:NNN
          \l_@@_flex_tl \c_false_bool
          #3 #2 \q_recursion_tail \q_recursion_stop
      }
  }
%    \end{macrocode}
%   Nothing is allowed after a complex root token: check and if there is
%   kill the parsing.
%    \begin{macrocode}
\cs_new_protected:Npn \@@_parse_loop_complex_cleanup:wN
  #1 \q_recursion_tail \q_recursion_stop
  {
    \tl_if_blank:nF {#1}
      { \@@_parse_loop_break:wN \q_recursion_stop }
  }
%    \end{macrocode}
%   Something is not right: remove all of the remaining tokens from the
%   number and clear the storage areas as a signal for the next part of the
%   code.
%    \begin{macrocode}
\cs_new_protected:Npn \@@_parse_loop_break:wN
  #1 \q_recursion_stop
  {
    \tl_clear:N \l_@@_imaginary_tl
    \tl_clear:N \l_@@_flex_tl
    \tl_clear:N \l_@@_real_tl
  }
%    \end{macrocode}
% \end{macro}
% \end{macro}
% \end{macro}
% \end{macro}
% \end{macro}
% \end{macro}
% \end{macro}
% \end{macro}
% \end{macro}
% \end{macro}
% \end{macro}
% \end{macro}
% \end{macro}
% \end{macro}
% \end{macro}
% \end{macro}
%
% \begin{macro}[int]{\@@_parse_sign:}
% \begin{macro}[aux]{\@@_parse_sign_aux:Nw}
%  The first token of a number after a comparator could be a sign. A quick
%  check is made and if found stored; if there is no sign then the internal
%  format requires that |+| is used. For the number to be valid it has to be
%  more than just a sign, so the next part of the chain is only called if that
%  is the case.
%    \begin{macrocode}
\cs_new_protected:Npn \@@_parse_sign:
  {
    \exp_after:wN \@@_parse_sign_aux:Nw
      \l_@@_arg_tl \q_stop
  }
\cs_new_protected:Npn \@@_parse_sign_aux:Nw #1#2 \q_stop
  {
    \tl_if_in:NnTF \l_@@_input_sign_tl {#1}
      {
        \tl_set:Nn \l_@@_arg_tl {#2}
        \tl_set:Nn \l_@@_real_tl {#1}
      }
      { \tl_set:Nn \l_@@_real_tl { + } }
    \tl_if_empty:NTF \l_@@_arg_tl
      { \tl_clear:N \l_@@_real_tl }
      { \@@_parse_exponent: }
  }
%    \end{macrocode}
% \end{macro}
% \end{macro}
%
% \subsection{Processing numbers}
%
% \subsection{Formatting parsed numbers}
%
% \begin{variable}[int]^^A
%   {
%      \l_@@_bracket_negative_bool  ,
%      \l_@@_bracket_close_tl       ,
%      \l_@@_explicit_plus_bool     ,
%      \l_@@_exponent_base_tl       ,
%      \l_@@_exponent_product_tl    ,
%      \l_@@_group_decimal_bool     ,
%      \l_@@_group_integer_bool     ,
%      \l_@@_group_minimum_int      ,
%      \l_@@_group_separator_tl     ,
%      \l_@@_negative_color_tl      ,
%      \l_@@_bracket_open_tl        ,
%      \l_@@_output_uncert_close_tl ,
%      \l_@@_output_complex_tl      ,
%      \l_@@_output_decimal_tl      ,
%      \l_@@_output_uncert_open_tl  ,
%      \l_@@_tight_bool             ,
%      \l_@@_unity_mantissa_bool    ,
%      \l_@@_zero_exponent_bool
%   }
%  Keys producing tokens in the output.
%    \begin{macrocode}
\keys_define:nn { siunitx / number }
  {
    bracket-negative         .bool_set:N = \l_@@_bracket_negative_bool  ,
    close-bracket            .tl_set:N   = \l_@@_bracket_close_tl       ,
    explicit-plus            .bool_set:N = \l_@@_explicit_plus_bool     ,
    exponent-base            .tl_set:N   = \l_@@_exponent_base_tl       ,
    exponent-product         .tl_set:N   = \l_@@_exponent_product_tl    ,
    group-digits             .choice:                                   ,
    group-digits / all       .code:n     =
      {
        \bool_set_true:N \l_@@_group_decimal_bool
        \bool_set_true:N \l_@@_group_integer_bool
      } ,
    group-digits / decimal   .code:n     =
      {
        \bool_set_true:N  \l_@@_group_decimal_bool
        \bool_set_false:N \l_@@_group_integer_bool

      } ,
    group-digits / integer   .code:n     =
      {
        \bool_set_false:N \l_@@_group_decimal_bool
        \bool_set_true:N  \l_@@_group_integer_bool
      } ,
    group-digits / none      .code:n     =
      {
        \bool_set_false:N \l_@@_group_decimal_bool
        \bool_set_false:N \l_@@_group_integer_bool
      } ,
    group-digits             .default:n  = all                          ,
    group-minimum-digits     .int_set:N  = \l_@@_group_minimum_int      ,
    group-separator          .tl_set:N   = \l_@@_group_separator_tl     ,
    negative-color           .tl_set:N   = \l_@@_negative_color_tl      ,
    open-bracket             .tl_set:N   = \l_@@_bracket_open_tl        ,
    output-close-uncertainty .tl_set:N   = \l_@@_output_uncert_close_tl ,
    output-complex-root      .tl_set:N   = \l_@@_output_complex_tl      ,
    output-decimal-marker    .tl_set:N   = \l_@@_output_decimal_tl      ,
    output-open-uncertainty  .tl_set:N   = \l_@@_output_uncert_open_tl  ,
    tight-spacing            .bool_set:N = \l_@@_tight_bool             ,
    unity-mantissa           .bool_set:N = \l_@@_unity_mantissa_bool    ,
    zero-exponent            .bool_set:N = \l_@@_zero_exponent_bool     ,
  }
\bool_new:N \l_@@_group_decimal_bool
\bool_new:N \l_@@_group_integer_bool
%    \end{macrocode}
% \end{variable}
%
% \begin{macro}[int]{\@@_format:}
%    \begin{macrocode}
\cs_new_protected:Npn \@@_format:
  {
    \tl_set:Nx \l_@@_formatted_tl
      {
        \tl_if_empty:NTF \l_@@_real_tl
          {
            \tl_if_empty:NF \l_@@_imaginary_tl
              { \@@_format:N \l_@@_imaginary_tl }
          }
          {
            \tl_if_empty:NTF \l_@@_imaginary_tl
              { \@@_format:N \l_@@_real_tl }
              { ??? }
          }
      }
  }
%    \end{macrocode}
% \end{macro}
%
% \begin{macro}[int, EXP]{\@@_format:N}
% \begin{macro}[aux, EXP]{\@@_format:nNnnnNn}
% \begin{macro}[aux, EXP]{\@@_format_comparator:n}
% \begin{macro}[aux, EXP]{\@@_format_sign:N, \@@_format_sign_aux:N}
% \begin{macro}[aux, EXP]
%   {\@@_format_sign_color:w, \@@_format_sign_brackets:w}
% \begin{macro}[aux, EXP]{\@@_format_integer:n}
% \begin{macro}[aux, EXP]{\@@_format_decimal:n}
% \begin{macro}[aux, EXP]{\@@_format_uncertainty:nnn}
% \begin{macro}[aux, EXP]{\@@_format_exponent:Nnn}
% \begin{macro}[aux, EXP]{\@@_format_end:}
%   The approach to formatting a single number is to split into
%   the constituent parts
%    \begin{macrocode}
\cs_new:Npn \@@_format:N #1
  { \exp_after:wN \@@_format:nNnnnNn #1 }
\cs_new:Npn \@@_format:nNnnnNn #1#2#3#4#5#6#7
  {
    \@@_format_comparator:n {#1}
    \@@_format_sign:N #2
    \@@_format_integer:n {#3}
    \@@_format_decimal:n {#4}
    \@@_format_uncertainty:nnn {#5} {#3} {#4}
    \@@_format_exponent:Nnn #6 {#7} { #3 . #4 }
    \@@_format_end:
  }
%    \end{macrocode}
%    Not much exciting here!
%    \begin{macrocode}
\cs_new:Npn \@@_format_comparator:n #1
  { \exp_not:n {#1} }
%    \end{macrocode}
%  Formatting signs has to deal with some additional formatting requirements
%  for negative numbers. Both making such numbers a fixed color and bracketing
%  them needs some rearrangement of the order of tokens, which is set up in the
%  main formatting macro by the dedicated do-nothing end function.
%    \begin{macrocode}
\cs_new:Npn \@@_format_sign:N #1
  {
    \str_if_eq:nnTF {#1} { + }
      {
        \bool_if:NT \l_@@_explicit_plus_bool
          { \@@_format_sign_aux:N #1 }
      }
      {
        \str_if_eq:nnTF {#1} { - }
          {
            \tl_if_empty:NF \l_@@_negative_color_tl
              { \@@_format_sign_color:w }
            \bool_if:NT \l_@@_bracket_negative_bool
              { \@@_format_sign_brackets:w }
            \@@_format_sign_aux:N #1
          }
          { \@@_format_sign_aux:N #1 }
      }
  }
\cs_new:Npn \@@_format_sign_aux:N #1
  {
    \bool_if:NTF \l_@@_tight_bool
      { { \exp_not:n {#1} } }
      { \exp_not:n {#1} }
  }
\cs_new:Npn \@@_format_sign_color:w #1 \@@_format_end:
  {
    \exp_not:N \textcolor { \exp_not:V \l_@@_negative_color_tl }
      {
        #1
        \@@_format_end:
      }
  }
\cs_new:Npn \@@_format_sign_brackets:w #1 \@@_format_end:
  {
    \l_@@_bracket_open_tl
    #1
    \l_@@_bracket_close_tl
    \@@_format_end:
  }
\cs_new:Npn \@@_format_integer:n #1
  {
  }
\cs_new:Npn \@@_format_decimal:n #1
  {
  }
\cs_new:Npn \@@_format_uncertainty:nnn #1#2#3
  {
  }
%    \end{macrocode}
%   Setting the exponent part requires some information about the mantissa:
%   was it there or not. This means that whilst only the sign and value for
%   the exponent are typeset here, there is a need to also have access to the
%   combined mantissa part (with a decimal marker). The rest of the work is
%   about picking up the various options and getting the combinations right.
%   For signs, the auxiliary from the main sign routine can be used, but not
%   the main function: negative exponents don't have special handling.
%    \begin{macrocode}
\cs_new:Npn \@@_format_exponent:Nnn #1#2#3
  {
    \bool_if:nT
      {
            \l_@@_zero_exponent_bool
        || ! ( \str_if_eq_p:nn {#2} { 0 } )
      }
      {
        \bool_if:nF
          {
               \str_if_eq_p:nn {#3} { 1. }
            && ! \l_@@_unity_mantissa_bool
          }
          {
            \bool_if:NTF \l_@@_tight_bool
              { { \exp_not:V \l_@@_exponent_product_tl } }
              { \exp_not:V \l_@@_exponent_product_tl }
          }
        \exp_not:V \l_@@_exponent_base_tl
        ^
          {
            \bool_if:nT
              {
                \l_@@_explicit_plus_bool
                || ! ( \str_if_eq_p:nn {#1} { + } )
              }
              { \@@_format_sign_aux:N #1 }
            \exp_not:n {#2}
          }
      }
  }
%    \end{macrocode}
%   A do-nothing marker used to allow shuffling of the output and so expandable
%   operations for formatting.
%    \begin{macrocode}
\cs_new:Npn \@@_format_end: { }
%    \end{macrocode}
% \end{macro}
% \end{macro}
% \end{macro}
% \end{macro}
% \end{macro}
% \end{macro}
% \end{macro}
% \end{macro}
% \end{macro}
% \end{macro}
%
% \subsection{Messages}
%
%    \begin{macrocode}
\msg_new:nnnn { siunitx / number } { invalid-input }
  { Invalid~number~'#1'. }
  {
    The~input~'#1'~could~not~be~parsed~as~a~number~following~the~
    format~defined~in~module~documentation.
  }
\msg_new:nnnn { siunitx / number } { uncertainty-invalid }
  { Uncertainty~part~in~'#1'~invalid. }
  {
    The~uncertainty~(second~part)~in~the~input~'#1'~has~more~figures~than~the~
    main~number~in~the~integer~part.~This~cannot~be~handled~by~siunitx.
  }
%    \end{macrocode}
%
% \subsection{Standard settings for module options}
%
% Some of these follow naturally from the point of definition
% (\emph{e.g.}~boolean variables are always |false| to begin with),
% but for clarity everything is set here.
%    \begin{macrocode}
\keys_set:nn { siunitx / number }
  {
    bracket-negative         = false                                  , % (
    close-bracket            = )                                      ,
    explicit-plus            = false                                  ,
    exponent-base            = 10                                     ,
    exponent-product         = \times                                 ,
    group-digits             = all                                    ,
    group-minimum-digits     = 4                                      ,
    group-separator          = \,                                     , % (
    input-close-uncertainty  = )                                      ,
    input-complex-roots      = ij                                     ,
    input-comparators        = { <=>\approx\ge\geq\gg\le\leq\ll\sim } ,
    input-decimal-markers    = { ., }                                 ,
    input-digits             = 0123456789                             ,
    input-exponent-markers   = dDeE                                   ,
    input-ignore             = \,                                     ,
    input-open-uncertainty   = (                                      , % )
    input-signs              = +-\mp\pm                               ,
    input-uncertainty-signs  = \pm                                    , % (
    negative-color           =                                        ,
    open-bracket             = (                                      , % )
    output-close-uncertainty = )                                      ,
    output-complex-root      = \mathrm { i }                          ,
    output-decimal-marker    = .                                      ,
    output-open-uncertainty  = (                                      , % )
    tight-spacing            = false                                  ,
    unity-mantissa           = false                                  ,
    zero-exponent            = false
  }
%    \end{macrocode}
%
%    \begin{macrocode}
%</package>
%    \end{macrocode}
%
% \end{implementation}
%
% \PrintIndex

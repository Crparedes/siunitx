% \iffalse meta-comment
%
% File: siunitx-number.dtx Copyright (C) 2014-2018 Joseph Wright
%
% It may be distributed and/or modified under the conditions of the
% LaTeX Project Public License (LPPL), either version 1.3c of this
% license or (at your option) any later version.  The latest version
% of this license is in the file
%
%    https://www.latex-project.org/lppl.txt
%
% This file is part of the "siunitx bundle" (The Work in LPPL)
% and all files in that bundle must be distributed together.
%
% The released version of this bundle is available from CTAN.
%
% -----------------------------------------------------------------------
%
% The development version of the bundle can be found at
%
%    https://github.com/josephwright/siunitx
%
% for those people who are interested.
%
% -----------------------------------------------------------------------
%
%<*driver>
\documentclass{l3doc}
% The next line is needed so that \GetFileInfo will be able to pick up
% version data
\usepackage{siunitx}
\begin{document}
  \DocInput{\jobname.dtx}
\end{document}
%</driver>
% \fi
%
% \GetFileInfo{siunitx.sty}
%
% \title{^^A
%   \pkg{siunitx-number} -- Parsing and formatting numbers^^A
%   \thanks{This file describes \fileversion,
%     last revised \filedate.}^^A
% }
%
% \author{^^A
%  Joseph Wright^^A
%  \thanks{^^A
%    E-mail:
%    \href{mailto:joseph.wright@morningstar2.co.uk}
%      {joseph.wright@morningstar2.co.uk}^^A
%   }^^A
% }
%
% \date{Released \filedate}
%
% \maketitle
%
% \begin{documentation}
%
% \begin{function}{\siunitx_number_format:nN, \siunitx_number_format:VN}
%   \begin{syntax}
%     \cs{siunitx_number_format:nN} \Arg{number} \meta{tl~var}
%   \end{syntax}
% \end{function}
%
% \begin{function}{\siunitx_number_format:nNN}
%   \begin{syntax}
%     \cs{siunitx_number_format:nNN} \Arg{number} \meta{tl~var} \meta{marker}
%   \end{syntax}
% \end{function}
%
% \begin{function}[TF]{\siunitx_if_number:n}
%   \begin{syntax}
%     \cs{siunitx_if_number_token:NTF} \Arg{tokens}
%       \Arg{true code} \Arg{false code}
%   \end{syntax}
%   Determines if the \meta{tokens} form a valid number which can be fully
%   parsed by \pkg{siunitx}.
% \end{function}
%
% \begin{function}[TF]{\siunitx_if_number_token:N}
%   \begin{syntax}
%     \cs{siunitx_if_number_token:NTF} \Arg{token}
%       \Arg{true code} \Arg{false code}
%   \end{syntax}
%   Determines if the \meta{token} is valid in a number based on those
%   tokens currently set up for detection in a number.
% \end{function}
%
% \subsection{Key--value options}
%
% The options defined by this submodule are available within the \pkg{l3keys}
% |siunitx| tree.
%
% \begin{function}{bracket-negative}
%   \begin{syntax}
%     |bracket-negative| = |true|\verb"|"|false|
%   \end{syntax}
% \end{function}
%
% \begin{function}{drop-uncertainty}
%   \begin{syntax}
%     |drop-uncertainty| = |true|\verb"|"|false|
%   \end{syntax}
% \end{function}
%
% \begin{function}{drop-zero-decimal}
%   \begin{syntax}
%     |drop-zero-decimal| = |true|\verb"|"|false|
%   \end{syntax}
% \end{function}
%
% \begin{function}{evaluate-expression}
%   \begin{syntax}
%     |evaluate-expression| = |true|\verb"|"|false|
%   \end{syntax}
% \end{function}
%
% \begin{function}{exponent-base}
%   \begin{syntax}
%     |exponent-base| = \meta{base}
%   \end{syntax}
% \end{function}
%
% \begin{function}{exponent-mode}
%   \begin{syntax}
%     |exponent-mode| = |engineering|\verb"|"|fixed|\verb"|"|input|\verb"|"|scientific|
%   \end{syntax}
% \end{function}
%
% \begin{function}{exponent-product}
%   \begin{syntax}
%     |exponent-product| = \meta{symbol}
%   \end{syntax}
% \end{function}
%
% \begin{function}{expression}
%   \begin{syntax}
%     |expression| = \meta{expression}
%   \end{syntax}
% \end{function}
%
% \begin{function}{fixed-exponent}
%   \begin{syntax}
%     |fixed-exponent| = \meta{exponent}
%   \end{syntax}
% \end{function}
%
% \begin{function}{group-digits}
%   \begin{syntax}
%     |group-digits| = |all|\verb"|"|decimal|\verb"|"|integer|\verb"|"|none|
%   \end{syntax}
% \end{function}
%
% \begin{function}{group-minimum-digits}
%   \begin{syntax}
%     |group-minimum-digits| = \meta{value}
%   \end{syntax}
% \end{function}
%
% \begin{function}{group-separator}
%   \begin{syntax}
%     |group-separator| = \meta{symbol}
%   \end{syntax}
% \end{function}
%
% \begin{function}{input-close-uncertainty}
%   \begin{syntax}
%     |input-close-uncertainty| = \meta{tokens}
%   \end{syntax}
% \end{function}
%
% \begin{function}{input-complex-roots}
%   \begin{syntax}
%     |input-complex-roots| = \meta{tokens}
%   \end{syntax}
% \end{function}
%
% \begin{function}{input-comparators}
%   \begin{syntax}
%     |input-comparators| = \meta{tokens}
%   \end{syntax}
% \end{function}
%
% \begin{function}{input-close-uncertainty}
%   \begin{syntax}
%     |input-close-uncertainty| = \meta{tokens}
%   \end{syntax}
% \end{function}
%
% \begin{function}{input-decimal-markers}
%   \begin{syntax}
%     |input-decimal-markers| = \meta{tokens}
%   \end{syntax}
% \end{function}
%
% \begin{function}{input-digits}
%   \begin{syntax}
%     |input-digits| = \meta{tokens}
%   \end{syntax}
% \end{function}
%
% \begin{function}{input-exponent-markers}
%   \begin{syntax}
%     |input-exponent-markers| = \meta{tokens}
%   \end{syntax}
% \end{function}
%
% \begin{function}{input-open-uncertainty}
%   \begin{syntax}
%     |input-open-uncertainty| = \meta{tokens}
%   \end{syntax}
% \end{function}
%
% \begin{function}{input-signs}
%   \begin{syntax}
%     |input-signs| = \meta{tokens}
%   \end{syntax}
% \end{function}
%
% \begin{function}{input-uncertainty-signs}
%   \begin{syntax}
%     |input-uncertainty-signs| = \meta{tokens}
%   \end{syntax}
% \end{function}
%
% \begin{function}{minimum-decimal-digits}
%   \begin{syntax}
%     |minimum-decimal-digits| = \meta{min}
%   \end{syntax}
% \end{function}
%
% \begin{function}{minimum-integer-digits}
%   \begin{syntax}
%     |minimum-integer-digits| = \meta{min}
%   \end{syntax}
% \end{function}
%
% \begin{function}{negative-color}
%   \begin{syntax}
%     |negative-color| = \meta{color}
%   \end{syntax}
% \end{function}
%
% \begin{function}{number-close-bracket}
%   \begin{syntax}
%     |number-close-bracket| = \meta{symbol}
%   \end{syntax}
% \end{function}
%
% \begin{function}{number-open-bracket}
%   \begin{syntax}
%     |number-open-bracket| = \meta{symbol}
%   \end{syntax}
% \end{function}
%
% \begin{function}{output-close-uncertainty}
%   \begin{syntax}
%     |output-close-uncertainty| = \meta{symbol}
%   \end{syntax}
% \end{function}
%
% \begin{function}{output-complex-root}
%   \begin{syntax}
%     |output-complex-root| = \meta{symbol}
%   \end{syntax}
% \end{function}
%
% \begin{function}{output-decimal-marker}
%   \begin{syntax}
%     |output-decimal-marker| = \meta{symbol}
%   \end{syntax}
% \end{function}
%
% \begin{function}{output-open-uncertainty}
%   \begin{syntax}
%     |output-open-uncertainty| = \meta{symbol}
%   \end{syntax}
% \end{function}
%
% \begin{function}{parse-numbers}
%   \begin{syntax}
%     |parse-numbers| = |true|\verb"|"|false|
%   \end{syntax}
% \end{function}
%
% \begin{function}{print-explicit-plus}
%   \begin{syntax}
%     |print-explicit-plus| = |true|\verb"|"|false|
%   \end{syntax}
% \end{function}
%
% \begin{function}{print-unity-mantissa}
%   \begin{syntax}
%     |print-unity-mantissa| = |true|\verb"|"|false|
%   \end{syntax}
% \end{function}
%
% \begin{function}{print-zero-exponent}
%   \begin{syntax}
%     |print-zero-exponent| = |true|\verb"|"|false|
%   \end{syntax}
% \end{function}
%
% \begin{function}{round-half}
%   \begin{syntax}
%     |round-half| = |even|\verb"|"|up|
%   \end{syntax}
% \end{function}
%
% \begin{function}{round-minimum}
%   \begin{syntax}
%     |round-minimum| = \meta{min}
%   \end{syntax}
% \end{function}
%
% \begin{function}{round-mode}
%   \begin{syntax}
%     |round-mode| = |figures|\verb"|"|none|\verb"|"|places|\verb"|"|uncertainty|
%   \end{syntax}
% \end{function}
%
% \begin{function}{round-pad}
%   \begin{syntax}
%     |round-pad| = |true|\verb"|"|false|
%   \end{syntax}
% \end{function}
%
% \begin{function}{round-precision}
%   \begin{syntax}
%     |round-precision| = \meta{precision}
%   \end{syntax}
% \end{function}
%
% \begin{function}{separate-uncertainty}
%   \begin{syntax}
%     |separate-uncertainty| = |true|\verb"|"|false|
%   \end{syntax}
% \end{function}
%
% \begin{function}{uncertainty-separator}
%   \begin{syntax}
%     |uncertainty-separator| = \meta{separator}
%   \end{syntax}
% \end{function}
%
% \begin{function}{tight-spacing}
%   \begin{syntax}
%     |tight-spacing| = |true|\verb"|"|false|
%   \end{syntax}
% \end{function}
%
% \end{documentation}
%
% \begin{implementation}
%
% \section{\pkg{siunitx-number} implementation}
%
% Start the \pkg{DocStrip} guards.
%    \begin{macrocode}
%<*package>
%    \end{macrocode}
%
% Identify the internal prefix (\LaTeX3 \pkg{DocStrip} convention): only
% internal material in this \emph{submodule} should be used directly.
%    \begin{macrocode}
%<@@=siunitx_number>
%    \end{macrocode}
%
% \subsection{Initial set-up}
%
%   Variants not provided by \pkg{expl3}.
%    \begin{macrocode}
\cs_generate_variant:Nn \tl_if_blank:nTF { f }
\cs_generate_variant:Nn \tl_if_blank_p:n { f }
\cs_generate_variant:Nn \tl_if_in:NnTF { NV }
%    \end{macrocode}
%
% \begin{variable}{\l_@@_tmp_tl}
%   Scratch space.
%    \begin{macrocode}
\tl_new:N \l_@@_tmp_tl
%    \end{macrocode}
% \end{variable}
%
% \subsection{Main formatting routine}
%
% \begin{variable}{\l_@@_formatted_tl}
%   A token list for the final formatted result: may or may not be generated
%   by the parser, depending on settings which are active.
%    \begin{macrocode}
\tl_new:N \l_@@_formatted_tl
%    \end{macrocode}
% \end{variable}
%
% \begin{variable}{\l_@@_tab_tl}
%   A token list for marking the position of tabular alignments in formatted
%   output.
%    \begin{macrocode}
\tl_new:N \l_@@_tab_tl
%    \end{macrocode}
% \end{variable}
%
% \begin{variable}{\l_@@_parse_bool}
%   Top-level options.
%    \begin{macrocode}
\keys_define:nn { siunitx }
  {
    parse-numbers .bool_set:N = \l_@@_parse_bool
  }
%    \end{macrocode}
% \end{variable}
%
% \begin{macro}{\siunitx_number_format:nN, \siunitx_number_format:VN}
% \begin{macro}{\siunitx_number_format:nNN}
% \begin{macro}{\@@_format:nN}
%    \begin{macrocode}
\cs_new_protected:Npn \siunitx_number_format:nN #1#2
  {
    \tl_clear:N \l_@@_tab_tl
    \@@_format:nN {#1} #2
  }
\cs_generate_variant:Nn \siunitx_number_format:nN { V }
\cs_new_protected:Npn \siunitx_number_format:nNN #1#2#3
  {
    \tl_set:Nn \l_@@_tab_tl {#3}
    \@@_format:nN {#1} #2
  }
\cs_new_protected:Npn \@@_format:nN #1#2
  {
    \group_begin:
      \bool_if:NTF \l_@@_parse_bool
        {
          \@@_parse:n {#1}
          \@@_process:
          \@@_format:
        }
        { \tl_set:Nn \l_@@_formatted_tl { \ensuremath {#1} } }
    \exp_args:NNNV \group_end:
    \tl_set:Nn #2 \l_@@_formatted_tl
  }
%    \end{macrocode}
% \end{macro}
% \end{macro}
% \end{macro}
%
% \subsection{Parsing numbers}
%
% Before numbers can be manipulated or formatted they need to be parsed into
% an internal form. In particular, if multiple code paths are to be avoided,
% it is necessary to do such parsing even for relatively simple cases such
% as converting |1e10| to |1 \times 10^{10}|.
%
% Storing the result of such parsing can be done in a number of ways. In the
% first version of \pkg{siunitx} a series of separate data stores were used.
% This is potentially quite fast (as recovery of items relies only on \TeX{}'s
% hash table) but makes managing the various data entries somewhat tedious and
% error-prone. For version two of the package, a single data structure
% (property list) was used for each part of the parsed number. Whilst this is
% easy to manage and extend, it is somewhat slower as at a \TeX{} level there
% are repeated pack--unpack steps. In particular, the fact that there are a
% limited number of items to track for a \enquote{number} means that a more
% efficient approach is desirable (contrast parsing units, which is open-ended
% and therefore fits well with using a property list).
%
% To allow for complex numbers, two parallel data structures are used, one for
% the real part and one for the imaginary part. If the part is entirely absent
% then the data structures are left empty. Within each part, the structure
% is
% \begin{quote}
%   \marg{comparator}\meta{sign}\marg{integer}\marg{decimal}
%     \marg{uncertainty}\\
%     \meta{exponent sign}\marg{exponent}
% \end{quote}
% where the two sign parts must be single tokens and all other components
% must be given in braces. \emph{All} of the components must be present in
% a stored number (\foreign{i.e.}~at the end of parsing). The number must have
% at least one digit for both the \meta{integer} and \meta{exponent} parts.
%
% \begin{variable}
%   {
%      \l_@@_expression_bool       ,
%      \l_@@_input_uncert_close_tl ,
%      \l_@@_input_comparator_tl   ,
%      \l_@@_input_complex_tl      ,
%      \l_@@_input_decimal_tl      ,
%      \l_@@_input_digit_tl        ,
%      \l_@@_input_exponent_tl     ,
%      \l_@@_input_ignore_tl       ,
%      \l_@@_input_uncert_open_tl  ,
%      \l_@@_input_sign_tl         ,
%      \l_@@_input_uncert_sign_tl
%   }
%  \begin{macro}[EXP]{\@@_expression:n}
%   Options which determine the various valid parts of a parsed number.
%    \begin{macrocode}
\keys_define:nn { siunitx }
  {
    evaluate-expression .bool_set:N =
      \l_@@_expression_bool ,
    expression .code:n =
      \cs_set:Npn \@@_expression:n ##1 {#1} ,
    input-close-uncertainty .tl_set:N =
      \l_@@_input_uncert_close_tl ,
    input-comparators .tl_set:N =
      \l_@@_input_comparator_tl ,
    input-complex-roots .tl_set:N =
      \l_@@_input_complex_tl ,
    input-decimal-markers .tl_set:N =
      \l_@@_input_decimal_tl ,
    input-digits .tl_set:N =
      \l_@@_input_digit_tl ,
    input-exponent-markers .tl_set:N =
      \l_@@_input_exponent_tl ,
    input-ignore .tl_set:N =
      \l_@@_input_ignore_tl ,
    input-open-uncertainty .tl_set:N =
      \l_@@_input_uncert_open_tl  ,
    input-signs .tl_set:N =
      \l_@@_input_sign_tl ,
    input-uncertainty-signs .code:n =
      {
        \tl_set:Nn \l_@@_input_uncert_sign_tl {#1}
        \tl_map_inline:nn {#1}
          {
            \tl_if_in:NnF \l_@@_input_sign_tl {##1}
              { \tl_put_right:Nn \l_@@_input_sign_tl {##1} }
          }
      } ,
    parse-numbers .bool_set:N =
      \l_@@_parse_bool
  }
\cs_new:Npn \@@_expression:n #1 { }
\tl_new:N \l_@@_input_uncert_sign_tl
%    \end{macrocode}
% \end{macro}
% \end{variable}
%
% \begin{variable}{\l_@@_arg_tl}
%   The input argument or a part thereof, depending on the position in
%   the parsing routine.
%    \begin{macrocode}
\tl_new:N \l_@@_arg_tl
%    \end{macrocode}
% \end{variable}
%
% \begin{variable}{\l_@@_comparator_tl}
%   A comparator, if found, is held here.
%    \begin{macrocode}
\tl_new:N \l_@@_comparator_tl
%    \end{macrocode}
% \end{variable}
%
% \begin{variable}{\l_@@_exponent_tl}
%   The exponent part of a parsed number. It is easiest to find this
%   relatively early in the parsing process, but as it needs to go at
%   the end of the internal format is held separately until required.
%    \begin{macrocode}
\tl_new:N \l_@@_exponent_tl
%    \end{macrocode}
% \end{variable}
%
% \begin{variable}{\l_@@_flex_tl}
%   When parsing for a separate uncertainty or complex number, the nature
%   of the grabbed part cannot be determined until the end of the number.
%   To avoid abusing the storage areas, this dedicated one is used for
%   \enquote{flexible} cases.
%    \begin{macrocode}
\tl_new:N \l_@@_flex_tl
%    \end{macrocode}
% \end{variable}
%
% \begin{variable}{\l_@@_imaginary_tl, \l_@@_real_tl}
%   Used to hold the real and imaginary parts of a number in the standardised
%   format.
%    \begin{macrocode}
\tl_new:N \l_@@_imaginary_tl
\tl_new:N \l_@@_real_tl
%    \end{macrocode}
% \end{variable}
%
% \begin{variable}{\l_@@_input_tl}
%   The numerical input exactly as given by the user.
%    \begin{macrocode}
\tl_new:N \l_@@_input_tl
%    \end{macrocode}
% \end{variable}
%
% \begin{variable}{\l_@@_partial_tl}
%   To avoid needing to worry about the fact that the final data stores are
%   somewhat tricky to add to token-by-token, a simple store is used to build
%   up the parsed part of a number before transferring in one go.
%    \begin{macrocode}
\tl_new:N \l_@@_partial_tl
%    \end{macrocode}
% \end{variable}
%
% \begin{variable}{\l_@@_validate_bool}
%   Used to set up for validation with no error production.
%    \begin{macrocode}
\bool_new:N \l_@@_validate_bool
%    \end{macrocode}
% \end{variable}
%
% \begin{macro}{\@@_parse:n}
%   After some initial set up, the parser expands the input and then replaces
%   as far as possible tricky tokens with ones that can be handled using
%   delimited arguments. The parser begins with the assumption that the input
%   is a real number. To avoid multiple conditionals here, the parser is
%   set up as a chain of commands initially, with a loop only later. This
%   avoids more conditionals than are necessary.
%    \begin{macrocode}q
\cs_new_protected:Npn \@@_parse:n #1
  {
    \tl_clear:N \l_@@_imaginary_tl
    \tl_clear:N \l_@@_real_tl
    \protected@edef \l_@@_arg_tl
      {
        \bool_if:NTF \l_@@_expression_bool
          { \fp_eval:n { \@@_expression:n {#1} } }
          {#1}
      }
    \tl_set_eq:NN \l_@@_input_tl \l_@@_arg_tl
    \@@_parse_replace:
    \tl_if_empty:NF \l_@@_arg_tl
      { \@@_parse_comparator: }
    \@@_parse_check:
  }
%    \end{macrocode}
% \end{macro}
%
% \begin{macro}{\@@_parse_check:}
%   After the loop there is one case that might need tidying up. If a
%   separated uncertainty was found it will be currently in \cs{l_@@_flex_tl}
%   and needs moving. A series of tests pick up that case, then the check is
%   made that some content was found for at least one of the real or imaginary
%   parts of the number.
%    \begin{macrocode}
\cs_new_protected:Npn \@@_parse_check:
  {
    \tl_if_empty:NF \l_@@_flex_tl
      {
        \bool_lazy_and:nnTF
          {
            \tl_if_blank_p:f
              { \exp_after:wN \use_iv:nnnn \l_@@_real_tl }
          }
          {
            \tl_if_blank_p:f
              { \exp_after:wN \use_iv:nnnn \l_@@_flex_tl }
          }
          {
            \tl_set:Nx \l_@@_tmp_tl
              { \exp_after:wN \use_i:nnnn \l_@@_flex_tl }
            \tl_if_in:NVTF \l_@@_input_uncert_sign_tl
              \l_@@_tmp_tl
              { \@@_parse_combine_uncert: }
              { \tl_clear:N \l_@@_real_tl }
          }
          { \tl_clear:N \l_@@_real_tl }
      }
    \bool_lazy_and:nnTF
      { \tl_if_empty_p:N \l_@@_real_tl  }
      { \tl_if_empty_p:N \l_@@_imaginary_tl }
      {
        \bool_if:NF \l_@@_validate_bool
          {
            \msg_error:nnx { siunitx } { invalid-number }
              { \exp_not:V \l_@@_input_tl }
          }
      }
      { \@@_parse_finalise: }
  }
%    \end{macrocode}
% \end{macro}
%
% \begin{macro}{\@@_parse_combine_uncert:}
% \begin{macro}{\@@_parse_combine_uncert_auxi:NnnnNnnn}
% \begin{macro}
%   {
%     \@@_parse_combine_uncert_auxii:nnnnn,
%     \@@_parse_combine_uncert_auxii:fnnnn
%   }
% \begin{macro}
%   {
%     \@@_parse_combine_uncert_auxiii:nnnnnn,
%     \@@_parse_combine_uncert_auxiii:fnnnnn
%   }
% \begin{macro}{\@@_parse_combine_uncert_auxiv:nnnn}
% \begin{macro}[EXP]{\@@_parse_combine_uncert_auxv:w}
% \begin{macro}[EXP]{\@@_parse_combine_uncert_auxvi:w}
%   Conversion of a second numerical part to an uncertainty needs a bit of
%   work. The first step is to extract the useful information from the two
%   stores: the sign, integer and decimal parts from the real number and the
%   integer and decimal parts from the second number. That is done using the
%   input stack to avoid lots of assignments.
%    \begin{macrocode}
\cs_new_protected:Npn \@@_parse_combine_uncert:
  {
    \exp_after:wN \exp_after:wN \exp_after:wN
    \@@_parse_combine_uncert_auxi:NnnnNnnn
      \exp_after:wN \l_@@_real_tl \l_@@_flex_tl
  }
%    \end{macrocode}
%   Here, |#4|, |#5| and |#8| are all junk arguments simply there to mop up
%   tokens, while |#1| will be recovered later from \cs{l_@@_real_tl} so does
%   not need to be passed about. The difference in places between the two
%   decimal parts is now found: this is done just once to avoid having to
%   parse token lists twice. The value is then used to generate a number of
%   filler |0| tokens, and these are added to the appropriate part of the
%   number. Finally, everything is recombined: the integer part only needs
%   a test to avoid an empty main number.
%    \begin{macrocode}
\cs_new_protected:Npn
  \@@_parse_combine_uncert_auxi:NnnnNnnn #1#2#3#4#5#6#7#8
  {
    \int_compare:nNnTF { \tl_count:n {#6} } > { \tl_count:n {#2} }
      {
        \tl_clear:N \l_@@_real_tl
        \tl_clear:N \l_@@_flex_tl
      }
      {
        \@@_parse_combine_uncert_auxii:fnnnn
          { \int_eval:n { \tl_count:n {#3} - \tl_count:n {#7} } }
          {#2} {#3} {#6} {#7}
      }
  }
\cs_new_protected:Npn
  \@@_parse_combine_uncert_auxii:nnnnn #1
  {
    \@@_parse_combine_uncert_auxiii:fnnnnn
      { \prg_replicate:nn { \int_abs:n {#1} } { 0 } }
      {#1}
  }
\cs_generate_variant:Nn \@@_parse_combine_uncert_auxii:nnnnn { f }
\cs_new_protected:Npn
  \@@_parse_combine_uncert_auxiii:nnnnnn #1#2#3#4#5#6
  {
    \int_compare:nNnTF {#2} > 0
      {
        \@@_parse_combine_uncert_auxiv:nnnn
          {#3} {#4} {#5} { #6 #1 }
      }
      {
        \@@_parse_combine_uncert_auxiv:nnnn
          {#3} { #4 #1 } {#5} {#6}
      }
  }
\cs_generate_variant:Nn
  \@@_parse_combine_uncert_auxiii:nnnnnn { f }
\cs_new_protected:Npn
  \@@_parse_combine_uncert_auxiv:nnnn #1#2#3#4
  {
    \tl_set:Nx \l_@@_real_tl
      {
        \tl_head:V \l_@@_real_tl
          { \exp_not:n {#1} }
          {
            \bool_lazy_and:nnTF
              { \tl_if_blank_p:n {#2} }
              { ! \tl_if_blank_p:n {#4} }
              { 0 }
              { \exp_not:n {#2} }
          }
          {
            \@@_parse_combine_uncert_auxv:w #3#4
              \q_recursion_tail \q_recursion_stop
          }
      }
  }
%    \end{macrocode}
%   A short routine to remove any leading zeros in the uncertainty part,
%   which are not needed for the compact representation used by the module.
%    \begin{macrocode}
\cs_new:Npn \@@_parse_combine_uncert_auxv:w #1
  {
    \quark_if_recursion_tail_stop:N #1
    \str_if_eq:nnTF {#1} { 0 }
      { \@@_parse_combine_uncert_auxv:w }
      { \@@_parse_combine_uncert_auxvi:w #1 }
  }
\cs_new:Npn \@@_parse_combine_uncert_auxvi:w
  #1 \q_recursion_tail \q_recursion_stop
  { \exp_not:n {#1} }
%    \end{macrocode}
% \end{macro}
% \end{macro}
% \end{macro}
% \end{macro}
% \end{macro}
% \end{macro}
% \end{macro}
%
% \begin{macro}{\@@_parse_comparator:}
% \begin{macro}{\@@_parse_comparator_aux:Nw}
%   A comparator has to be the very first token in the input. A such, the
%   test for this can be very fast: grab the first token, do a check and
%   if appropriate store the result.
%    \begin{macrocode}
\cs_new_protected:Npn \@@_parse_comparator:
  {
    \exp_after:wN \@@_parse_comparator_aux:Nw
      \l_@@_arg_tl \q_stop
  }
\cs_new_protected:Npn \@@_parse_comparator_aux:Nw #1#2 \q_stop
  {
    \tl_if_in:NnTF \l_@@_input_comparator_tl {#1}
      {
        \tl_set:Nn \l_@@_comparator_tl {#1}
        \tl_set:Nn \l_@@_arg_tl {#2}
      }
      { \tl_clear:N \l_@@_comparator_tl }
    \tl_if_empty:NF \l_@@_arg_tl
      { \@@_parse_sign: }
  }
%    \end{macrocode}
% \end{macro}
% \end{macro}
%
% \begin{macro}{\@@_parse_exponent:}
% \begin{macro}{\@@_parse_exponent_aux:w}
% \begin{macro}{\@@_parse_exponent_aux:nn}
% \begin{macro}{\@@_parse_exponent_aux:Nw}
% \begin{macro}{\@@_parse_exponent_aux:Nn}
% \begin{macro}
%   {\@@_parse_exponent_zero_test:N, \@@_parse_exponent_check:N}
% \begin{macro}{\@@_parse_exponent_cleanup:N}
%   An exponent part of a number has to come at the end and can only occur
%   once. Thus it is relatively easy to parse. First, there is a check that
%   an exponent part is allowed, and if so a split is made (the previous
%   part of the chain checks that there is some content in \cs{l_@@_arg_tl}
%   before calling this function). After splitting, if there is no exponent
%   then simply save a default. Otherwise, check for a sign and then store
%   either this or an assumed |+| and the digits after a check that nothing
%   else is present after the~|e|. The only slight complication to all of
%   this is allowing an arbitrary token in the input to represent the exponent:
%   this is done by setting any exponent tokens to the first of the allowed
%   list, then using that in a delimited argument set up. Once an exponent
%   part is found, there is a loop to check that each of the tokens is a digit
%   then a tidy up step to remove any leading zeros.
%    \begin{macrocode}
\cs_new_protected:Npn \@@_parse_exponent:
  {
    \tl_if_empty:NTF \l_@@_input_exponent_tl
      {
        \tl_set:Nn \l_@@_exponent_tl { +0 }
        \tl_if_empty:NF \l_@@_real_tl
          { \@@_parse_loop: }
      }
      {
        \tl_set:Nx \l_@@_tmp_tl
          { \tl_head:V \l_@@_input_exponent_tl }
        \tl_map_inline:Nn \l_@@_input_exponent_tl
          {
            \tl_replace_all:NnV \l_@@_arg_tl
              {##1} \l_@@_tmp_tl
          }
        \use:x
          {
            \cs_set_protected:Npn
              \exp_not:N \@@_parse_exponent_aux:w
              ####1 \exp_not:V \l_@@_tmp_tl
              ####2 \exp_not:V \l_@@_tmp_tl
              ####3 \exp_not:N \q_stop
          }
            { \@@_parse_exponent_aux:nn {##1} {##2} }
        \use:x
          {
            \@@_parse_exponent_aux:w
              \exp_not:V \l_@@_arg_tl
              \exp_not:V \l_@@_tmp_tl \exp_not:N \q_nil
              \exp_not:V \l_@@_tmp_tl \exp_not:N \q_stop
          }
      }
  }
\cs_new_protected:Npn \@@_parse_exponent_aux:w  { }
\cs_new_protected:Npn \@@_parse_exponent_aux:nn #1#2
  {
    \quark_if_nil:nTF {#2}
      { \tl_set:Nn \l_@@_exponent_tl { +0 } }
      {
        \tl_set:Nn \l_@@_arg_tl {#1}
        \tl_if_blank:nTF {#2}
          { \tl_clear:N \l_@@_real_tl }
          { \@@_parse_exponent_aux:Nw #2 \q_stop }
      }
    \tl_if_empty:NF \l_@@_real_tl
      { \@@_parse_loop: }
  }
\cs_new_protected:Npn \@@_parse_exponent_aux:Nw #1#2 \q_stop
  {
    \tl_if_in:NnTF \l_@@_input_sign_tl {#1}
      { \@@_parse_exponent_aux:Nn #1 {#2} }
      { \@@_parse_exponent_aux:Nn + {#1#2} }
    \tl_if_empty:NT \l_@@_exponent_tl
      { \tl_clear:N \l_@@_real_tl }
  }
\cs_new_protected:Npn \@@_parse_exponent_aux:Nn #1#2
  {
    \tl_set:Nn \l_@@_exponent_tl { #1 }
    \tl_if_blank:nTF {#2}
      { \tl_clear:N \l_@@_real_tl }
      {
        \@@_parse_exponent_zero_test:N #2
          \q_recursion_tail \q_recursion_stop
      }
  }
\cs_new_protected:Npn \@@_parse_exponent_zero_test:N #1
  {
    \quark_if_recursion_tail_stop_do:Nn #1
      { \tl_set:Nn \l_@@_exponent_tl { +0 } }
    \str_if_eq:nnTF {#1} { 0 }
      { \@@_parse_exponent_zero_test:N }
      { \@@_parse_exponent_check:N #1 }
  }
\cs_new_protected:Npn \@@_parse_exponent_check:N #1
  {
    \quark_if_recursion_tail_stop:N #1
    \tl_if_in:NnTF \l_@@_input_digit_tl {#1}
      {
        \tl_put_right:Nn \l_@@_exponent_tl {#1}
        \@@_parse_exponent_check:N
      }
      { \@@_parse_exponent_cleanup:wN }
  }
\cs_new_protected:Npn \@@_parse_exponent_cleanup:wN
  #1 \q_recursion_stop
  { \tl_clear:N \l_@@_real_tl }
%    \end{macrocode}
% \end{macro}
% \end{macro}
% \end{macro}
% \end{macro}
% \end{macro}
% \end{macro}
%
% \begin{macro}{\@@_parse_replace:}
% \begin{macro}{\@@_parse_replace_aux:nN}
% \begin{macro}{\@@_parse_replace_sign:}
% \begin{variable}{\c_@@_parse_sign_replacement_tl}
%   There are two parts to the replacement code. First, any active
%   hyphens signs are normalised: these can come up with some packages and
%   cause issues. Multi-token signs then are converted to the single token
%   equivalents so that everything else can work on a one token basis.
%    \begin{macrocode}
\cs_new_protected:Npn \@@_parse_replace:
  {
    \@@_parse_replace_minus:
    \exp_last_unbraced:NV \@@_parse_replace_aux:nN
      \c_@@_parse_sign_replacement_tl
      { ? } \q_recursion_tail
        \q_recursion_stop
  }
\cs_set_protected:Npn \@@_parse_replace_aux:nN #1#2
  {
    \quark_if_recursion_tail_stop:N #2
    \tl_replace_all:Nnn \l_@@_arg_tl {#1} {#2}
    \@@_parse_replace_aux:nN
  }
\tl_const:Nn \c_@@_parse_sign_replacement_tl
  {
    { -+ } \mp
    { +- } \pm
    { << } \ll
    { <= } \le
    { >> } \gg
    { >= } \ge
  }
\group_begin:
  \char_set_catcode_active:N \-
  \cs_new_protected:Npx \@@_parse_replace_minus:
    {
      \tl_replace_all:Nnn \exp_not:N \l_@@_arg_tl
        { \exp_not:N - }  { \token_to_str:N - }
    }
\group_end:
%    \end{macrocode}
% \end{variable}
% \end{macro}
% \end{macro}
% \end{macro}
% \end{macro}
%
% \begin{macro}{\@@_parse_finalise:}
% \begin{macro}{\@@_parse_finalise_aux:N}
% \begin{macro}{\@@_parse_finalise_aux:Nw}
%   Combine all of the bits of a number together: both the real and
%   imaginary parts contain all of the data.
%    \begin{macrocode}
\cs_new_protected:Npn \@@_parse_finalise:
  {
    \@@_parse_finalise_aux:N \l_@@_real_tl
    \@@_parse_finalise_aux:N \l_@@_imaginary_tl
  }
\cs_new_protected:Npn \@@_parse_finalise_aux:N #1
  {
    \tl_if_empty:NF #1
      {
        \tl_set:Nx #1
          {
            { \exp_not:V \l_@@_comparator_tl }
            \exp_not:V #1
            \exp_after:wN \@@_parse_finalise_aux:Nw
              \l_@@_exponent_tl \q_stop
          }
      }
  }
\cs_new:Npn \@@_parse_finalise_aux:Nw #1#2 \q_stop
  {
    \exp_not:N #1
    { \exp_not:n {#2} }
  }
%    \end{macrocode}
% \end{macro}
% \end{macro}
% \end{macro}
%
% \begin{macro}{\@@_parse_loop:}
% \begin{macro}{\@@_parse_loop_first:N}
% \begin{macro}{\@@_parse_loop_main:NNNNN}
% \begin{macro}{\@@_parse_loop_main_end:NN}
% \begin{macro}{\@@_parse_loop_main_digit:NNNNN}
% \begin{macro}{\@@_parse_loop_main_decimal:NN}
% \begin{macro}{\@@_parse_loop_main_uncert:NNN}
% \begin{macro}{\@@_parse_loop_main_complex:N}
% \begin{macro}{\@@_parse_loop_main_sign:NNN}
% \begin{macro}{\@@_parse_loop_main_store:NNN}
% \begin{macro}{\@@_parse_loop_after_decimal:NNN}
% \begin{macro}{\@@_parse_loop_uncert:NNNNN}
% \begin{macro}{\@@_parse_loop_after_uncert:NNN}
% \begin{macro}{\@@_parse_loop_root_swap:NNwNN}
% \begin{macro}{\@@_parse_loop_complex_cleanup:wN}
% \begin{macro}{\@@_parse_loop_break:wN}
%   At this stage, the partial input \cs{l_@@_arg_tl} will contain any
%   mantissa, which may contain an uncertainty or complex part. Parsing this
%   and allowing for all of the different formats possible is best done using
%   a token-by-token approach. However, as at each stage only a subset of
%   tokens are valid, the approach take is to use a set of semi-dedicated
%   functions to parse different components along with switches to allow a
%   sensible amount of code sharing.
%    \begin{macrocode}
\cs_new_protected:Npn \@@_parse_loop:
  {
    \tl_clear:N \l_@@_partial_tl
    \exp_after:wN \@@_parse_loop_first:NNN
      \exp_after:wN \l_@@_real_tl \exp_after:wN \c_true_bool
        \l_@@_arg_tl
        \q_recursion_tail \q_recursion_stop
  }
%    \end{macrocode}
%   The very first token of the input is handled with a dedicated function.
%   Valid cases here are
%   \begin{itemize}
%     \item Entirely blank if the original input was for example |+e10|:
%       simply clean up if in the integer part of issue an error if in
%       a second part (complex number, \foreign{etc.}).
%     \item An integer part digit: pass through to the main collection
%       routine.
%     \item A decimal marker: store an empty integer part and move to
%       the main collection routine for a decimal part.
%     \item A complex root token: shuffle to the end of the input.
%   \end{itemize}
%   Anything else is invalid and sends the code to the abort function.
%    \begin{macrocode}
\cs_new_protected:Npn \@@_parse_loop_first:NNN #1#2#3
  {
    \quark_if_recursion_tail_stop_do:Nn #3
      {
        \bool_if:NTF #2
          { \tl_put_right:Nn #1 { { 1 } { } { } } }
          { \@@_parse_loop_break:wN \q_recursion_stop }
      }
    \tl_if_in:NnTF \l_@@_input_digit_tl {#3}
      {
        \@@_parse_loop_main:NNNNN
          #1 \c_true_bool \c_false_bool #2 #3
      }
      {
        \tl_if_in:NnTF \l_@@_input_decimal_tl {#3}
          {
            \tl_put_right:Nn #1 { { 0 } }
            \@@_parse_loop_after_decimal:NNN #1 #2
          }
          {
            \tl_if_in:NnTF \l_@@_input_complex_tl {#3}
              { \@@_parse_loop_root_swap:NNwNN #1 #3 }
              { \@@_parse_loop_break:wN }
          }
      }
  }
%    \end{macrocode}
%   A single function is used to cover the \enquote{main} part of numbers:
%   finding real, complex or separated uncertainty parts and covering both
%   the integer and decimal components. This works because these elements
%   share a lot of concepts: a small number of switches can be used to
%   differentiate between them. To keep the code at least somewhat readable,
%   this main function deals with the validity testing but hands off other
%   tasks to dedicated auxiliaries for each case.
%
%   The possibilities are
%   \begin{itemize}
%     \item The number terminates, meaning that some digits were collected
%       and everything is simply tidied up (as far as the loop is concerned).
%     \item A digit is found: this is the common case and leads to a storage
%       auxiliary (which handles non-significant zeros).
%     \item A decimal marker is found: only valid in the integer part and
%       there leading to a store-and-switch situation.
%     \item An open-uncertainty token: switch to the dedicated collector
%       for uncertainties.
%     \item A complex root token: store the current number as an imaginary
%       part and terminate the loop.
%     \item A sign token (if allowed): stop collecting this number and
%       restart collection for the second part.
%   \end{itemize}
%    \begin{macrocode}
\cs_new_protected:Npn \@@_parse_loop_main:NNNNN #1#2#3#4#5
  {
    \quark_if_recursion_tail_stop_do:Nn #5
      { \@@_parse_loop_main_end:NN #1#2 }
    \tl_if_in:NnTF \l_@@_input_digit_tl {#5}
      { \@@_parse_loop_main_digit:NNNNN #1#2#3#4#5 }
      {
        \tl_if_in:NnTF \l_@@_input_decimal_tl {#5}
          {
            \bool_if:NTF #2
              { \@@_parse_loop_main_decimal:NN #1 #4 }
              { \@@_parse_loop_break:wN }
          }
          {
            \tl_if_in:NnTF \l_@@_input_uncert_open_tl {#5}
              { \@@_parse_loop_main_uncert:NNN #1#2 #4 }
              {
                \tl_if_in:NnTF \l_@@_input_complex_tl {#5}
                  {
                    \@@_parse_loop_main_store:NNN
                      #1 #2 \c_true_bool
                    \@@_parse_loop_main_complex:N #1
                  }
                  {
                    \bool_if:NTF #4
                      {
                        \tl_if_in:NnTF \l_@@_input_sign_tl {#5}
                          {
                            \@@_parse_loop_main_sign:NNN
                              #1#2 #5
                          }
                          { \@@_parse_loop_break:wN }
                      }
                      { \@@_parse_loop_break:wN }
                  }
              }
          }
      }
  }
%    \end{macrocode}
%   If the main loop finds the end marker then there is a tidy up phase.
%   The current partial number is stored either as the integer or decimal,
%   depending on the setting for the indicator switch. For the integer
%   part, if no number has been collected then one or more non-significant
%   zeros have been dropped. Exactly one zero is therefore needed to make
%   sure the parsed result is correct.
%    \begin{macrocode}
\cs_new_protected:Npn \@@_parse_loop_main_end:NN #1#2
  {
    \bool_lazy_and:nnT
      {#2} { \tl_if_empty_p:N \l_@@_partial_tl }
      { \tl_set:Nn \l_@@_partial_tl { 0 } }
    \tl_put_right:Nx #1
      {
        { \exp_not:V \l_@@_partial_tl }
        \bool_if:NT #2 { { } }
        { }
      }
  }
%    \end{macrocode}
%   The most common case for the main loop collector is to find a digit.
%   Here, in the integer part it is possible that zeros are non-significant:
%   that is handled using a combination of a switch and a string test. Other
%   than that, the situation here is simple: store the input and loop.
%    \begin{macrocode}
\cs_new_protected:Npn \@@_parse_loop_main_digit:NNNNN #1#2#3#4#5
  {
    \bool_lazy_or:nnTF
      {#3} { ! \str_if_eq_p:nn {#5} { 0 } }
      {
        \tl_put_right:Nn \l_@@_partial_tl {#5}
        \@@_parse_loop_main:NNNNN #1 #2 \c_true_bool #4
      }
      { \@@_parse_loop_main:NNNNN #1 #2 \c_false_bool #4 }
  }
%    \end{macrocode}
%   When a decimal marker was found, move the integer part to the
%   store and then go back to the loop with the flags set correctly.
%   There is the case of non-significant zeros to cover before that, of course.
%    \begin{macrocode}
\cs_new_protected:Npn \@@_parse_loop_main_decimal:NN #1#2
  {
    \@@_parse_loop_main_store:NNN #1 \c_false_bool \c_false_bool
    \@@_parse_loop_after_decimal:NNN #1 #2
  }
%    \end{macrocode}
%   Starting an uncertainty part means storing the number to date as in other
%   cases, with the possibility of a blank decimal part allowed for. The
%   uncertainty itself is collected by a dedicated function as it is extremely
%   restricted.
%    \begin{macrocode}
\cs_new_protected:Npn \@@_parse_loop_main_uncert:NNN #1#2#3
  {
    \@@_parse_loop_main_store:NNN #1 #2 \c_false_bool
    \@@_parse_loop_uncert:NNNNN
      #1 \c_true_bool \c_false_bool #3
  }
%    \end{macrocode}
%   A complex root token has to be at the end of the input (leading ones
%   are dealt with specially). Thus after moving the data to the correct
%   place there is a hand-off to a cleanup function. The case where only the
%   complex root token was given is covered by
%   \cs{@@_parse_loop_root_swap:NNwNN}.
%    \begin{macrocode}
\cs_new_protected:Npn \@@_parse_loop_main_complex:N #1
  {
    \tl_set_eq:NN \l_@@_imaginary_tl #1
    \tl_clear:N #1
    \@@_parse_loop_complex_cleanup:wN
  }
%    \end{macrocode}
%   If a sign is found, terminate the current number, store the sign as the
%   first token of the second part and go back to do the dedicated first-token
%   function.
%    \begin{macrocode}
\cs_new_protected:Npn \@@_parse_loop_main_sign:NNN #1#2#3
  {
    \@@_parse_loop_main_store:NNN #1 #2 \c_true_bool
    \tl_set:Nn \l_@@_flex_tl {#3}
    \@@_parse_loop_first:NNN
      \l_@@_flex_tl \c_false_bool
  }
%    \end{macrocode}
%   A common auxiliary for the various non-digit token functions: tidy up the
%   integer and decimal parts of a number. Here, the two flags are used to
%   indicate if empty decimal and uncertainty parts should be included in
%   the storage cycle.
%    \begin{macrocode}
\cs_new_protected:Npn \@@_parse_loop_main_store:NNN #1#2#3
  {
    \tl_if_empty:NT \l_@@_partial_tl
      { \tl_set:Nn \l_@@_partial_tl { 0 } }
    \tl_put_right:Nx #1
      {
        { \exp_not:V \l_@@_partial_tl }
        \bool_if:NT #2 { { } }
        \bool_if:NT #3 { { } }
      }
    \tl_clear:N \l_@@_partial_tl
  }
%    \end{macrocode}
%   After a decimal marker there has to be a digit if there wasn't one before
%   it. That is handled by using a dedicated function, which checks for
%   an empty integer part first then either simply hands off or looks for
%   a digit.
%    \begin{macrocode}
\cs_new_protected:Npn \@@_parse_loop_after_decimal:NNN #1#2#3
  {
    \tl_if_blank:fTF { \exp_after:wN \use_none:n #1 }
      {
        \quark_if_recursion_tail_stop_do:Nn #3
          { \@@_parse_loop_break:wN \q_recursion_stop }
        \tl_if_in:NnTF \l_@@_input_digit_tl {#1}
          {
            \tl_put_right:Nn \l_@@_partial_tl {#3}
            \@@_parse_loop_main:NNNNN
              #1 \c_false_bool \c_true_bool #2
          }
          { \@@_parse_loop_break:wN }
      }
      {
        \@@_parse_loop_main:NNNNN
          #1 \c_false_bool \c_true_bool #2 #3
      }
  }
%    \end{macrocode}
%   Inside the brackets for an uncertainty the range of valid choices is
%   very limited. Either the token is a digit, in which case there is a
%   test to look for non-significant zeros, or it is a closing bracket. The
%   latter is not valid for the very first token, which is handled using a
%   switch (it's a simple enough difference).
%    \begin{macrocode}
\cs_new_protected:Npn \@@_parse_loop_uncert:NNNNN #1#2#3#4#5
  {
    \quark_if_recursion_tail_stop_do:Nn #5
      { \@@_parse_loop_break:wN \q_recursion_stop }
    \tl_if_in:NnTF \l_@@_input_digit_tl {#5}
      {
        \bool_lazy_or:nnTF
          {#3} { ! \str_if_eq_p:nn {#5} { 0 } }
          {
            \tl_put_right:Nn \l_@@_partial_tl {#5}
            \@@_parse_loop_uncert:NNNNN
              #1 \c_false_bool \c_true_bool #4
          }
          {
            \@@_parse_loop_uncert:NNNNN
              #1 \c_false_bool \c_false_bool #4
          }
      }
      {
        \tl_if_in:NnTF \l_@@_input_uncert_close_tl {#5}
          {
            \bool_if:NTF #2
              { \@@_parse_loop_break:wN }
              {
                \@@_parse_loop_main_store:NNN #1
                  \c_false_bool \c_false_bool
                \@@_parse_loop_after_uncert:NNN #1 #3
              }
          }
          { \@@_parse_loop_break:wN }
      }
  }
%    \end{macrocode}
%   After a bracketed uncertainty there are only a very small number of
%   valid choices. The number can end, there can be a complex root token
%   or there can be a sign. The latter is only allowed if the part being
%   parsed at the moment was the first part of the number. The case where
%   there is no root symbol but there should have been is cleared up after
%   the loop code, so at this stage there is no check.
%    \begin{macrocode}
\cs_new_protected:Npn \@@_parse_loop_after_uncert:NNN #1#2#3
  {
    \quark_if_recursion_tail_stop:N #3
    \tl_if_in:NnTF \l_@@_input_complex_tl {#3}
      { \@@_parse_loop_main_complex:N #1 }
      {
        \bool_if:NTF #2
          {
            \tl_if_in:NnTF \l_@@_input_sign_tl {#3}
              {
                \tl_set:Nn \l_@@_flex_tl {#3}
                \@@_parse_loop_first:NNN
                  \l_@@_flex_tl \c_false_bool
              }
              { \@@_parse_loop_break:wN }
          }
          { \@@_parse_loop_break:wN }
      }
  }
%    \end{macrocode}
%   When the complex root symbol comes at the start of the number rather than
%   at the end, the easiest approach is to shuffle it to the \enquote{normal}
%   position. As the exponent has already been removed, this must be the last
%   token of the input and any duplication will be picked up. The case where
%   just a complex root token has to be covered: in that situation, there is
%   an implicit |1| to store after which the loop stops.
%    \begin{macrocode}
\cs_new_protected:Npn \@@_parse_loop_root_swap:NNwNN #1#2#3
  \q_recursion_tail \q_recursion_stop
  {
    \tl_if_blank:nTF {#3}
      {
        \tl_set:Nx \l_@@_imaginary_tl
          {
            \exp_not:V #1
            { 1 } { } { }
          }
        \tl_clear:N #1
      }
      {
        \use:x
          {
            \tl_clear:N \exp_not:N #1
            \tl_set:Nn \exp_not:N \l_@@_flex_tl { \exp_not:V #1 }
          }
        \@@_parse_loop_first:NNN
          \l_@@_flex_tl \c_false_bool
          #3 #2 \q_recursion_tail \q_recursion_stop
      }
  }
%    \end{macrocode}
%   Nothing is allowed after a complex root token: check and if there is
%   kill the parsing.
%    \begin{macrocode}
\cs_new_protected:Npn \@@_parse_loop_complex_cleanup:wN
  #1 \q_recursion_tail \q_recursion_stop
  {
    \tl_if_blank:nF {#1}
      { \@@_parse_loop_break:wN \q_recursion_stop }
  }
%    \end{macrocode}
%   Something is not right: remove all of the remaining tokens from the
%   number and clear the storage areas as a signal for the next part of the
%   code.
%    \begin{macrocode}
\cs_new_protected:Npn \@@_parse_loop_break:wN
  #1 \q_recursion_stop
  {
    \tl_clear:N \l_@@_imaginary_tl
    \tl_clear:N \l_@@_flex_tl
    \tl_clear:N \l_@@_real_tl
  }
%    \end{macrocode}
% \end{macro}
% \end{macro}
% \end{macro}
% \end{macro}
% \end{macro}
% \end{macro}
% \end{macro}
% \end{macro}
% \end{macro}
% \end{macro}
% \end{macro}
% \end{macro}
% \end{macro}
% \end{macro}
% \end{macro}
% \end{macro}
%
% \begin{macro}{\@@_parse_sign:}
% \begin{macro}{\@@_parse_sign_aux:Nw}
%   The first token of a number after a comparator could be a sign. A quick
%   check is made and if found stored; if there is no sign then the internal
%   format requires that |+| is used. For the number to be valid it has to be
%   more than just a sign, so the next part of the chain is only called if that
%   is the case.
%    \begin{macrocode}
\cs_new_protected:Npn \@@_parse_sign:
  {
    \exp_after:wN \@@_parse_sign_aux:Nw
      \l_@@_arg_tl \q_stop
  }
\cs_new_protected:Npn \@@_parse_sign_aux:Nw #1#2 \q_stop
  {
    \tl_if_in:NnTF \l_@@_input_sign_tl {#1}
      {
        \tl_set:Nn \l_@@_arg_tl {#2}
        \tl_set:Nn \l_@@_real_tl {#1}
      }
      { \tl_set:Nn \l_@@_real_tl { + } }
    \tl_if_empty:NTF \l_@@_arg_tl
      { \tl_clear:N \l_@@_real_tl }
      { \@@_parse_exponent: }
  }
%    \end{macrocode}
% \end{macro}
% \end{macro}
%
% \subsection{Processing numbers}
%
% \begin{variable}
%   {
%     \l_@@_drop_uncertainty_bool  ,
%     \l_@@_drop_zero_decimal_bool ,
%     \l_@@_exponent_mode_tl       ,
%     \l_@@_exponent_fixed_int     ,
%     \l_@@_min_decimal_int        ,
%     \l_@@_min_integer_int        ,
%     \l_@@_round_half_even_bool   ,
%     \l_@@_round_min_tl           ,
%     \l_@@_round_mode_tl          ,
%     \l_@@_round_pad_bool         ,
%     \l_@@_round_precision_int
%   }
%    \begin{macrocode}
\keys_define:nn { siunitx }
  {
    drop-uncertainty .bool_set:N =
      \l_@@_drop_uncertainty_bool ,
    drop-zero-decimal .bool_set:N =
      \l_@@_drop_zero_decimal_bool ,
    exponent-mode .choices:nn =
      { engineering , fixed , input , scientific }
      { \tl_set_eq:NN \l_@@_exponent_mode_tl \l_keys_choice_tl } ,
    fixed-exponent .int_set:N =
      \l_@@_exponent_fixed_int ,
    minimum-decimal-digits .int_set:N =
      \l_@@_min_decimal_int ,
    minimum-integer-digits .int_set:N =
      \l_@@_min_integer_int ,
    round-half .choice: ,
    round-half / even .code:n =
      { \bool_set_true:N \l_@@_round_half_even_bool } ,
    round-half / up .code:n =
      { \bool_set_false:N \l_@@_round_half_even_bool } ,
    round-minimum .tl_set:N =
      \l_@@_round_min_tl ,
    round-mode .choices:nn =
      { figures , none , places, uncertainty }
      { \tl_set_eq:NN \l_@@_round_mode_tl \l_keys_choice_tl } ,
    round-pad .bool_set:N =
      \l_@@_round_pad_bool ,
    round-precision .int_set:N =
      \l_@@_round_precision_int ,
  }
\bool_new:N \l_@@_round_half_even_bool
\tl_new:N \l_@@_exponent_mode_tl
\tl_new:N \l_@@_round_mode_tl
%    \end{macrocode}
% \end{variable}
%
% \begin{variable}{\c_@@_zero_tl, \c_@@_zero_decimal_tl}
%   The two possible forms for zero.
%    \begin{macrocode}
\tl_const:Nn \c_@@_zero_tl { { } + { 0 } { } { } + { 0 } }
\tl_const:Nn \c_@@_zero_decimal_tl { { } + { 0 } { 0 } { } + { 0 } }
%    \end{macrocode}
% \end{variable}
%
% \begin{macro}{\@@_process:}
% \begin{macro}{\@@_process:N}
%   A top-level interface for the processing tools.
%    \begin{macrocode}
\cs_new_protected:Npn \@@_process:
  {
    \@@_process:N \l_@@_real_tl
    \@@_process:N \l_@@_imaginary_tl
  }
\cs_new_protected:Npn \@@_process:N #1
  {
    \tl_if_empty:NF #1
      {
        \bool_lazy_or:nnF
          { \tl_if_eq_p:NN #1 \c_@@_zero_tl }
          { \tl_if_eq_p:NN #1 \c_@@_zero_decimal_tl }
          {
            \@@_uncertainty:N #1
            \@@_exponent:N #1
            \@@_round:N #1
            \@@_zero_decimal:N #1
            \@@_digits:N #1
          }
      }
  }
%    \end{macrocode}
% \end{macro}
% \end{macro}
%
% \begin{macro}{\@@_exponent:N}
% \begin{macro}[EXP]
%   {
%     \@@_exponent_engineering:nNnnnNn ,
%     \@@_exponent_fixed:nNnnnNn       ,
%     \@@_exponent_input:nNnnnNn       ,
%     \@@_exponent_scientific:nNnnnNn
%   }
% \begin{macro}[EXP]{\@@_exponent_scientific:nnnw}
% \begin{macro}[EXP]{\@@_exponent_shift:nnn}
% \begin{macro}[EXP]{\@@_exponent_shift_down:nnnw}
% \begin{macro}[EXP]{\@@_exponent_shift_down:nnn}
% \begin{macro}[EXP]{\@@_exponent_shift_down:nw}
% \begin{macro}[EXP]{\@@_exponent_shift_up:nnn}
% \begin{macro}[EXP]{\@@_exponent_shift_up:nnw}
% \begin{macro}[EXP]{\@@_exponent_finaliise:n}
% \begin{macro}[EXP]{\@@_exponent_engineering_aux:nNnnnNn}
% \begin{macro}[EXP]
%   {
%     \@@_exponent_engineering_0:nnnn ,
%     \@@_exponent_engineering_1:nnnn ,
%     \@@_exponent_engineering_2:nnnn
%   }
%  \begin{macro}[EXP]{\@@_exponent_engineering:nNw}
%   Manipulating an exponent is done using a single expansion function
%   \emph{unless} dealing with engineering-style output. The latter is easier
%   to handle by first converting to scientific output, then post-processing.
%   (Once \texttt{e}-type expansion is generally available, this will be
%   handling using a single \cs{tl_set:Nx}.)
%    \begin{macrocode}
\cs_new_protected:Npn \@@_exponent:N #1
  {
    \tl_set:Nx #1
      {
        \cs:w
          @@_exponent_ \l_@@_exponent_mode_tl :nNnnnNn
          \exp_after:wN
        \cs_end: #1
      }
    \str_if_eq:VnT \l_@@_exponent_mode_tl { engineering }
      {
        \tl_set:Nx #1
         { \exp_after:wN \@@_exponent_engineering_aux:nNnnnNn #1 }
      }
  }
\cs_new:Npn \@@_exponent_fixed:nNnnnNn #1#2#3#4#5#6#7
  {
    \exp_not:n { {#1} #2 }
    \@@_exponent_shift:nnn
      { \l_@@_exponent_fixed_int - (#6#7) } {#3} {#4}
    \exp_not:n { {#5} }
    \exp_not:N #6 { \int_use:N \l_@@_exponent_fixed_int }
  }
\cs_new:Npn \@@_exponent_input:nNnnnNn #1#2#3#4#5#6#7
  { \exp_not:n { {#1} #2 {#3} {#4} {#5} #6 {#7} } }
%    \end{macrocode}
%   To convert to scientific notation, the key question is to find the number
%   of significant places. That is easy enough if the number has a non-zero
%   integer component. For a pure decimal, we have to trim off leading
%   zeros in a loop.
%    \begin{macrocode}
\cs_new:Npn \@@_exponent_scientific:nNnnnNn #1#2#3#4#5#6#7
  {
    \exp_not:n { {#1} #2 }
    \int_compare:nNnTF { \tl_count:n {#3} } = 1
      {
        \str_if_eq:nnTF {#3} { 0 }
          {
            \@@_exponent_scientific:nnnw
              { 0 } {#5} { #6#7 } #4 \q_stop
          }
          { \exp_not:n { {#3} {#4} {#5} #6 {#7} } }
      }
      {
        \@@_exponent_shift:nnn { \tl_count:n {#3} - 1 } {#3} {#4}
        \exp_not:n { {#5} }
        \@@_exponent_finalise:n { \tl_count:n {#3} + #6#7 - 1 }
      }
  }
\cs_new_eq:NN \@@_exponent_engineering:nNnnnNn
  \@@_exponent_scientific:nNnnnNn
\cs_new:Npn \@@_exponent_scientific:nnnw #1#2#3#4#5 \q_stop
  {
    \str_if_eq:nnTF {#4} { 0 }
      {
        \@@_exponent_scientific:nnnw
          { #1 - 1 } {#2} {#3} #5 \q_stop
      }
      {
        \exp_not:n { {#4} {#5} {#2} }
        \@@_exponent_finalise:n { #1 + #3 - 1 }
      }
  }
%    \end{macrocode}
%   When adjusting the exponent position, there are two paths depending on
%   which way the shift takes place.
%    \begin{macrocode}
\cs_new:Npn \@@_exponent_shift:nnn #1#2#3
  {
    \int_compare:nNnTF {#1} > 0
      { \@@_exponent_shift_down:nnnw {#1} {#3} { } #2 \q_stop }
      { \@@_exponent_shift_up:nnw {#1} {#2} #3 \q_stop}
  }
%    \end{macrocode}
%   For shifting the exponent down, there is first a loop to reserve the
%   integer part before doing the work: that of course has to be undone
%   for any remainder at he end of the process.
%    \begin{macrocode}
\cs_new:Npn \@@_exponent_shift_down:nnnw #1#2#3#4#5 \q_stop
  {
    \tl_if_blank:nTF {#5}
      { \@@_exponent_shift_down:nnn {#1} { #4 #3 } {#2} }
      { \@@_exponent_shift_down:nnnw {#1} {#2} { #4 #3 } #5 \q_stop }
  }
\cs_new:Npn \@@_exponent_shift_down:nnn #1#2#3
  {
    \int_compare:nNnTF {#1} = 0
      { { \tl_reverse:n {#2} } \exp_not:n { {#3} } }
      { \@@_exponent_shift_down:nw {#1} #2 \q_stop {#3} }
  }
\cs_new:Npn \@@_exponent_shift_down:nw #1#2#3 \q_stop #4
  {
    \tl_if_blank:nTF {#3}
      { \@@_exponent_shift_down:nnn { #1 - 1 } { 0 } { #2#4 } }
      { \@@_exponent_shift_down:nnn { #1 - 1 } {#3} { #2#4 } }
  }
%    \end{macrocode}
%   For shifting the exponent up, we can run out of decimal digits, at which
%   point filling is easy. Other than that a simple loop as we are picking
%   input off the front of the decimal part.
%    \begin{macrocode}
\cs_new:Npn \@@_exponent_shift_up:nnn #1#2#3
  {
    \int_compare:nNnTF {#1} = 0
      { \exp_not:n { {#2} {#3} } }
      {
        \tl_if_blank:nTF {#3}
          {
            {
              \exp_not:n {#2}
              \prg_replicate:nn { \int_abs:n {#1} } { 0 }
            }
            { }
         }
         { \@@_exponent_shift_up:nnw {#1} {#2} #3 \q_stop }
      }
  }
\cs_new:Npn \@@_exponent_shift_up:nnw #1 #2#3#4 \q_stop
  { \@@_exponent_shift_up:nnn { #1 + 1 } { #2 #3 } {#4} }
%    \end{macrocode}
%   Tidy up the exponent to put the sign in the right place.
%    \begin{macrocode}
\cs_new:Npn \@@_exponent_finalise:n #1
  {
    \int_compare:nNnTF {#1} > 0
      { + }
      { - }
      { \int_abs:n {#1} }
  }
%    \end{macrocode}
%   This could (and eventually will) be combined with the main function above:
%   that will need \texttt{e}-type expansion. The input has already been
%   normalised such that the integer part is in the range $1 \le n < 10$.
%   Thus there are only three cases to deal with, depending on the required
%   adjustment to the exponent.
%    \begin{macrocode}
\cs_new:Npn \@@_exponent_engineering_aux:nNnnnNn #1#2#3#4#5#6#7
  {
    \exp_not:n { {#1} #2 }
    \use:c
      {
        @@_exponent_engineering_
	     \int_compare:nNnTF {#6#7} < 0
	       {
	         \int_case:nnF { \int_mod:nn { #7 } { 3 } }
	           {
	             { 1 } { 2 }
	             { 2 } { 1 }
	           }
	           { 0 }
	       }
	       { \int_mod:nn {#7} { 3 } }
	       :nnnn
      }
        {#3} {#4} {#5} {#6#7}
  }
\cs_new:cpn { @@_exponent_engineering_0:nnnn } #1#2#3#4
  {
    \exp_not:n { {#1} {#2} {#3} }
    \@@_exponent_finalise:n {#4}
  }
\cs_new:cpn { @@_exponent_engineering_1:nnnn } #1#2#3#4
  {
    \tl_if_blank:nTF {#2}
      { { \exp_not:n { #1 0 } } { } }
      {
        { \exp_not:n {#1} \exp_not:o { \tl_head:w #2 \q_stop } }
        { \exp_not:f { \tl_tail:n {#2} } }
      }
    \exp_not:n { {#3} }
    \@@_exponent_finalise:n { #4 - 1 }
  }
\cs_new:cpn { @@_exponent_engineering_2:nnnn } #1#2#3#4
  {
    \tl_if_blank:nTF {#2}
      { { \exp_not:n { #1 00 } } { } }
      { \@@_exponent_engineering:nNw {#1} #2 \q_stop }
    \exp_not:n { {#3} }
    \@@_exponent_finalise:n { #4 - 2 }
  }
\cs_new:Npn \@@_exponent_engineering:nNw #1#2#3 \q_stop
  {
    \tl_if_blank:nTF {#3}
      { { \exp_not:n { #1#2 0 } } { } }
      {
        { \exp_not:n {#1#2} \exp_not:o { \tl_head:w #3 \q_stop } }
        { \exp_not:f { \tl_tail:n {#3} } }
      }
  }
%    \end{macrocode}
% \end{macro}
% \end{macro}
% \end{macro}
% \end{macro}
% \end{macro}
% \end{macro}
% \end{macro}
% \end{macro}
% \end{macro}
% \end{macro}
% \end{macro}
% \end{macro}
% \end{macro}
%
% \begin{macro}{\@@_digits:N}
% \begin{macro}[EXP]{\@@_digits:nNnnnNn}
% \begin{macro}[EXP]{\@@_digits:Nn}
%   Forcing a minimum number of digits in each part is quite easy. As
%   the common case is that we don't do anything here, there is no real need
%   to optimise the calculation (normally also numbers have only a few digits).
%    \begin{macrocode}
\cs_new_protected:Npn \@@_digits:N #1
  {
    \tl_set:Nx #1
      { \exp_after:wN \@@_digits:nNnnnNn #1 }
  }
\cs_new:Npn \@@_digits:nNnnnNn #1#2#3#4#5#6#7
  {
    \exp_not:n { {#1} #2 }
    {
      \@@_digits:Nn \l_@@_min_integer_int {#3}
      \exp_not:n {#3}
    }
    {
      \exp_not:n {#4}
      \@@_digits:Nn \l_@@_min_decimal_int {#4}
    }
    \exp_not:n { {#5} #6 {#7} }
  }
\cs_new:Npn \@@_digits:Nn #1#2
  {
    \int_compare:nNnT
      { #1 - \tl_count:n {#2} } > 0
      { \prg_replicate:nn { #1 - \tl_count:n {#2}  } { 0 } }
  }
%    \end{macrocode}
% \end{macro}
% \end{macro}
% \end{macro}
%
% \begin{macro}{\@@_round:N}
% \begin{macro}[EXP]{\@@_round_none:nNnnnNn}
%   Rounding is at the top level simple enough: fire off the expandable
%   set up which does the work.
%    \begin{macrocode}
\cs_new_protected:Npn \@@_round:N #1
  {
    \tl_set:Nx #1
      {
        \cs:w
          @@_round_ \l_@@_round_mode_tl :nNnnnNn
          \exp_after:wN
        \cs_end: #1
      }
  }
\cs_new:Npn \@@_round_none:nNnnnNn #1#2#3#4#5#6#7
  { \exp_not:n { {#1} #2 {#3} {#4} {#5} #6 {#7} } }
%    \end{macrocode}
% \end{macro}
% \end{macro}
%
% \begin{macro}[EXP]{\@@_round:nnn, \@@_round:fnn}
% \begin{macro}[EXP]
%   {
%     \@@_round_auxi:nnnN  ,
%     \@@_round_auxii:nnnN ,
%     \@@_round_auxiii:nnnN
%   }
%  \begin{macro}[EXP]{\@@_round_auxiv:nnN, \@@_round_auxv:nnN}
%  \begin{macro}[EXP]{\@@_round_auxvi:nN}
%  \begin{macro}[EXP]{\@@_round_auxvii:nnN, \@@_round_auxviii:nnN}
%  \begin{macro}[EXP]{\@@_round_final_integer:nnw, \@@_round_final_decimal:nnw}
%  \begin{macro}[EXP]{\@@_round_final:nn, \@@_round_final:fn}
%  \begin{macro}[EXP]{\@@_round_truncate:n, \@@_round_truncate_direct:n}
%  \begin{macro}[EXP]{\@@_round_truncate:nnN}
%   Actually doing the rounding needs us to work from the least significant
%   digit, so we start by reversing the input. We \emph{could} also drop
%   digits in this phase, but tracking everything would be horrible, so
%   we go slightly slower but clearer and split the steps. First we reverse
%   the decimal part, then the integer.
%    \begin{macrocode}
\cs_new:Npn \@@_round:nnn #1#2#3
  {
    \@@_round_auxi:nnnN {#1} {#2} { }
      #3 \q_recursion_tail \q_recursion_stop
  }
\cs_generate_variant:Nn \@@_round:nnn { f }
\cs_new:Npn \@@_round_auxi:nnnN #1#2#3#4
  {
    \quark_if_recursion_tail_stop_do:Nn #4
      {
        \@@_round_auxii:nnnN {#1} {#3} { } #2
          \q_recursion_tail \q_recursion_stop
      }
    \@@_round_auxi:nnnN {#1} {#2} {#4#3}
  }
\cs_new:Npn \@@_round_auxii:nnnN #1#2#3#4
  {
    \quark_if_recursion_tail_stop_do:Nn #4
      {
        \tl_if_blank:nTF {#2}
          {
            \@@_round_auxiv:nnnN {#1} { } { } #3
              \q_recursion_tail \q_recursion_stop
          }
          {
            \@@_round_auxiii:nnnN {#1} {#3} { } #2
              \q_recursion_tail \q_recursion_stop
          }
      }
    \@@_round_auxii:nnnN {#1} {#2} {#4#3}
  }
%    \end{macrocode}
%   We now have the input reversed plus how many digits we need to discard
%   (|#1|).  We have two functions, one which deals with the decimal part,
%   one of which deals with the integer. In the latter, we should never hit
%   the end before we've dropped all the digits: the fixed-zero is a
%   fall-back in case something weird happens. For the integer case, we need
%   to collect up zeros to pad the length back out correctly later.
%    \begin{macrocode}
\cs_new:Npn \@@_round_auxiii:nnnN #1#2#3#4
  {
    \quark_if_recursion_tail_stop_do:Nn #4
      {
        \@@_round_auxiv:nnnN {#1} { } {#3} #2
          \q_recursion_tail \q_recursion_stop
      }
    \int_compare:nNnTF {#1} > 0
      {
        \exp_args:Nf \@@_round_auxiii:nnnN
          { \int_eval:n { #1 - 1 } } {#2} { #4#3 }
      }
      { \@@_round_auxv:nnN {#3} {#2} #4 }
  }
\cs_new:Npn \@@_round_auxiv:nnnN #1#2#3#4
  {
    \quark_if_recursion_tail_stop_do:Nn #4
      { { 0 } { } }
    \int_compare:nNnTF {#1} > 0
      {
        \exp_args:Nf \@@_round_auxiv:nnnN
          { \int_eval:n { #1 - 1 } } { #2 0 } { #4#3 }
      }
      { \@@_round_auxvi:nnnN {#3} {#2} #4 }
  }
%    \end{macrocode}
%   The lead off to rounding proper needs to deal with the half-even rule:
%   it can only apply at this stage, when the \emph{discarded} value can
%   be exactly half.
%    \begin{macrocode}
\cs_new:Npn \@@_round_auxv:nnN #1#2#3
  {
    \quark_if_recursion_tail_stop_do:Nn #3
      {
        \@@_round_auxvi:nnN
          {#1} { } #2 \q_recursion_tail \q_recursion_stop
      }
    \bool_lazy_or:nnTF
      { \int_compare_p:nNn { 0 \tl_head:n {#1} } < 5 }
      {
        \bool_lazy_all_p:n
          {
            { \l_@@_round_half_even_bool }
            { \int_if_odd_p:n {#3} }
            { \@@_round_if_half_p:n {#1} }
          }
      }
      { \@@_round_final_decimal:nnw }
      { \@@_round_auxvii:nnN }
        {#2} { } #3
  }
\cs_new:Npn \@@_round_auxvi:nnnN #1#2#3
  {
    \quark_if_recursion_tail_stop_do:Nn #3
      { { 0 } { } }
    \bool_lazy_or:nnTF
      { \int_compare_p:nNn { \tl_head:n {#1} } < 5 }
      {
        \bool_lazy_all_p:n
          {
            { \l_@@_round_half_even_bool }
            { \int_if_odd_p:n {#3} }
            { \@@_round_if_half_p:n {#1} }
          }
      }
      { \@@_round_final_integer:nnw }
      { \@@_round_auxviii:nnN }
        { } {#2} #3
  }
%    \end{macrocode}
%   The main rounding routines. These are only every called when there is
%   rounding to do, so there is no need to carry a flag forward. Thus the
%   question to ask is simple: is the next value a $9$ or not (as that
%   continues the sequence). There is a general need to handle the case
%   where a zero is rounded up: that automatically means a need to trim
%   the other end.
%    \begin{macrocode}
\cs_new:Npn \@@_round_auxvii:nnN #1#2#3
  {
    \quark_if_recursion_tail_stop_do:Nn #3
      {
        \str_if_eq:nnTF {#1} { 0 }
          {
            { 1 }
            { \@@_round_trunctate:n {#2} }
          }
          {
            \@@_round_auxviii:nnN {#2} { } #1
              \q_recursion_tail \q_recursion_stop
          }
      }
    \int_compare:nNnTF {#3} = 9
      { \@@_round_auxvii:nnN {#1} { 0 #2 } }
      {
        \int_compare:nNnTF {#3} = 0
          {
            \@@_round_final_decimal:nnw
              {#1} { 1 \@@_round_trunctate:n {#2} }
          }
          {
            \@@_round_final:fn
              { \int_eval:n { #3 + 1 } }
              { \@@_round_final_decimal:nnw {#1} {#2} }
          }
      }
  }
\cs_new:Npn \@@_round_auxviii:nnN #1#2#3
  {
    \quark_if_recursion_tail_stop_do:Nn #3
      {
        \tl_if_blank:nTF {#1}
          {
            { 1 \@@_round_trunctate_direct:n {#2} 0 }
            { }
          }
          {
            { 1 #2 }
            { \@@_round_trunctate:n {#1} }
          }
      }
    \int_compare:nNnTF {#3} = 9
      { \@@_round_auxviii:nnN {#1} { 0 #2 } }
      {
        \@@_round_final:fn
          { \int_eval:n { #3 + 1 } }
          { \@@_round_final_integer:nnw {#1} {#2} }
      }
  }
%    \end{macrocode}
%   Tidying up means grabbing the remaining digits and undoing the reversal.
%    \begin{macrocode}
\cs_new:Npn \@@_round_final_decimal:nnw
  #1#2#3 \q_recursion_tail \q_recursion_stop
  {
    { \tl_reverse:n {#1} }
    { \tl_reverse:n {#3} #2 }
  }
\cs_new:Npn \@@_round_final_integer:nnw
  #1#2#3 \q_recursion_tail \q_recursion_stop
  {
    { \tl_reverse:n {#3} #2 }
    {#1}
  }
\cs_new:Npn \@@_round_final:nn #1#2
  { #2 #1 }
\cs_generate_variant:Nn \@@_round_final:nn { f }
%    \end{macrocode}
%   When we have rounded up to the next power of ten, we need to go back and
%   remove one more digit. That only happens when rounding to a number of
%   figures or when dealing with an integer part.
%    \begin{macrocode}
\cs_new:Npn \@@_round_trunctate:n #1
  {
    \str_if_eq:VnTF \l_@@_round_mode_tl { figures }
      { \@@_round_trunctate_direct:n {#1} }
      {#1}
  }
\cs_new:Npn \@@_round_trunctate_direct:n #1
  {
    \@@_round_truncate:nnN { } { }
      #1 \q_recursion_tail \q_recursion_stop
  }
\cs_new:Npn \@@_round_truncate:nnN #1#2#3
  {
    \quark_if_recursion_tail_stop_do:Nn #3 { #1 }
    \@@_round_truncate:nnN {#1#2} {#3}   
  }
%    \end{macrocode}
% \end{macro}
% \end{macro}
% \end{macro}
% \end{macro}
% \end{macro}
% \end{macro}
% \end{macro}
% \end{macro}
% \end{macro}
%
% \begin{macro}[EXP]{\@@_round_if_half_p:n}
% \begin{macro}[EXP]{\@@_round_if_half:N}
%   A simple test for a valuing being exactly half: we can only test
%   digit-by-digit as there is no limit on the size of the value given.
%    \begin{macrocode}
\prg_new_conditional:Npnn \@@_round_if_half:n #1 { p }
  {
    \int_compare:nNnTF { \tl_head:n { #1 0 } } = 5
      {
        \exp_after:wN \@@_round_if_half:N \use_none:n #1 0
          \q_recursion_tail \q_recursion_stop
      }
      { \prg_return_false: }
  }
\cs_new:Npn \@@_round_if_half:N #1
  {
    \quark_if_recursion_tail_stop_do:Nn #1
      { \prg_return_true: }
    \int_compare:nNnTF {#1} = 0
      { \@@_round_if_half:N }
      { \use_i_delimit_by_q_recursion_stop:nw { \prg_return_false: } }
  }
%    \end{macrocode}
% \end{macro}
% \end{macro}
%
% \begin{macro}[EXP]{\@@_round_pad:nnn}
%   The case where we are short of digits is easy enough to handle:
%   generate zeros to pad it out.
%    \begin{macrocode}
\cs_new:Npn \@@_round_pad:nnn #1#2#3
  {
    {#2}
    {
      #3
      \bool_if:NT \l_@@_round_pad_bool
        { \prg_replicate:nn {#1} { 0 } } 
    }
  }
%    \end{macrocode}
% \end{macro}
%
% \begin{macro}[EXP]{\@@_round_figures:nNnnnNn}
% \begin{macro}[EXP]{\@@_round_figures_count:nnN}
% \begin{macro}[EXP]{\@@_round_figures_count:nnnN}
%   Rounding to a fixed number of significant figures starts by checking that
%   there is no uncertainty, and that the number of figures requested is
%   positive: if not, the result is always fixed at zero.
%    \begin{macrocode}
\cs_new:Npn \@@_round_figures:nNnnnNn #1#2#3#4#5#6#7
  {
    \tl_if_blank:nTF {#5}
      {
        \int_compare:nNnTF \l_@@_round_precision_int > 0
          {
            \exp_not:n { {#1} #2 }
            \@@_round_figures_count:nnN {#3} {#4} #3#4
              \q_recursion_tail \q_recursion_stop
            \exp_not:n { { } #6 {#7} }
          }
          { { } + { 0 } { } { } + { 0 } }
      }
      { \exp_not:n { {#1} #2 {#3} {#4} {#5} #6 {#7} } }
  }
%    \end{macrocode}
%   The first real step is to count up the number of significant figures.
%   The only tricky issue here is dealing with leading zeros.
%    \begin{macrocode}
\cs_new:Npn \@@_round_figures_count:nnN #1#2#3
  {
    \quark_if_recursion_tail_stop_do:Nn #3
      { { } + { 0 } { } { } + { 0 } }
    \int_compare:nNnTF {#3} = 0
      { \@@_round_figures_count:nnN {#1} {#2} }
      { \@@_round_figures_count:nnnN { 1 } {#1} {#2} }
  }
\cs_new:Npn \@@_round_figures_count:nnnN #1#2#3#4
  {
    \quark_if_recursion_tail_stop_do:Nn #4
      {
        \int_compare:nNnTF {#1} > \l_@@_round_precision_int
          {
            \@@_round:fnn
              { \int_eval:n { #1 - \l_@@_round_precision_int } }
              {#2} {#3}
          }
          {
            \@@_round_pad:nnn
              { \l_@@_round_precision_int - (#1) } {#2} {#3}
          }
      }
    \exp_args:Nf \@@_round_figures_count:nnnN
      { \int_eval:n { #1 + 1 } } {#2} {#3}
  }
%    \end{macrocode}
% \end{macro}
% \end{macro}
% \end{macro}
%
% \begin{macro}[EXP]{\@@_round_places:nNnnnNn}
% \begin{macro}[EXP]{\@@_round_places_decimal:nn, \@@_round_places_integer:nn}
%   The first step when rounding to a fixed number of places is to establish
%   if this is in the decimal or integer parts. The two require different
%   calculations for how many digits to drop from the input.
%    \begin{macrocode}
\cs_new:Npn \@@_round_places:nNnnnNn #1#2#3#4#5#6#7
  {
    \tl_if_blank:nTF {#5}
      {
        \exp_not:n { {#1} #2 }
        \int_compare:nNnTF \l_@@_round_precision_int > 0
          { \@@_round_places_decimal:nn }
          { \@@_round_places_integer:nn }
            {#3} {#4}
        { }
        \exp_not:n { #6 {#7} }
      }
      { \exp_not:n { {#1} #2 {#3} {#4} {#5} #6 {#7} } }
  }
\cs_new:Npn \@@_round_places_decimal:nn #1#2
  {
    \int_compare:nNnTF
      { \l_@@_round_precision_int - 0 \tl_count:n {#2} } > 0
      {
        \@@_round_pad:nnn
          { \l_@@_round_precision_int - 0 \tl_count:n {#2} }
          {#1} {#2}
      }
      {
        \@@_round:fnn
           {
             \int_eval:n
               { 0 \tl_count:n {#2} - \l_@@_round_precision_int }
           }
           {#1} {#2}
      }
  }
\cs_new:Npn \@@_round_places_integer:nn #1#2
  {
    \@@_round:fnn
       {
         \int_eval:n
           { 0 \tl_count:n {#2} - \l_@@_round_precision_int }
       }
       {#1} {#2}
  }
%    \end{macrocode}
% \end{macro}
% \end{macro}
%
% \begin{macro}[EXP]{\@@_round_uncertainty:nNnnnNn}
%    \begin{macrocode}
\cs_new:Npn \@@_round_uncertainty:nNnnnNn #1#2#3#4#5#6#7
  {
    \tl_if_blank:nTF {#5}
      { \exp_not:n { {#1} #2 {#3} {#4} { } #6 {#7} } }
      { }
  }
%    \end{macrocode}
% \end{macro}
%
% \begin{macro}{\@@_uncertainty:N}
% \begin{macro}[EXP]{\@@_uncertainty:nNnnnNn}
%   Simple stripping of the uncertainty.
%    \begin{macrocode}
\cs_new_protected:Npn \@@_uncertainty:N #1
  {
    \bool_if:NT \l_@@_drop_uncertainty_bool
      {
        \tl_set:Nx #1
          { \exp_after:wN \@@_uncertainty:nNnnnNn #1 }
      }
  }
\cs_new:Npn \@@_uncertainty:nNnnnNn #1#2#3#4#5#6#7
  { \exp_not:n { {#1} #2 {#3} {#4} { } #6 {#7} } }
%    \end{macrocode}
% \end{macro}
% \end{macro}
%
% \begin{macro}{\@@_zero_decimal:N}
% \begin{macro}[EXP]{\@@_zero_decimal:nNnnnNn}
%   Simple stripping of the decimal part if zero.
%    \begin{macrocode}
\cs_new_protected:Npn \@@_zero_decimal:N #1
  {
    \bool_if:NT \l_@@_drop_zero_decimal_bool
      {
        \tl_set:Nx #1
          { \exp_after:wN \@@_zero_decimal:nNnnnNn #1 }
      }
  }
\cs_new:Npn \@@_zero_decimal:nNnnnNn #1#2#3#4#5#6#7
  {
    \exp_not:n { {#1} #2 {#3} }
    \str_if_eq_x:nnTF
      { \exp_not:n {#4} }
      { \prg_replicate:nn { \tl_count:n {#4} } { 0 } }
      { { } }
      { \exp_not:n { {#4} } }
    \exp_not:n { {#5}  #6 {#7} }
  }
%    \end{macrocode}
% \end{macro}
% \end{macro}
%
% \subsection{Formatting parsed numbers}
%
% \begin{variable}
%   {
%      \l_@@_bracket_negative_bool  ,
%      \l_@@_bracket_close_tl       ,
%      \l_@@_explicit_plus_bool     ,
%      \l_@@_exponent_base_tl       ,
%      \l_@@_exponent_product_tl    ,
%      \l_@@_group_decimal_bool     ,
%      \l_@@_group_integer_bool     ,
%      \l_@@_group_minimum_int      ,
%      \l_@@_group_separator_tl     ,
%      \l_@@_negative_color_tl      ,
%      \l_@@_bracket_open_tl        ,
%      \l_@@_output_uncert_close_tl ,
%      \l_@@_output_complex_tl      ,
%      \l_@@_output_decimal_tl      ,
%      \l_@@_output_uncert_open_tl  ,
%      \l_@@_uncert_separate_bool   ,
%      \l_@@_uncert_separator_tl    ,
%      \l_@@_tight_bool             ,
%      \l_@@_unity_mantissa_bool    ,
%      \l_@@_zero_exponent_bool
%   }
%   Keys producing tokens in the output.
%    \begin{macrocode}
\keys_define:nn { siunitx }
  {
    bracket-negative .bool_set:N =
      \l_@@_bracket_negative_bool ,
    exponent-base .tl_set:N =
      \l_@@_exponent_base_tl ,
    exponent-product .tl_set:N =
      \l_@@_exponent_product_tl ,
    group-digits .choice: ,
    group-digits / all .code:n =
      {
        \bool_set_true:N \l_@@_group_decimal_bool
        \bool_set_true:N \l_@@_group_integer_bool
      } ,
    group-digits / decimal .code:n =
      {
        \bool_set_true:N  \l_@@_group_decimal_bool
        \bool_set_false:N \l_@@_group_integer_bool
      } ,
    group-digits / integer .code:n =
      {
        \bool_set_false:N \l_@@_group_decimal_bool
        \bool_set_true:N  \l_@@_group_integer_bool
      } ,
    group-digits / none .code:n =
      {
        \bool_set_false:N \l_@@_group_decimal_bool
        \bool_set_false:N \l_@@_group_integer_bool
      } ,
    group-digits .default:n  = all ,
    group-minimum-digits .int_set:N  =
      \l_@@_group_minimum_int ,
    group-separator .tl_set:N =
      \l_@@_group_separator_tl ,
    negative-color .tl_set:N =
    \l_@@_negative_color_tl ,
    number-close-bracket .tl_set:N =
      \l_@@_bracket_close_tl ,
    number-open-bracket .tl_set:N =
      \l_@@_bracket_open_tl ,
    output-close-uncertainty .tl_set:N =
      \l_@@_output_uncert_close_tl ,
    output-complex-root .tl_set:N =
      \l_@@_output_complex_tl ,
    output-decimal-marker .tl_set:N =
      \l_@@_output_decimal_tl ,
    output-open-uncertainty .tl_set:N =
      \l_@@_output_uncert_open_tl ,
    print-explicit-plus .bool_set:N =
      \l_@@_explicit_plus_bool ,
    print-unity-mantissa .bool_set:N =
      \l_@@_unity_mantissa_bool ,
    print-zero-exponent .bool_set:N =
      \l_@@_zero_exponent_bool ,
    separate-uncertainty .bool_set:N =
      \l_@@_uncert_separate_bool ,
    uncertainty-separator .tl_set:N =
      \l_@@_uncert_separator_tl ,
    tight-spacing .bool_set:N =
      \l_@@_tight_bool
  }
\bool_new:N \l_@@_group_decimal_bool
\bool_new:N \l_@@_group_integer_bool
%    \end{macrocode}
% \end{variable}
%
% \begin{macro}{\@@_format:}
%    \begin{macrocode}
\cs_new_protected:Npn \@@_format:
  {
    \tl_set:Nx \l_@@_formatted_tl
      {
        \tl_if_empty:NTF \l_@@_real_tl
          {
            \tl_if_empty:NF \l_@@_imaginary_tl
              { \@@_format:N \l_@@_imaginary_tl }
          }
          {
            \tl_if_empty:NTF \l_@@_imaginary_tl
              { \@@_format:N \l_@@_real_tl }
              { ??? }
          }
      }
  }
%    \end{macrocode}
% \end{macro}
%
% \begin{macro}[EXP]{\@@_format:N}
% \begin{macro}[EXP]{\@@_format:nNnnnNn}
% \begin{macro}[EXP]{\@@_format_comparator:n}
% \begin{macro}[EXP]{\@@_format_sign:N, \@@_format_sign_aux:N}
% \begin{macro}[EXP]
%   {\@@_format_sign_color:w, \@@_format_sign_brackets:w}
% \begin{macro}[EXP]{\@@_format_integer:nnn}
% \begin{macro}[EXP]{\@@_format_decimal:n, \@@_format_decimal:f}
% \begin{macro}[EXP]{\@@_format_digits:nn}
% \begin{macro}[EXP]{\@@_format_integer_aux:n}
% \begin{macro}[EXP]
%   {
%     \@@_format_integer_aux_0:n,
%     \@@_format_integer_aux_1:n,
%     \@@_format_integer_aux_2:n
%   }
% \begin{macro}[EXP]{\@@_format_decimal_aux:n}
% \begin{macro}[EXP]{\@@_format_decimal_loop:NNNN}
% \begin{macro}[EXP]{\@@_format_integer_first:nnNN}
% \begin{macro}[EXP]{\@@_format_integer_loop:NNNN}
% \begin{macro}[EXP]{\@@_format_uncertainty:nn}
% \begin{macro}[EXP]{\@@_format_uncertainty_unaligned:}
% \begin{macro}[EXP]
%   {\@@_format_uncertainty_aux:nn, \@@_format_uncertainty_aux:fn}
% \begin{macro}[EXP]
%   {\@@_format_uncertainty:nnw, \@@_format_uncertainty:fnw}
% \begin{macro}[EXP]{\@@_format_uncertainty:nw}
% \begin{macro}[EXP]{\@@_format_exponent:Nnn}
% \begin{macro}[EXP]{\@@_format_end:}
%   The approach to formatting a single number is to split into
%   the constituent parts. All of the parts are assembled including
%   inserting tabular alignment markers (which may be empty) for each
%   separate unit.
%    \begin{macrocode}
\cs_new:Npn \@@_format:N #1
  { \exp_after:wN \@@_format:nNnnnNn #1 }
\cs_new:Npn \@@_format:nNnnnNn #1#2#3#4#5#6#7
  {
    \@@_format_comparator:n {#1}
    \@@_format_sign:N #2
    \@@_format_integer:nnn {#3} {#4} {#7}
    \@@_format_decimal:n {#4}
    \@@_format_uncertainty:nn {#5} {#4}
    \@@_format_exponent:Nnn #6 {#7} { #3 . #4 }
    \@@_format_end:
  }
%    \end{macrocode}
%   To get the spacing correct this needs to be an ordinary math character.
%    \begin{macrocode}
\cs_new:Npn \@@_format_comparator:n #1
  {
    \tl_if_blank:nF {#1}
      { \exp_not:n { \mathord {#1} } }
    \exp_not:V \l_@@_tab_tl
  }
%    \end{macrocode}
%   Formatting signs has to deal with some additional formatting requirements
%   for negative numbers. Both making such numbers a fixed color and bracketing
%   them needs some rearrangement of the order of tokens, which is set up in
%   the main formatting macro by the dedicated do-nothing end function.
%    \begin{macrocode}
\cs_new:Npn \@@_format_sign:N #1
  {
    \str_if_eq:nnTF {#1} { + }
      {
        \bool_if:NT \l_@@_explicit_plus_bool
          { \@@_format_sign_aux:N #1 }
      }
      {
        \str_if_eq:nnTF {#1} { - }
          {
            \tl_if_empty:NF \l_@@_negative_color_tl
              { \@@_format_sign_color:w }
            \bool_if:NTF \l_@@_bracket_negative_bool
              { \@@_format_sign_brackets:w }
              { \@@_format_sign_aux:N #1 }
          }
          { \@@_format_sign_aux:N #1 }
      }
  }
\cs_new:Npn \@@_format_sign_aux:N #1
  {
    \bool_if:NTF \l_@@_tight_bool
      { \exp_not:n { \mathord {#1} } }
      { \exp_not:n {#1} }
  }
\cs_new:Npn
  \@@_format_sign_color:w #1 \@@_format_end:
  {
    \exp_not:N \textcolor { \exp_not:V \l_@@_negative_color_tl }
      {
        #1
        \@@_format_end:
      }
  }
\cs_new:Npn
  \@@_format_sign_brackets:w #1 \@@_format_end:
  {
    \exp_not:V \l_@@_bracket_open_tl
    #1
    \exp_not:V \l_@@_bracket_close_tl
    \@@_format_end:
  }
%    \end{macrocode}
%   Digit formatting leads off with separate functions to allow for a few
%   \enquote{up front} items before using a common set of tests for some common
%   cases. The code then splits again as the two types of grouping need
%   different strategies.
%    \begin{macrocode}
\cs_new:Npn \@@_format_integer:nnn #1#2#3
  {
    \bool_lazy_all:nF
      {
        { \str_if_eq_p:nn {#1} { 1 } }
        { \tl_if_blank_p:n {#2} }
        { ! \str_if_eq_p:nn {#3} { 0 } }
        { ! \l_@@_unity_mantissa_bool }
      }
      { \@@_format_digits:nn { integer } {#1} }
  }
\cs_new:Npn \@@_format_decimal:n #1
  {
    \exp_not:V \l_@@_tab_tl
    \tl_if_blank:nF {#1}
      {
        \str_if_eq:VnTF \l_@@_output_decimal_tl { , }
          { \exp_not:N \mathord }
          { \use:n }
            { \exp_not:V \l_@@_output_decimal_tl }
      }
    \exp_not:V \l_@@_tab_tl
    \@@_format_digits:nn { decimal } {#1}
  }
\cs_generate_variant:Nn \@@_format_decimal:n { f }
\cs_new:Npn \@@_format_digits:nn #1#2
  {
    \bool_if:cTF { l_@@_group_ #1 _ bool }
      {
        \int_compare:nNnTF
          { \tl_count:n {#2} } < \l_@@_group_minimum_int
          { \exp_not:n {#2} }
          { \use:c { @@_format_ #1 _aux:n } {#2} }
      }
      { \exp_not:n {#2} }
  }
%    \end{macrocode}
%   For integers, we need to know how many digits there are to allow for the
%   correct insertion of separators. That is done using a two-part set up such
%   that there is no separator on the first pass.
%    \begin{macrocode}
\cs_new:Npn \@@_format_integer_aux:n #1
  {
     \use:c
       {
         @@_format_integer_aux_
         \int_eval:n { \int_mod:nn { \tl_count:n {#1} } { 3 } }
         :n
       } {#1}
  }
\cs_new:cpn { @@_format_integer_aux_0:n } #1
  { \@@_format_integer_first:nnNN #1 \q_nil }
\cs_new:cpn { @@_format_integer_aux_1:n } #1
  { \@@_format_integer_first:nnNN { } { } #1 \q_nil }
\cs_new:cpn { @@_format_integer_aux_2:n } #1
  { \@@_format_integer_first:nnNN { } #1 \q_nil }
\cs_new:Npn \@@_format_integer_first:nnNN #1#2#3#4
  {
    \exp_not:n {#1#2#3}
    \quark_if_nil:NF #4
      { \@@_format_integer_loop:NNNN #4 }
  }
\cs_new:Npn \@@_format_integer_loop:NNNN #1#2#3#4
  {
    \exp_not:V \l_@@_group_separator_tl
    \exp_not:n {#1#2#3}
    \quark_if_nil:NF #4
      { \@@_format_integer_loop:NNNN #4 }
  }
%    \end{macrocode}
%   For decimals, no need to do any counting, just loop using enough markers to
%   find the end of the list. By passing the decimal marker, it is possible not
%   to have to use a check on the content of the rest of the number. The
%   |\use_none:n(n)| mop up the remaining |\q_nil| tokens.
%    \begin{macrocode}
\cs_new:Npn \@@_format_decimal_aux:n #1
  {
    \@@_format_decimal_loop:NNNN \c_empty_tl
      #1 \q_nil \q_nil \q_nil
  }
\cs_new:Npn \@@_format_decimal_loop:NNNN #1#2#3#4
  {
    \quark_if_nil:NF #2
      {
        \exp_not:V #1
        \exp_not:n {#2}
        \quark_if_nil:NTF #3
          { \use_none:n }
          {
            \exp_not:n {#3}
            \quark_if_nil:NTF #4
              { \use_none:nn }
              {
                \exp_not:n {#4}
                \@@_format_decimal_loop:NNNN
                  \l_@@_group_separator_tl
              }
          }
      }
  }
%    \end{macrocode}
%   Uncertainties which are directly attached are easy to deal with. For those
%   that are separated, the first step is to find if they are entirely
%   contained within the decimal part, and to pad if they are. For the case
%   where the boundary is crossed to the integer part, the correct number of
%   digit tokens need to be removed from the start of the uncertainty and
%   the split result sent to the appropriate auxiliaries.
%    \begin{macrocode}
\cs_new:Npn \@@_format_uncertainty:nn #1#2
  {
    \tl_if_blank:nTF {#1}
      { \@@_format_uncertainty_unaligned: }
      {
        \bool_if:NTF \l_@@_uncert_separate_bool
          {
            \exp_not:V \l_@@_tab_tl
            \@@_format_sign_aux:N \pm
            \exp_not:V \l_@@_tab_tl
            \@@_format_uncertainty_aux:fn
              { \int_eval:n { \tl_count:n {#1} - \tl_count:n {#2} } }
              {#1}
          }
          {
            \exp_not:V \l_@@_uncert_separator_tl
            \exp_not:V \l_@@_output_uncert_open_tl
            \exp_not:n {#1}
            \exp_not:V \l_@@_output_uncert_close_tl
            \@@_format_uncertainty_unaligned:
          }
      }
  }
\cs_new:Npn \@@_format_uncertainty_unaligned:
  {
    \exp_not:V \l_@@_tab_tl
    \exp_not:V \l_@@_tab_tl
    \exp_not:V \l_@@_tab_tl
    \exp_not:V \l_@@_tab_tl
  }
\cs_new:Npn \@@_format_uncertainty_aux:nn #1#2
  {
    \int_compare:nNnTF {#1} > 0
      {
        \@@_format_uncertainty_aux:fnw
          { \int_eval:n { #1 - 1 } }
          { }
          #2 \q_nil
      }
      {
        0
        \@@_format_decimal:f
          {
            \prg_replicate:nn { \int_abs:n {#1} } { 0 }
            #2
          }
      }
  }
\cs_generate_variant:Nn \@@_format_uncertainty_aux:nn { f }
\cs_new:Npn \@@_format_uncertainty_aux:nnw #1#2#3
  {
    \quark_if_nil:NF #3
      {
        \int_compare:nNnTF {#1} = 0
          { \@@_format_uncertainty_aux:nw {#2#3} }
          {
            \@@_format_uncertainty_aux:fnw
              { \int_eval:n { #1 - 1 } }
              {#2#3}
          }
      }
  }
\cs_generate_variant:Nn \@@_format_uncertainty_aux:nnw { f }
\cs_new:Npn \@@_format_uncertainty_aux:nw #1#2 \q_nil
  {
    \@@_format_digits:nn { integer } {#1}
    \@@_format_decimal:n {#2}
  }
%    \end{macrocode}
%   Setting the exponent part requires some information about the mantissa:
%   was it there or not. This means that whilst only the sign and value for
%   the exponent are typeset here, there is a need to also have access to the
%   combined mantissa part (with a decimal marker). The rest of the work is
%   about picking up the various options and getting the combinations right.
%   For signs, the auxiliary from the main sign routine can be used, but not
%   the main function: negative exponents don't have special handling.
%    \begin{macrocode}
\cs_new:Npn \@@_format_exponent:Nnn #1#2#3
  {
    \exp_not:V \l_@@_tab_tl
    \bool_lazy_or:nnTF
      { \l_@@_zero_exponent_bool }
      { ! \str_if_eq_p:nn {#2} { 0 } }
      {
        \bool_lazy_and:nnTF
          { \str_if_eq_p:nn {#3} { 1. } }
          { ! \l_@@_unity_mantissa_bool }
          { \exp_not:V \l_@@_tab_tl }
          {
            \bool_if:NTF \l_@@_tight_bool
              {
                \exp_not:N \mathord
                  { \exp_not:V \l_@@_exponent_product_tl }
              }
              { \exp_not:V \l_@@_exponent_product_tl }
            \exp_not:V \l_@@_tab_tl
          }
        \exp_not:V \l_@@_exponent_base_tl
        ^
          {
            \bool_lazy_or:nnT
              { \l_@@_explicit_plus_bool }
              { ! \str_if_eq_p:nn {#1} { + } }
              { \@@_format_sign_aux:N #1 }
            \@@_format_digits:nn { integer } {#2}
          }
      }
      { \exp_not:V \l_@@_tab_tl }
  }
%    \end{macrocode}
%   A do-nothing marker used to allow shuffling of the output and so expandable
%   operations for formatting.
%    \begin{macrocode}
\cs_new:Npn \@@_format_end: { }
%    \end{macrocode}
% \end{macro}
% \end{macro}
% \end{macro}
% \end{macro}
% \end{macro}
% \end{macro}
% \end{macro}
% \end{macro}
% \end{macro}
% \end{macro}
% \end{macro}
% \end{macro}
% \end{macro}
% \end{macro}
% \end{macro}
% \end{macro}
% \end{macro}
% \end{macro}
% \end{macro}
% \end{macro}
% \end{macro}
%
% \subsection{Miscellaneous tools}
%
% \begin{variable}{\l_@@_valid_tl}
%   The list of valid tokens.
%    \begin{macrocode}
\tl_new:N \l_@@_valid_tl
%    \end{macrocode}
% \end{variable}
%
% \begin{macro}[TF]{\siunitx_if_number:n}
%   Test if an entire number is valid: this means parsing the number but not
%   returning anything.
%    \begin{macrocode}
\prg_new_protected_conditional:Npnn \siunitx_if_number:n #1
  { T , F , TF }
  {
    \group_begin:
      \bool_set_true:N \l_@@_validate_bool
      \@@_parse:n {#1}
      \bool_lazy_and:nnTF
        { \tl_if_empty_p:N \l_@@_real_tl }
        { \tl_if_empty_p:N \l_@@_imaginary_tl }
        {
          \group_end:
          \prg_return_false:
        }
        {
          \group_end:
          \prg_return_true:
        }
  }
%    \end{macrocode}
% \end{macro}
%
% \begin{macro}[TF]{\siunitx_if_number_token:N}
%   A simple conditional to answer the question of whether a specific token is
%   possibly valid in a number.
%    \begin{macrocode}
\prg_new_protected_conditional:Npnn \siunitx_if_number_token:N #1
  { T , F , TF }
  {
    \tl_set:Nx \l_@@_valid_tl
      {
        \exp_not:V \l_@@_input_uncert_close_tl
        \exp_not:V \l_@@_input_complex_tl
        \exp_not:V \l_@@_input_comparator_tl
        \exp_not:V \l_@@_input_decimal_tl
        \exp_not:V \l_@@_input_digit_tl
        \exp_not:V \l_@@_input_exponent_tl
        \exp_not:V \l_@@_input_ignore_tl
        \exp_not:V \l_@@_input_uncert_open_tl
        \exp_not:V \l_@@_input_sign_tl
        \exp_not:V \l_@@_input_uncert_sign_tl
      }
    \tl_if_in:VnTF \l_@@_valid_tl {#1}
      { \prg_return_true: }
      { \prg_return_false: }
  }
%    \end{macrocode}
% \end{macro}
%
% \subsection{Messages}
%
%    \begin{macrocode}
\msg_new:nnnn { siunitx } { invalid-number }
  { Invalid~number~'#1'. }
  {
    The~input~'#1'~could~not~be~parsed~as~a~number~following~the~
    format~defined~in~module~documentation.
  }
%    \end{macrocode}
%
% \subsection{Standard settings for module options}
%
% Some of these follow naturally from the point of definition
% (\foreign{e.g.}~boolean variables are always |false| to begin with),
% but for clarity everything is set here.
%    \begin{macrocode}
\keys_set:nn { siunitx }
  {
    bracket-negative         = false                                  ,
    drop-uncertainty         = false                                  ,
    drop-zero-decimal        = false                                  ,
    evaluate-expression      = false                                  ,
    exponent-base            = 10                                     ,
    exponent-mode            = input                                  ,
    exponent-product         = \times                                 ,
    expression               = #1                                     ,
    fixed-exponent           = 0                                      ,
    group-digits             = all                                    ,
    group-minimum-digits     = 4                                      ,
    group-separator          = \,                                     , % (
    input-close-uncertainty  = )                                      ,
    input-complex-roots      = ij                                     ,
    input-comparators        = { <=>\approx\ge\geq\gg\le\leq\ll\sim } ,
    input-decimal-markers    = { ., }                                 ,
    input-digits             = 0123456789                             ,
    input-exponent-markers   = dDeE                                   ,
    input-ignore             = \,                                     ,
    input-open-uncertainty   = (                                      , % )
    input-signs              = +-\mp\pm                               ,
    input-uncertainty-signs  = \pm                                    ,
    minimum-decimal-digits   = 0                                      ,
    minimum-integer-digits   = 0                                      ,
    negative-color           =                                        , % (
    number-close-bracket     = )                                      ,
    number-open-bracket      = (                                      , % ) (
    output-close-uncertainty = )                                      ,
    output-complex-root      = \mathrm { i }                          ,
    output-decimal-marker    = .                                      ,
    output-open-uncertainty  = (                                      , % )
    parse-numbers            = true                                   ,
    print-explicit-plus      = false                                  ,
    print-unity-mantissa     = false                                  ,
    print-zero-exponent      = false                                  ,
    round-half               = up                                     ,
    round-minimum            = 0                                      ,
    round-mode               = none                                   ,
    round-pad                = true                                   ,
    round-precision          = 2                                      ,
    separate-uncertainty     = false                                  ,
    uncertainty-separator    =                                        ,
    tight-spacing            = false
  }
%    \end{macrocode}
%
%    \begin{macrocode}
%</package>
%    \end{macrocode}
%
% \end{implementation}
%
% \PrintIndex

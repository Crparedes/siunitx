% \iffalse meta-comment
%
% File: siunitx-complex.dtx Copyright (C) 2021 Joseph Wright
%
% It may be distributed and/or modified under the conditions of the
% LaTeX Project Public License (LPPL), either version 1.3c of this
% license or (at your option) any later version.  The latest version
% of this license is in the file
%
%    https://www.latex-project.org/lppl.txt
%
% This file is part of the "siunitx bundle" (The Work in LPPL)
% and all files in that bundle must be distributed together.
%
% The released version of this bundle is available from CTAN.
%
% -----------------------------------------------------------------------
%
% The development version of the bundle can be found at
%
%    https://github.com/josephwright/siunitx
%
% for those people who are interested.
%
% -----------------------------------------------------------------------
%
%<*driver>
\documentclass{l3doc}
\ProvideDocumentCommand\foreign{m}{\textit{#1}}
% The next line is needed so that \GetFileInfo will be able to pick up
% version data
\usepackage{siunitx}
\begin{document}
  \DocInput{\jobname.dtx}
\end{document}
%</driver>
% \fi
%
% \GetFileInfo{siunitx.sty}
%
% \title{^^A
%   \pkg{siunitx-complex} -- Complex numbers^^A
%   \thanks{This file describes \fileversion,
%     last revised \filedate.}^^A
% }
%
% \author{^^A
%  Joseph Wright^^A
%  \thanks{^^A
%    E-mail:
%    \href{mailto:joseph.wright@morningstar2.co.uk}
%      {joseph.wright@morningstar2.co.uk}^^A
%   }^^A
% }
%
% \date{Released \filedate}
%
% \maketitle
%
% \begin{documentation}
%
% This submodule is concerned with formatting complex numbers. It augments the
% standard functions \cs{siunitx_number_format:nN} and \cs{siunitx_quantity:nn}
% by allowing parsing of numbers with a complex part. There are no additional
% assumptions concerning \LaTeXe{} commands in the submodule beyond those in the
% core number and unit submodules.
%
% \begin{function}{\siunitx_complex_number:n}
%   \begin{syntax}
%     \cs{siunitx_complex_number:n} \Arg{number}
%   \end{syntax}
%   Parses the \meta{number} and splits into real and complex parts, which are
%   then formatted as described for \cs{siunitx_number_format:nN}. The results
%   are combined and printed using the standard functions in the module.
% \end{function}
%
% \begin{function}{\siunitx_complex_quantity:nn}
%   \begin{syntax}
%     \cs{siunitx_complex_quantity:n} \Arg{number} \Arg{units}
%   \end{syntax}
%   Parses the \meta{number} and splits into real and complex parts, which are
%   then formatted as described for \cs{siunitx_quantity:nn}. The results
%   are combined and printed using the standard functions in the module.
% \end{function}
%
% \begin{function}{complex-root-position}
%   \begin{syntax}
%     |complex-root-position| = |after-number|\verb"|"|before-number|
%   \end{syntax}
%   Choice which determines where the complex root symbol is printed relative
%   to the numbers. The standard setting is |after-number|.
% \end{function}
%
% \begin{function}{input-complex-root}
%   \begin{syntax}
%     |input-complex-root| = \meta{tokens}
%   \end{syntax}
%   The token(s) considered as complexes roots for number parsing.
%   The standard setting is |ij|.
% \end{function}
%
% \begin{function}{output-complex-root}
%   \begin{syntax}
%     |output-complex-root| = \meta{tokens}
%   \end{syntax}
%   The token(s) used to show the complex root in output. The standard setting
%   is |\mathrm{i}|.
% \end{function}
%
% \end{documentation}
%
% \begin{implementation}
%
% Start the \pkg{DocStrip} guards.
%    \begin{macrocode}
%<*package>
%    \end{macrocode}
%
% \section{\pkg{siunitx-complex} implementation}
%
% Identify the internal prefix (\LaTeX3 \pkg{DocStrip} convention): only
% internal material in this \emph{submodule} should be used directly.
%    \begin{macrocode}
%<@@=siunitx_complex>
%    \end{macrocode}
%
% \subsection{General setup}
%
% \begin{variable}{\l_@@_tmp_tl}
%    \begin{macrocode}
\tl_new:N \l_@@_tmp_tl
%    \end{macrocode}
% \end{variable}
%
% \begin{variable}{\l_@@_input_tl}
%   The numerical input exactly as given by the user.
%    \begin{macrocode}
\tl_new:N \l_@@_input_tl
%    \end{macrocode}
% \end{variable}
%
% \begin{variable}{\l_@@_comparator_tl}
%   A comparator, if found, is held here.
%    \begin{macrocode}
\tl_new:N \l_@@_comparator_tl
%    \end{macrocode}
% \end{variable}
%
% \begin{variable}{\l_@@_exp_tl}
%   The exponent part of a parsed number.
%    \begin{macrocode}
\tl_new:N \l_@@_exp_tl
%    \end{macrocode}
% \end{variable}
%
% \begin{variable}{\l_@@_real_tl, \l_@@_img_tl}
%   The real and imaginary parts of the number, respectively.
%    \begin{macrocode}
\tl_new:N \l_@@_real_tl
\tl_new:N \l_@@_img_tl
%    \end{macrocode}
% \end{variable}
%
% \begin{variable}{\l_@@_join_tl, \l_@@_sign_tl}
%   Staging posts for a joining and leading sign, respectively.
%    \begin{macrocode}
\tl_new:N \l_@@_join_tl
\tl_new:N \l_@@_sign_tl
%    \end{macrocode}
% \end{variable}
%
% \begin{variable}{\l_@@_input_root_tl, \l_@@_output_root_tl}
%    \begin{macrocode}
\bool_new:N \l_@@_root_after_bool
\keys_define:nn { siunitx }
  {
    complex-root-position .choice: ,
    complex-root-position / after-number .code:n =
      { \bool_set_true:N \l_@@_root_after_bool } ,
    complex-root-position / before-number .code:n =
      { \bool_set_false:N \l_@@_root_after_bool } ,
    input-complex-root .tl_set:N =
      \l_@@_input_root_tl ,
    output-complex-root .tl_set:N =
      \l_@@_output_root_tl
  }
%    \end{macrocode}
% \end{variable}
%
% \subsection{Parsing}
%
% \begin{macro}{\@@_parse:nNN}
% \begin{macro}{\@@_parse_end:}
% \begin{macro}{\@@_parse_clear:}
%   Parsing for complex numbers needs some of the same approaches as the
%   general parser. However, as the aim here is to do only enough to split
%   the real and imaginary parts before handing off the the usual code,
%   it's not a full repeat. Instead, we shortcut where we can. The |clear|
%   function here is not only there to make this function shorter: it
%   also allows a single way to zap any stored data if a parse error occurs.
%    \begin{macrocode}
\cs_new_protected:Npn \@@_parse:nNN #1#2#3
  {
    \group_begin:
      \@@_parse_clear:
      \protected@edef \l_@@_arg_tl {#1}
      \tl_set_eq:NN \l_@@_input_tl \l_@@_arg_tl
      \siunitx_number_normalize_symbols:N \l_@@_arg_tl
      \tl_if_empty:NF \l_@@_arg_tl
        { \@@_parse_comparator: }
      \@@_parse_check:
      \cs_set_protected:Npx \@@_parse_end:
        {
          \tl_set:Nn \exp_not:N #2 { \exp_not:V \l_@@_real_tl }
          \tl_set:Nn \exp_not:N #3 { \exp_not:V \l_@@_img_tl }
        }
    \exp_after:wN \group_end:
    \@@_parse_end:
  }
\cs_new_protected:Npn \@@_parse_end: { }
\cs_new_protected:Npn \@@_parse_clear:
  {
    \tl_clear:N \l_@@_real_tl
    \tl_clear:N \l_@@_img_tl
    \tl_clear:N \l_@@_exp_tl
    \tl_clear:N \l_@@_sign_tl
    \tl_clear:N \l_@@_join_tl
  }
%    \end{macrocode}
% \end{macro}
% \end{macro}
% \end{macro}
%
% \begin{macro}{\@@_parse_check:, \@@_parse_finalise:}
% \begin{macro}{\@@_parse_finalise:N}
%   Now we tidy up and do the main work: passing to the standard formatter for
%   final parsing.
%    \begin{macrocode}
\cs_new_protected:Npn \@@_parse_check:
  {
    \bool_lazy_all:nTF
      {
        { \tl_if_empty_p:N \l_@@_real_tl }
        { \tl_if_empty_p:N \l_@@_img_tl }
        { \tl_if_empty_p:N \l_@@_exp_tl }
      }
      {
        \msg_error:nnx { siunitx } { invalid-complex-number }
          { \exp_not:V \l_@@_input_tl }
      }
      { \@@_parse_finalise: }
  }
\cs_new_protected:Npn \@@_parse_finalise:
  {
    \tl_if_empty:NTF \l_@@_img_tl
      { \@@_parse_finalise:N \l_@@_real_tl }
      {
        \tl_if_empty:NTF \l_@@_real_tl
          { \@@_parse_finalise:N \l_@@_img_tl }
          {
            \@@_parse_finalise:N \l_@@_real_tl
            \tl_set_eq:NN \l_@@_sign_tl \l_@@_join_tl
            \@@_parse_finalise:N \l_@@_img_tl
          }
      }
  }
\cs_new_protected:Npn \@@_parse_finalise:N #1
  {
    \tl_set:Nx #1
      {
        \exp_not:V \l_@@_comparator_tl
        \exp_not:V \l_@@_sign_tl
        \exp_not:V #1
        \exp_not:V \l_@@_exp_tl
      }
    \tl_clear:N \l_@@_comparator_tl
    \tl_clear:N \l_@@_sign_tl
    \siunitx_number_parse:VN #1 #1
  }
%    \end{macrocode}
% \end{macro}
% \end{macro}
% \end{macro}
%
% \begin{macro}{\@@_parse_comparator:}
% \begin{macro}{\@@_parse_comparator_aux:Nw}
%   The first step is to extract any comparator: this is the same as
%   for a full number parse.
%    \begin{macrocode}
\cs_new_protected:Npn \@@_parse_comparator:
  {
    \exp_after:wN \@@_parse_comparator_aux:Nw
      \l_@@_arg_tl \q_stop
  }
\cs_new_protected:Npn \@@_parse_comparator_aux:Nw #1#2 \q_stop
  {
    \tl_if_in:NnTF \l_siunitx_number_input_comparator_tl {#1}
      {
        \tl_set:Nn \l_@@_comparator_tl {#1}
        \tl_set:Nn \l_@@_arg_tl {#2}
      }
      { \tl_clear:N \l_@@_comparator_tl }
    \tl_if_empty:NF \l_@@_arg_tl
      { \@@_parse_sign: }
  }
%    \end{macrocode}
% \end{macro}
% \end{macro}
%
% \begin{macro}{\@@_parse_exponent:}
% \begin{macro}{\@@_parse_exponent_auxi:w}
% \begin{macro}{\@@_parse_exponent_auxii:nn}
%   An exponent part of a number has to come at the end and can only occur
%   once. Thus it is relatively easy to parse. The code here is a simplified
%   version of that in \pkg{siunitx-number}: we only need to find \emph{some}
%   exponent, not check on the detail. Notice that we need to retain the
%   exponent marker here: that is done using the short-lived temporary
%   variable.
%    \begin{macrocode}
\cs_new_protected:Npn \@@_parse_exponent:
  {
    \tl_if_empty:NTF \l_siunitx_number_input_exponent_tl
      { \@@_parse_root: }
      {
        \tl_set:Nx \l_@@_tmp_tl
          { \tl_head:V \l_siunitx_number_input_exponent_tl }
        \tl_map_inline:Nn \l_siunitx_number_input_exponent_tl
          {
            \tl_replace_all:NnV \l_@@_arg_tl
              {##1} \l_@@_tmp_tl
          }
        \use:x
          {
            \cs_set_protected:Npn
              \exp_not:N \@@_parse_exponent_auxi:w
              ####1 \exp_not:V \l_@@_tmp_tl
              ####2 \exp_not:V \l_@@_tmp_tl
              ####3 \exp_not:N \q_stop
          }
            { \@@_parse_exponent_auxii:nn {##1} {##2} }
        \use:x
          {
            \@@_parse_exponent_auxi:w
              \exp_not:V \l_@@_arg_tl
              \exp_not:V \l_@@_tmp_tl \exp_not:N \q_nil
              \exp_not:V \l_@@_tmp_tl \exp_not:N \q_stop
          }
      }
  }
\cs_new_protected:Npn \@@_parse_exponent_auxi:w  { }
\cs_new_protected:Npn \@@_parse_exponent_auxii:nn #1#2
  {
    \quark_if_nil:nF {#2}
      {
        \tl_set:Nn \l_@@_arg_tl {#1}
        \tl_set:Nx \l_@@_exp_tl
          { \exp_not:V \l_@@_tmp_tl \exp_not:n {#2} }
      }
    \@@_parse_root:
  }
%    \end{macrocode}
% \end{macro}
% \end{macro}
% \end{macro}
%
% \begin{macro}{\@@_parse_root:}
% \begin{macro}{\@@_parse_root_auxi:w}
% \begin{macro}{\@@_parse_root_auxii:nn}
%   Splitting at the complex root is much like splitting the exponent.
%   After dealing with the case where there is no complex root allowed,
%   use the first possible symbol to do the work.
%    \begin{macrocode}
\cs_new_protected:Npn \@@_parse_root:
  {
    \tl_if_empty:NTF \l_@@_input_root_tl
      { \tl_set_eq:NN \l_@@_real_tl \l_@@_arg_tl }
      {
        \tl_set:Nx \l_@@_tmp_tl
          { \tl_head:V \l_@@_input_root_tl }
        \tl_map_inline:Nn \l_@@_input_root_tl
          {
            \tl_replace_all:NnV \l_@@_arg_tl
              {##1} \l_@@_tmp_tl
          }
        \use:x
          {
            \cs_set_protected:Npn
              \exp_not:N \@@_parse_root_auxi:w
              ####1 \exp_not:V \l_@@_tmp_tl
              ####2 \exp_not:V \l_@@_tmp_tl
              ####3 \exp_not:N \q_stop
          }
            { \@@_parse_root_auxii:nn {##1} {##2} }
        \use:x
          {
            \@@_parse_root_auxi:w
              \exp_not:V \l_@@_arg_tl
              \exp_not:V \l_@@_tmp_tl \exp_not:N \q_nil
              \exp_not:V \l_@@_tmp_tl \exp_not:N \q_stop
          }
      }
  }
\cs_new_protected:Npn \@@_parse_root_auxi:w  { }
%    \end{macrocode}
%   This is where the business end lies. We have four possibilities:
%   \begin{itemize}
%     \item There was no complex root at all: |#2| will be |\q_nil|
%     \item All of the number is in the complex part with a leading
%       root: |#1| will be empty. This includes the case where
%       the input was \emph{just} a root symbol (plus possibly sign,
%       exponent): we need to cover that.
%     \item All of the number was before the complex root: |#2| will
%       be empty and we need to check |#1| fully to split out the two
%       parts
%     \item The input has a a real part with the complex part starting
%       with the root symbol: just the last token needs to be separated.
%   \end{itemize}
%    \begin{macrocode}
\cs_new_protected:Npn \@@_parse_root_auxii:nn #1#2
  {
    \quark_if_nil:nTF {#2}
      { \tl_set:Nn \l_@@_real_tl {#1} }
      {
        \tl_set:Nn \l_@@_img_tl {#2}
        \tl_if_blank:nTF {#1}
          {
            \tl_if_blank:nT {#2}
              { \tl_set:Nn \l_@@_img_tl { 1 } }
          }
          {
            \tl_if_blank:nTF {#2}
              { \@@_parse_split:n {#1} }
              { \@@_parse_sign_check:n {#1} }
          }
      }
  }
%    \end{macrocode}
% \end{macro}
% \end{macro}
% \end{macro}
%
% \begin{macro}{\@@_parse_sign:}
% \begin{macro}{\@@_parse_sign_aux:Nw}
%   The first token of a number after a comparator could be a sign. A quick
%   check is made and if found stored. There is no need to worry about the
%   nature of the sign: we keep them regardless.
%    \begin{macrocode}
\cs_new_protected:Npn \@@_parse_sign:
  {
    \exp_after:wN \@@_parse_sign_aux:Nw
      \l_@@_arg_tl \q_stop
  }
\cs_new_protected:Npn \@@_parse_sign_aux:Nw #1#2 \q_stop
  {
    \tl_if_in:NnT \l_siunitx_number_input_sign_tl {#1}
      {
        \tl_set:Nn \l_@@_sign_tl {#1}
        \tl_set:Nn \l_@@_arg_tl {#2}
      }
    \tl_if_empty:NF \l_@@_arg_tl
      { \@@_parse_exponent: }
  }
%    \end{macrocode}
% \end{macro}
% \end{macro}
%
% \begin{macro}{\@@_parse_sign_check:n}
% \begin{macro}{\@@_parse_sign_check:nN}
% \begin{macro}{\@@_parse_sign_check:nNw}
%   Here, we want to check that the last token in the input is a sign.
%   There cannot be anything after the sign, and there has to be at one
%   token before the sign: we can therefore signal a parsing error if we
%   need to.
%    \begin{macrocode}
\cs_new_protected:Npn \@@_parse_sign_check:n #1
  {
    \@@_parse_sign_check:nN { } #1 \q_recursion_tail \q_recursion_stop
  }
\cs_new_protected:Npn \@@_parse_sign_check:nN #1#2
  {
    \quark_if_recursion_tail_stop_do:Nn #2
      { \@@_parse_clear: }
    \tl_if_in:NnTF \l_siunitx_number_input_sign_tl {#2}
      { \@@_parse_sign_check:nNw {#1} #2 }
      { \@@_parse_sign_check:nN {#1#2} }
  }
\cs_new_protected:Npn \@@_parse_sign_check:nNw
  #1#2 #3 \q_recursion_tail \q_recursion_stop
  {
    \tl_if_blank:nTF {#3}
      {
        \tl_if_blank:nTF {#1}
          { \@@_parse_clear: }
          {
            \tl_set:Nn \l_@@_real_tl {#1}
            \tl_set:Nn \l_@@_join_tl {#2}
          }
      }
      { \@@_parse_clear: }
  }
%    \end{macrocode}
% \end{macro}
% \end{macro}
% \end{macro}
%
% \begin{macro}{\@@_parse_split:n}
% \begin{macro}{\@@_parse_split:nN}
% \begin{macro}{\@@_parse_split:w}
%   Checking for a sign inside the leading part of the number is a simple loop.
%   There is the possibility that there is no number in the imaginary part
%   needs to be allowed for. Notice that we do a check that there is some
%   real part: this covers for example an original input |++1i|, which
%   otherwise would not be trapped.
%    \begin{macrocode}
\cs_new_protected:Npn \@@_parse_split:n #1
  {
    \@@_parse_split:nN { } #1 \q_recursion_tail \q_recursion_stop
  }
\cs_new_protected:Npn \@@_parse_split:nN #1#2
  {
    \quark_if_recursion_tail_stop_do:Nn #2
      { \tl_set:Nn \l_@@_img_tl {#1} }
    \tl_if_in:NnTF \l_siunitx_number_input_sign_tl {#2}
      {
        \tl_set:Nn \l_@@_real_tl {#1}
        \tl_set:Nn \l_@@_join_tl {#2}
        \@@_parse_split:w
      }
      { \@@_parse_split:nN {#1#2} }
  }
\cs_new_protected:Npn \@@_parse_split:w #1 \q_recursion_tail \q_recursion_stop
  {
    \tl_set:Nx \l_@@_img_tl
      {
        \tl_if_blank:nTF {#1}
          { 1 }
          { \exp_not:n {#1} }
     }
    \tl_if_empty:NT \l_@@_real_tl
      { \@@_parse_clear: }
  }
%    \end{macrocode}
% \end{macro}
% \end{macro}
% \end{macro}
%
% \section{Formatting}
%
% \begin{variable}{\l_@@_bracket_close_tl, \l_@@_bracket_open_tl}
%   Purely internal for the present.
%    \begin{macrocode}
\tl_new:N \l_@@_bracket_close_tl
\tl_new:N \l_@@_bracket_open_tl
\tl_set:Nn \l_@@_bracket_open_tl { ( }
\tl_set:Nn \l_@@_bracket_close_tl { ) }
%    \end{macrocode}
% \end{variable}
%
% \begin{variable}{\l_@@_unit_tl}
%    \begin{macrocode}
\tl_new:N \l_@@_unit_tl
%    \end{macrocode}
% \end{variable}
%
% \begin{macro}{\siunitx_complex_number:n}
% \begin{macro}{\siunitx_complex_quantity:nn}
%   The work here is pretty trivial.
%    \begin{macrocode}
\cs_new_protected:Npn \siunitx_complex_number:n #1
  {
    \group_begin:
      \bool_if:NTF \l_siunitx_number_parse_bool
        {
          \@@_parse:nNN {#1} \l_@@_real_tl \l_@@_img_tl
          \@@_format:n { }
        }
        {
          \siunitx_number_format:nN {#1} \l_@@_tmp_tl
          \siunitx_print_number:V \l_@@_tmp_tl
        }
    \group_end:
  }
\cs_new_protected:Npn \siunitx_complex_quantity:nn #1#2
  {
    \group_begin:
      \bool_if:NTF \l_siunitx_number_parse_bool
        {
          \@@_parse:nNN {#1} \l_@@_real_tl \l_@@_img_tl
          \@@_format:n {#2}
        }
        { \siunitx_quantity:nn {#1} {#2} }
    \group_end:
  }
%    \end{macrocode}
% \end{macro}
% \end{macro}
%
% \begin{macro}{\@@_format:n}
% \begin{macro}{\@@_format_auxi:n}
% \begin{macro}{\@@_format_unary:nnnnnnn}
% \begin{macro}{\@@_format_auxii:n}
% \begin{macro}{\@@_drop_exponent:nnnnnnn}
% \begin{macro}{\@@_format_sign:nnnnnnn}
% \begin{macro}{\@@_extract_exponent:nw}
% \begin{macro}{\@@_extract_exponent_aux:w}
% \begin{macro}[EXP]{\@@_format_bracket:n}
%   We start here checking that there is something to do.
%    \begin{macrocode}
\cs_new_protected:Npn \@@_format:n #1
  {
    \bool_lazy_and:nnF
      { \tl_if_empty_p:N \l_@@_real_tl }
      { \tl_if_empty_p:N \l_@@_img_tl }
      { \@@_format_auxi:n {#1} }
  }
%    \end{macrocode}
%    \begin{macrocode}
\cs_new_protected:Npn \@@_format_auxi:n #1
  {
    \tl_clear:N \l_@@_tmp_tl
    \tl_if_empty:NTF \l_@@_real_tl
      {
        \@@_format_units:n {#1}
        \exp_after:wN \@@_format_unary:nnnnnnn \l_@@_img_tl
        \tl_set:Nx \l_@@_tmp_tl
          {
            \siunitx_number_output:N \l_@@_img_tl
            \exp_not:V \l_@@_output_root_tl
          }
      }
      {
        \tl_if_empty:NTF \l_@@_img_tl
          {
            \siunitx_number_process:NN \l_@@_real_tl \l_@@_real_tl
            \tl_set:Nx \l_@@_tmp_tl
              { \siunitx_number_output:N \l_@@_real_tl }
          }
          { \@@_format_auxii:n {#1} }
      }
    \tl_if_blank:nTF {#1}
      { \siunitx_print_number:V \l_@@_tmp_tl }
      { \siunitx_quantity_print:VV \l_@@_tmp_tl \l_@@_unit_tl }
  }
%    \end{macrocode}
%   An imaginary part that is exactly $1$ is omitted, with only the complex
%   root printed. That means checking and removing a lone $1$ here.
%    \begin{macrocode}
\cs_new_protected:Npn \@@_format_unary:nnnnnnn #1#2#3#4#5#6#7
  {
    \tl_set:Nx \l_@@_img_tl
      {
        \exp_not:n { {#1} {#2} }
        \tl_if_blank:nTF {#4}
          {
            \str_if_eq:nnTF {#3} { 1 }
              { { } { } }
              { \exp_not:n { {#3} {#4} } }
          }
          { \exp_not:n { {#3} {#4} } }
        \exp_not:n { {#5} {#6} {#7} }
      }
  }
%    \end{macrocode}
%   If we get to this stage we have both parts to a complex number. We
%   need to process both and do some massaging, then it's just a question
%   of reassembly with the right parts in the right places.
%    \begin{macrocode}
\cs_new_protected:Npn \@@_format_auxii:n #1
  {
    \@@_format_units:n {#1}
    \exp_after:wN \@@_drop_exponent:nnnnnnn \l_@@_real_tl
    \exp_after:wN \@@_format_sign:nnnnnnn \l_@@_img_tl
    \tl_set:Nx \l_@@_tmp_tl
      { \siunitx_number_output:NN \l_@@_img_tl \q_nil }
    \exp_after:wN \@@_extract_exponent:w \l_@@_tmp_tl \q_stop
    \tl_set:Nx \l_@@_tmp_tl
      {
        \bool_lazy_and:nnTF
          { \l_siunitx_number_bracket_ambiguous_bool }
          { ! \tl_if_empty_p:N \l_@@_exp_tl }
          { \@@_format_bracket:n }
          { \use:n }
          {
            \siunitx_number_output:N \l_@@_real_tl
            \exp_not:V \l_@@_sign_tl
            \bool_if:NF \l_@@_root_after_bool
              { \exp_not:V \l_@@_output_root_tl }
             \exp_not:V \l_@@_tmp_tl
            \bool_if:NT \l_@@_root_after_bool
              { \exp_not:V \l_@@_output_root_tl }
           }
         \exp_not:V \l_@@_exp_tl
      }
  }
%    \end{macrocode}
%   No exponent for the real part. 
%    \begin{macrocode}
\cs_new_protected:Npn \@@_drop_exponent:nnnnnnn #1#2#3#4#5#6#7
  { \tl_set:Nn \l_@@_real_tl { {#1} {#2} {#3} {#4} {#5} { } { 0 } } }
%    \end{macrocode}
%   Ensure the imaginary part has a sign, and also deal with the case
%   where there is no mantissa to print (as it is $1$).
%    \begin{macrocode}
\cs_new_protected:Npn \@@_format_sign:nnnnnnn #1#2#3#4#5#6#7
  {
    \tl_set:Nx \l_@@_img_tl
      {
        { }
        {  \tl_if_blank:nTF {#2} { + } { \exp_not:n {#2} } }
        \tl_if_blank:nTF {#4}
          {
            \str_if_eq:nnTF {#3} { 1 }
              { { } { } }
              { \exp_not:n { {#3} {#4} } }
          }
          { \exp_not:n { {#3} {#4} } }
        \exp_not:n { {#5} {#6} {#7} }
      }
  }
%    \end{macrocode}
%   Pull out the formatted exponent: we also need the sign.
%    \begin{macrocode}
\cs_new_protected:Npn \@@_extract_exponent:w
  #1 \q_nil #2 \q_nil #3 \q_nil #4 \q_nil #5 \q_nil #6 \q_nil #7 \q_nil #8
  \q_nil #9 \q_stop
  {
    \tl_set:Nn \l_@@_sign_tl {#1#2}
    \@@_extract_exponent_aux:nw {#3#4#5#6#7#8} #9 \q_stop
  }
\cs_new:Npn \@@_extract_exponent_aux:nw
  #1#2 \q_nil #3 \q_nil #4 \q_stop
  {
    \tl_set:Nn \l_@@_tmp_tl {#1#2}
    \tl_set:Nn \l_@@_exp_tl {#3#4}
  }
\cs_new_protected:Npn \@@_format_bracket:n #1
  {
    \exp_not:V \l_@@_bracket_open_tl
    #1
    \exp_not:V \l_@@_bracket_close_tl
  }
%    \end{macrocode}
% \end{macro}
% \end{macro}
% \end{macro}
% \end{macro}
% \end{macro}
% \end{macro}
% \end{macro}
% \end{macro}
% \end{macro}
%
% \begin{macro}{\@@_format_units:n}
% \begin{macro}
%   {
%     \@@_format_combine-exponent:n ,
%     \@@_format_extract-exponent:n ,
%     \@@_format_input:n
%   }
% \begin{macro}{\@@_format_extract-exponent:N}
% \begin{macro}[EXP]{\@@_extract_exp:nnnnnnn}
%   Formatting units needs to know the settings from the main module, and
%   the flow is then much the same as in \pkg{siunitx-compound}. We only
%   have to watch the fact there are two numbers to format.
%    \begin{macrocode}
\cs_new_protected:Npn \@@_format_units:n #1
  {
    \tl_if_blank:nTF {#1}
      {
        \siunitx_number_process:NN \l_@@_real_tl \l_@@_real_tl
        \siunitx_number_process:NN \l_@@_img_tl \l_@@_img_tl
      }
      {
        \use:c { @@_format_ \l_siunitx_quantity_prefix_mode_tl :n } {#1}
      }
  }
\cs_new_protected:cpn { @@_format_combine-exponent:n } #1
  {
    \tl_if_empty:NF \l_@@_real_tl
      { \siunitx_number_process:NN \l_@@_real_tl \l_@@_real_tl }
    \siunitx_number_process:NN \l_@@_img_tl \l_@@_img_tl
    \fp_set:Nn \l_@@_tmp_fp
     { \exp_after:wN \@@_extract_exp:nnnnnnn \l_@@_img_tl }
    \siunitx_unit_format_combine_exponent:nnN {#1}
      \l_@@_tmp_fp \l_@@_unit_tl
  }
\cs_new_protected:cpx { @@_format_extract-exponent:n } #1
  {
    \exp_not:N \siunitx_unit_format_extract_prefixes:nNN {#1}
      \exp_not:N \l_@@_unit_tl \exp_not:N \l_@@_tmp_fp
    \exp_not:c { @@_format_extract-exponent:N }
      \exp_not:N \l_@@_img_tl
    \exp_not:N \tl_if_empty:NF \exp_not:N \l_@@_real_tl
      {
        \exp_not:c { @@_format_extract-exponent:N }
          \exp_not:N \l_@@_real_tl
      }
  }
\cs_new_protected:cpn { @@_format_extract-exponent:N } #1
  {
    \tl_set:Nx #1
      { \siunitx_number_adjust_exponent:Nn #1 \l_@@_tmp_fp }
    \siunitx_number_process:NN #1 #1
  }
\cs_new_protected:Npn \@@_format_input:n #1
  {
    \siunitx_number_process:NN \l_@@_real_tl \l_@@_real_tl
    \siunitx_number_process:NN \l_@@_img_tl \l_@@_img_tl
    \siunitx_unit_format:nN {#1} \l_@@_unit_tl
  }
\cs_new:Npn \@@_extract_exp:nnnnnnn #1#2#3#4#5#6#7 { #6#7 }
%    \end{macrocode}
% \end{macro}
% \end{macro}
% \end{macro}
% \end{macro}
%
% \subsection{Messages}
%
%    \begin{macrocode}
\msg_new:nnnn { siunitx } { invalid-complex-number }
  { Invalid~complex-number~'#1'. }
  {
    The~input~'#1'~could~not~be~parsed~as~a~complex|number~following~the~
    format~defined~in~module~documentation.
  }
%    \end{macrocode}
%
% \subsection{Standard settings for module options}
%
% Some of these follow naturally from the point of definition
% (\foreign{e.g.}~boolean variables are always |false| to begin with),
% but for clarity everything is set here.
%    \begin{macrocode}
\keys_set:nn { siunitx }
  {
    complex-root-position = after-number ,
    input-complex-root    = ij           ,
    output-complex-root   = \mathrm { i }
  }
%    \end{macrocode}
%
%    \begin{macrocode}
%</package>
%    \end{macrocode}
%
% \end{implementation}
%
% \PrintIndex
